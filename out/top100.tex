\documentclass{article}
\usepackage{amsfonts}
\usepackage{amssymb}
\usepackage{amsmath}
\setlength\parindent{0pt}
\setlength\parskip{8pt}
\begin{document}
\newcommand{\meets}{\mathrel{\supset\!\!\!\subset}}
\newcommand{\notmeets}{\mathrel{\not\meets}}

Thm01. Let $m , n \in N$. Then if $n \neq 0$, then $$(\frac{ m}{n})^ {2}\neq 2.$$
%% (Scores {tree_length = 50, tree_depth = 10, characters = 81, tokens = 38, subsequent_dollars = 0, initial_dollars = 0, parses = 1},180)

Thm01. Let $m , n \in N$. Then $n \neq 0$, only if $$(\frac{ m}{n})^ {2}\neq 2.$$
%% (Scores {tree_length = 50, tree_depth = 10, characters = 81, tokens = 38, subsequent_dollars = 0, initial_dollars = 0, parses = 1},180)

Thm01. Let $m , n \in N$. Assume that $$n \neq 0.$$ then $$(\frac{ m}{n})^ {2}\neq 2.$$
%% (Scores {tree_length = 51, tree_depth = 10, characters = 87, tokens = 38, subsequent_dollars = 0, initial_dollars = 0, parses = 1},187)

Thm01. If $n \neq 0$, then $$(\frac{ m}{n})^ {2}\neq 2$$ for all natural numbers $m$ and $n$.
%% (Scores {tree_length = 47, tree_depth = 11, characters = 93, tokens = 39, subsequent_dollars = 0, initial_dollars = 0, parses = 1},191)

Thm01. For all natural numbers $m$ and $n$, if $n \neq 0$, then $$(\frac{ m}{n})^ {2}\neq 2.$$
%% (Scores {tree_length = 48, tree_depth = 11, characters = 94, tokens = 40, subsequent_dollars = 0, initial_dollars = 0, parses = 1},194)

Thm01. Let $m$ and $n$ be natural numbers. Then if $n \neq 0$, then $$(\frac{ m}{n})^ {2}\neq 2.$$
%% (Scores {tree_length = 47, tree_depth = 10, characters = 98, tokens = 41, subsequent_dollars = 0, initial_dollars = 0, parses = 1},197)

Thm01. Let $m$ and $n$ be natural numbers. Then $n \neq 0$, only if $$(\frac{ m}{n})^ {2}\neq 2.$$
%% (Scores {tree_length = 47, tree_depth = 10, characters = 98, tokens = 41, subsequent_dollars = 0, initial_dollars = 0, parses = 1},197)

Thm01. Let $m$ and $n$ be natural numbers. Assume that $n \neq 0$. Then $(\frac{ m}{n})^ {2}\neq 2$.
%% (Scores {tree_length = 48, tree_depth = 10, characters = 100, tokens = 41, subsequent_dollars = 0, initial_dollars = 0, parses = 1},200)

Thm01. Let $m$ and $n$ be natural numbers. Assume that $$n \neq 0.$$ then $$(\frac{ m}{n})^ {2}\neq 2.$$
%% (Scores {tree_length = 48, tree_depth = 10, characters = 104, tokens = 41, subsequent_dollars = 0, initial_dollars = 0, parses = 1},204)

Thm01. Let $m , n \in N$. Then if $n$ is not equal to $0$, then the exponentiation of the quotient of $m$ and $n$ and $2$ is not equal to $2$.
%% (Scores {tree_length = 56, tree_depth = 12, characters = 142, tokens = 50, subsequent_dollars = 0, initial_dollars = 0, parses = 1},261)

Thm01. Let $m , n \in N$. Then $n$ is not equal to $0$, only if the exponentiation of the quotient of $m$ and $n$ and $2$ is not equal to $2$.
%% (Scores {tree_length = 56, tree_depth = 12, characters = 142, tokens = 50, subsequent_dollars = 0, initial_dollars = 0, parses = 1},261)

Thm01. Let $m , n \in N$. Assume that $n$ is not equal to $0$. Then the exponentiation of the quotient of $m$ and $n$ and $2$ is not equal to $2$.
%% (Scores {tree_length = 57, tree_depth = 11, characters = 146, tokens = 50, subsequent_dollars = 0, initial_dollars = 0, parses = 1},265)

Thm01. If $n$ is not equal to $0$, then the exponentiation of the quotient of $m$ and $n$ and $2$ is not equal to $2$ for all natural numbers $m$ and $n$.
%% (Scores {tree_length = 53, tree_depth = 13, characters = 154, tokens = 51, subsequent_dollars = 0, initial_dollars = 0, parses = 1},272)

Thm01. For all natural numbers $m$ and $n$, if $n$ is not equal to $0$, then the exponentiation of the quotient of $m$ and $n$ and $2$ is not equal to $2$.
%% (Scores {tree_length = 54, tree_depth = 13, characters = 155, tokens = 52, subsequent_dollars = 0, initial_dollars = 0, parses = 1},275)

Thm01. Let $m$ and $n$ be natural numbers. Then if $n$ is not equal to $0$, then the exponentiation of the quotient of $m$ and $n$ and $2$ is not equal to $2$.
%% (Scores {tree_length = 53, tree_depth = 12, characters = 159, tokens = 53, subsequent_dollars = 0, initial_dollars = 0, parses = 1},278)

Thm01. Let $m$ and $n$ be natural numbers. Then $n$ is not equal to $0$, only if the exponentiation of the quotient of $m$ and $n$ and $2$ is not equal to $2$.
%% (Scores {tree_length = 53, tree_depth = 12, characters = 159, tokens = 53, subsequent_dollars = 0, initial_dollars = 0, parses = 1},278)

Thm01. Let $m$ and $n$ be natural numbers. Assume that $n$ is not equal to $0$. Then the exponentiation of the quotient of $m$ and $n$ and $2$ is not equal to $2$.
%% (Scores {tree_length = 54, tree_depth = 11, characters = 163, tokens = 53, subsequent_dollars = 0, initial_dollars = 0, parses = 1},282)

Thm01. If we can prove that $n$ is not equal to $0$, then we can prove that the exponentiation of the quotient of $m$ and $n$ and $2$ is not equal to $2$ for all instances $m$ and $n$ of natural numbers.
%% (Scores {tree_length = 56, tree_depth = 14, characters = 203, tokens = 61, subsequent_dollars = 0, initial_dollars = 0, parses = 1},335)

Thm01. For all instances $m$ and $n$ of natural numbers, if we can prove that $n$ is not equal to $0$, then we can prove that the exponentiation of the quotient of $m$ and $n$ and $2$ is not equal to $2$.
%% (Scores {tree_length = 57, tree_depth = 14, characters = 204, tokens = 62, subsequent_dollars = 0, initial_dollars = 0, parses = 1},338)

Thm01. Let $m$ and $n$ be instances of natural numbers. Then if we can prove that $n$ is not equal to $0$, then we can prove that the exponentiation of the quotient of $m$ and $n$ and $2$ is not equal to $2$.
%% (Scores {tree_length = 56, tree_depth = 13, characters = 208, tokens = 63, subsequent_dollars = 0, initial_dollars = 0, parses = 1},341)

Thm01. Let $m$ and $n$ be instances of natural numbers. Then we can prove that $n$ is not equal to $0$, only if we can prove that the exponentiation of the quotient of $m$ and $n$ and $2$ is not equal to $2$.
%% (Scores {tree_length = 56, tree_depth = 13, characters = 208, tokens = 63, subsequent_dollars = 0, initial_dollars = 0, parses = 1},341)

Thm01. Let $m$ and $n$ be instances of natural numbers. Assume that we can prove that $n$ is not equal to $0$. Then we can prove that the exponentiation of the quotient of $m$ and $n$ and $2$ is not equal to $2$.
%% (Scores {tree_length = 57, tree_depth = 12, characters = 212, tokens = 63, subsequent_dollars = 0, initial_dollars = 0, parses = 1},345)

Thm01a. Let $m , n \in N$. Then $$(\frac{ m}{n + 1})^ {2}\neq 2.$$
%% (Scores {tree_length = 43, tree_depth = 10, characters = 66, tokens = 32, subsequent_dollars = 0, initial_dollars = 0, parses = 1},152)

Thm01a. $$(\frac{ m}{n + 1})^ {2}\neq 2$$ for all natural numbers $m$ and $n$.
%% (Scores {tree_length = 40, tree_depth = 11, characters = 78, tokens = 33, subsequent_dollars = 0, initial_dollars = 0, parses = 1},163)

Thm01a. For all natural numbers $m$ and $n$, $$(\frac{ m}{n + 1})^ {2}\neq 2.$$
%% (Scores {tree_length = 41, tree_depth = 11, characters = 79, tokens = 34, subsequent_dollars = 0, initial_dollars = 0, parses = 1},166)

Thm01a. Let $m$ and $n$ be natural numbers. Then $(\frac{ m}{n + 1})^ {2}\neq 2$.
%% (Scores {tree_length = 40, tree_depth = 10, characters = 81, tokens = 35, subsequent_dollars = 0, initial_dollars = 0, parses = 1},167)

Thm01a. Let $m$ and $n$ be natural numbers. Then $$(\frac{ m}{n + 1})^ {2}\neq 2.$$
%% (Scores {tree_length = 40, tree_depth = 10, characters = 83, tokens = 35, subsequent_dollars = 0, initial_dollars = 0, parses = 1},169)

Thm01a. Let $m , n \in N$. Then the exponentiation of the quotient of $m$ and the sum of $n$ and $1$ and $2$ is not equal to $2$.
%% (Scores {tree_length = 52, tree_depth = 14, characters = 129, tokens = 44, subsequent_dollars = 0, initial_dollars = 0, parses = 1},240)

Thm01a. The exponentiation of the quotient of $m$ and the sum of $n$ and $1$ and $2$ is not equal to $2$ for all natural numbers $m$ and $n$.
%% (Scores {tree_length = 49, tree_depth = 15, characters = 141, tokens = 45, subsequent_dollars = 0, initial_dollars = 0, parses = 1},251)

Thm01a. For all natural numbers $m$ and $n$, the exponentiation of the quotient of $m$ and the sum of $n$ and $1$ and $2$ is not equal to $2$.
%% (Scores {tree_length = 50, tree_depth = 15, characters = 142, tokens = 46, subsequent_dollars = 0, initial_dollars = 0, parses = 1},254)

Thm01a. Let $m$ and $n$ be natural numbers. Then the exponentiation of the quotient of $m$ and the sum of $n$ and $1$ and $2$ is not equal to $2$.
%% (Scores {tree_length = 49, tree_depth = 14, characters = 146, tokens = 47, subsequent_dollars = 0, initial_dollars = 0, parses = 1},257)

Thm01a. We can prove that the exponentiation of the quotient of $m$ and the sum of $n$ and $1$ and $2$ is not equal to $2$ for all instances $m$ and $n$ of natural numbers.
%% (Scores {tree_length = 51, tree_depth = 16, characters = 172, tokens = 51, subsequent_dollars = 0, initial_dollars = 0, parses = 1},291)

Thm01a. For all instances $m$ and $n$ of natural numbers, we can prove that the exponentiation of the quotient of $m$ and the sum of $n$ and $1$ and $2$ is not equal to $2$.
%% (Scores {tree_length = 52, tree_depth = 16, characters = 173, tokens = 52, subsequent_dollars = 0, initial_dollars = 0, parses = 1},294)

Thm01a. Let $m$ and $n$ be instances of natural numbers. Then we can prove that the exponentiation of the quotient of $m$ and the sum of $n$ and $1$ and $2$ is not equal to $2$.
%% (Scores {tree_length = 51, tree_depth = 15, characters = 177, tokens = 53, subsequent_dollars = 0, initial_dollars = 0, parses = 1},297)

Thm01b. Let $q \in Q$. Then $$q ^ {2}\neq 2.$$
%% (Scores {tree_length = 29, tree_depth = 8, characters = 46, tokens = 20, subsequent_dollars = 0, initial_dollars = 0, parses = 1},104)

Thm01b. $$q ^ {2}\neq 2$$ for every rational number $q$.
%% (Scores {tree_length = 26, tree_depth = 9, characters = 56, tokens = 19, subsequent_dollars = 0, initial_dollars = 0, parses = 1},111)

Thm01b. $$q ^ {2}\neq 2$$ for all rational numbers $q$.
%% (Scores {tree_length = 27, tree_depth = 9, characters = 55, tokens = 19, subsequent_dollars = 0, initial_dollars = 0, parses = 1},111)

Thm01b. For all rational numbers $q$, $$q ^ {2}\neq 2.$$
%% (Scores {tree_length = 28, tree_depth = 9, characters = 56, tokens = 20, subsequent_dollars = 0, initial_dollars = 0, parses = 1},114)

Thm01b. Let $q$ be a rational number. Then $q ^ {2}\neq 2$.
%% (Scores {tree_length = 27, tree_depth = 8, characters = 59, tokens = 22, subsequent_dollars = 0, initial_dollars = 0, parses = 1},117)

Thm01b. Let $q$ be a rational number. Then $$q ^ {2}\neq 2.$$
%% (Scores {tree_length = 27, tree_depth = 8, characters = 61, tokens = 22, subsequent_dollars = 0, initial_dollars = 0, parses = 1},119)

Thm01b. Let $q \in Q$. Then the exponentiation of $q$ and $2$ is not equal to $2$.
%% (Scores {tree_length = 32, tree_depth = 9, characters = 82, tokens = 28, subsequent_dollars = 0, initial_dollars = 0, parses = 1},152)

Thm01b. The exponentiation of $q$ and $2$ is not equal to $2$ for every rational number $q$.
%% (Scores {tree_length = 29, tree_depth = 10, characters = 92, tokens = 27, subsequent_dollars = 0, initial_dollars = 0, parses = 1},159)

Thm01b. The exponentiation of $q$ and $2$ is not equal to $2$ for all rational numbers $q$.
%% (Scores {tree_length = 30, tree_depth = 10, characters = 91, tokens = 27, subsequent_dollars = 0, initial_dollars = 0, parses = 1},159)

Thm01b. For all rational numbers $q$, the exponentiation of $q$ and $2$ is not equal to $2$.
%% (Scores {tree_length = 31, tree_depth = 10, characters = 92, tokens = 28, subsequent_dollars = 0, initial_dollars = 0, parses = 1},162)

Thm01b. Let $q$ be a rational number. Then the exponentiation of $q$ and $2$ is not equal to $2$.
%% (Scores {tree_length = 30, tree_depth = 9, characters = 97, tokens = 30, subsequent_dollars = 0, initial_dollars = 0, parses = 1},167)

Thm01b. We can prove that the exponentiation of $q$ and $2$ is not equal to $2$ for every instance $q$ of rational numbers.
%% (Scores {tree_length = 31, tree_depth = 11, characters = 123, tokens = 33, subsequent_dollars = 0, initial_dollars = 0, parses = 1},199)

Thm01b. We can prove that the exponentiation of $q$ and $2$ is not equal to $2$ for all instances $q$ of rational numbers.
%% (Scores {tree_length = 32, tree_depth = 11, characters = 122, tokens = 33, subsequent_dollars = 0, initial_dollars = 0, parses = 1},199)

Thm01b. For all instances $q$ of rational numbers, we can prove that the exponentiation of $q$ and $2$ is not equal to $2$.
%% (Scores {tree_length = 33, tree_depth = 11, characters = 123, tokens = 34, subsequent_dollars = 0, initial_dollars = 0, parses = 1},202)

Thm01b. Let $q$ be an instance of rational numbers. Then we can prove that the exponentiation of $q$ and $2$ is not equal to $2$.
%% (Scores {tree_length = 32, tree_depth = 10, characters = 129, tokens = 36, subsequent_dollars = 0, initial_dollars = 0, parses = 1},208)

Thm01c. Let $q \in Q$. Then $$\sqrt{ 2}\neq q.$$
%% (Scores {tree_length = 26, tree_depth = 8, characters = 48, tokens = 19, subsequent_dollars = 0, initial_dollars = 0, parses = 1},102)

Thm01c. $$\sqrt{ 2}\neq q$$ for every rational number $q$.
%% (Scores {tree_length = 23, tree_depth = 9, characters = 58, tokens = 18, subsequent_dollars = 0, initial_dollars = 0, parses = 1},109)

Thm01c. $$\sqrt{ 2}\neq q$$ for all rational numbers $q$.
%% (Scores {tree_length = 24, tree_depth = 9, characters = 57, tokens = 18, subsequent_dollars = 0, initial_dollars = 0, parses = 1},109)

Thm01c. For all rational numbers $q$, $$\sqrt{ 2}\neq q.$$
%% (Scores {tree_length = 25, tree_depth = 9, characters = 58, tokens = 19, subsequent_dollars = 0, initial_dollars = 0, parses = 1},112)

Thm01c. Let $q$ be a rational number. Then $\sqrt{ 2}\neq q$.
%% (Scores {tree_length = 24, tree_depth = 8, characters = 61, tokens = 21, subsequent_dollars = 0, initial_dollars = 0, parses = 1},115)

Thm01c. Let $q$ be a rational number. Then $$\sqrt{ 2}\neq q.$$
%% (Scores {tree_length = 24, tree_depth = 8, characters = 63, tokens = 21, subsequent_dollars = 0, initial_dollars = 0, parses = 1},117)

Thm01c. Let $q \in Q$. Then the square root of $2$ is not equal to $q$.
%% (Scores {tree_length = 27, tree_depth = 8, characters = 71, tokens = 25, subsequent_dollars = 0, initial_dollars = 0, parses = 1},132)

Thm01c. The square root of $2$ is not equal to $q$ for every rational number $q$.
%% (Scores {tree_length = 24, tree_depth = 9, characters = 81, tokens = 24, subsequent_dollars = 0, initial_dollars = 0, parses = 1},139)

Thm01c. The square root of $2$ is not equal to $q$ for all rational numbers $q$.
%% (Scores {tree_length = 25, tree_depth = 9, characters = 80, tokens = 24, subsequent_dollars = 0, initial_dollars = 0, parses = 1},139)

Thm01c. For all rational numbers $q$, the square root of $2$ is not equal to $q$.
%% (Scores {tree_length = 26, tree_depth = 9, characters = 81, tokens = 25, subsequent_dollars = 0, initial_dollars = 0, parses = 1},142)

Thm01c. Let $q$ be a rational number. Then the square root of $2$ is not equal to $q$.
%% (Scores {tree_length = 25, tree_depth = 8, characters = 86, tokens = 27, subsequent_dollars = 0, initial_dollars = 0, parses = 1},147)

Thm01c. We can prove that the square root of $2$ is not equal to $q$ for every instance $q$ of rational numbers.
%% (Scores {tree_length = 26, tree_depth = 10, characters = 112, tokens = 30, subsequent_dollars = 0, initial_dollars = 0, parses = 1},179)

Thm01c. We can prove that the square root of $2$ is not equal to $q$ for all instances $q$ of rational numbers.
%% (Scores {tree_length = 27, tree_depth = 10, characters = 111, tokens = 30, subsequent_dollars = 0, initial_dollars = 0, parses = 1},179)

Thm01c. For all instances $q$ of rational numbers, we can prove that the square root of $2$ is not equal to $q$.
%% (Scores {tree_length = 28, tree_depth = 10, characters = 112, tokens = 31, subsequent_dollars = 0, initial_dollars = 0, parses = 1},182)

Thm01c. Let $q$ be an instance of rational numbers. Then we can prove that the square root of $2$ is not equal to $q$.
%% (Scores {tree_length = 27, tree_depth = 9, characters = 118, tokens = 33, subsequent_dollars = 0, initial_dollars = 0, parses = 1},188)

Thm01d. $\sqrt{ 2}$ is not rational.
%% (Scores {tree_length = 12, tree_depth = 7, characters = 36, tokens = 12, subsequent_dollars = 0, initial_dollars = 1, parses = 1},69)

Thm01d. The square root of $2$ is not rational.
%% (Scores {tree_length = 13, tree_depth = 8, characters = 47, tokens = 13, subsequent_dollars = 0, initial_dollars = 0, parses = 1},82)

Thm01d. We can prove that the square root of $2$ is not rational.
%% (Scores {tree_length = 14, tree_depth = 9, characters = 65, tokens = 17, subsequent_dollars = 0, initial_dollars = 0, parses = 1},106)

Thm01e. $\sqrt{ 2}$ is irrational.
%% (Scores {tree_length = 12, tree_depth = 7, characters = 34, tokens = 11, subsequent_dollars = 0, initial_dollars = 1, parses = 1},66)

Thm01e. The square root of $2$ is irrational.
%% (Scores {tree_length = 13, tree_depth = 8, characters = 45, tokens = 12, subsequent_dollars = 0, initial_dollars = 0, parses = 1},79)

Thm01e. We can prove that the square root of $2$ is irrational.
%% (Scores {tree_length = 14, tree_depth = 9, characters = 63, tokens = 16, subsequent_dollars = 0, initial_dollars = 0, parses = 1},103)

Thm01f. There exist no natural numbers $p$ and $q$, such that $p ^{ 2}= 2 (q ^{ 2})$.
%% (Scores {tree_length = 34, tree_depth = 10, characters = 85, tokens = 34, subsequent_dollars = 0, initial_dollars = 0, parses = 1},164)

Thm01f. There exist no natural numbers $p$ and $q$, such that $$p ^{ 2}= 2 (q ^{ 2}).$$
%% (Scores {tree_length = 34, tree_depth = 10, characters = 87, tokens = 34, subsequent_dollars = 0, initial_dollars = 0, parses = 1},166)

Thm01f. There exists no natural number $p$, such that $p ^{ 2}= 2 (q ^{ 2})$ for a natural number $q$.
%% (Scores {tree_length = 37, tree_depth = 11, characters = 102, tokens = 37, subsequent_dollars = 0, initial_dollars = 0, parses = 1},188)

Thm01f. There exists no natural number $p$, such that $p ^{ 2}= 2 (q ^{ 2})$ for some natural number $q$.
%% (Scores {tree_length = 38, tree_depth = 11, characters = 105, tokens = 37, subsequent_dollars = 0, initial_dollars = 0, parses = 1},192)

Thm01f. It is not the case that there exist natural numbers $p$ and $q$, such that $p ^{ 2}= 2 (q ^{ 2})$.
%% (Scores {tree_length = 35, tree_depth = 11, characters = 106, tokens = 39, subsequent_dollars = 0, initial_dollars = 0, parses = 1},192)

Thm01f. There exists no natural number $p$, such that there exists a natural number $q$, such that $p ^{ 2}= 2 (q ^{ 2})$.
%% (Scores {tree_length = 39, tree_depth = 11, characters = 122, tokens = 41, subsequent_dollars = 0, initial_dollars = 0, parses = 1},214)

Thm01f. It is not the case that there exists a natural number $p$, such that there exists a natural number $q$, such that $p ^{ 2}= 2 (q ^{ 2})$.
%% (Scores {tree_length = 40, tree_depth = 12, characters = 145, tokens = 47, subsequent_dollars = 0, initial_dollars = 0, parses = 1},245)

Thm01f. There exists no natural number $p$, such that the square of $p$ is equal to the product of $2$ and the square of $q$ for a natural number $q$.
%% (Scores {tree_length = 43, tree_depth = 14, characters = 150, tokens = 43, subsequent_dollars = 0, initial_dollars = 0, parses = 1},251)

Thm01f. There exists no natural number $p$, such that the square of $p$ is equal to the product of $2$ and the square of $q$ for some natural number $q$.
%% (Scores {tree_length = 44, tree_depth = 14, characters = 153, tokens = 43, subsequent_dollars = 0, initial_dollars = 0, parses = 1},255)

Thm01f. We can prove that there exists no natural number $p$, such that the square of $p$ is equal to the product of $2$ and the square of $q$ for a natural number $q$.
%% (Scores {tree_length = 44, tree_depth = 15, characters = 168, tokens = 47, subsequent_dollars = 0, initial_dollars = 0, parses = 1},275)

Thm01f. There exists no natural number $p$, such that there exists a natural number $q$, such that the square of $p$ is equal to the product of $2$ and the square of $q$.
%% (Scores {tree_length = 45, tree_depth = 14, characters = 170, tokens = 47, subsequent_dollars = 0, initial_dollars = 0, parses = 1},277)

Thm01f. We can prove that there exists no natural number $p$, such that the square of $p$ is equal to the product of $2$ and the square of $q$ for some natural number $q$.
%% (Scores {tree_length = 45, tree_depth = 15, characters = 171, tokens = 47, subsequent_dollars = 0, initial_dollars = 0, parses = 1},279)

Thm01f. We can prove that there exists no natural number $p$, such that there exists a natural number $q$, such that the square of $p$ is equal to the product of $2$ and the square of $q$.
%% (Scores {tree_length = 46, tree_depth = 15, characters = 188, tokens = 51, subsequent_dollars = 0, initial_dollars = 0, parses = 1},301)

Thm01f. It is not the case that there exists a natural number $p$, such that there exists a natural number $q$, such that the square of $p$ is equal to the product of $2$ and the square of $q$.
%% (Scores {tree_length = 46, tree_depth = 15, characters = 193, tokens = 53, subsequent_dollars = 0, initial_dollars = 0, parses = 1},308)

Thm01f. We can prove that it is not the case that there exists a natural number $p$, such that there exists a natural number $q$, such that the square of $p$ is equal to the product of $2$ and the square of $q$.
%% (Scores {tree_length = 47, tree_depth = 16, characters = 211, tokens = 57, subsequent_dollars = 0, initial_dollars = 0, parses = 1},332)

Thm02. If the degree of $P$ is greater than $0$, then $c$ is a root of $P$ for a complex number $c$ for every polynomial $P$.
%% (Scores {tree_length = 41, tree_depth = 10, characters = 125, tokens = 41, subsequent_dollars = 0, initial_dollars = 0, parses = 1},218)

Thm02. If the degree of $P$ is greater than $0$, then $c$ is a root of $P$ for a complex number $c$ for all polynomials $P$.
%% (Scores {tree_length = 42, tree_depth = 10, characters = 124, tokens = 41, subsequent_dollars = 0, initial_dollars = 0, parses = 1},218)

Thm02. For all polynomials $P$, if the degree of $P$ is greater than $0$, then $c$ is a root of $P$ for a complex number $c$.
%% (Scores {tree_length = 43, tree_depth = 10, characters = 125, tokens = 42, subsequent_dollars = 0, initial_dollars = 0, parses = 1},221)

Thm02. If the degree of $P$ is greater than $0$, then $c$ is a root of $P$ for some complex number $c$ for every polynomial $P$.
%% (Scores {tree_length = 42, tree_depth = 10, characters = 128, tokens = 41, subsequent_dollars = 0, initial_dollars = 0, parses = 1},222)

Thm02. If the degree of $P$ is greater than $0$, then $c$ is a root of $P$ for some complex number $c$ for all polynomials $P$.
%% (Scores {tree_length = 43, tree_depth = 10, characters = 127, tokens = 41, subsequent_dollars = 0, initial_dollars = 0, parses = 1},222)

Thm02. For all polynomials $P$, if the degree of $P$ is greater than $0$, then $c$ is a root of $P$ for some complex number $c$.
%% (Scores {tree_length = 44, tree_depth = 10, characters = 128, tokens = 42, subsequent_dollars = 0, initial_dollars = 0, parses = 1},225)

Thm02. Let $P$ be a polynomial. Then if the degree of $P$ is greater than $0$, then $c$ is a root of $P$ for a complex number $c$.
%% (Scores {tree_length = 42, tree_depth = 9, characters = 130, tokens = 44, subsequent_dollars = 0, initial_dollars = 0, parses = 1},226)

Thm02. Let $P$ be a polynomial. Then the degree of $P$ is greater than $0$, only if $c$ is a root of $P$ for a complex number $c$.
%% (Scores {tree_length = 42, tree_depth = 9, characters = 130, tokens = 44, subsequent_dollars = 0, initial_dollars = 0, parses = 1},226)

Thm02. Let $P$ be a polynomial. Then if the degree of $P$ is greater than $0$, then $c$ is a root of $P$ for some complex number $c$.
%% (Scores {tree_length = 43, tree_depth = 9, characters = 133, tokens = 44, subsequent_dollars = 0, initial_dollars = 0, parses = 1},230)

Thm02. Let $P$ be a polynomial. Then the degree of $P$ is greater than $0$, only if $c$ is a root of $P$ for some complex number $c$.
%% (Scores {tree_length = 43, tree_depth = 9, characters = 133, tokens = 44, subsequent_dollars = 0, initial_dollars = 0, parses = 1},230)

Thm02. Let $P$ be a polynomial. Assume that the degree of $P$ is greater than $0$. Then $c$ is a root of $P$ for a complex number $c$.
%% (Scores {tree_length = 43, tree_depth = 11, characters = 134, tokens = 44, subsequent_dollars = 0, initial_dollars = 0, parses = 1},233)

Thm02. Let $P$ be a polynomial. Assume that the degree of $P$ is greater than $0$. Then $c$ is a root of $P$ for some complex number $c$.
%% (Scores {tree_length = 44, tree_depth = 11, characters = 137, tokens = 44, subsequent_dollars = 0, initial_dollars = 0, parses = 1},237)

Thm02. If the degree of $P$ is greater than $0$, then there exists a complex number $c$, such that $c$ is a root of $P$ for every polynomial $P$.
%% (Scores {tree_length = 43, tree_depth = 10, characters = 145, tokens = 45, subsequent_dollars = 0, initial_dollars = 0, parses = 1},244)

Thm02. If the degree of $P$ is greater than $0$, then there exists a complex number $c$, such that $c$ is a root of $P$ for all polynomials $P$.
%% (Scores {tree_length = 44, tree_depth = 10, characters = 144, tokens = 45, subsequent_dollars = 0, initial_dollars = 0, parses = 1},244)

Thm02. For all polynomials $P$, if the degree of $P$ is greater than $0$, then there exists a complex number $c$, such that $c$ is a root of $P$.
%% (Scores {tree_length = 45, tree_depth = 10, characters = 145, tokens = 46, subsequent_dollars = 0, initial_dollars = 0, parses = 1},247)

Thm02. Let $P$ be a polynomial. Then if the degree of $P$ is greater than $0$, then there exists a complex number $c$, such that $c$ is a root of $P$.
%% (Scores {tree_length = 44, tree_depth = 9, characters = 150, tokens = 48, subsequent_dollars = 0, initial_dollars = 0, parses = 1},252)

Thm02. Let $P$ be a polynomial. Then the degree of $P$ is greater than $0$, only if there exists a complex number $c$, such that $c$ is a root of $P$.
%% (Scores {tree_length = 44, tree_depth = 9, characters = 150, tokens = 48, subsequent_dollars = 0, initial_dollars = 0, parses = 1},252)

Thm02. Let $P$ be a polynomial. Assume that the degree of $P$ is greater than $0$. Then there exists a complex number $c$, such that $c$ is a root of $P$.
%% (Scores {tree_length = 45, tree_depth = 11, characters = 154, tokens = 48, subsequent_dollars = 0, initial_dollars = 0, parses = 1},259)

Thm02. If we can prove that the degree of $P$ is greater than $0$, then we can prove that $c$ is a root of $P$ for a complex number $c$ for every polynomial $P$.
%% (Scores {tree_length = 43, tree_depth = 11, characters = 161, tokens = 49, subsequent_dollars = 0, initial_dollars = 0, parses = 1},265)

Thm02. If we can prove that the degree of $P$ is greater than $0$, then we can prove that $c$ is a root of $P$ for a complex number $c$ for all polynomials $P$.
%% (Scores {tree_length = 44, tree_depth = 11, characters = 160, tokens = 49, subsequent_dollars = 0, initial_dollars = 0, parses = 1},265)

Thm02. For all polynomials $P$, if we can prove that the degree of $P$ is greater than $0$, then we can prove that $c$ is a root of $P$ for a complex number $c$.
%% (Scores {tree_length = 45, tree_depth = 11, characters = 161, tokens = 50, subsequent_dollars = 0, initial_dollars = 0, parses = 1},268)

Thm02. If we can prove that the degree of $P$ is greater than $0$, then we can prove that $c$ is a root of $P$ for some complex number $c$ for every polynomial $P$.
%% (Scores {tree_length = 44, tree_depth = 11, characters = 164, tokens = 49, subsequent_dollars = 0, initial_dollars = 0, parses = 1},269)

Thm02. If we can prove that the degree of $P$ is greater than $0$, then we can prove that $c$ is a root of $P$ for some complex number $c$ for all polynomials $P$.
%% (Scores {tree_length = 45, tree_depth = 11, characters = 163, tokens = 49, subsequent_dollars = 0, initial_dollars = 0, parses = 1},269)

Thm02. For all polynomials $P$, if we can prove that the degree of $P$ is greater than $0$, then we can prove that $c$ is a root of $P$ for some complex number $c$.
%% (Scores {tree_length = 46, tree_depth = 11, characters = 164, tokens = 50, subsequent_dollars = 0, initial_dollars = 0, parses = 1},272)

Thm02. Let $P$ be a polynomial. Then if we can prove that the degree of $P$ is greater than $0$, then we can prove that $c$ is a root of $P$ for a complex number $c$.
%% (Scores {tree_length = 44, tree_depth = 10, characters = 166, tokens = 52, subsequent_dollars = 0, initial_dollars = 0, parses = 1},273)

Thm02. Let $P$ be a polynomial. Then we can prove that the degree of $P$ is greater than $0$, only if we can prove that $c$ is a root of $P$ for a complex number $c$.
%% (Scores {tree_length = 44, tree_depth = 10, characters = 166, tokens = 52, subsequent_dollars = 0, initial_dollars = 0, parses = 1},273)

Thm02. Let $P$ be a polynomial. Then if we can prove that the degree of $P$ is greater than $0$, then we can prove that $c$ is a root of $P$ for some complex number $c$.
%% (Scores {tree_length = 45, tree_depth = 10, characters = 169, tokens = 52, subsequent_dollars = 0, initial_dollars = 0, parses = 1},277)

Thm02. Let $P$ be a polynomial. Then we can prove that the degree of $P$ is greater than $0$, only if we can prove that $c$ is a root of $P$ for some complex number $c$.
%% (Scores {tree_length = 45, tree_depth = 10, characters = 169, tokens = 52, subsequent_dollars = 0, initial_dollars = 0, parses = 1},277)

Thm02. Let $P$ be a polynomial. Assume that we can prove that the degree of $P$ is greater than $0$. Then we can prove that $c$ is a root of $P$ for a complex number $c$.
%% (Scores {tree_length = 45, tree_depth = 12, characters = 170, tokens = 52, subsequent_dollars = 0, initial_dollars = 0, parses = 1},280)

Thm02. Let $P$ be a polynomial. Assume that we can prove that the degree of $P$ is greater than $0$. Then we can prove that $c$ is a root of $P$ for some complex number $c$.
%% (Scores {tree_length = 46, tree_depth = 12, characters = 173, tokens = 52, subsequent_dollars = 0, initial_dollars = 0, parses = 1},284)

Thm02. If we can prove that the degree of $P$ is greater than $0$, then we can prove that there exists a complex number $c$, such that $c$ is a root of $P$ for every polynomial $P$.
%% (Scores {tree_length = 45, tree_depth = 11, characters = 181, tokens = 53, subsequent_dollars = 0, initial_dollars = 0, parses = 1},291)

Thm02. If we can prove that the degree of $P$ is greater than $0$, then we can prove that there exists a complex number $c$, such that $c$ is a root of $P$ for all polynomials $P$.
%% (Scores {tree_length = 46, tree_depth = 11, characters = 180, tokens = 53, subsequent_dollars = 0, initial_dollars = 0, parses = 1},291)

Thm02. For all polynomials $P$, if we can prove that the degree of $P$ is greater than $0$, then we can prove that there exists a complex number $c$, such that $c$ is a root of $P$.
%% (Scores {tree_length = 47, tree_depth = 11, characters = 181, tokens = 54, subsequent_dollars = 0, initial_dollars = 0, parses = 1},294)

Thm02. Let $P$ be a polynomial. Then if we can prove that the degree of $P$ is greater than $0$, then we can prove that there exists a complex number $c$, such that $c$ is a root of $P$.
%% (Scores {tree_length = 46, tree_depth = 10, characters = 186, tokens = 56, subsequent_dollars = 0, initial_dollars = 0, parses = 1},299)

Thm02. Let $P$ be a polynomial. Then we can prove that the degree of $P$ is greater than $0$, only if we can prove that there exists a complex number $c$, such that $c$ is a root of $P$.
%% (Scores {tree_length = 46, tree_depth = 10, characters = 186, tokens = 56, subsequent_dollars = 0, initial_dollars = 0, parses = 1},299)

Thm02. Let $P$ be a polynomial. Assume that we can prove that the degree of $P$ is greater than $0$. Then we can prove that there exists a complex number $c$, such that $c$ is a root of $P$.
%% (Scores {tree_length = 47, tree_depth = 12, characters = 190, tokens = 56, subsequent_dollars = 0, initial_dollars = 0, parses = 1},306)

Thm03. $Rat$ is denumerable.
%% (Scores {tree_length = 10, tree_depth = 6, characters = 28, tokens = 8, subsequent_dollars = 0, initial_dollars = 1, parses = 1},54)

Thm03. We can prove that $Rat$ is denumerable.
%% (Scores {tree_length = 11, tree_depth = 7, characters = 46, tokens = 12, subsequent_dollars = 0, initial_dollars = 0, parses = 1},77)

Thm03a. $| Nat | = | Rat |$.
%% (Scores {tree_length = 19, tree_depth = 8, characters = 28, tokens = 12, subsequent_dollars = 0, initial_dollars = 1, parses = 1},69)

Thm03a. $$| Nat | = | Rat |.$$
%% (Scores {tree_length = 19, tree_depth = 8, characters = 30, tokens = 11, subsequent_dollars = 0, initial_dollars = 0, parses = 1},69)

Thm03a. The cardinality of $Nat$ is equal to the cardinality of $Rat$.
%% (Scores {tree_length = 21, tree_depth = 9, characters = 70, tokens = 18, subsequent_dollars = 0, initial_dollars = 0, parses = 1},119)

Thm03a. We can prove that the cardinality of $Nat$ is equal to the cardinality of $Rat$.
%% (Scores {tree_length = 22, tree_depth = 10, characters = 88, tokens = 22, subsequent_dollars = 0, initial_dollars = 0, parses = 1},143)

Thm04. If $u \perp v$, then $$\| u + v \| = \sqrt{ \| u \| ^{ 2}+ \| v \| ^{ 2}}$$ for all vectors $u$ and $v$.
%% (Scores {tree_length = 59, tree_depth = 13, characters = 111, tokens = 47, subsequent_dollars = 0, initial_dollars = 0, parses = 1},231)

Thm04. For all vectors $u$ and $v$, if $u \perp v$, then $$\| u + v \| = \sqrt{ \| u \| ^{ 2}+ \| v \| ^{ 2}}.$$
%% (Scores {tree_length = 60, tree_depth = 13, characters = 112, tokens = 48, subsequent_dollars = 0, initial_dollars = 0, parses = 1},234)

Thm04. Let $u$ and $v$ be vectors. Then if $u \perp v$, then $$\| u + v \| = \sqrt{ \| u \| ^{ 2}+ \| v \| ^{ 2}}.$$
%% (Scores {tree_length = 59, tree_depth = 12, characters = 116, tokens = 49, subsequent_dollars = 0, initial_dollars = 0, parses = 1},237)

Thm04. Let $u$ and $v$ be vectors. Then $u \perp v$, only if $$\| u + v \| = \sqrt{ \| u \| ^{ 2}+ \| v \| ^{ 2}}.$$
%% (Scores {tree_length = 59, tree_depth = 12, characters = 116, tokens = 49, subsequent_dollars = 0, initial_dollars = 0, parses = 1},237)

Thm04. Let $u$ and $v$ be vectors. Assume that $u \perp v$. Then $\| u + v \| = \sqrt{ \| u \| ^{ 2}+ \| v \| ^{ 2}}$.
%% (Scores {tree_length = 60, tree_depth = 11, characters = 118, tokens = 49, subsequent_dollars = 0, initial_dollars = 0, parses = 1},239)

Thm04. Let $u$ and $v$ be vectors. Assume that $$u \perp v.$$ then $$\| u + v \| = \sqrt{ \| u \| ^{ 2}+ \| v \| ^{ 2}}.$$
%% (Scores {tree_length = 60, tree_depth = 11, characters = 122, tokens = 49, subsequent_dollars = 0, initial_dollars = 0, parses = 1},243)

Thm04. If $u$ is perpendicular to $v$, then the length of the resultant of $u$ and $v$ is equal to the square root of the sum of the square of the length of $u$ and the square of the length of $v$ for all vectors $u$ and $v$.
%% (Scores {tree_length = 71, tree_depth = 18, characters = 225, tokens = 67, subsequent_dollars = 0, initial_dollars = 0, parses = 1},382)

Thm04. For all vectors $u$ and $v$, if $u$ is perpendicular to $v$, then the length of the resultant of $u$ and $v$ is equal to the square root of the sum of the square of the length of $u$ and the square of the length of $v$.
%% (Scores {tree_length = 72, tree_depth = 18, characters = 226, tokens = 68, subsequent_dollars = 0, initial_dollars = 0, parses = 1},385)

Thm04. Let $u$ and $v$ be vectors. Then if $u$ is perpendicular to $v$, then the length of the resultant of $u$ and $v$ is equal to the square root of the sum of the square of the length of $u$ and the square of the length of $v$.
%% (Scores {tree_length = 71, tree_depth = 17, characters = 230, tokens = 69, subsequent_dollars = 0, initial_dollars = 0, parses = 1},388)

Thm04. Let $u$ and $v$ be vectors. Then $u$ is perpendicular to $v$, only if the length of the resultant of $u$ and $v$ is equal to the square root of the sum of the square of the length of $u$ and the square of the length of $v$.
%% (Scores {tree_length = 71, tree_depth = 17, characters = 230, tokens = 69, subsequent_dollars = 0, initial_dollars = 0, parses = 1},388)

Thm04. Let $u$ and $v$ be vectors. Assume that $u$ is perpendicular to $v$. Then the length of the resultant of $u$ and $v$ is equal to the square root of the sum of the square of the length of $u$ and the square of the length of $v$.
%% (Scores {tree_length = 72, tree_depth = 16, characters = 234, tokens = 69, subsequent_dollars = 0, initial_dollars = 0, parses = 1},392)

Thm04. If we can prove that $u$ is perpendicular to $v$, then we can prove that the length of the resultant of $u$ and $v$ is equal to the square root of the sum of the square of the length of $u$ and the square of the length of $v$ for all instances $u$ and $v$ of vectors.
%% (Scores {tree_length = 74, tree_depth = 19, characters = 274, tokens = 77, subsequent_dollars = 0, initial_dollars = 0, parses = 1},445)

Thm04. For all instances $u$ and $v$ of vectors, if we can prove that $u$ is perpendicular to $v$, then we can prove that the length of the resultant of $u$ and $v$ is equal to the square root of the sum of the square of the length of $u$ and the square of the length of $v$.
%% (Scores {tree_length = 75, tree_depth = 19, characters = 275, tokens = 78, subsequent_dollars = 0, initial_dollars = 0, parses = 1},448)

Thm04. Let $u$ and $v$ be instances of vectors. Then if we can prove that $u$ is perpendicular to $v$, then we can prove that the length of the resultant of $u$ and $v$ is equal to the square root of the sum of the square of the length of $u$ and the square of the length of $v$.
%% (Scores {tree_length = 74, tree_depth = 18, characters = 279, tokens = 79, subsequent_dollars = 0, initial_dollars = 0, parses = 1},451)

Thm04. Let $u$ and $v$ be instances of vectors. Then we can prove that $u$ is perpendicular to $v$, only if we can prove that the length of the resultant of $u$ and $v$ is equal to the square root of the sum of the square of the length of $u$ and the square of the length of $v$.
%% (Scores {tree_length = 74, tree_depth = 18, characters = 279, tokens = 79, subsequent_dollars = 0, initial_dollars = 0, parses = 1},451)

Thm04. Let $u$ and $v$ be instances of vectors. Assume that we can prove that $u$ is perpendicular to $v$. Then we can prove that the length of the resultant of $u$ and $v$ is equal to the square root of the sum of the square of the length of $u$ and the square of the length of $v$.
%% (Scores {tree_length = 75, tree_depth = 17, characters = 283, tokens = 79, subsequent_dollars = 0, initial_dollars = 0, parses = 1},455)

Thm07. Let $p , q \in N$. Then if $p$ and $q$ are prime, then $$\left(\frac{ p }{ q }\right) \left(\frac{ q }{ p }\right) = (- 1)^ {\frac{ p - 1}{2}\frac{ q - 1}{2}}.$$
%% (Scores {tree_length = 85, tree_depth = 12, characters = 168, tokens = 75, subsequent_dollars = 0, initial_dollars = 0, parses = 1},341)

Thm07. Let $p , q \in N$. Then $p$ and $q$ are prime, only if $$\left(\frac{ p }{ q }\right) \left(\frac{ q }{ p }\right) = (- 1)^ {\frac{ p - 1}{2}\frac{ q - 1}{2}}.$$
%% (Scores {tree_length = 85, tree_depth = 12, characters = 168, tokens = 75, subsequent_dollars = 0, initial_dollars = 0, parses = 1},341)

Thm07. Let $p , q \in N$. Assume that $p$ and $q$ are prime. Then $$\left(\frac{ p }{ q }\right) \left(\frac{ q }{ p }\right) = (- 1)^ {\frac{ p - 1}{2}\frac{ q - 1}{2}}.$$
%% (Scores {tree_length = 86, tree_depth = 11, characters = 172, tokens = 75, subsequent_dollars = 0, initial_dollars = 0, parses = 1},345)

Thm07. Let $p , q \in N$. Assume that both $p$ and $q$ are prime. Then $$\left(\frac{ p }{ q }\right) \left(\frac{ q }{ p }\right) = (- 1)^ {\frac{ p - 1}{2}\frac{ q - 1}{2}}.$$
%% (Scores {tree_length = 85, tree_depth = 11, characters = 177, tokens = 76, subsequent_dollars = 0, initial_dollars = 0, parses = 1},350)

Thm07. If $p$ and $q$ are prime, then $$\left(\frac{ p }{ q }\right) \left(\frac{ q }{ p }\right) = (- 1)^ {\frac{ p - 1}{2}\frac{ q - 1}{2}}$$ for all natural numbers $p$ and $q$.
%% (Scores {tree_length = 82, tree_depth = 13, characters = 180, tokens = 76, subsequent_dollars = 0, initial_dollars = 0, parses = 1},352)

Thm07. Let $p , q \in N$. Then if $p$ is prime and $q$ is prime, then $$\left(\frac{ p }{ q }\right) \left(\frac{ q }{ p }\right) = (- 1)^ {\frac{ p - 1}{2}\frac{ q - 1}{2}}.$$
%% (Scores {tree_length = 87, tree_depth = 12, characters = 176, tokens = 77, subsequent_dollars = 0, initial_dollars = 0, parses = 1},353)

Thm07. Let $p , q \in N$. Then $p$ is prime and $q$ is prime, only if $$\left(\frac{ p }{ q }\right) \left(\frac{ q }{ p }\right) = (- 1)^ {\frac{ p - 1}{2}\frac{ q - 1}{2}}.$$
%% (Scores {tree_length = 87, tree_depth = 12, characters = 176, tokens = 77, subsequent_dollars = 0, initial_dollars = 0, parses = 1},353)

Thm07. For all natural numbers $p$ and $q$, if $p$ and $q$ are prime, then $$\left(\frac{ p }{ q }\right) \left(\frac{ q }{ p }\right) = (- 1)^ {\frac{ p - 1}{2}\frac{ q - 1}{2}}.$$
%% (Scores {tree_length = 83, tree_depth = 13, characters = 181, tokens = 77, subsequent_dollars = 0, initial_dollars = 0, parses = 1},355)

Thm07. Let $p , q \in N$. Assume that $p$ is prime and $q$ is prime. Then $$\left(\frac{ p }{ q }\right) \left(\frac{ q }{ p }\right) = (- 1)^ {\frac{ p - 1}{2}\frac{ q - 1}{2}}.$$
%% (Scores {tree_length = 88, tree_depth = 11, characters = 180, tokens = 77, subsequent_dollars = 0, initial_dollars = 0, parses = 1},357)

Thm07. Let $p$ and $q$ be natural numbers. Then if $p$ and $q$ are prime, then $$\left(\frac{ p }{ q }\right) \left(\frac{ q }{ p }\right) = (- 1)^ {\frac{ p - 1}{2}\frac{ q - 1}{2}}.$$
%% (Scores {tree_length = 82, tree_depth = 12, characters = 185, tokens = 78, subsequent_dollars = 0, initial_dollars = 0, parses = 1},358)

Thm07. Let $p$ and $q$ be natural numbers. Then $p$ and $q$ are prime, only if $$\left(\frac{ p }{ q }\right) \left(\frac{ q }{ p }\right) = (- 1)^ {\frac{ p - 1}{2}\frac{ q - 1}{2}}.$$
%% (Scores {tree_length = 82, tree_depth = 12, characters = 185, tokens = 78, subsequent_dollars = 0, initial_dollars = 0, parses = 1},358)

Thm07. Let $p$ and $q$ be natural numbers. Assume that $p$ and $q$ are prime. Then $\left(\frac{ p }{ q }\right) \left(\frac{ q }{ p }\right) = (- 1)^ {\frac{ p - 1}{2}\frac{ q - 1}{2}}$.
%% (Scores {tree_length = 83, tree_depth = 11, characters = 187, tokens = 78, subsequent_dollars = 0, initial_dollars = 0, parses = 1},360)

Thm07. Let $p , q \in N$. Assume that both $p$ is prime and $q$ is prime. Then $$\left(\frac{ p }{ q }\right) \left(\frac{ q }{ p }\right) = (- 1)^ {\frac{ p - 1}{2}\frac{ q - 1}{2}}.$$
%% (Scores {tree_length = 87, tree_depth = 11, characters = 185, tokens = 78, subsequent_dollars = 0, initial_dollars = 0, parses = 1},362)

Thm07. Let $p$ and $q$ be natural numbers. Assume that $p$ and $q$ are prime. Then $$\left(\frac{ p }{ q }\right) \left(\frac{ q }{ p }\right) = (- 1)^ {\frac{ p - 1}{2}\frac{ q - 1}{2}}.$$
%% (Scores {tree_length = 83, tree_depth = 11, characters = 189, tokens = 78, subsequent_dollars = 0, initial_dollars = 0, parses = 1},362)

Thm07. If $p$ is prime and $q$ is prime, then $$\left(\frac{ p }{ q }\right) \left(\frac{ q }{ p }\right) = (- 1)^ {\frac{ p - 1}{2}\frac{ q - 1}{2}}$$ for all natural numbers $p$ and $q$.
%% (Scores {tree_length = 84, tree_depth = 13, characters = 188, tokens = 78, subsequent_dollars = 0, initial_dollars = 0, parses = 1},364)

Thm07. For all natural numbers $p$ and $q$, if $p$ is prime and $q$ is prime, then $$\left(\frac{ p }{ q }\right) \left(\frac{ q }{ p }\right) = (- 1)^ {\frac{ p - 1}{2}\frac{ q - 1}{2}}.$$
%% (Scores {tree_length = 85, tree_depth = 13, characters = 189, tokens = 79, subsequent_dollars = 0, initial_dollars = 0, parses = 1},367)

Thm07. Let $p$ and $q$ be natural numbers. Assume that both $p$ and $q$ are prime. Then $$\left(\frac{ p }{ q }\right) \left(\frac{ q }{ p }\right) = (- 1)^ {\frac{ p - 1}{2}\frac{ q - 1}{2}}.$$
%% (Scores {tree_length = 82, tree_depth = 11, characters = 194, tokens = 79, subsequent_dollars = 0, initial_dollars = 0, parses = 1},367)

Thm07. Let $p$ and $q$ be natural numbers. Then if $p$ is prime and $q$ is prime, then $$\left(\frac{ p }{ q }\right) \left(\frac{ q }{ p }\right) = (- 1)^ {\frac{ p - 1}{2}\frac{ q - 1}{2}}.$$
%% (Scores {tree_length = 84, tree_depth = 12, characters = 193, tokens = 80, subsequent_dollars = 0, initial_dollars = 0, parses = 1},370)

Thm07. Let $p$ and $q$ be natural numbers. Then $p$ is prime and $q$ is prime, only if $$\left(\frac{ p }{ q }\right) \left(\frac{ q }{ p }\right) = (- 1)^ {\frac{ p - 1}{2}\frac{ q - 1}{2}}.$$
%% (Scores {tree_length = 84, tree_depth = 12, characters = 193, tokens = 80, subsequent_dollars = 0, initial_dollars = 0, parses = 1},370)

Thm07. Let $p$ and $q$ be natural numbers. Assume that $p$ is prime and $q$ is prime. Then $\left(\frac{ p }{ q }\right) \left(\frac{ q }{ p }\right) = (- 1)^ {\frac{ p - 1}{2}\frac{ q - 1}{2}}$.
%% (Scores {tree_length = 85, tree_depth = 11, characters = 195, tokens = 80, subsequent_dollars = 0, initial_dollars = 0, parses = 1},372)

Thm07. Let $p$ and $q$ be natural numbers. Assume that $p$ is prime and $q$ is prime. Then $$\left(\frac{ p }{ q }\right) \left(\frac{ q }{ p }\right) = (- 1)^ {\frac{ p - 1}{2}\frac{ q - 1}{2}}.$$
%% (Scores {tree_length = 85, tree_depth = 11, characters = 197, tokens = 80, subsequent_dollars = 0, initial_dollars = 0, parses = 1},374)

Thm07. Let $p$ and $q$ be natural numbers. Assume that both $p$ is prime and $q$ is prime. Then $$\left(\frac{ p }{ q }\right) \left(\frac{ q }{ p }\right) = (- 1)^ {\frac{ p - 1}{2}\frac{ q - 1}{2}}.$$
%% (Scores {tree_length = 84, tree_depth = 11, characters = 202, tokens = 81, subsequent_dollars = 0, initial_dollars = 0, parses = 1},379)

Thm07. Let $p , q \in N$. Then if $p$ is prime and $q$ is prime, then the product of the Legendre symbol of $p$ and $q$ and the Legendre symbol of $q$ and $p$ is equal to the exponentiation of the negation of $1$ and the product of the quotient of the difference of $p$ and $1$ and $2$ and the quotient of the difference of $q$ and $1$ and $2$.
%% (Scores {tree_length = 118, tree_depth = 19, characters = 344, tokens = 104, subsequent_dollars = 0, initial_dollars = 0, parses = 1},586)

Thm07. Let $p , q \in N$. Then $p$ is prime and $q$ is prime, only if the product of the Legendre symbol of $p$ and $q$ and the Legendre symbol of $q$ and $p$ is equal to the exponentiation of the negation of $1$ and the product of the quotient of the difference of $p$ and $1$ and $2$ and the quotient of the difference of $q$ and $1$ and $2$.
%% (Scores {tree_length = 118, tree_depth = 19, characters = 344, tokens = 104, subsequent_dollars = 0, initial_dollars = 0, parses = 1},586)

Thm07. Let $p , q \in N$. Assume that $p$ is prime and $q$ is prime. Then the product of the Legendre symbol of $p$ and $q$ and the Legendre symbol of $q$ and $p$ is equal to the exponentiation of the negation of $1$ and the product of the quotient of the difference of $p$ and $1$ and $2$ and the quotient of the difference of $q$ and $1$ and $2$.
%% (Scores {tree_length = 119, tree_depth = 18, characters = 348, tokens = 104, subsequent_dollars = 0, initial_dollars = 0, parses = 1},590)

Thm07. Let $p , q \in N$. Assume that both $p$ is prime and $q$ is prime. Then the product of the Legendre symbol of $p$ and $q$ and the Legendre symbol of $q$ and $p$ is equal to the exponentiation of the negation of $1$ and the product of the quotient of the difference of $p$ and $1$ and $2$ and the quotient of the difference of $q$ and $1$ and $2$.
%% (Scores {tree_length = 118, tree_depth = 18, characters = 353, tokens = 105, subsequent_dollars = 0, initial_dollars = 0, parses = 1},595)

Thm07. If $p$ is prime and $q$ is prime, then the product of the Legendre symbol of $p$ and $q$ and the Legendre symbol of $q$ and $p$ is equal to the exponentiation of the negation of $1$ and the product of the quotient of the difference of $p$ and $1$ and $2$ and the quotient of the difference of $q$ and $1$ and $2$ for all natural numbers $p$ and $q$.
%% (Scores {tree_length = 115, tree_depth = 20, characters = 356, tokens = 105, subsequent_dollars = 0, initial_dollars = 0, parses = 1},597)

Thm07. For all natural numbers $p$ and $q$, if $p$ is prime and $q$ is prime, then the product of the Legendre symbol of $p$ and $q$ and the Legendre symbol of $q$ and $p$ is equal to the exponentiation of the negation of $1$ and the product of the quotient of the difference of $p$ and $1$ and $2$ and the quotient of the difference of $q$ and $1$ and $2$.
%% (Scores {tree_length = 116, tree_depth = 20, characters = 357, tokens = 106, subsequent_dollars = 0, initial_dollars = 0, parses = 1},600)

Thm07. Let $p$ and $q$ be natural numbers. Then if $p$ is prime and $q$ is prime, then the product of the Legendre symbol of $p$ and $q$ and the Legendre symbol of $q$ and $p$ is equal to the exponentiation of the negation of $1$ and the product of the quotient of the difference of $p$ and $1$ and $2$ and the quotient of the difference of $q$ and $1$ and $2$.
%% (Scores {tree_length = 115, tree_depth = 19, characters = 361, tokens = 107, subsequent_dollars = 0, initial_dollars = 0, parses = 1},603)

Thm07. Let $p$ and $q$ be natural numbers. Then $p$ is prime and $q$ is prime, only if the product of the Legendre symbol of $p$ and $q$ and the Legendre symbol of $q$ and $p$ is equal to the exponentiation of the negation of $1$ and the product of the quotient of the difference of $p$ and $1$ and $2$ and the quotient of the difference of $q$ and $1$ and $2$.
%% (Scores {tree_length = 115, tree_depth = 19, characters = 361, tokens = 107, subsequent_dollars = 0, initial_dollars = 0, parses = 1},603)

Thm07. Let $p$ and $q$ be natural numbers. Assume that $p$ is prime and $q$ is prime. Then the product of the Legendre symbol of $p$ and $q$ and the Legendre symbol of $q$ and $p$ is equal to the exponentiation of the negation of $1$ and the product of the quotient of the difference of $p$ and $1$ and $2$ and the quotient of the difference of $q$ and $1$ and $2$.
%% (Scores {tree_length = 116, tree_depth = 18, characters = 365, tokens = 107, subsequent_dollars = 0, initial_dollars = 0, parses = 1},607)

Thm07. Let $p$ and $q$ be natural numbers. Assume that both $p$ is prime and $q$ is prime. Then the product of the Legendre symbol of $p$ and $q$ and the Legendre symbol of $q$ and $p$ is equal to the exponentiation of the negation of $1$ and the product of the quotient of the difference of $p$ and $1$ and $2$ and the quotient of the difference of $q$ and $1$ and $2$.
%% (Scores {tree_length = 115, tree_depth = 18, characters = 370, tokens = 108, subsequent_dollars = 0, initial_dollars = 0, parses = 1},612)

Thm07. If we can prove that $p$ is prime and $q$ is prime, then we can prove that the product of the Legendre symbol of $p$ and $q$ and the Legendre symbol of $q$ and $p$ is equal to the exponentiation of the negation of $1$ and the product of the quotient of the difference of $p$ and $1$ and $2$ and the quotient of the difference of $q$ and $1$ and $2$ for all instances $p$ and $q$ of natural numbers.
%% (Scores {tree_length = 118, tree_depth = 21, characters = 405, tokens = 115, subsequent_dollars = 0, initial_dollars = 0, parses = 1},660)

Thm07. For all instances $p$ and $q$ of natural numbers, if we can prove that $p$ is prime and $q$ is prime, then we can prove that the product of the Legendre symbol of $p$ and $q$ and the Legendre symbol of $q$ and $p$ is equal to the exponentiation of the negation of $1$ and the product of the quotient of the difference of $p$ and $1$ and $2$ and the quotient of the difference of $q$ and $1$ and $2$.
%% (Scores {tree_length = 119, tree_depth = 21, characters = 406, tokens = 116, subsequent_dollars = 0, initial_dollars = 0, parses = 1},663)

Thm07. Let $p$ and $q$ be instances of natural numbers. Then if we can prove that $p$ is prime and $q$ is prime, then we can prove that the product of the Legendre symbol of $p$ and $q$ and the Legendre symbol of $q$ and $p$ is equal to the exponentiation of the negation of $1$ and the product of the quotient of the difference of $p$ and $1$ and $2$ and the quotient of the difference of $q$ and $1$ and $2$.
%% (Scores {tree_length = 118, tree_depth = 20, characters = 410, tokens = 117, subsequent_dollars = 0, initial_dollars = 0, parses = 1},666)

Thm07. Let $p$ and $q$ be instances of natural numbers. Then we can prove that $p$ is prime and $q$ is prime, only if we can prove that the product of the Legendre symbol of $p$ and $q$ and the Legendre symbol of $q$ and $p$ is equal to the exponentiation of the negation of $1$ and the product of the quotient of the difference of $p$ and $1$ and $2$ and the quotient of the difference of $q$ and $1$ and $2$.
%% (Scores {tree_length = 118, tree_depth = 20, characters = 410, tokens = 117, subsequent_dollars = 0, initial_dollars = 0, parses = 1},666)

Thm07. Let $p$ and $q$ be instances of natural numbers. Assume that we can prove that $p$ is prime and $q$ is prime. Then we can prove that the product of the Legendre symbol of $p$ and $q$ and the Legendre symbol of $q$ and $p$ is equal to the exponentiation of the negation of $1$ and the product of the quotient of the difference of $p$ and $1$ and $2$ and the quotient of the difference of $q$ and $1$ and $2$.
%% (Scores {tree_length = 119, tree_depth = 19, characters = 414, tokens = 117, subsequent_dollars = 0, initial_dollars = 0, parses = 1},670)

Thm07. Let $p$ and $q$ be instances of natural numbers. Assume that we can prove that both $p$ is prime and $q$ is prime. Then we can prove that the product of the Legendre symbol of $p$ and $q$ and the Legendre symbol of $q$ and $p$ is equal to the exponentiation of the negation of $1$ and the product of the quotient of the difference of $p$ and $1$ and $2$ and the quotient of the difference of $q$ and $1$ and $2$.
%% (Scores {tree_length = 118, tree_depth = 19, characters = 419, tokens = 118, subsequent_dollars = 0, initial_dollars = 0, parses = 1},675)

Thm09. If $r$ is a real number equal to the radius of $c$, then the area of $c$ is equal to $\pi r ^ {2}$ for every circle $c$.
%% (Scores {tree_length = 50, tree_depth = 12, characters = 127, tokens = 44, subsequent_dollars = 0, initial_dollars = 0, parses = 1},234)

Thm09. If $r$ is a real number equal to the radius of $c$, then the area of $c$ is equal to $\pi r ^ {2}$ for all circles $c$.
%% (Scores {tree_length = 51, tree_depth = 12, characters = 126, tokens = 44, subsequent_dollars = 0, initial_dollars = 0, parses = 1},234)

Thm09. For all circles $c$, if $r$ is a real number equal to the radius of $c$, then the area of $c$ is equal to $\pi r ^ {2}$.
%% (Scores {tree_length = 52, tree_depth = 12, characters = 127, tokens = 45, subsequent_dollars = 0, initial_dollars = 0, parses = 1},237)

Thm09. Let $c$ be a circle. Then if $r$ is a real number equal to the radius of $c$, then the area of $c$ is equal to $\pi r ^ {2}$.
%% (Scores {tree_length = 51, tree_depth = 11, characters = 132, tokens = 47, subsequent_dollars = 0, initial_dollars = 0, parses = 1},242)

Thm09. Let $c$ be a circle. Then $r$ is a real number equal to the radius of $c$, only if the area of $c$ is equal to $\pi r ^ {2}$.
%% (Scores {tree_length = 51, tree_depth = 11, characters = 132, tokens = 47, subsequent_dollars = 0, initial_dollars = 0, parses = 1},242)

Thm09. Let $c$ be a circle. Assume that $r$ is a real number equal to the radius of $c$. Then the area of $c$ is equal to $\pi r ^ {2}$.
%% (Scores {tree_length = 52, tree_depth = 13, characters = 136, tokens = 47, subsequent_dollars = 0, initial_dollars = 0, parses = 1},249)

Thm09. Let $c$ be a circle. Let $r \in R$. Then if $r$ is equal to the radius of $c$, then the area of $c$ is equal to $\pi r ^ {2}$.
%% (Scores {tree_length = 57, tree_depth = 10, characters = 133, tokens = 51, subsequent_dollars = 0, initial_dollars = 0, parses = 1},252)

Thm09. Let $c$ be a circle. Let $r \in R$. Then $r$ is equal to the radius of $c$, only if the area of $c$ is equal to $\pi r ^ {2}$.
%% (Scores {tree_length = 57, tree_depth = 10, characters = 133, tokens = 51, subsequent_dollars = 0, initial_dollars = 0, parses = 1},252)

Thm09. For all real numbers $r$, if $r$ is equal to the radius of $c$, then the area of $c$ is equal to $\pi r ^ {2}$ for every circle $c$.
%% (Scores {tree_length = 55, tree_depth = 12, characters = 139, tokens = 49, subsequent_dollars = 0, initial_dollars = 0, parses = 1},256)

Thm09. For all real numbers $r$, if $r$ is equal to the radius of $c$, then the area of $c$ is equal to $\pi r ^ {2}$ for all circles $c$.
%% (Scores {tree_length = 56, tree_depth = 12, characters = 138, tokens = 49, subsequent_dollars = 0, initial_dollars = 0, parses = 1},256)

Thm09. For all circles $c$, for all real numbers $r$, if $r$ is equal to the radius of $c$, then the area of $c$ is equal to $\pi r ^ {2}$.
%% (Scores {tree_length = 57, tree_depth = 12, characters = 139, tokens = 50, subsequent_dollars = 0, initial_dollars = 0, parses = 1},259)

Thm09. Let $c$ be a circle. Let $r \in R$. Assume that $r$ is equal to the radius of $c$. Then the area of $c$ is equal to $\pi r ^ {2}$.
%% (Scores {tree_length = 58, tree_depth = 13, characters = 137, tokens = 51, subsequent_dollars = 0, initial_dollars = 0, parses = 1},260)

Thm09. Let $c$ be a circle. Then if $r$ is equal to the radius of $c$, then the area of $c$ is equal to $\pi r ^ {2}$ for every real number $r$.
%% (Scores {tree_length = 54, tree_depth = 11, characters = 144, tokens = 51, subsequent_dollars = 0, initial_dollars = 0, parses = 1},261)

Thm09. Let $c$ be a circle. Then if $r$ is equal to the radius of $c$, then the area of $c$ is equal to $\pi r ^ {2}$ for all real numbers $r$.
%% (Scores {tree_length = 55, tree_depth = 11, characters = 143, tokens = 51, subsequent_dollars = 0, initial_dollars = 0, parses = 1},261)

Thm09. Let $c$ be a circle. Let $r$ be a real number. Then if $r$ is equal to the radius of $c$, then the area of $c$ is equal to $\pi r ^ {2}$.
%% (Scores {tree_length = 55, tree_depth = 10, characters = 144, tokens = 53, subsequent_dollars = 0, initial_dollars = 0, parses = 1},263)

Thm09. Let $c$ be a circle. Let $r$ be a real number. Then $r$ is equal to the radius of $c$, only if the area of $c$ is equal to $\pi r ^ {2}$.
%% (Scores {tree_length = 55, tree_depth = 10, characters = 144, tokens = 53, subsequent_dollars = 0, initial_dollars = 0, parses = 1},263)

Thm09. Let $c$ be a circle. Then for all real numbers $r$, if $r$ is equal to the radius of $c$, then the area of $c$ is equal to $\pi r ^ {2}$.
%% (Scores {tree_length = 56, tree_depth = 11, characters = 144, tokens = 52, subsequent_dollars = 0, initial_dollars = 0, parses = 1},264)

Thm09. Let $c$ be a circle. Let $r$ be a real number. Assume that $r$ is equal to the radius of $c$. Then the area of $c$ is equal to $\pi r ^ {2}$.
%% (Scores {tree_length = 56, tree_depth = 13, characters = 148, tokens = 53, subsequent_dollars = 0, initial_dollars = 0, parses = 1},271)

Thm09. Let $c$ be a circle. Let $r \in R$. Then if $r$ is equal to the radius of $c$, then the area of $c$ is equal to the product of the number \(\pi\) and the exponentiation of $r$ and $2$.
%% (Scores {tree_length = 63, tree_depth = 14, characters = 191, tokens = 62, subsequent_dollars = 0, initial_dollars = 0, parses = 1},331)

Thm09. Let $c$ be a circle. Let $r \in R$. Then $r$ is equal to the radius of $c$, only if the area of $c$ is equal to the product of the number \(\pi\) and the exponentiation of $r$ and $2$.
%% (Scores {tree_length = 63, tree_depth = 14, characters = 191, tokens = 62, subsequent_dollars = 0, initial_dollars = 0, parses = 1},331)

Thm09. For all real numbers $r$, if $r$ is equal to the radius of $c$, then the area of $c$ is equal to the product of the number \(\pi\) and the exponentiation of $r$ and $2$ for every circle $c$.
%% (Scores {tree_length = 61, tree_depth = 16, characters = 197, tokens = 60, subsequent_dollars = 0, initial_dollars = 0, parses = 1},335)

Thm09. For all real numbers $r$, if $r$ is equal to the radius of $c$, then the area of $c$ is equal to the product of the number \(\pi\) and the exponentiation of $r$ and $2$ for all circles $c$.
%% (Scores {tree_length = 62, tree_depth = 16, characters = 196, tokens = 60, subsequent_dollars = 0, initial_dollars = 0, parses = 1},335)

Thm09. Let $c$ be a circle. Let $r \in R$. Assume that $r$ is equal to the radius of $c$. Then the area of $c$ is equal to the product of the number \(\pi\) and the exponentiation of $r$ and $2$.
%% (Scores {tree_length = 64, tree_depth = 13, characters = 195, tokens = 62, subsequent_dollars = 0, initial_dollars = 0, parses = 1},335)

Thm09. For all circles $c$, for all real numbers $r$, if $r$ is equal to the radius of $c$, then the area of $c$ is equal to the product of the number \(\pi\) and the exponentiation of $r$ and $2$.
%% (Scores {tree_length = 63, tree_depth = 16, characters = 197, tokens = 61, subsequent_dollars = 0, initial_dollars = 0, parses = 1},338)

Thm09. Let $c$ be a circle. Then if $r$ is equal to the radius of $c$, then the area of $c$ is equal to the product of the number \(\pi\) and the exponentiation of $r$ and $2$ for every real number $r$.
%% (Scores {tree_length = 60, tree_depth = 15, characters = 202, tokens = 62, subsequent_dollars = 0, initial_dollars = 0, parses = 1},340)

Thm09. Let $c$ be a circle. Then if $r$ is equal to the radius of $c$, then the area of $c$ is equal to the product of the number \(\pi\) and the exponentiation of $r$ and $2$ for all real numbers $r$.
%% (Scores {tree_length = 61, tree_depth = 15, characters = 201, tokens = 62, subsequent_dollars = 0, initial_dollars = 0, parses = 1},340)

Thm09. Let $c$ be a circle. Let $r$ be a real number. Then if $r$ is equal to the radius of $c$, then the area of $c$ is equal to the product of the number \(\pi\) and the exponentiation of $r$ and $2$.
%% (Scores {tree_length = 61, tree_depth = 14, characters = 202, tokens = 64, subsequent_dollars = 0, initial_dollars = 0, parses = 1},342)

Thm09. Let $c$ be a circle. Let $r$ be a real number. Then $r$ is equal to the radius of $c$, only if the area of $c$ is equal to the product of the number \(\pi\) and the exponentiation of $r$ and $2$.
%% (Scores {tree_length = 61, tree_depth = 14, characters = 202, tokens = 64, subsequent_dollars = 0, initial_dollars = 0, parses = 1},342)

Thm09. Let $c$ be a circle. Then for all real numbers $r$, if $r$ is equal to the radius of $c$, then the area of $c$ is equal to the product of the number \(\pi\) and the exponentiation of $r$ and $2$.
%% (Scores {tree_length = 62, tree_depth = 15, characters = 202, tokens = 63, subsequent_dollars = 0, initial_dollars = 0, parses = 1},343)

Thm09. Let $c$ be a circle. Let $r$ be a real number. Assume that $r$ is equal to the radius of $c$. Then the area of $c$ is equal to the product of the number \(\pi\) and the exponentiation of $r$ and $2$.
%% (Scores {tree_length = 62, tree_depth = 13, characters = 206, tokens = 64, subsequent_dollars = 0, initial_dollars = 0, parses = 1},346)

Thm09. For all instances $r$ of real numbers, if we can prove that $r$ is equal to the radius of $c$, then we can prove that the area of $c$ is equal to the product of the number \(\pi\) and the exponentiation of $r$ and $2$ for every circle $c$.
%% (Scores {tree_length = 64, tree_depth = 17, characters = 246, tokens = 70, subsequent_dollars = 0, initial_dollars = 0, parses = 1},398)

Thm09. For all instances $r$ of real numbers, if we can prove that $r$ is equal to the radius of $c$, then we can prove that the area of $c$ is equal to the product of the number \(\pi\) and the exponentiation of $r$ and $2$ for all circles $c$.
%% (Scores {tree_length = 65, tree_depth = 17, characters = 245, tokens = 70, subsequent_dollars = 0, initial_dollars = 0, parses = 1},398)

Thm09. For all circles $c$, for all instances $r$ of real numbers, if we can prove that $r$ is equal to the radius of $c$, then we can prove that the area of $c$ is equal to the product of the number \(\pi\) and the exponentiation of $r$ and $2$.
%% (Scores {tree_length = 66, tree_depth = 17, characters = 246, tokens = 71, subsequent_dollars = 0, initial_dollars = 0, parses = 1},401)

Thm09. Let $c$ be a circle. Then if we can prove that $r$ is equal to the radius of $c$, then we can prove that the area of $c$ is equal to the product of the number \(\pi\) and the exponentiation of $r$ and $2$ for every instance $r$ of real numbers.
%% (Scores {tree_length = 63, tree_depth = 16, characters = 251, tokens = 72, subsequent_dollars = 0, initial_dollars = 0, parses = 1},403)

Thm09. Let $c$ be a circle. Then if we can prove that $r$ is equal to the radius of $c$, then we can prove that the area of $c$ is equal to the product of the number \(\pi\) and the exponentiation of $r$ and $2$ for all instances $r$ of real numbers.
%% (Scores {tree_length = 64, tree_depth = 16, characters = 250, tokens = 72, subsequent_dollars = 0, initial_dollars = 0, parses = 1},403)

Thm09. Let $c$ be a circle. Then for all instances $r$ of real numbers, if we can prove that $r$ is equal to the radius of $c$, then we can prove that the area of $c$ is equal to the product of the number \(\pi\) and the exponentiation of $r$ and $2$.
%% (Scores {tree_length = 65, tree_depth = 16, characters = 251, tokens = 73, subsequent_dollars = 0, initial_dollars = 0, parses = 1},406)

Thm09. Let $c$ be a circle. Let $r$ be an instance of real numbers. Then if we can prove that $r$ is equal to the radius of $c$, then we can prove that the area of $c$ is equal to the product of the number \(\pi\) and the exponentiation of $r$ and $2$.
%% (Scores {tree_length = 64, tree_depth = 15, characters = 252, tokens = 74, subsequent_dollars = 0, initial_dollars = 0, parses = 1},406)

Thm09. Let $c$ be a circle. Let $r$ be an instance of real numbers. Then we can prove that $r$ is equal to the radius of $c$, only if we can prove that the area of $c$ is equal to the product of the number \(\pi\) and the exponentiation of $r$ and $2$.
%% (Scores {tree_length = 64, tree_depth = 15, characters = 252, tokens = 74, subsequent_dollars = 0, initial_dollars = 0, parses = 1},406)

Thm09. Let $c$ be a circle. Let $r$ be an instance of real numbers. Assume that we can prove that $r$ is equal to the radius of $c$. Then we can prove that the area of $c$ is equal to the product of the number \(\pi\) and the exponentiation of $r$ and $2$.
%% (Scores {tree_length = 65, tree_depth = 14, characters = 256, tokens = 74, subsequent_dollars = 0, initial_dollars = 0, parses = 1},410)

Thm10FermatLittle. If $p$ is a prime natural number, then for all integers $a$, $a ^ {p}- a = p q$ for an integer $q$.
%% (Scores {tree_length = 53, tree_depth = 12, characters = 118, tokens = 39, subsequent_dollars = 1, initial_dollars = 0, parses = 1},224)

Thm10FermatLittle. $p$ is a prime natural number, only if for all integers $a$, $a ^ {p}- a = p q$ for an integer $q$.
%% (Scores {tree_length = 53, tree_depth = 12, characters = 118, tokens = 39, subsequent_dollars = 1, initial_dollars = 1, parses = 1},225)

Thm10FermatLittle. Assume that $p$ is a prime natural number. Let $a \in Z$. Then $a ^ {p}- a = p q$ for an integer $q$.
%% (Scores {tree_length = 55, tree_depth = 10, characters = 120, tokens = 40, subsequent_dollars = 0, initial_dollars = 0, parses = 1},226)

Thm10FermatLittle. If $p$ is a prime natural number, then for all integers $a$, $a ^ {p}- a = p q$ for some integer $q$.
%% (Scores {tree_length = 54, tree_depth = 12, characters = 120, tokens = 39, subsequent_dollars = 1, initial_dollars = 0, parses = 1},227)

Thm10FermatLittle. $p$ is a prime natural number, only if for all integers $a$, $a ^ {p}- a = p q$ for some integer $q$.
%% (Scores {tree_length = 54, tree_depth = 12, characters = 120, tokens = 39, subsequent_dollars = 1, initial_dollars = 1, parses = 1},228)

Thm10FermatLittle. Assume that $p$ is a prime natural number. Let $a \in Z$. Then $$a ^ {p}- a = p q$$ for an integer $q$.
%% (Scores {tree_length = 55, tree_depth = 10, characters = 122, tokens = 40, subsequent_dollars = 0, initial_dollars = 0, parses = 1},228)

Thm10FermatLittle. Assume that $p$ is a prime natural number. Let $a \in Z$. Then $a ^ {p}- a = p q$ for some integer $q$.
%% (Scores {tree_length = 56, tree_depth = 10, characters = 122, tokens = 40, subsequent_dollars = 0, initial_dollars = 0, parses = 1},229)

Thm10FermatLittle. Assume that $p$ is a prime natural number. Then $a ^ {p}- a = p q$ for an integer $q$ for every integer $a$.
%% (Scores {tree_length = 52, tree_depth = 11, characters = 127, tokens = 39, subsequent_dollars = 0, initial_dollars = 0, parses = 1},230)

Thm10FermatLittle. Assume that $p$ is a prime natural number. Then $a ^ {p}- a = p q$ for an integer $q$ for all integers $a$.
%% (Scores {tree_length = 53, tree_depth = 11, characters = 126, tokens = 39, subsequent_dollars = 0, initial_dollars = 0, parses = 1},230)

Thm10FermatLittle. Assume that $p$ is a prime natural number. Let $a \in Z$. Then $$a ^ {p}- a = p q$$ for some integer $q$.
%% (Scores {tree_length = 56, tree_depth = 10, characters = 124, tokens = 40, subsequent_dollars = 0, initial_dollars = 0, parses = 1},231)

Thm10FermatLittle. Assume that $p$ is a prime natural number. Then $a ^ {p}- a = p q$ for some integer $q$ for every integer $a$.
%% (Scores {tree_length = 53, tree_depth = 11, characters = 129, tokens = 39, subsequent_dollars = 0, initial_dollars = 0, parses = 1},233)

Thm10FermatLittle. Assume that $p$ is a prime natural number. Then $a ^ {p}- a = p q$ for some integer $q$ for all integers $a$.
%% (Scores {tree_length = 54, tree_depth = 11, characters = 128, tokens = 39, subsequent_dollars = 0, initial_dollars = 0, parses = 1},233)

Thm10FermatLittle. Assume that $p$ is a prime natural number. Let $a$ be an integer. Then $a ^ {p}- a = p q$ for an integer $q$.
%% (Scores {tree_length = 53, tree_depth = 10, characters = 128, tokens = 41, subsequent_dollars = 0, initial_dollars = 0, parses = 1},233)

Thm10FermatLittle. Let $p \in N$. Assume that $p$ is prime. Let $a \in Z$. Then $a ^ {p}- a = p q$ for an integer $q$.
%% (Scores {tree_length = 61, tree_depth = 10, characters = 118, tokens = 44, subsequent_dollars = 0, initial_dollars = 0, parses = 1},234)

Thm10FermatLittle. Assume that $p$ is a prime natural number. Then for all integers $a$, $a ^ {p}- a = p q$ for an integer $q$.
%% (Scores {tree_length = 54, tree_depth = 11, characters = 127, tokens = 40, subsequent_dollars = 1, initial_dollars = 0, parses = 1},234)

Thm10FermatLittle. Assume that $p$ is a prime natural number. Let $a$ be an integer. Then $$a ^ {p}- a = p q$$ for an integer $q$.
%% (Scores {tree_length = 53, tree_depth = 10, characters = 130, tokens = 41, subsequent_dollars = 0, initial_dollars = 0, parses = 1},235)

Thm10FermatLittle. Let $p \in N$. Assume that $p$ is prime. Let $a \in Z$. Then $$a ^ {p}- a = p q$$ for an integer $q$.
%% (Scores {tree_length = 61, tree_depth = 10, characters = 120, tokens = 44, subsequent_dollars = 0, initial_dollars = 0, parses = 1},236)

Thm10FermatLittle. Assume that $p$ is a prime natural number. Let $a$ be an integer. Then $a ^ {p}- a = p q$ for some integer $q$.
%% (Scores {tree_length = 54, tree_depth = 10, characters = 130, tokens = 41, subsequent_dollars = 0, initial_dollars = 0, parses = 1},236)

Thm10FermatLittle. Let $p \in N$. Assume that $p$ is prime. Let $a \in Z$. Then $a ^ {p}- a = p q$ for some integer $q$.
%% (Scores {tree_length = 62, tree_depth = 10, characters = 120, tokens = 44, subsequent_dollars = 0, initial_dollars = 0, parses = 1},237)

Thm10FermatLittle. Assume that $p$ is a prime natural number. Then for all integers $a$, $a ^ {p}- a = p q$ for some integer $q$.
%% (Scores {tree_length = 55, tree_depth = 11, characters = 129, tokens = 40, subsequent_dollars = 1, initial_dollars = 0, parses = 1},237)

Thm10FermatLittle. Let $p \in N$. Then if $p$ is prime, then for all integers $a$, $a ^ {p}- a = p q$ for an integer $q$.
%% (Scores {tree_length = 59, tree_depth = 12, characters = 121, tokens = 44, subsequent_dollars = 1, initial_dollars = 0, parses = 1},238)

Thm10FermatLittle. Let $p \in N$. Then $p$ is prime, only if for all integers $a$, $a ^ {p}- a = p q$ for an integer $q$.
%% (Scores {tree_length = 59, tree_depth = 12, characters = 121, tokens = 44, subsequent_dollars = 1, initial_dollars = 0, parses = 1},238)

Thm10FermatLittle. Let $p \in N$. Assume that $p$ is prime. Then $a ^ {p}- a = p q$ for an integer $q$ for every integer $a$.
%% (Scores {tree_length = 58, tree_depth = 11, characters = 125, tokens = 43, subsequent_dollars = 0, initial_dollars = 0, parses = 1},238)

Thm10FermatLittle. Let $p \in N$. Assume that $p$ is prime. Then $a ^ {p}- a = p q$ for an integer $q$ for all integers $a$.
%% (Scores {tree_length = 59, tree_depth = 11, characters = 124, tokens = 43, subsequent_dollars = 0, initial_dollars = 0, parses = 1},238)

Thm10FermatLittle. Assume that $p$ is a prime natural number. Let $a$ be an integer. Then $$a ^ {p}- a = p q$$ for some integer $q$.
%% (Scores {tree_length = 54, tree_depth = 10, characters = 132, tokens = 41, subsequent_dollars = 0, initial_dollars = 0, parses = 1},238)

Thm10FermatLittle. Let $p \in N$. Assume that $p$ is prime. Let $a \in Z$. Then $$a ^ {p}- a = p q$$ for some integer $q$.
%% (Scores {tree_length = 62, tree_depth = 10, characters = 122, tokens = 44, subsequent_dollars = 0, initial_dollars = 0, parses = 1},239)

Thm10FermatLittle. Let $p \in N$. Then if $p$ is prime, then for all integers $a$, $a ^ {p}- a = p q$ for some integer $q$.
%% (Scores {tree_length = 60, tree_depth = 12, characters = 123, tokens = 44, subsequent_dollars = 1, initial_dollars = 0, parses = 1},241)

Thm10FermatLittle. Let $p \in N$. Then $p$ is prime, only if for all integers $a$, $a ^ {p}- a = p q$ for some integer $q$.
%% (Scores {tree_length = 60, tree_depth = 12, characters = 123, tokens = 44, subsequent_dollars = 1, initial_dollars = 0, parses = 1},241)

Thm10FermatLittle. Let $p \in N$. Assume that $p$ is prime. Then $a ^ {p}- a = p q$ for some integer $q$ for every integer $a$.
%% (Scores {tree_length = 59, tree_depth = 11, characters = 127, tokens = 43, subsequent_dollars = 0, initial_dollars = 0, parses = 1},241)

Thm10FermatLittle. Let $p \in N$. Assume that $p$ is prime. Then $a ^ {p}- a = p q$ for some integer $q$ for all integers $a$.
%% (Scores {tree_length = 60, tree_depth = 11, characters = 126, tokens = 43, subsequent_dollars = 0, initial_dollars = 0, parses = 1},241)

Thm10FermatLittle. Let $p \in N$. Assume that $p$ is prime. Let $a$ be an integer. Then $a ^ {p}- a = p q$ for an integer $q$.
%% (Scores {tree_length = 59, tree_depth = 10, characters = 126, tokens = 45, subsequent_dollars = 0, initial_dollars = 0, parses = 1},241)

Thm10FermatLittle. Let $p \in N$. Assume that $p$ is prime. Then for all integers $a$, $a ^ {p}- a = p q$ for an integer $q$.
%% (Scores {tree_length = 60, tree_depth = 11, characters = 125, tokens = 44, subsequent_dollars = 1, initial_dollars = 0, parses = 1},242)

Thm10FermatLittle. Let $p \in N$. Assume that $p$ is prime. Let $a$ be an integer. Then $$a ^ {p}- a = p q$$ for an integer $q$.
%% (Scores {tree_length = 59, tree_depth = 10, characters = 128, tokens = 45, subsequent_dollars = 0, initial_dollars = 0, parses = 1},243)

Thm10FermatLittle. If $p$ is prime, then for all integers $a$, $a ^ {p}- a = p q$ for an integer $q$ for every natural number $p$.
%% (Scores {tree_length = 56, tree_depth = 13, characters = 130, tokens = 43, subsequent_dollars = 1, initial_dollars = 0, parses = 1},244)

Thm10FermatLittle. If $p$ is prime, then for all integers $a$, $a ^ {p}- a = p q$ for an integer $q$ for all natural numbers $p$.
%% (Scores {tree_length = 57, tree_depth = 13, characters = 129, tokens = 43, subsequent_dollars = 1, initial_dollars = 0, parses = 1},244)

Thm10FermatLittle. Let $p \in N$. Assume that $p$ is prime. Let $a$ be an integer. Then $a ^ {p}- a = p q$ for some integer $q$.
%% (Scores {tree_length = 60, tree_depth = 10, characters = 128, tokens = 45, subsequent_dollars = 0, initial_dollars = 0, parses = 1},244)

Thm10FermatLittle. Let $p \in N$. Assume that $p$ is prime. Then for all integers $a$, $a ^ {p}- a = p q$ for some integer $q$.
%% (Scores {tree_length = 61, tree_depth = 11, characters = 127, tokens = 44, subsequent_dollars = 1, initial_dollars = 0, parses = 1},245)

Thm10FermatLittle. Let $p \in N$. Assume that $p$ is prime. Let $a$ be an integer. Then $$a ^ {p}- a = p q$$ for some integer $q$.
%% (Scores {tree_length = 60, tree_depth = 10, characters = 130, tokens = 45, subsequent_dollars = 0, initial_dollars = 0, parses = 1},246)

Thm10FermatLittle. For all natural numbers $p$, if $p$ is prime, then for all integers $a$, $a ^ {p}- a = p q$ for an integer $q$.
%% (Scores {tree_length = 58, tree_depth = 13, characters = 130, tokens = 44, subsequent_dollars = 1, initial_dollars = 0, parses = 1},247)

Thm10FermatLittle. If $p$ is prime, then for all integers $a$, $a ^ {p}- a = p q$ for some integer $q$ for every natural number $p$.
%% (Scores {tree_length = 57, tree_depth = 13, characters = 132, tokens = 43, subsequent_dollars = 1, initial_dollars = 0, parses = 1},247)

Thm10FermatLittle. If $p$ is prime, then for all integers $a$, $a ^ {p}- a = p q$ for some integer $q$ for all natural numbers $p$.
%% (Scores {tree_length = 58, tree_depth = 13, characters = 131, tokens = 43, subsequent_dollars = 1, initial_dollars = 0, parses = 1},247)

Thm10FermatLittle. Let $p$ be a natural number. Assume that $p$ is prime. Let $a \in Z$. Then $a ^ {p}- a = p q$ for an integer $q$.
%% (Scores {tree_length = 59, tree_depth = 10, characters = 132, tokens = 46, subsequent_dollars = 0, initial_dollars = 0, parses = 1},248)

Thm10FermatLittle. If $p$ is a prime natural number, then for all integers $a$, there exists an integer $q$, such that $a ^ {p}- a = p q$.
%% (Scores {tree_length = 55, tree_depth = 12, characters = 138, tokens = 43, subsequent_dollars = 0, initial_dollars = 0, parses = 1},249)

Thm10FermatLittle. For all natural numbers $p$, if $p$ is prime, then for all integers $a$, $a ^ {p}- a = p q$ for some integer $q$.
%% (Scores {tree_length = 59, tree_depth = 13, characters = 132, tokens = 44, subsequent_dollars = 1, initial_dollars = 0, parses = 1},250)

Thm10FermatLittle. Let $p$ be a natural number. Assume that $p$ is prime. Let $a \in Z$. Then $$a ^ {p}- a = p q$$ for an integer $q$.
%% (Scores {tree_length = 59, tree_depth = 10, characters = 134, tokens = 46, subsequent_dollars = 0, initial_dollars = 0, parses = 1},250)

Thm10FermatLittle. $p$ is a prime natural number, only if for all integers $a$, there exists an integer $q$, such that $a ^ {p}- a = p q$.
%% (Scores {tree_length = 55, tree_depth = 12, characters = 138, tokens = 43, subsequent_dollars = 0, initial_dollars = 1, parses = 1},250)

Thm10FermatLittle. Let $p$ be a natural number. Assume that $p$ is prime. Let $a \in Z$. Then $a ^ {p}- a = p q$ for some integer $q$.
%% (Scores {tree_length = 60, tree_depth = 10, characters = 134, tokens = 46, subsequent_dollars = 0, initial_dollars = 0, parses = 1},251)

Thm10FermatLittle. Let $p$ be a natural number. Then if $p$ is prime, then for all integers $a$, $a ^ {p}- a = p q$ for an integer $q$.
%% (Scores {tree_length = 57, tree_depth = 12, characters = 135, tokens = 46, subsequent_dollars = 1, initial_dollars = 0, parses = 1},252)

Thm10FermatLittle. Let $p$ be a natural number. Then $p$ is prime, only if for all integers $a$, $a ^ {p}- a = p q$ for an integer $q$.
%% (Scores {tree_length = 57, tree_depth = 12, characters = 135, tokens = 46, subsequent_dollars = 1, initial_dollars = 0, parses = 1},252)

Thm10FermatLittle. Let $p$ be a natural number. Assume that $p$ is prime. Then $a ^ {p}- a = p q$ for an integer $q$ for every integer $a$.
%% (Scores {tree_length = 56, tree_depth = 11, characters = 139, tokens = 45, subsequent_dollars = 0, initial_dollars = 0, parses = 1},252)

Thm10FermatLittle. Let $p$ be a natural number. Assume that $p$ is prime. Then $a ^ {p}- a = p q$ for an integer $q$ for all integers $a$.
%% (Scores {tree_length = 57, tree_depth = 11, characters = 138, tokens = 45, subsequent_dollars = 0, initial_dollars = 0, parses = 1},252)

Thm10FermatLittle. Assume that $p$ is a prime natural number. Let $a \in Z$. Then there exists an integer $q$, such that $a ^ {p}- a = p q$.
%% (Scores {tree_length = 57, tree_depth = 10, characters = 140, tokens = 44, subsequent_dollars = 0, initial_dollars = 0, parses = 1},252)

Thm10FermatLittle. Let $p$ be a natural number. Assume that $p$ is prime. Let $a \in Z$. Then $$a ^ {p}- a = p q$$ for some integer $q$.
%% (Scores {tree_length = 60, tree_depth = 10, characters = 136, tokens = 46, subsequent_dollars = 0, initial_dollars = 0, parses = 1},253)

Thm10FermatLittle. Let $p$ be a natural number. Then if $p$ is prime, then for all integers $a$, $a ^ {p}- a = p q$ for some integer $q$.
%% (Scores {tree_length = 58, tree_depth = 12, characters = 137, tokens = 46, subsequent_dollars = 1, initial_dollars = 0, parses = 1},255)

Thm10FermatLittle. Let $p$ be a natural number. Then $p$ is prime, only if for all integers $a$, $a ^ {p}- a = p q$ for some integer $q$.
%% (Scores {tree_length = 58, tree_depth = 12, characters = 137, tokens = 46, subsequent_dollars = 1, initial_dollars = 0, parses = 1},255)

Thm10FermatLittle. Let $p$ be a natural number. Assume that $p$ is prime. Then $a ^ {p}- a = p q$ for some integer $q$ for every integer $a$.
%% (Scores {tree_length = 57, tree_depth = 11, characters = 141, tokens = 45, subsequent_dollars = 0, initial_dollars = 0, parses = 1},255)

Thm10FermatLittle. Let $p$ be a natural number. Assume that $p$ is prime. Then $a ^ {p}- a = p q$ for some integer $q$ for all integers $a$.
%% (Scores {tree_length = 58, tree_depth = 11, characters = 140, tokens = 45, subsequent_dollars = 0, initial_dollars = 0, parses = 1},255)

Thm10FermatLittle. Let $p$ be a natural number. Assume that $p$ is prime. Let $a$ be an integer. Then $a ^ {p}- a = p q$ for an integer $q$.
%% (Scores {tree_length = 57, tree_depth = 10, characters = 140, tokens = 47, subsequent_dollars = 0, initial_dollars = 0, parses = 1},255)

Thm10FermatLittle. Let $p$ be a natural number. Assume that $p$ is prime. Then for all integers $a$, $a ^ {p}- a = p q$ for an integer $q$.
%% (Scores {tree_length = 58, tree_depth = 11, characters = 139, tokens = 46, subsequent_dollars = 1, initial_dollars = 0, parses = 1},256)

Thm10FermatLittle. Assume that $p$ is a prime natural number. Then there exists an integer $q$, such that $a ^ {p}- a = p q$ for every integer $a$.
%% (Scores {tree_length = 54, tree_depth = 11, characters = 147, tokens = 43, subsequent_dollars = 0, initial_dollars = 0, parses = 1},256)

Thm10FermatLittle. Assume that $p$ is a prime natural number. Then there exists an integer $q$, such that $a ^ {p}- a = p q$ for all integers $a$.
%% (Scores {tree_length = 55, tree_depth = 11, characters = 146, tokens = 43, subsequent_dollars = 0, initial_dollars = 0, parses = 1},256)

Thm10FermatLittle. Let $p$ be a natural number. Assume that $p$ is prime. Let $a$ be an integer. Then $$a ^ {p}- a = p q$$ for an integer $q$.
%% (Scores {tree_length = 57, tree_depth = 10, characters = 142, tokens = 47, subsequent_dollars = 0, initial_dollars = 0, parses = 1},257)

Thm10FermatLittle. Let $p$ be a natural number. Assume that $p$ is prime. Let $a$ be an integer. Then $a ^ {p}- a = p q$ for some integer $q$.
%% (Scores {tree_length = 58, tree_depth = 10, characters = 142, tokens = 47, subsequent_dollars = 0, initial_dollars = 0, parses = 1},258)

Thm10FermatLittle. Let $p$ be a natural number. Assume that $p$ is prime. Then for all integers $a$, $a ^ {p}- a = p q$ for some integer $q$.
%% (Scores {tree_length = 59, tree_depth = 11, characters = 141, tokens = 46, subsequent_dollars = 1, initial_dollars = 0, parses = 1},259)

Thm10FermatLittle. Assume that $p$ is a prime natural number. Let $a$ be an integer. Then there exists an integer $q$, such that $a ^ {p}- a = p q$.
%% (Scores {tree_length = 55, tree_depth = 10, characters = 148, tokens = 45, subsequent_dollars = 0, initial_dollars = 0, parses = 1},259)

Thm10FermatLittle. Assume that $p$ is a prime natural number. Then for all integers $a$, there exists an integer $q$, such that $a ^ {p}- a = p q$.
%% (Scores {tree_length = 56, tree_depth = 11, characters = 147, tokens = 44, subsequent_dollars = 0, initial_dollars = 0, parses = 1},259)

Thm10FermatLittle. Let $p$ be a natural number. Assume that $p$ is prime. Let $a$ be an integer. Then $$a ^ {p}- a = p q$$ for some integer $q$.
%% (Scores {tree_length = 58, tree_depth = 10, characters = 144, tokens = 47, subsequent_dollars = 0, initial_dollars = 0, parses = 1},260)

Thm10FermatLittle. Let $p \in N$. Assume that $p$ is prime. Let $a \in Z$. Then there exists an integer $q$, such that $a ^ {p}- a = p q$.
%% (Scores {tree_length = 63, tree_depth = 10, characters = 138, tokens = 48, subsequent_dollars = 0, initial_dollars = 0, parses = 1},260)

Thm10FermatLittle. Let $p \in N$. Then if $p$ is prime, then for all integers $a$, there exists an integer $q$, such that $a ^ {p}- a = p q$.
%% (Scores {tree_length = 61, tree_depth = 12, characters = 141, tokens = 48, subsequent_dollars = 0, initial_dollars = 0, parses = 1},263)

Thm10FermatLittle. Let $p \in N$. Then $p$ is prime, only if for all integers $a$, there exists an integer $q$, such that $a ^ {p}- a = p q$.
%% (Scores {tree_length = 61, tree_depth = 12, characters = 141, tokens = 48, subsequent_dollars = 0, initial_dollars = 0, parses = 1},263)

Thm10FermatLittle. Let $p \in N$. Assume that $p$ is prime. Then there exists an integer $q$, such that $a ^ {p}- a = p q$ for every integer $a$.
%% (Scores {tree_length = 60, tree_depth = 11, characters = 145, tokens = 47, subsequent_dollars = 0, initial_dollars = 0, parses = 1},264)

Thm10FermatLittle. Let $p \in N$. Assume that $p$ is prime. Then there exists an integer $q$, such that $a ^ {p}- a = p q$ for all integers $a$.
%% (Scores {tree_length = 61, tree_depth = 11, characters = 144, tokens = 47, subsequent_dollars = 0, initial_dollars = 0, parses = 1},264)

Thm10FermatLittle. Let $p \in N$. Assume that $p$ is prime. Then for all integers $a$, there exists an integer $q$, such that $a ^ {p}- a = p q$.
%% (Scores {tree_length = 62, tree_depth = 11, characters = 145, tokens = 48, subsequent_dollars = 0, initial_dollars = 0, parses = 1},267)

Thm10FermatLittle. Let $p \in N$. Assume that $p$ is prime. Let $a$ be an integer. Then there exists an integer $q$, such that $a ^ {p}- a = p q$.
%% (Scores {tree_length = 61, tree_depth = 10, characters = 146, tokens = 49, subsequent_dollars = 0, initial_dollars = 0, parses = 1},267)

Thm10FermatLittle. If $p$ is prime, then for all integers $a$, there exists an integer $q$, such that $a ^ {p}- a = p q$ for every natural number $p$.
%% (Scores {tree_length = 58, tree_depth = 13, characters = 150, tokens = 47, subsequent_dollars = 0, initial_dollars = 0, parses = 1},269)

Thm10FermatLittle. If $p$ is prime, then for all integers $a$, there exists an integer $q$, such that $a ^ {p}- a = p q$ for all natural numbers $p$.
%% (Scores {tree_length = 59, tree_depth = 13, characters = 149, tokens = 47, subsequent_dollars = 0, initial_dollars = 0, parses = 1},269)

Thm10FermatLittle. For all natural numbers $p$, if $p$ is prime, then for all integers $a$, there exists an integer $q$, such that $a ^ {p}- a = p q$.
%% (Scores {tree_length = 60, tree_depth = 13, characters = 150, tokens = 48, subsequent_dollars = 0, initial_dollars = 0, parses = 1},272)

Thm10FermatLittle. Let $p$ be a natural number. Assume that $p$ is prime. Let $a \in Z$. Then there exists an integer $q$, such that $a ^ {p}- a = p q$.
%% (Scores {tree_length = 61, tree_depth = 10, characters = 152, tokens = 50, subsequent_dollars = 0, initial_dollars = 0, parses = 1},274)

Thm10FermatLittle. Let $p$ be a natural number. Then if $p$ is prime, then for all integers $a$, there exists an integer $q$, such that $a ^ {p}- a = p q$.
%% (Scores {tree_length = 59, tree_depth = 12, characters = 155, tokens = 50, subsequent_dollars = 0, initial_dollars = 0, parses = 1},277)

Thm10FermatLittle. Let $p$ be a natural number. Then $p$ is prime, only if for all integers $a$, there exists an integer $q$, such that $a ^ {p}- a = p q$.
%% (Scores {tree_length = 59, tree_depth = 12, characters = 155, tokens = 50, subsequent_dollars = 0, initial_dollars = 0, parses = 1},277)

Thm10FermatLittle. Let $p$ be a natural number. Assume that $p$ is prime. Then there exists an integer $q$, such that $a ^ {p}- a = p q$ for every integer $a$.
%% (Scores {tree_length = 58, tree_depth = 11, characters = 159, tokens = 49, subsequent_dollars = 0, initial_dollars = 0, parses = 1},278)

Thm10FermatLittle. Let $p$ be a natural number. Assume that $p$ is prime. Then there exists an integer $q$, such that $a ^ {p}- a = p q$ for all integers $a$.
%% (Scores {tree_length = 59, tree_depth = 11, characters = 158, tokens = 49, subsequent_dollars = 0, initial_dollars = 0, parses = 1},278)

Thm10FermatLittle. Let $p$ be a natural number. Assume that $p$ is prime. Let $a$ be an integer. Then there exists an integer $q$, such that $a ^ {p}- a = p q$.
%% (Scores {tree_length = 59, tree_depth = 10, characters = 160, tokens = 51, subsequent_dollars = 0, initial_dollars = 0, parses = 1},281)

Thm10FermatLittle. Let $p$ be a natural number. Assume that $p$ is prime. Then for all integers $a$, there exists an integer $q$, such that $a ^ {p}- a = p q$.
%% (Scores {tree_length = 60, tree_depth = 11, characters = 159, tokens = 50, subsequent_dollars = 0, initial_dollars = 0, parses = 1},281)

Thm10FermatLittle. Let $p \in N$. Assume that $p$ is prime. Let $a \in Z$. Then the difference of the exponentiation of $a$ and $p$ and $a$ is equal to the product of $p$ and $q$ for an integer $q$.
%% (Scores {tree_length = 71, tree_depth = 12, characters = 198, tokens = 62, subsequent_dollars = 0, initial_dollars = 0, parses = 1},344)

Thm10FermatLittle. Let $p \in N$. Then if $p$ is prime, then for all integers $a$, the difference of the exponentiation of $a$ and $p$ and $a$ is equal to the product of $p$ and $q$ for an integer $q$.
%% (Scores {tree_length = 69, tree_depth = 14, characters = 201, tokens = 62, subsequent_dollars = 0, initial_dollars = 0, parses = 1},347)

Thm10FermatLittle. Let $p \in N$. Then $p$ is prime, only if for all integers $a$, the difference of the exponentiation of $a$ and $p$ and $a$ is equal to the product of $p$ and $q$ for an integer $q$.
%% (Scores {tree_length = 69, tree_depth = 14, characters = 201, tokens = 62, subsequent_dollars = 0, initial_dollars = 0, parses = 1},347)

Thm10FermatLittle. Let $p \in N$. Assume that $p$ is prime. Let $a \in Z$. Then the difference of the exponentiation of $a$ and $p$ and $a$ is equal to the product of $p$ and $q$ for some integer $q$.
%% (Scores {tree_length = 72, tree_depth = 12, characters = 200, tokens = 62, subsequent_dollars = 0, initial_dollars = 0, parses = 1},347)

Thm10FermatLittle. Let $p \in N$. Assume that $p$ is prime. Then the difference of the exponentiation of $a$ and $p$ and $a$ is equal to the product of $p$ and $q$ for an integer $q$ for every integer $a$.
%% (Scores {tree_length = 68, tree_depth = 13, characters = 205, tokens = 61, subsequent_dollars = 0, initial_dollars = 0, parses = 1},348)

Thm10FermatLittle. Let $p \in N$. Assume that $p$ is prime. Then the difference of the exponentiation of $a$ and $p$ and $a$ is equal to the product of $p$ and $q$ for an integer $q$ for all integers $a$.
%% (Scores {tree_length = 69, tree_depth = 13, characters = 204, tokens = 61, subsequent_dollars = 0, initial_dollars = 0, parses = 1},348)

Thm10FermatLittle. Let $p \in N$. Then if $p$ is prime, then for all integers $a$, the difference of the exponentiation of $a$ and $p$ and $a$ is equal to the product of $p$ and $q$ for some integer $q$.
%% (Scores {tree_length = 70, tree_depth = 14, characters = 203, tokens = 62, subsequent_dollars = 0, initial_dollars = 0, parses = 1},350)

Thm10FermatLittle. Let $p \in N$. Then $p$ is prime, only if for all integers $a$, the difference of the exponentiation of $a$ and $p$ and $a$ is equal to the product of $p$ and $q$ for some integer $q$.
%% (Scores {tree_length = 70, tree_depth = 14, characters = 203, tokens = 62, subsequent_dollars = 0, initial_dollars = 0, parses = 1},350)

Thm10FermatLittle. Let $p \in N$. Assume that $p$ is prime. Then for all integers $a$, the difference of the exponentiation of $a$ and $p$ and $a$ is equal to the product of $p$ and $q$ for an integer $q$.
%% (Scores {tree_length = 70, tree_depth = 13, characters = 205, tokens = 62, subsequent_dollars = 0, initial_dollars = 0, parses = 1},351)

Thm10FermatLittle. Let $p \in N$. Assume that $p$ is prime. Then the difference of the exponentiation of $a$ and $p$ and $a$ is equal to the product of $p$ and $q$ for some integer $q$ for every integer $a$.
%% (Scores {tree_length = 69, tree_depth = 13, characters = 207, tokens = 61, subsequent_dollars = 0, initial_dollars = 0, parses = 1},351)

Thm10FermatLittle. Let $p \in N$. Assume that $p$ is prime. Then the difference of the exponentiation of $a$ and $p$ and $a$ is equal to the product of $p$ and $q$ for some integer $q$ for all integers $a$.
%% (Scores {tree_length = 70, tree_depth = 13, characters = 206, tokens = 61, subsequent_dollars = 0, initial_dollars = 0, parses = 1},351)

Thm10FermatLittle. Let $p \in N$. Assume that $p$ is prime. Let $a$ be an integer. Then the difference of the exponentiation of $a$ and $p$ and $a$ is equal to the product of $p$ and $q$ for an integer $q$.
%% (Scores {tree_length = 69, tree_depth = 12, characters = 206, tokens = 63, subsequent_dollars = 0, initial_dollars = 0, parses = 1},351)

Thm10FermatLittle. If $p$ is prime, then for all integers $a$, the difference of the exponentiation of $a$ and $p$ and $a$ is equal to the product of $p$ and $q$ for an integer $q$ for every natural number $p$.
%% (Scores {tree_length = 66, tree_depth = 15, characters = 210, tokens = 61, subsequent_dollars = 0, initial_dollars = 0, parses = 1},353)

Thm10FermatLittle. If $p$ is prime, then for all integers $a$, the difference of the exponentiation of $a$ and $p$ and $a$ is equal to the product of $p$ and $q$ for an integer $q$ for all natural numbers $p$.
%% (Scores {tree_length = 67, tree_depth = 15, characters = 209, tokens = 61, subsequent_dollars = 0, initial_dollars = 0, parses = 1},353)

Thm10FermatLittle. Let $p \in N$. Assume that $p$ is prime. Then for all integers $a$, the difference of the exponentiation of $a$ and $p$ and $a$ is equal to the product of $p$ and $q$ for some integer $q$.
%% (Scores {tree_length = 71, tree_depth = 13, characters = 207, tokens = 62, subsequent_dollars = 0, initial_dollars = 0, parses = 1},354)

Thm10FermatLittle. Let $p \in N$. Assume that $p$ is prime. Let $a$ be an integer. Then the difference of the exponentiation of $a$ and $p$ and $a$ is equal to the product of $p$ and $q$ for some integer $q$.
%% (Scores {tree_length = 70, tree_depth = 12, characters = 208, tokens = 63, subsequent_dollars = 0, initial_dollars = 0, parses = 1},354)

Thm10FermatLittle. For all natural numbers $p$, if $p$ is prime, then for all integers $a$, the difference of the exponentiation of $a$ and $p$ and $a$ is equal to the product of $p$ and $q$ for an integer $q$.
%% (Scores {tree_length = 68, tree_depth = 15, characters = 210, tokens = 62, subsequent_dollars = 0, initial_dollars = 0, parses = 1},356)

Thm10FermatLittle. If $p$ is prime, then for all integers $a$, the difference of the exponentiation of $a$ and $p$ and $a$ is equal to the product of $p$ and $q$ for some integer $q$ for every natural number $p$.
%% (Scores {tree_length = 67, tree_depth = 15, characters = 212, tokens = 61, subsequent_dollars = 0, initial_dollars = 0, parses = 1},356)

Thm10FermatLittle. If $p$ is prime, then for all integers $a$, the difference of the exponentiation of $a$ and $p$ and $a$ is equal to the product of $p$ and $q$ for some integer $q$ for all natural numbers $p$.
%% (Scores {tree_length = 68, tree_depth = 15, characters = 211, tokens = 61, subsequent_dollars = 0, initial_dollars = 0, parses = 1},356)

Thm10FermatLittle. Let $p$ be a natural number. Assume that $p$ is prime. Let $a \in Z$. Then the difference of the exponentiation of $a$ and $p$ and $a$ is equal to the product of $p$ and $q$ for an integer $q$.
%% (Scores {tree_length = 69, tree_depth = 12, characters = 212, tokens = 64, subsequent_dollars = 0, initial_dollars = 0, parses = 1},358)

Thm10FermatLittle. For all natural numbers $p$, if $p$ is prime, then for all integers $a$, the difference of the exponentiation of $a$ and $p$ and $a$ is equal to the product of $p$ and $q$ for some integer $q$.
%% (Scores {tree_length = 69, tree_depth = 15, characters = 212, tokens = 62, subsequent_dollars = 0, initial_dollars = 0, parses = 1},359)

Thm10FermatLittle. Let $p$ be a natural number. Then if $p$ is prime, then for all integers $a$, the difference of the exponentiation of $a$ and $p$ and $a$ is equal to the product of $p$ and $q$ for an integer $q$.
%% (Scores {tree_length = 67, tree_depth = 14, characters = 215, tokens = 64, subsequent_dollars = 0, initial_dollars = 0, parses = 1},361)

Thm10FermatLittle. Let $p$ be a natural number. Then $p$ is prime, only if for all integers $a$, the difference of the exponentiation of $a$ and $p$ and $a$ is equal to the product of $p$ and $q$ for an integer $q$.
%% (Scores {tree_length = 67, tree_depth = 14, characters = 215, tokens = 64, subsequent_dollars = 0, initial_dollars = 0, parses = 1},361)

Thm10FermatLittle. Let $p$ be a natural number. Assume that $p$ is prime. Let $a \in Z$. Then the difference of the exponentiation of $a$ and $p$ and $a$ is equal to the product of $p$ and $q$ for some integer $q$.
%% (Scores {tree_length = 70, tree_depth = 12, characters = 214, tokens = 64, subsequent_dollars = 0, initial_dollars = 0, parses = 1},361)

Thm10FermatLittle. Let $p$ be a natural number. Assume that $p$ is prime. Then the difference of the exponentiation of $a$ and $p$ and $a$ is equal to the product of $p$ and $q$ for an integer $q$ for every integer $a$.
%% (Scores {tree_length = 66, tree_depth = 13, characters = 219, tokens = 63, subsequent_dollars = 0, initial_dollars = 0, parses = 1},362)

Thm10FermatLittle. Let $p$ be a natural number. Assume that $p$ is prime. Then the difference of the exponentiation of $a$ and $p$ and $a$ is equal to the product of $p$ and $q$ for an integer $q$ for all integers $a$.
%% (Scores {tree_length = 67, tree_depth = 13, characters = 218, tokens = 63, subsequent_dollars = 0, initial_dollars = 0, parses = 1},362)

Thm10FermatLittle. Let $p$ be a natural number. Then if $p$ is prime, then for all integers $a$, the difference of the exponentiation of $a$ and $p$ and $a$ is equal to the product of $p$ and $q$ for some integer $q$.
%% (Scores {tree_length = 68, tree_depth = 14, characters = 217, tokens = 64, subsequent_dollars = 0, initial_dollars = 0, parses = 1},364)

Thm10FermatLittle. Let $p$ be a natural number. Then $p$ is prime, only if for all integers $a$, the difference of the exponentiation of $a$ and $p$ and $a$ is equal to the product of $p$ and $q$ for some integer $q$.
%% (Scores {tree_length = 68, tree_depth = 14, characters = 217, tokens = 64, subsequent_dollars = 0, initial_dollars = 0, parses = 1},364)

Thm10FermatLittle. Let $p$ be a natural number. Assume that $p$ is prime. Then for all integers $a$, the difference of the exponentiation of $a$ and $p$ and $a$ is equal to the product of $p$ and $q$ for an integer $q$.
%% (Scores {tree_length = 68, tree_depth = 13, characters = 219, tokens = 64, subsequent_dollars = 0, initial_dollars = 0, parses = 1},365)

Thm10FermatLittle. Let $p$ be a natural number. Assume that $p$ is prime. Then the difference of the exponentiation of $a$ and $p$ and $a$ is equal to the product of $p$ and $q$ for some integer $q$ for every integer $a$.
%% (Scores {tree_length = 67, tree_depth = 13, characters = 221, tokens = 63, subsequent_dollars = 0, initial_dollars = 0, parses = 1},365)

Thm10FermatLittle. Let $p$ be a natural number. Assume that $p$ is prime. Then the difference of the exponentiation of $a$ and $p$ and $a$ is equal to the product of $p$ and $q$ for some integer $q$ for all integers $a$.
%% (Scores {tree_length = 68, tree_depth = 13, characters = 220, tokens = 63, subsequent_dollars = 0, initial_dollars = 0, parses = 1},365)

Thm10FermatLittle. Let $p$ be a natural number. Assume that $p$ is prime. Let $a$ be an integer. Then the difference of the exponentiation of $a$ and $p$ and $a$ is equal to the product of $p$ and $q$ for an integer $q$.
%% (Scores {tree_length = 67, tree_depth = 12, characters = 220, tokens = 65, subsequent_dollars = 0, initial_dollars = 0, parses = 1},365)

Thm10FermatLittle. Let $p$ be a natural number. Assume that $p$ is prime. Then for all integers $a$, the difference of the exponentiation of $a$ and $p$ and $a$ is equal to the product of $p$ and $q$ for some integer $q$.
%% (Scores {tree_length = 69, tree_depth = 13, characters = 221, tokens = 64, subsequent_dollars = 0, initial_dollars = 0, parses = 1},368)

Thm10FermatLittle. Let $p$ be a natural number. Assume that $p$ is prime. Let $a$ be an integer. Then the difference of the exponentiation of $a$ and $p$ and $a$ is equal to the product of $p$ and $q$ for some integer $q$.
%% (Scores {tree_length = 68, tree_depth = 12, characters = 222, tokens = 65, subsequent_dollars = 0, initial_dollars = 0, parses = 1},368)

Thm10FermatLittle. Let $p \in N$. Assume that $p$ is prime. Let $a \in Z$. Then there exists an integer $q$, such that the difference of the exponentiation of $a$ and $p$ and $a$ is equal to the product of $p$ and $q$.
%% (Scores {tree_length = 73, tree_depth = 12, characters = 218, tokens = 66, subsequent_dollars = 0, initial_dollars = 0, parses = 1},370)

Thm10FermatLittle. Let $p \in N$. Then if $p$ is prime, then for all integers $a$, there exists an integer $q$, such that the difference of the exponentiation of $a$ and $p$ and $a$ is equal to the product of $p$ and $q$.
%% (Scores {tree_length = 71, tree_depth = 14, characters = 221, tokens = 66, subsequent_dollars = 0, initial_dollars = 0, parses = 1},373)

Thm10FermatLittle. Let $p \in N$. Then $p$ is prime, only if for all integers $a$, there exists an integer $q$, such that the difference of the exponentiation of $a$ and $p$ and $a$ is equal to the product of $p$ and $q$.
%% (Scores {tree_length = 71, tree_depth = 14, characters = 221, tokens = 66, subsequent_dollars = 0, initial_dollars = 0, parses = 1},373)

Thm10FermatLittle. Let $p \in N$. Assume that $p$ is prime. Then there exists an integer $q$, such that the difference of the exponentiation of $a$ and $p$ and $a$ is equal to the product of $p$ and $q$ for every integer $a$.
%% (Scores {tree_length = 70, tree_depth = 13, characters = 225, tokens = 65, subsequent_dollars = 0, initial_dollars = 0, parses = 1},374)

Thm10FermatLittle. Let $p \in N$. Assume that $p$ is prime. Then there exists an integer $q$, such that the difference of the exponentiation of $a$ and $p$ and $a$ is equal to the product of $p$ and $q$ for all integers $a$.
%% (Scores {tree_length = 71, tree_depth = 13, characters = 224, tokens = 65, subsequent_dollars = 0, initial_dollars = 0, parses = 1},374)

Thm10FermatLittle. Let $p \in N$. Assume that $p$ is prime. Then for all integers $a$, there exists an integer $q$, such that the difference of the exponentiation of $a$ and $p$ and $a$ is equal to the product of $p$ and $q$.
%% (Scores {tree_length = 72, tree_depth = 13, characters = 225, tokens = 66, subsequent_dollars = 0, initial_dollars = 0, parses = 1},377)

Thm10FermatLittle. Let $p \in N$. Assume that $p$ is prime. Let $a$ be an integer. Then there exists an integer $q$, such that the difference of the exponentiation of $a$ and $p$ and $a$ is equal to the product of $p$ and $q$.
%% (Scores {tree_length = 71, tree_depth = 12, characters = 226, tokens = 67, subsequent_dollars = 0, initial_dollars = 0, parses = 1},377)

Thm10FermatLittle. If $p$ is prime, then for all integers $a$, there exists an integer $q$, such that the difference of the exponentiation of $a$ and $p$ and $a$ is equal to the product of $p$ and $q$ for every natural number $p$.
%% (Scores {tree_length = 68, tree_depth = 15, characters = 230, tokens = 65, subsequent_dollars = 0, initial_dollars = 0, parses = 1},379)

Thm10FermatLittle. If $p$ is prime, then for all integers $a$, there exists an integer $q$, such that the difference of the exponentiation of $a$ and $p$ and $a$ is equal to the product of $p$ and $q$ for all natural numbers $p$.
%% (Scores {tree_length = 69, tree_depth = 15, characters = 229, tokens = 65, subsequent_dollars = 0, initial_dollars = 0, parses = 1},379)

Thm10FermatLittle. For all natural numbers $p$, if $p$ is prime, then for all integers $a$, there exists an integer $q$, such that the difference of the exponentiation of $a$ and $p$ and $a$ is equal to the product of $p$ and $q$.
%% (Scores {tree_length = 70, tree_depth = 15, characters = 230, tokens = 66, subsequent_dollars = 0, initial_dollars = 0, parses = 1},382)

Thm10FermatLittle. Let $p$ be a natural number. Assume that $p$ is prime. Let $a \in Z$. Then there exists an integer $q$, such that the difference of the exponentiation of $a$ and $p$ and $a$ is equal to the product of $p$ and $q$.
%% (Scores {tree_length = 71, tree_depth = 12, characters = 232, tokens = 68, subsequent_dollars = 0, initial_dollars = 0, parses = 1},384)

Thm10FermatLittle. Let $p$ be a natural number. Then if $p$ is prime, then for all integers $a$, there exists an integer $q$, such that the difference of the exponentiation of $a$ and $p$ and $a$ is equal to the product of $p$ and $q$.
%% (Scores {tree_length = 69, tree_depth = 14, characters = 235, tokens = 68, subsequent_dollars = 0, initial_dollars = 0, parses = 1},387)

Thm10FermatLittle. Let $p$ be a natural number. Then $p$ is prime, only if for all integers $a$, there exists an integer $q$, such that the difference of the exponentiation of $a$ and $p$ and $a$ is equal to the product of $p$ and $q$.
%% (Scores {tree_length = 69, tree_depth = 14, characters = 235, tokens = 68, subsequent_dollars = 0, initial_dollars = 0, parses = 1},387)

Thm10FermatLittle. Let $p$ be a natural number. Assume that $p$ is prime. Then there exists an integer $q$, such that the difference of the exponentiation of $a$ and $p$ and $a$ is equal to the product of $p$ and $q$ for every integer $a$.
%% (Scores {tree_length = 68, tree_depth = 13, characters = 239, tokens = 67, subsequent_dollars = 0, initial_dollars = 0, parses = 1},388)

Thm10FermatLittle. Let $p$ be a natural number. Assume that $p$ is prime. Then there exists an integer $q$, such that the difference of the exponentiation of $a$ and $p$ and $a$ is equal to the product of $p$ and $q$ for all integers $a$.
%% (Scores {tree_length = 69, tree_depth = 13, characters = 238, tokens = 67, subsequent_dollars = 0, initial_dollars = 0, parses = 1},388)

Thm10FermatLittle. Let $p$ be a natural number. Assume that $p$ is prime. Let $a$ be an integer. Then there exists an integer $q$, such that the difference of the exponentiation of $a$ and $p$ and $a$ is equal to the product of $p$ and $q$.
%% (Scores {tree_length = 69, tree_depth = 12, characters = 240, tokens = 69, subsequent_dollars = 0, initial_dollars = 0, parses = 1},391)

Thm10FermatLittle. Let $p$ be a natural number. Assume that $p$ is prime. Then for all integers $a$, there exists an integer $q$, such that the difference of the exponentiation of $a$ and $p$ and $a$ is equal to the product of $p$ and $q$.
%% (Scores {tree_length = 70, tree_depth = 13, characters = 239, tokens = 68, subsequent_dollars = 0, initial_dollars = 0, parses = 1},391)

Thm10FermatLittle. If we can prove that $p$ is prime, then for all instances $a$ of integers, we can prove that the difference of the exponentiation of $a$ and $p$ and $a$ is equal to the product of $p$ and $q$ for an integer $q$ for every instance $p$ of natural numbers.
%% (Scores {tree_length = 70, tree_depth = 16, characters = 272, tokens = 73, subsequent_dollars = 0, initial_dollars = 0, parses = 1},432)

Thm10FermatLittle. If we can prove that $p$ is prime, then for all instances $a$ of integers, we can prove that the difference of the exponentiation of $a$ and $p$ and $a$ is equal to the product of $p$ and $q$ for an integer $q$ for all instances $p$ of natural numbers.
%% (Scores {tree_length = 71, tree_depth = 16, characters = 271, tokens = 73, subsequent_dollars = 0, initial_dollars = 0, parses = 1},432)

Thm10FermatLittle. For all instances $p$ of natural numbers, if we can prove that $p$ is prime, then for all instances $a$ of integers, we can prove that the difference of the exponentiation of $a$ and $p$ and $a$ is equal to the product of $p$ and $q$ for an integer $q$.
%% (Scores {tree_length = 72, tree_depth = 16, characters = 272, tokens = 74, subsequent_dollars = 0, initial_dollars = 0, parses = 1},435)

Thm10FermatLittle. If we can prove that $p$ is prime, then for all instances $a$ of integers, we can prove that the difference of the exponentiation of $a$ and $p$ and $a$ is equal to the product of $p$ and $q$ for some integer $q$ for every instance $p$ of natural numbers.
%% (Scores {tree_length = 71, tree_depth = 16, characters = 274, tokens = 73, subsequent_dollars = 0, initial_dollars = 0, parses = 1},435)

Thm10FermatLittle. If we can prove that $p$ is prime, then for all instances $a$ of integers, we can prove that the difference of the exponentiation of $a$ and $p$ and $a$ is equal to the product of $p$ and $q$ for some integer $q$ for all instances $p$ of natural numbers.
%% (Scores {tree_length = 72, tree_depth = 16, characters = 273, tokens = 73, subsequent_dollars = 0, initial_dollars = 0, parses = 1},435)

Thm10FermatLittle. For all instances $p$ of natural numbers, if we can prove that $p$ is prime, then for all instances $a$ of integers, we can prove that the difference of the exponentiation of $a$ and $p$ and $a$ is equal to the product of $p$ and $q$ for some integer $q$.
%% (Scores {tree_length = 73, tree_depth = 16, characters = 274, tokens = 74, subsequent_dollars = 0, initial_dollars = 0, parses = 1},438)

Thm10FermatLittle. Let $p$ be an instance of natural numbers. Then if we can prove that $p$ is prime, then for all instances $a$ of integers, we can prove that the difference of the exponentiation of $a$ and $p$ and $a$ is equal to the product of $p$ and $q$ for an integer $q$.
%% (Scores {tree_length = 71, tree_depth = 15, characters = 278, tokens = 76, subsequent_dollars = 0, initial_dollars = 0, parses = 1},441)

Thm10FermatLittle. Let $p$ be an instance of natural numbers. Then we can prove that $p$ is prime, only if for all instances $a$ of integers, we can prove that the difference of the exponentiation of $a$ and $p$ and $a$ is equal to the product of $p$ and $q$ for an integer $q$.
%% (Scores {tree_length = 71, tree_depth = 15, characters = 278, tokens = 76, subsequent_dollars = 0, initial_dollars = 0, parses = 1},441)

Thm10FermatLittle. Let $p$ be an instance of natural numbers. Assume that we can prove that $p$ is prime. Then we can prove that the difference of the exponentiation of $a$ and $p$ and $a$ is equal to the product of $p$ and $q$ for an integer $q$ for every instance $a$ of integers.
%% (Scores {tree_length = 70, tree_depth = 14, characters = 282, tokens = 75, subsequent_dollars = 0, initial_dollars = 0, parses = 1},442)

Thm10FermatLittle. Let $p$ be an instance of natural numbers. Assume that we can prove that $p$ is prime. Then we can prove that the difference of the exponentiation of $a$ and $p$ and $a$ is equal to the product of $p$ and $q$ for an integer $q$ for all instances $a$ of integers.
%% (Scores {tree_length = 71, tree_depth = 14, characters = 281, tokens = 75, subsequent_dollars = 0, initial_dollars = 0, parses = 1},442)

Thm10FermatLittle. Let $p$ be an instance of natural numbers. Then if we can prove that $p$ is prime, then for all instances $a$ of integers, we can prove that the difference of the exponentiation of $a$ and $p$ and $a$ is equal to the product of $p$ and $q$ for some integer $q$.
%% (Scores {tree_length = 72, tree_depth = 15, characters = 280, tokens = 76, subsequent_dollars = 0, initial_dollars = 0, parses = 1},444)

Thm10FermatLittle. Let $p$ be an instance of natural numbers. Then we can prove that $p$ is prime, only if for all instances $a$ of integers, we can prove that the difference of the exponentiation of $a$ and $p$ and $a$ is equal to the product of $p$ and $q$ for some integer $q$.
%% (Scores {tree_length = 72, tree_depth = 15, characters = 280, tokens = 76, subsequent_dollars = 0, initial_dollars = 0, parses = 1},444)

Thm10FermatLittle. Let $p$ be an instance of natural numbers. Assume that we can prove that $p$ is prime. Then for all instances $a$ of integers, we can prove that the difference of the exponentiation of $a$ and $p$ and $a$ is equal to the product of $p$ and $q$ for an integer $q$.
%% (Scores {tree_length = 72, tree_depth = 14, characters = 282, tokens = 76, subsequent_dollars = 0, initial_dollars = 0, parses = 1},445)

Thm10FermatLittle. Let $p$ be an instance of natural numbers. Assume that we can prove that $p$ is prime. Then we can prove that the difference of the exponentiation of $a$ and $p$ and $a$ is equal to the product of $p$ and $q$ for some integer $q$ for every instance $a$ of integers.
%% (Scores {tree_length = 71, tree_depth = 14, characters = 284, tokens = 75, subsequent_dollars = 0, initial_dollars = 0, parses = 1},445)

Thm10FermatLittle. Let $p$ be an instance of natural numbers. Assume that we can prove that $p$ is prime. Then we can prove that the difference of the exponentiation of $a$ and $p$ and $a$ is equal to the product of $p$ and $q$ for some integer $q$ for all instances $a$ of integers.
%% (Scores {tree_length = 72, tree_depth = 14, characters = 283, tokens = 75, subsequent_dollars = 0, initial_dollars = 0, parses = 1},445)

Thm10FermatLittle. Let $p$ be an instance of natural numbers. Assume that we can prove that $p$ is prime. Let $a$ be an instance of integers. Then we can prove that the difference of the exponentiation of $a$ and $p$ and $a$ is equal to the product of $p$ and $q$ for an integer $q$.
%% (Scores {tree_length = 71, tree_depth = 13, characters = 283, tokens = 77, subsequent_dollars = 0, initial_dollars = 0, parses = 1},445)

Thm10FermatLittle. Let $p$ be an instance of natural numbers. Assume that we can prove that $p$ is prime. Then for all instances $a$ of integers, we can prove that the difference of the exponentiation of $a$ and $p$ and $a$ is equal to the product of $p$ and $q$ for some integer $q$.
%% (Scores {tree_length = 73, tree_depth = 14, characters = 284, tokens = 76, subsequent_dollars = 0, initial_dollars = 0, parses = 1},448)

Thm10FermatLittle. Let $p$ be an instance of natural numbers. Assume that we can prove that $p$ is prime. Let $a$ be an instance of integers. Then we can prove that the difference of the exponentiation of $a$ and $p$ and $a$ is equal to the product of $p$ and $q$ for some integer $q$.
%% (Scores {tree_length = 72, tree_depth = 13, characters = 285, tokens = 77, subsequent_dollars = 0, initial_dollars = 0, parses = 1},448)

Thm10FermatLittle. If we can prove that $p$ is prime, then for all instances $a$ of integers, we can prove that there exists an integer $q$, such that the difference of the exponentiation of $a$ and $p$ and $a$ is equal to the product of $p$ and $q$ for every instance $p$ of natural numbers.
%% (Scores {tree_length = 72, tree_depth = 16, characters = 292, tokens = 77, subsequent_dollars = 0, initial_dollars = 0, parses = 1},458)

Thm10FermatLittle. If we can prove that $p$ is prime, then for all instances $a$ of integers, we can prove that there exists an integer $q$, such that the difference of the exponentiation of $a$ and $p$ and $a$ is equal to the product of $p$ and $q$ for all instances $p$ of natural numbers.
%% (Scores {tree_length = 73, tree_depth = 16, characters = 291, tokens = 77, subsequent_dollars = 0, initial_dollars = 0, parses = 1},458)

Thm10FermatLittle. For all instances $p$ of natural numbers, if we can prove that $p$ is prime, then for all instances $a$ of integers, we can prove that there exists an integer $q$, such that the difference of the exponentiation of $a$ and $p$ and $a$ is equal to the product of $p$ and $q$.
%% (Scores {tree_length = 74, tree_depth = 16, characters = 292, tokens = 78, subsequent_dollars = 0, initial_dollars = 0, parses = 1},461)

Thm10FermatLittle. Let $p$ be an instance of natural numbers. Then if we can prove that $p$ is prime, then for all instances $a$ of integers, we can prove that there exists an integer $q$, such that the difference of the exponentiation of $a$ and $p$ and $a$ is equal to the product of $p$ and $q$.
%% (Scores {tree_length = 73, tree_depth = 15, characters = 298, tokens = 80, subsequent_dollars = 0, initial_dollars = 0, parses = 1},467)

Thm10FermatLittle. Let $p$ be an instance of natural numbers. Then we can prove that $p$ is prime, only if for all instances $a$ of integers, we can prove that there exists an integer $q$, such that the difference of the exponentiation of $a$ and $p$ and $a$ is equal to the product of $p$ and $q$.
%% (Scores {tree_length = 73, tree_depth = 15, characters = 298, tokens = 80, subsequent_dollars = 0, initial_dollars = 0, parses = 1},467)

Thm10FermatLittle. Let $p$ be an instance of natural numbers. Assume that we can prove that $p$ is prime. Then we can prove that there exists an integer $q$, such that the difference of the exponentiation of $a$ and $p$ and $a$ is equal to the product of $p$ and $q$ for every instance $a$ of integers.
%% (Scores {tree_length = 72, tree_depth = 14, characters = 302, tokens = 79, subsequent_dollars = 0, initial_dollars = 0, parses = 1},468)

Thm10FermatLittle. Let $p$ be an instance of natural numbers. Assume that we can prove that $p$ is prime. Then we can prove that there exists an integer $q$, such that the difference of the exponentiation of $a$ and $p$ and $a$ is equal to the product of $p$ and $q$ for all instances $a$ of integers.
%% (Scores {tree_length = 73, tree_depth = 14, characters = 301, tokens = 79, subsequent_dollars = 0, initial_dollars = 0, parses = 1},468)

Thm10FermatLittle. Let $p$ be an instance of natural numbers. Assume that we can prove that $p$ is prime. Let $a$ be an instance of integers. Then we can prove that there exists an integer $q$, such that the difference of the exponentiation of $a$ and $p$ and $a$ is equal to the product of $p$ and $q$.
%% (Scores {tree_length = 73, tree_depth = 13, characters = 303, tokens = 81, subsequent_dollars = 0, initial_dollars = 0, parses = 1},471)

Thm10FermatLittle. Let $p$ be an instance of natural numbers. Assume that we can prove that $p$ is prime. Then for all instances $a$ of integers, we can prove that there exists an integer $q$, such that the difference of the exponentiation of $a$ and $p$ and $a$ is equal to the product of $p$ and $q$.
%% (Scores {tree_length = 74, tree_depth = 14, characters = 302, tokens = 80, subsequent_dollars = 0, initial_dollars = 0, parses = 1},471)

Thm11. Let $n \in N$. Then $p \geq n$ and $p$ is prime for a natural number $p$.
%% (Scores {tree_length = 38, tree_depth = 10, characters = 80, tokens = 29, subsequent_dollars = 0, initial_dollars = 0, parses = 1},158)

Thm11. Let $n \in N$. Then $p \geq n$ and $p$ is prime for some natural number $p$.
%% (Scores {tree_length = 39, tree_depth = 10, characters = 83, tokens = 29, subsequent_dollars = 0, initial_dollars = 0, parses = 1},162)

Thm11. Let $n \in N$. Then both $$p \geq n$$ and $p$ is prime for a natural number $p$.
%% (Scores {tree_length = 37, tree_depth = 9, characters = 87, tokens = 30, subsequent_dollars = 0, initial_dollars = 0, parses = 1},164)

Thm11. $p \geq n$ and $p$ is prime for a natural number $p$ for every natural number $n$.
%% (Scores {tree_length = 35, tree_depth = 11, characters = 89, tokens = 28, subsequent_dollars = 0, initial_dollars = 1, parses = 1},165)

Thm11. $p \geq n$ and $p$ is prime for a natural number $p$ for all natural numbers $n$.
%% (Scores {tree_length = 36, tree_depth = 11, characters = 88, tokens = 28, subsequent_dollars = 0, initial_dollars = 1, parses = 1},165)

Thm11. For all natural numbers $n$, $p \geq n$ and $p$ is prime for a natural number $p$.
%% (Scores {tree_length = 37, tree_depth = 11, characters = 89, tokens = 29, subsequent_dollars = 1, initial_dollars = 0, parses = 1},168)

Thm11. Let $n \in N$. Then both $$p \geq n$$ and $p$ is prime for some natural number $p$.
%% (Scores {tree_length = 38, tree_depth = 9, characters = 90, tokens = 30, subsequent_dollars = 0, initial_dollars = 0, parses = 1},168)

Thm11. $p \geq n$ and $p$ is prime for some natural number $p$ for every natural number $n$.
%% (Scores {tree_length = 36, tree_depth = 11, characters = 92, tokens = 28, subsequent_dollars = 0, initial_dollars = 1, parses = 1},169)

Thm11. $p \geq n$ and $p$ is prime for some natural number $p$ for all natural numbers $n$.
%% (Scores {tree_length = 37, tree_depth = 11, characters = 91, tokens = 28, subsequent_dollars = 0, initial_dollars = 1, parses = 1},169)

Thm11. For all natural numbers $n$, $p \geq n$ and $p$ is prime for some natural number $p$.
%% (Scores {tree_length = 38, tree_depth = 11, characters = 92, tokens = 29, subsequent_dollars = 1, initial_dollars = 0, parses = 1},172)

Thm11. Let $n$ be a natural number. Then $p \geq n$ and $p$ is prime for a natural number $p$.
%% (Scores {tree_length = 36, tree_depth = 10, characters = 94, tokens = 31, subsequent_dollars = 0, initial_dollars = 0, parses = 1},172)

Thm11. Let $n$ be a natural number. Then $p \geq n$ and $p$ is prime for some natural number $p$.
%% (Scores {tree_length = 37, tree_depth = 10, characters = 97, tokens = 31, subsequent_dollars = 0, initial_dollars = 0, parses = 1},176)

Thm11. Let $n$ be a natural number. Then both $$p \geq n$$ and $p$ is prime for a natural number $p$.
%% (Scores {tree_length = 35, tree_depth = 9, characters = 101, tokens = 32, subsequent_dollars = 0, initial_dollars = 0, parses = 1},178)

Thm11. Let $n$ be a natural number. Then both $$p \geq n$$ and $p$ is prime for some natural number $p$.
%% (Scores {tree_length = 36, tree_depth = 9, characters = 104, tokens = 32, subsequent_dollars = 0, initial_dollars = 0, parses = 1},182)

Thm11. Let $n \in N$. Then there exists a natural number $p$, such that $p \geq n$ and $p$ is prime.
%% (Scores {tree_length = 40, tree_depth = 10, characters = 100, tokens = 33, subsequent_dollars = 0, initial_dollars = 0, parses = 1},184)

Thm11. Let $n \in N$. Then $p$ is greater than or equal to $n$ and $p$ is prime for a natural number $p$.
%% (Scores {tree_length = 38, tree_depth = 10, characters = 105, tokens = 36, subsequent_dollars = 0, initial_dollars = 0, parses = 1},190)

Thm11. There exists a natural number $p$, such that $p \geq n$ and $p$ is prime for every natural number $n$.
%% (Scores {tree_length = 37, tree_depth = 11, characters = 109, tokens = 32, subsequent_dollars = 0, initial_dollars = 0, parses = 1},190)

Thm11. There exists a natural number $p$, such that $p \geq n$ and $p$ is prime for all natural numbers $n$.
%% (Scores {tree_length = 38, tree_depth = 11, characters = 108, tokens = 32, subsequent_dollars = 0, initial_dollars = 0, parses = 1},190)

Thm11. For all natural numbers $n$, there exists a natural number $p$, such that $p \geq n$ and $p$ is prime.
%% (Scores {tree_length = 39, tree_depth = 11, characters = 109, tokens = 33, subsequent_dollars = 0, initial_dollars = 0, parses = 1},193)

Thm11. Let $n \in N$. Then $p$ is greater than or equal to $n$ and $p$ is prime for some natural number $p$.
%% (Scores {tree_length = 39, tree_depth = 10, characters = 108, tokens = 36, subsequent_dollars = 0, initial_dollars = 0, parses = 1},194)

Thm11. Let $n \in N$. Then both $p$ is greater than or equal to $n$ and $p$ is prime for a natural number $p$.
%% (Scores {tree_length = 37, tree_depth = 9, characters = 110, tokens = 37, subsequent_dollars = 0, initial_dollars = 0, parses = 1},194)

Thm11. $p$ is greater than or equal to $n$ and $p$ is prime for a natural number $p$ for every natural number $n$.
%% (Scores {tree_length = 35, tree_depth = 11, characters = 114, tokens = 35, subsequent_dollars = 0, initial_dollars = 1, parses = 1},197)

Thm11. $p$ is greater than or equal to $n$ and $p$ is prime for a natural number $p$ for all natural numbers $n$.
%% (Scores {tree_length = 36, tree_depth = 11, characters = 113, tokens = 35, subsequent_dollars = 0, initial_dollars = 1, parses = 1},197)

Thm11. Let $n \in N$. Then both $p$ is greater than or equal to $n$ and $p$ is prime for some natural number $p$.
%% (Scores {tree_length = 38, tree_depth = 9, characters = 113, tokens = 37, subsequent_dollars = 0, initial_dollars = 0, parses = 1},198)

Thm11. Let $n$ be a natural number. Then there exists a natural number $p$, such that $p \geq n$ and $p$ is prime.
%% (Scores {tree_length = 38, tree_depth = 10, characters = 114, tokens = 35, subsequent_dollars = 0, initial_dollars = 0, parses = 1},198)

Thm11. For all natural numbers $n$, $p$ is greater than or equal to $n$ and $p$ is prime for a natural number $p$.
%% (Scores {tree_length = 37, tree_depth = 11, characters = 114, tokens = 36, subsequent_dollars = 1, initial_dollars = 0, parses = 1},200)

Thm11. $p$ is greater than or equal to $n$ and $p$ is prime for some natural number $p$ for every natural number $n$.
%% (Scores {tree_length = 36, tree_depth = 11, characters = 117, tokens = 35, subsequent_dollars = 0, initial_dollars = 1, parses = 1},201)

Thm11. $p$ is greater than or equal to $n$ and $p$ is prime for some natural number $p$ for all natural numbers $n$.
%% (Scores {tree_length = 37, tree_depth = 11, characters = 116, tokens = 35, subsequent_dollars = 0, initial_dollars = 1, parses = 1},201)

Thm11. For all natural numbers $n$, $p$ is greater than or equal to $n$ and $p$ is prime for some natural number $p$.
%% (Scores {tree_length = 38, tree_depth = 11, characters = 117, tokens = 36, subsequent_dollars = 1, initial_dollars = 0, parses = 1},204)

Thm11. Let $n$ be a natural number. Then $p$ is greater than or equal to $n$ and $p$ is prime for a natural number $p$.
%% (Scores {tree_length = 36, tree_depth = 10, characters = 119, tokens = 38, subsequent_dollars = 0, initial_dollars = 0, parses = 1},204)

Thm11. Let $n$ be a natural number. Then $p$ is greater than or equal to $n$ and $p$ is prime for some natural number $p$.
%% (Scores {tree_length = 37, tree_depth = 10, characters = 122, tokens = 38, subsequent_dollars = 0, initial_dollars = 0, parses = 1},208)

Thm11. Let $n$ be a natural number. Then both $p$ is greater than or equal to $n$ and $p$ is prime for a natural number $p$.
%% (Scores {tree_length = 35, tree_depth = 9, characters = 124, tokens = 39, subsequent_dollars = 0, initial_dollars = 0, parses = 1},208)

Thm11. Let $n$ be a natural number. Then both $p$ is greater than or equal to $n$ and $p$ is prime for some natural number $p$.
%% (Scores {tree_length = 36, tree_depth = 9, characters = 127, tokens = 39, subsequent_dollars = 0, initial_dollars = 0, parses = 1},212)

Thm11. Let $n \in N$. Then there exists a natural number $p$, such that $p$ is greater than or equal to $n$ and $p$ is prime.
%% (Scores {tree_length = 40, tree_depth = 10, characters = 125, tokens = 40, subsequent_dollars = 0, initial_dollars = 0, parses = 1},216)

Thm11. There exists a natural number $p$, such that $p$ is greater than or equal to $n$ and $p$ is prime for every natural number $n$.
%% (Scores {tree_length = 37, tree_depth = 11, characters = 134, tokens = 39, subsequent_dollars = 0, initial_dollars = 0, parses = 1},222)

Thm11. There exists a natural number $p$, such that $p$ is greater than or equal to $n$ and $p$ is prime for all natural numbers $n$.
%% (Scores {tree_length = 38, tree_depth = 11, characters = 133, tokens = 39, subsequent_dollars = 0, initial_dollars = 0, parses = 1},222)

Thm11. For all natural numbers $n$, there exists a natural number $p$, such that $p$ is greater than or equal to $n$ and $p$ is prime.
%% (Scores {tree_length = 39, tree_depth = 11, characters = 134, tokens = 40, subsequent_dollars = 0, initial_dollars = 0, parses = 1},225)

Thm11. Let $n$ be a natural number. Then there exists a natural number $p$, such that $p$ is greater than or equal to $n$ and $p$ is prime.
%% (Scores {tree_length = 38, tree_depth = 10, characters = 139, tokens = 42, subsequent_dollars = 0, initial_dollars = 0, parses = 1},230)

Thm11. We can prove that $p$ is greater than or equal to $n$ and $p$ is prime for a natural number $p$ for every instance $n$ of natural numbers.
%% (Scores {tree_length = 37, tree_depth = 12, characters = 145, tokens = 41, subsequent_dollars = 0, initial_dollars = 0, parses = 1},236)

Thm11. We can prove that $p$ is greater than or equal to $n$ and $p$ is prime for a natural number $p$ for all instances $n$ of natural numbers.
%% (Scores {tree_length = 38, tree_depth = 12, characters = 144, tokens = 41, subsequent_dollars = 0, initial_dollars = 0, parses = 1},236)

Thm11. For all instances $n$ of natural numbers, we can prove that $p$ is greater than or equal to $n$ and $p$ is prime for a natural number $p$.
%% (Scores {tree_length = 39, tree_depth = 12, characters = 145, tokens = 42, subsequent_dollars = 0, initial_dollars = 0, parses = 1},239)

Thm11. We can prove that $p$ is greater than or equal to $n$ and $p$ is prime for some natural number $p$ for every instance $n$ of natural numbers.
%% (Scores {tree_length = 38, tree_depth = 12, characters = 148, tokens = 41, subsequent_dollars = 0, initial_dollars = 0, parses = 1},240)

Thm11. We can prove that $p$ is greater than or equal to $n$ and $p$ is prime for some natural number $p$ for all instances $n$ of natural numbers.
%% (Scores {tree_length = 39, tree_depth = 12, characters = 147, tokens = 41, subsequent_dollars = 0, initial_dollars = 0, parses = 1},240)

Thm11. For all instances $n$ of natural numbers, we can prove that $p$ is greater than or equal to $n$ and $p$ is prime for some natural number $p$.
%% (Scores {tree_length = 40, tree_depth = 12, characters = 148, tokens = 42, subsequent_dollars = 0, initial_dollars = 0, parses = 1},243)

Thm11. Let $n$ be an instance of natural numbers. Then we can prove that $p$ is greater than or equal to $n$ and $p$ is prime for a natural number $p$.
%% (Scores {tree_length = 38, tree_depth = 11, characters = 151, tokens = 44, subsequent_dollars = 0, initial_dollars = 0, parses = 1},245)

Thm11. Let $n$ be an instance of natural numbers. Then we can prove that $p$ is greater than or equal to $n$ and $p$ is prime for some natural number $p$.
%% (Scores {tree_length = 39, tree_depth = 11, characters = 154, tokens = 44, subsequent_dollars = 0, initial_dollars = 0, parses = 1},249)

Thm11. Let $n$ be an instance of natural numbers. Then we can prove that both $p$ is greater than or equal to $n$ and $p$ is prime for a natural number $p$.
%% (Scores {tree_length = 37, tree_depth = 10, characters = 156, tokens = 45, subsequent_dollars = 0, initial_dollars = 0, parses = 1},249)

Thm11. Let $n$ be an instance of natural numbers. Then we can prove that both $p$ is greater than or equal to $n$ and $p$ is prime for some natural number $p$.
%% (Scores {tree_length = 38, tree_depth = 10, characters = 159, tokens = 45, subsequent_dollars = 0, initial_dollars = 0, parses = 1},253)

Thm11. We can prove that there exists a natural number $p$, such that $p$ is greater than or equal to $n$ and $p$ is prime for every instance $n$ of natural numbers.
%% (Scores {tree_length = 39, tree_depth = 12, characters = 165, tokens = 45, subsequent_dollars = 0, initial_dollars = 0, parses = 1},262)

Thm11. We can prove that there exists a natural number $p$, such that $p$ is greater than or equal to $n$ and $p$ is prime for all instances $n$ of natural numbers.
%% (Scores {tree_length = 40, tree_depth = 12, characters = 164, tokens = 45, subsequent_dollars = 0, initial_dollars = 0, parses = 1},262)

Thm11. For all instances $n$ of natural numbers, we can prove that there exists a natural number $p$, such that $p$ is greater than or equal to $n$ and $p$ is prime.
%% (Scores {tree_length = 41, tree_depth = 12, characters = 165, tokens = 46, subsequent_dollars = 0, initial_dollars = 0, parses = 1},265)

Thm11. Let $n$ be an instance of natural numbers. Then we can prove that there exists a natural number $p$, such that $p$ is greater than or equal to $n$ and $p$ is prime.
%% (Scores {tree_length = 40, tree_depth = 11, characters = 171, tokens = 48, subsequent_dollars = 0, initial_dollars = 0, parses = 1},271)

Thm19. Let $n \in N$. Then $n = a ^{ 2}+ b ^{ 2}+ c ^{ 2}+ d ^{ 2}$ for some natural numbers $a$, $b$, $c$ and $d$.
%% (Scores {tree_length = 63, tree_depth = 12, characters = 115, tokens = 57, subsequent_dollars = 2, initial_dollars = 0, parses = 1},250)

Thm19. Let $n \in N$. Then $$n = a ^{ 2}+ b ^{ 2}+ c ^{ 2}+ d ^{ 2}$$ for some natural numbers $a$, $b$, $c$ and $d$.
%% (Scores {tree_length = 63, tree_depth = 12, characters = 117, tokens = 57, subsequent_dollars = 2, initial_dollars = 0, parses = 1},252)

Thm19. $n = a ^{ 2}+ b ^{ 2}+ c ^{ 2}+ d ^{ 2}$ for some natural numbers $a$, $b$, $c$ and $d$ for every natural number $n$.
%% (Scores {tree_length = 60, tree_depth = 13, characters = 124, tokens = 56, subsequent_dollars = 2, initial_dollars = 1, parses = 1},257)

Thm19. $n = a ^{ 2}+ b ^{ 2}+ c ^{ 2}+ d ^{ 2}$ for some natural numbers $a$, $b$, $c$ and $d$ for all natural numbers $n$.
%% (Scores {tree_length = 61, tree_depth = 13, characters = 123, tokens = 56, subsequent_dollars = 2, initial_dollars = 1, parses = 1},257)

Thm19. For all natural numbers $n$, $n = a ^{ 2}+ b ^{ 2}+ c ^{ 2}+ d ^{ 2}$ for some natural numbers $a$, $b$, $c$ and $d$.
%% (Scores {tree_length = 62, tree_depth = 13, characters = 124, tokens = 57, subsequent_dollars = 3, initial_dollars = 0, parses = 1},260)

Thm19. Let $n$ be a natural number. Then $n = a ^{ 2}+ b ^{ 2}+ c ^{ 2}+ d ^{ 2}$ for some natural numbers $a$, $b$, $c$ and $d$.
%% (Scores {tree_length = 61, tree_depth = 12, characters = 129, tokens = 59, subsequent_dollars = 2, initial_dollars = 0, parses = 1},264)

Thm19. Let $n$ be a natural number. Then $$n = a ^{ 2}+ b ^{ 2}+ c ^{ 2}+ d ^{ 2}$$ for some natural numbers $a$, $b$, $c$ and $d$.
%% (Scores {tree_length = 61, tree_depth = 12, characters = 131, tokens = 59, subsequent_dollars = 2, initial_dollars = 0, parses = 1},266)

Thm19. Let $n \in N$. Then there exist natural numbers $a$, $b$, $c$ and $d$, such that $n = a ^{ 2}+ b ^{ 2}+ c ^{ 2}+ d ^{ 2}$.
%% (Scores {tree_length = 64, tree_depth = 12, characters = 129, tokens = 60, subsequent_dollars = 2, initial_dollars = 0, parses = 1},268)

Thm19. There exist natural numbers $a$, $b$, $c$ and $d$, such that $n = a ^{ 2}+ b ^{ 2}+ c ^{ 2}+ d ^{ 2}$ for every natural number $n$.
%% (Scores {tree_length = 61, tree_depth = 13, characters = 138, tokens = 59, subsequent_dollars = 2, initial_dollars = 0, parses = 1},274)

Thm19. There exist natural numbers $a$, $b$, $c$ and $d$, such that $n = a ^{ 2}+ b ^{ 2}+ c ^{ 2}+ d ^{ 2}$ for all natural numbers $n$.
%% (Scores {tree_length = 62, tree_depth = 13, characters = 137, tokens = 59, subsequent_dollars = 2, initial_dollars = 0, parses = 1},274)

Thm19. For all natural numbers $n$, there exist natural numbers $a$, $b$, $c$ and $d$, such that $n = a ^{ 2}+ b ^{ 2}+ c ^{ 2}+ d ^{ 2}$.
%% (Scores {tree_length = 63, tree_depth = 13, characters = 138, tokens = 60, subsequent_dollars = 2, initial_dollars = 0, parses = 1},277)

Thm19. Let $n$ be a natural number. Then there exist natural numbers $a$, $b$, $c$ and $d$, such that $n = a ^{ 2}+ b ^{ 2}+ c ^{ 2}+ d ^{ 2}$.
%% (Scores {tree_length = 62, tree_depth = 12, characters = 143, tokens = 62, subsequent_dollars = 2, initial_dollars = 0, parses = 1},282)

Thm19. Let $n \in N$. Then there exists a natural number $c$, such that there exists a natural number $d$, such that $n = a ^{ 2}+ b ^{ 2}+ c ^{ 2}+ d ^{ 2}$ for a natural number $b$ for a natural number $a$.
%% (Scores {tree_length = 75, tree_depth = 15, characters = 208, tokens = 74, subsequent_dollars = 0, initial_dollars = 0, parses = 1},373)

Thm19. Let $n \in N$. Then there exists a natural number $c$, such that there exists a natural number $d$, such that $n = a ^{ 2}+ b ^{ 2}+ c ^{ 2}+ d ^{ 2}$ for some natural number $b$ for a natural number $a$.
%% (Scores {tree_length = 76, tree_depth = 15, characters = 211, tokens = 74, subsequent_dollars = 0, initial_dollars = 0, parses = 1},377)

Thm19. Let $n \in N$. Then there exists a natural number $c$, such that there exists a natural number $d$, such that $n = a ^{ 2}+ b ^{ 2}+ c ^{ 2}+ d ^{ 2}$ for a natural number $b$ for some natural number $a$.
%% (Scores {tree_length = 76, tree_depth = 15, characters = 211, tokens = 74, subsequent_dollars = 0, initial_dollars = 0, parses = 1},377)

Thm19. Let $n \in N$. Then there exists a natural number $c$, such that there exists a natural number $d$, such that $n = a ^{ 2}+ b ^{ 2}+ c ^{ 2}+ d ^{ 2}$ for some natural number $b$ for some natural number $a$.
%% (Scores {tree_length = 77, tree_depth = 15, characters = 214, tokens = 74, subsequent_dollars = 0, initial_dollars = 0, parses = 1},381)

Thm19. Let $n$ be a natural number. Then there exists a natural number $c$, such that there exists a natural number $d$, such that $n = a ^{ 2}+ b ^{ 2}+ c ^{ 2}+ d ^{ 2}$ for a natural number $b$ for a natural number $a$.
%% (Scores {tree_length = 73, tree_depth = 15, characters = 222, tokens = 76, subsequent_dollars = 0, initial_dollars = 0, parses = 1},387)

Thm19. Let $n$ be a natural number. Then there exists a natural number $c$, such that there exists a natural number $d$, such that $n = a ^{ 2}+ b ^{ 2}+ c ^{ 2}+ d ^{ 2}$ for some natural number $b$ for a natural number $a$.
%% (Scores {tree_length = 74, tree_depth = 15, characters = 225, tokens = 76, subsequent_dollars = 0, initial_dollars = 0, parses = 1},391)

Thm19. Let $n$ be a natural number. Then there exists a natural number $c$, such that there exists a natural number $d$, such that $n = a ^{ 2}+ b ^{ 2}+ c ^{ 2}+ d ^{ 2}$ for a natural number $b$ for some natural number $a$.
%% (Scores {tree_length = 74, tree_depth = 15, characters = 225, tokens = 76, subsequent_dollars = 0, initial_dollars = 0, parses = 1},391)

Thm19. Let $n$ be a natural number. Then there exists a natural number $c$, such that there exists a natural number $d$, such that $n = a ^{ 2}+ b ^{ 2}+ c ^{ 2}+ d ^{ 2}$ for some natural number $b$ for some natural number $a$.
%% (Scores {tree_length = 75, tree_depth = 15, characters = 228, tokens = 76, subsequent_dollars = 0, initial_dollars = 0, parses = 1},395)

Thm19. Let $n \in N$. Then there exists a natural number $b$, such that there exists a natural number $c$, such that there exists a natural number $d$, such that $n = a ^{ 2}+ b ^{ 2}+ c ^{ 2}+ d ^{ 2}$ for a natural number $a$.
%% (Scores {tree_length = 77, tree_depth = 15, characters = 228, tokens = 78, subsequent_dollars = 0, initial_dollars = 0, parses = 1},399)

Thm19. Let $n \in N$. Then there exists a natural number $b$, such that there exists a natural number $c$, such that there exists a natural number $d$, such that $n = a ^{ 2}+ b ^{ 2}+ c ^{ 2}+ d ^{ 2}$ for some natural number $a$.
%% (Scores {tree_length = 78, tree_depth = 15, characters = 231, tokens = 78, subsequent_dollars = 0, initial_dollars = 0, parses = 1},403)

Thm19. There exists a natural number $b$, such that there exists a natural number $c$, such that there exists a natural number $d$, such that $n = a ^{ 2}+ b ^{ 2}+ c ^{ 2}+ d ^{ 2}$ for a natural number $a$ for every natural number $n$.
%% (Scores {tree_length = 74, tree_depth = 16, characters = 237, tokens = 77, subsequent_dollars = 0, initial_dollars = 0, parses = 1},405)

Thm19. There exists a natural number $b$, such that there exists a natural number $c$, such that there exists a natural number $d$, such that $n = a ^{ 2}+ b ^{ 2}+ c ^{ 2}+ d ^{ 2}$ for a natural number $a$ for all natural numbers $n$.
%% (Scores {tree_length = 75, tree_depth = 16, characters = 236, tokens = 77, subsequent_dollars = 0, initial_dollars = 0, parses = 1},405)

Thm19. For all natural numbers $n$, there exists a natural number $b$, such that there exists a natural number $c$, such that there exists a natural number $d$, such that $n = a ^{ 2}+ b ^{ 2}+ c ^{ 2}+ d ^{ 2}$ for a natural number $a$.
%% (Scores {tree_length = 76, tree_depth = 16, characters = 237, tokens = 78, subsequent_dollars = 0, initial_dollars = 0, parses = 1},408)

Thm19. There exists a natural number $b$, such that there exists a natural number $c$, such that there exists a natural number $d$, such that $n = a ^{ 2}+ b ^{ 2}+ c ^{ 2}+ d ^{ 2}$ for some natural number $a$ for every natural number $n$.
%% (Scores {tree_length = 75, tree_depth = 16, characters = 240, tokens = 77, subsequent_dollars = 0, initial_dollars = 0, parses = 1},409)

Thm19. There exists a natural number $b$, such that there exists a natural number $c$, such that there exists a natural number $d$, such that $n = a ^{ 2}+ b ^{ 2}+ c ^{ 2}+ d ^{ 2}$ for some natural number $a$ for all natural numbers $n$.
%% (Scores {tree_length = 76, tree_depth = 16, characters = 239, tokens = 77, subsequent_dollars = 0, initial_dollars = 0, parses = 1},409)

Thm19. For all natural numbers $n$, there exists a natural number $b$, such that there exists a natural number $c$, such that there exists a natural number $d$, such that $n = a ^{ 2}+ b ^{ 2}+ c ^{ 2}+ d ^{ 2}$ for some natural number $a$.
%% (Scores {tree_length = 77, tree_depth = 16, characters = 240, tokens = 78, subsequent_dollars = 0, initial_dollars = 0, parses = 1},412)

Thm19. Let $n$ be a natural number. Then there exists a natural number $b$, such that there exists a natural number $c$, such that there exists a natural number $d$, such that $n = a ^{ 2}+ b ^{ 2}+ c ^{ 2}+ d ^{ 2}$ for a natural number $a$.
%% (Scores {tree_length = 75, tree_depth = 15, characters = 242, tokens = 80, subsequent_dollars = 0, initial_dollars = 0, parses = 1},413)

Thm19. Let $n$ be a natural number. Then there exists a natural number $b$, such that there exists a natural number $c$, such that there exists a natural number $d$, such that $n = a ^{ 2}+ b ^{ 2}+ c ^{ 2}+ d ^{ 2}$ for some natural number $a$.
%% (Scores {tree_length = 76, tree_depth = 15, characters = 245, tokens = 80, subsequent_dollars = 0, initial_dollars = 0, parses = 1},417)

Thm19. Let $n \in N$. Then there exists a natural number $a$, such that there exists a natural number $b$, such that there exists a natural number $c$, such that there exists a natural number $d$, such that $n = a ^{ 2}+ b ^{ 2}+ c ^{ 2}+ d ^{ 2}$.
%% (Scores {tree_length = 79, tree_depth = 15, characters = 248, tokens = 82, subsequent_dollars = 0, initial_dollars = 0, parses = 1},425)

Thm19. There exists a natural number $a$, such that there exists a natural number $b$, such that there exists a natural number $c$, such that there exists a natural number $d$, such that $n = a ^{ 2}+ b ^{ 2}+ c ^{ 2}+ d ^{ 2}$ for every natural number $n$.
%% (Scores {tree_length = 76, tree_depth = 16, characters = 257, tokens = 81, subsequent_dollars = 0, initial_dollars = 0, parses = 1},431)

Thm19. There exists a natural number $a$, such that there exists a natural number $b$, such that there exists a natural number $c$, such that there exists a natural number $d$, such that $n = a ^{ 2}+ b ^{ 2}+ c ^{ 2}+ d ^{ 2}$ for all natural numbers $n$.
%% (Scores {tree_length = 77, tree_depth = 16, characters = 256, tokens = 81, subsequent_dollars = 0, initial_dollars = 0, parses = 1},431)

Thm19. For all natural numbers $n$, there exists a natural number $a$, such that there exists a natural number $b$, such that there exists a natural number $c$, such that there exists a natural number $d$, such that $n = a ^{ 2}+ b ^{ 2}+ c ^{ 2}+ d ^{ 2}$.
%% (Scores {tree_length = 78, tree_depth = 16, characters = 257, tokens = 82, subsequent_dollars = 0, initial_dollars = 0, parses = 1},434)

Thm19. Let $n$ be a natural number. Then there exists a natural number $a$, such that there exists a natural number $b$, such that there exists a natural number $c$, such that there exists a natural number $d$, such that $n = a ^{ 2}+ b ^{ 2}+ c ^{ 2}+ d ^{ 2}$.
%% (Scores {tree_length = 77, tree_depth = 15, characters = 262, tokens = 84, subsequent_dollars = 0, initial_dollars = 0, parses = 1},439)

Thm19. Let $n \in N$. Then there exists a natural number $c$, such that there exists a natural number $d$, such that $n$ is equal to the sum of the sum of the sum of the square of $a$ and the square of $b$ and the square of $c$ and the square of $d$ for a natural number $b$ for a natural number $a$.
%% (Scores {tree_length = 88, tree_depth = 20, characters = 300, tokens = 89, subsequent_dollars = 0, initial_dollars = 0, parses = 1},498)

Thm19. Let $n \in N$. Then there exists a natural number $c$, such that there exists a natural number $d$, such that $n$ is equal to the sum of the sum of the sum of the square of $a$ and the square of $b$ and the square of $c$ and the square of $d$ for some natural number $b$ for a natural number $a$.
%% (Scores {tree_length = 89, tree_depth = 20, characters = 303, tokens = 89, subsequent_dollars = 0, initial_dollars = 0, parses = 1},502)

Thm19. Let $n \in N$. Then there exists a natural number $c$, such that there exists a natural number $d$, such that $n$ is equal to the sum of the sum of the sum of the square of $a$ and the square of $b$ and the square of $c$ and the square of $d$ for a natural number $b$ for some natural number $a$.
%% (Scores {tree_length = 89, tree_depth = 20, characters = 303, tokens = 89, subsequent_dollars = 0, initial_dollars = 0, parses = 1},502)

Thm19. Let $n \in N$. Then there exists a natural number $c$, such that there exists a natural number $d$, such that $n$ is equal to the sum of the sum of the sum of the square of $a$ and the square of $b$ and the square of $c$ and the square of $d$ for some natural number $b$ for some natural number $a$.
%% (Scores {tree_length = 90, tree_depth = 20, characters = 306, tokens = 89, subsequent_dollars = 0, initial_dollars = 0, parses = 1},506)

Thm19. Let $n$ be a natural number. Then there exists a natural number $c$, such that there exists a natural number $d$, such that $n$ is equal to the sum of the sum of the sum of the square of $a$ and the square of $b$ and the square of $c$ and the square of $d$ for a natural number $b$ for a natural number $a$.
%% (Scores {tree_length = 86, tree_depth = 20, characters = 314, tokens = 91, subsequent_dollars = 0, initial_dollars = 0, parses = 1},512)

Thm19. Let $n$ be a natural number. Then there exists a natural number $c$, such that there exists a natural number $d$, such that $n$ is equal to the sum of the sum of the sum of the square of $a$ and the square of $b$ and the square of $c$ and the square of $d$ for some natural number $b$ for a natural number $a$.
%% (Scores {tree_length = 87, tree_depth = 20, characters = 317, tokens = 91, subsequent_dollars = 0, initial_dollars = 0, parses = 1},516)

Thm19. Let $n$ be a natural number. Then there exists a natural number $c$, such that there exists a natural number $d$, such that $n$ is equal to the sum of the sum of the sum of the square of $a$ and the square of $b$ and the square of $c$ and the square of $d$ for a natural number $b$ for some natural number $a$.
%% (Scores {tree_length = 87, tree_depth = 20, characters = 317, tokens = 91, subsequent_dollars = 0, initial_dollars = 0, parses = 1},516)

Thm19. Let $n$ be a natural number. Then there exists a natural number $c$, such that there exists a natural number $d$, such that $n$ is equal to the sum of the sum of the sum of the square of $a$ and the square of $b$ and the square of $c$ and the square of $d$ for some natural number $b$ for some natural number $a$.
%% (Scores {tree_length = 88, tree_depth = 20, characters = 320, tokens = 91, subsequent_dollars = 0, initial_dollars = 0, parses = 1},520)

Thm19. Let $n \in N$. Then there exists a natural number $b$, such that there exists a natural number $c$, such that there exists a natural number $d$, such that $n$ is equal to the sum of the sum of the sum of the square of $a$ and the square of $b$ and the square of $c$ and the square of $d$ for a natural number $a$.
%% (Scores {tree_length = 90, tree_depth = 20, characters = 320, tokens = 93, subsequent_dollars = 0, initial_dollars = 0, parses = 1},524)

Thm19. Let $n \in N$. Then there exists a natural number $b$, such that there exists a natural number $c$, such that there exists a natural number $d$, such that $n$ is equal to the sum of the sum of the sum of the square of $a$ and the square of $b$ and the square of $c$ and the square of $d$ for some natural number $a$.
%% (Scores {tree_length = 91, tree_depth = 20, characters = 323, tokens = 93, subsequent_dollars = 0, initial_dollars = 0, parses = 1},528)

Thm19. There exists a natural number $b$, such that there exists a natural number $c$, such that there exists a natural number $d$, such that $n$ is equal to the sum of the sum of the sum of the square of $a$ and the square of $b$ and the square of $c$ and the square of $d$ for a natural number $a$ for every natural number $n$.
%% (Scores {tree_length = 87, tree_depth = 21, characters = 329, tokens = 92, subsequent_dollars = 0, initial_dollars = 0, parses = 1},530)

Thm19. There exists a natural number $b$, such that there exists a natural number $c$, such that there exists a natural number $d$, such that $n$ is equal to the sum of the sum of the sum of the square of $a$ and the square of $b$ and the square of $c$ and the square of $d$ for a natural number $a$ for all natural numbers $n$.
%% (Scores {tree_length = 88, tree_depth = 21, characters = 328, tokens = 92, subsequent_dollars = 0, initial_dollars = 0, parses = 1},530)

Thm19. For all natural numbers $n$, there exists a natural number $b$, such that there exists a natural number $c$, such that there exists a natural number $d$, such that $n$ is equal to the sum of the sum of the sum of the square of $a$ and the square of $b$ and the square of $c$ and the square of $d$ for a natural number $a$.
%% (Scores {tree_length = 89, tree_depth = 21, characters = 329, tokens = 93, subsequent_dollars = 0, initial_dollars = 0, parses = 1},533)

Thm19. There exists a natural number $b$, such that there exists a natural number $c$, such that there exists a natural number $d$, such that $n$ is equal to the sum of the sum of the sum of the square of $a$ and the square of $b$ and the square of $c$ and the square of $d$ for some natural number $a$ for every natural number $n$.
%% (Scores {tree_length = 88, tree_depth = 21, characters = 332, tokens = 92, subsequent_dollars = 0, initial_dollars = 0, parses = 1},534)

Thm19. There exists a natural number $b$, such that there exists a natural number $c$, such that there exists a natural number $d$, such that $n$ is equal to the sum of the sum of the sum of the square of $a$ and the square of $b$ and the square of $c$ and the square of $d$ for some natural number $a$ for all natural numbers $n$.
%% (Scores {tree_length = 89, tree_depth = 21, characters = 331, tokens = 92, subsequent_dollars = 0, initial_dollars = 0, parses = 1},534)

Thm19. For all natural numbers $n$, there exists a natural number $b$, such that there exists a natural number $c$, such that there exists a natural number $d$, such that $n$ is equal to the sum of the sum of the sum of the square of $a$ and the square of $b$ and the square of $c$ and the square of $d$ for some natural number $a$.
%% (Scores {tree_length = 90, tree_depth = 21, characters = 332, tokens = 93, subsequent_dollars = 0, initial_dollars = 0, parses = 1},537)

Thm19. Let $n$ be a natural number. Then there exists a natural number $b$, such that there exists a natural number $c$, such that there exists a natural number $d$, such that $n$ is equal to the sum of the sum of the sum of the square of $a$ and the square of $b$ and the square of $c$ and the square of $d$ for a natural number $a$.
%% (Scores {tree_length = 88, tree_depth = 20, characters = 334, tokens = 95, subsequent_dollars = 0, initial_dollars = 0, parses = 1},538)

Thm19. Let $n$ be a natural number. Then there exists a natural number $b$, such that there exists a natural number $c$, such that there exists a natural number $d$, such that $n$ is equal to the sum of the sum of the sum of the square of $a$ and the square of $b$ and the square of $c$ and the square of $d$ for some natural number $a$.
%% (Scores {tree_length = 89, tree_depth = 20, characters = 337, tokens = 95, subsequent_dollars = 0, initial_dollars = 0, parses = 1},542)

Thm19. Let $n \in N$. Then there exists a natural number $a$, such that there exists a natural number $b$, such that there exists a natural number $c$, such that there exists a natural number $d$, such that $n$ is equal to the sum of the sum of the sum of the square of $a$ and the square of $b$ and the square of $c$ and the square of $d$.
%% (Scores {tree_length = 92, tree_depth = 20, characters = 340, tokens = 97, subsequent_dollars = 0, initial_dollars = 0, parses = 1},550)

Thm19. Let $n$ be an instance of natural numbers. Then we can prove that there exists a natural number $c$, such that there exists a natural number $d$, such that $n$ is equal to the sum of the sum of the sum of the square of $a$ and the square of $b$ and the square of $c$ and the square of $d$ for a natural number $b$ for a natural number $a$.
%% (Scores {tree_length = 88, tree_depth = 21, characters = 346, tokens = 97, subsequent_dollars = 0, initial_dollars = 0, parses = 1},553)

Thm19. There exists a natural number $a$, such that there exists a natural number $b$, such that there exists a natural number $c$, such that there exists a natural number $d$, such that $n$ is equal to the sum of the sum of the sum of the square of $a$ and the square of $b$ and the square of $c$ and the square of $d$ for every natural number $n$.
%% (Scores {tree_length = 89, tree_depth = 21, characters = 349, tokens = 96, subsequent_dollars = 0, initial_dollars = 0, parses = 1},556)

Thm19. There exists a natural number $a$, such that there exists a natural number $b$, such that there exists a natural number $c$, such that there exists a natural number $d$, such that $n$ is equal to the sum of the sum of the sum of the square of $a$ and the square of $b$ and the square of $c$ and the square of $d$ for all natural numbers $n$.
%% (Scores {tree_length = 90, tree_depth = 21, characters = 348, tokens = 96, subsequent_dollars = 0, initial_dollars = 0, parses = 1},556)

Thm19. Let $n$ be an instance of natural numbers. Then we can prove that there exists a natural number $c$, such that there exists a natural number $d$, such that $n$ is equal to the sum of the sum of the sum of the square of $a$ and the square of $b$ and the square of $c$ and the square of $d$ for some natural number $b$ for a natural number $a$.
%% (Scores {tree_length = 89, tree_depth = 21, characters = 349, tokens = 97, subsequent_dollars = 0, initial_dollars = 0, parses = 1},557)

Thm19. Let $n$ be an instance of natural numbers. Then we can prove that there exists a natural number $c$, such that there exists a natural number $d$, such that $n$ is equal to the sum of the sum of the sum of the square of $a$ and the square of $b$ and the square of $c$ and the square of $d$ for a natural number $b$ for some natural number $a$.
%% (Scores {tree_length = 89, tree_depth = 21, characters = 349, tokens = 97, subsequent_dollars = 0, initial_dollars = 0, parses = 1},557)

Thm19. For all natural numbers $n$, there exists a natural number $a$, such that there exists a natural number $b$, such that there exists a natural number $c$, such that there exists a natural number $d$, such that $n$ is equal to the sum of the sum of the sum of the square of $a$ and the square of $b$ and the square of $c$ and the square of $d$.
%% (Scores {tree_length = 91, tree_depth = 21, characters = 349, tokens = 97, subsequent_dollars = 0, initial_dollars = 0, parses = 1},559)

Thm19. Let $n$ be an instance of natural numbers. Then we can prove that there exists a natural number $c$, such that there exists a natural number $d$, such that $n$ is equal to the sum of the sum of the sum of the square of $a$ and the square of $b$ and the square of $c$ and the square of $d$ for some natural number $b$ for some natural number $a$.
%% (Scores {tree_length = 90, tree_depth = 21, characters = 352, tokens = 97, subsequent_dollars = 0, initial_dollars = 0, parses = 1},561)

Thm19. Let $n$ be a natural number. Then there exists a natural number $a$, such that there exists a natural number $b$, such that there exists a natural number $c$, such that there exists a natural number $d$, such that $n$ is equal to the sum of the sum of the sum of the square of $a$ and the square of $b$ and the square of $c$ and the square of $d$.
%% (Scores {tree_length = 90, tree_depth = 20, characters = 354, tokens = 99, subsequent_dollars = 0, initial_dollars = 0, parses = 1},564)

Thm19. We can prove that there exists a natural number $b$, such that there exists a natural number $c$, such that there exists a natural number $d$, such that $n$ is equal to the sum of the sum of the sum of the square of $a$ and the square of $b$ and the square of $c$ and the square of $d$ for a natural number $a$ for every instance $n$ of natural numbers.
%% (Scores {tree_length = 89, tree_depth = 22, characters = 360, tokens = 98, subsequent_dollars = 0, initial_dollars = 0, parses = 1},570)

Thm19. We can prove that there exists a natural number $b$, such that there exists a natural number $c$, such that there exists a natural number $d$, such that $n$ is equal to the sum of the sum of the sum of the square of $a$ and the square of $b$ and the square of $c$ and the square of $d$ for a natural number $a$ for all instances $n$ of natural numbers.
%% (Scores {tree_length = 90, tree_depth = 22, characters = 359, tokens = 98, subsequent_dollars = 0, initial_dollars = 0, parses = 1},570)

Thm19. For all instances $n$ of natural numbers, we can prove that there exists a natural number $b$, such that there exists a natural number $c$, such that there exists a natural number $d$, such that $n$ is equal to the sum of the sum of the sum of the square of $a$ and the square of $b$ and the square of $c$ and the square of $d$ for a natural number $a$.
%% (Scores {tree_length = 91, tree_depth = 22, characters = 360, tokens = 99, subsequent_dollars = 0, initial_dollars = 0, parses = 1},573)

Thm19. We can prove that there exists a natural number $b$, such that there exists a natural number $c$, such that there exists a natural number $d$, such that $n$ is equal to the sum of the sum of the sum of the square of $a$ and the square of $b$ and the square of $c$ and the square of $d$ for some natural number $a$ for every instance $n$ of natural numbers.
%% (Scores {tree_length = 90, tree_depth = 22, characters = 363, tokens = 98, subsequent_dollars = 0, initial_dollars = 0, parses = 1},574)

Thm19. We can prove that there exists a natural number $b$, such that there exists a natural number $c$, such that there exists a natural number $d$, such that $n$ is equal to the sum of the sum of the sum of the square of $a$ and the square of $b$ and the square of $c$ and the square of $d$ for some natural number $a$ for all instances $n$ of natural numbers.
%% (Scores {tree_length = 91, tree_depth = 22, characters = 362, tokens = 98, subsequent_dollars = 0, initial_dollars = 0, parses = 1},574)

Thm19. For all instances $n$ of natural numbers, we can prove that there exists a natural number $b$, such that there exists a natural number $c$, such that there exists a natural number $d$, such that $n$ is equal to the sum of the sum of the sum of the square of $a$ and the square of $b$ and the square of $c$ and the square of $d$ for some natural number $a$.
%% (Scores {tree_length = 92, tree_depth = 22, characters = 363, tokens = 99, subsequent_dollars = 0, initial_dollars = 0, parses = 1},577)

Thm19. Let $n$ be an instance of natural numbers. Then we can prove that there exists a natural number $b$, such that there exists a natural number $c$, such that there exists a natural number $d$, such that $n$ is equal to the sum of the sum of the sum of the square of $a$ and the square of $b$ and the square of $c$ and the square of $d$ for a natural number $a$.
%% (Scores {tree_length = 90, tree_depth = 21, characters = 366, tokens = 101, subsequent_dollars = 0, initial_dollars = 0, parses = 1},579)

Thm19. Let $n$ be an instance of natural numbers. Then we can prove that there exists a natural number $b$, such that there exists a natural number $c$, such that there exists a natural number $d$, such that $n$ is equal to the sum of the sum of the sum of the square of $a$ and the square of $b$ and the square of $c$ and the square of $d$ for some natural number $a$.
%% (Scores {tree_length = 91, tree_depth = 21, characters = 369, tokens = 101, subsequent_dollars = 0, initial_dollars = 0, parses = 1},583)

Thm19. We can prove that there exists a natural number $a$, such that there exists a natural number $b$, such that there exists a natural number $c$, such that there exists a natural number $d$, such that $n$ is equal to the sum of the sum of the sum of the square of $a$ and the square of $b$ and the square of $c$ and the square of $d$ for every instance $n$ of natural numbers.
%% (Scores {tree_length = 91, tree_depth = 22, characters = 380, tokens = 102, subsequent_dollars = 0, initial_dollars = 0, parses = 1},596)

Thm19. We can prove that there exists a natural number $a$, such that there exists a natural number $b$, such that there exists a natural number $c$, such that there exists a natural number $d$, such that $n$ is equal to the sum of the sum of the sum of the square of $a$ and the square of $b$ and the square of $c$ and the square of $d$ for all instances $n$ of natural numbers.
%% (Scores {tree_length = 92, tree_depth = 22, characters = 379, tokens = 102, subsequent_dollars = 0, initial_dollars = 0, parses = 1},596)

Thm19. For all instances $n$ of natural numbers, we can prove that there exists a natural number $a$, such that there exists a natural number $b$, such that there exists a natural number $c$, such that there exists a natural number $d$, such that $n$ is equal to the sum of the sum of the sum of the square of $a$ and the square of $b$ and the square of $c$ and the square of $d$.
%% (Scores {tree_length = 93, tree_depth = 22, characters = 380, tokens = 103, subsequent_dollars = 0, initial_dollars = 0, parses = 1},599)

Thm19. Let $n$ be an instance of natural numbers. Then we can prove that there exists a natural number $a$, such that there exists a natural number $b$, such that there exists a natural number $c$, such that there exists a natural number $d$, such that $n$ is equal to the sum of the sum of the sum of the square of $a$ and the square of $b$ and the square of $c$ and the square of $d$.
%% (Scores {tree_length = 92, tree_depth = 21, characters = 386, tokens = 105, subsequent_dollars = 0, initial_dollars = 0, parses = 1},605)

Thm20a. Assume that $p$ is a prime natural number. Let $k \in N$. Then if $p = 4 k + 1$, then $p = x ^{ 2}+ y ^{ 2}$ for some natural numbers $x$ and $y$.
%% (Scores {tree_length = 75, tree_depth = 11, characters = 154, tokens = 59, subsequent_dollars = 0, initial_dollars = 0, parses = 1},300)

Thm20a. Assume that $p$ is a prime natural number. Let $k \in N$. Then $p = 4 k + 1$, only if $p = x ^{ 2}+ y ^{ 2}$ for some natural numbers $x$ and $y$.
%% (Scores {tree_length = 75, tree_depth = 11, characters = 154, tokens = 59, subsequent_dollars = 0, initial_dollars = 0, parses = 1},300)

Thm20a. If $p$ is a prime natural number, then for all natural numbers $k$, if $p = 4 k + 1$, then $p = x ^{ 2}+ y ^{ 2}$ for some natural numbers $x$ and $y$.
%% (Scores {tree_length = 73, tree_depth = 13, characters = 159, tokens = 59, subsequent_dollars = 0, initial_dollars = 0, parses = 1},305)

Thm20a. $p$ is a prime natural number, only if for all natural numbers $k$, if $p = 4 k + 1$, then $p = x ^{ 2}+ y ^{ 2}$ for some natural numbers $x$ and $y$.
%% (Scores {tree_length = 73, tree_depth = 13, characters = 159, tokens = 59, subsequent_dollars = 0, initial_dollars = 1, parses = 1},306)

Thm20a. Assume that $p$ is a prime natural number. Let $k \in N$. Assume that $$p = 4 k + 1.$$ then $p = x ^{ 2}+ y ^{ 2}$ for some natural numbers $x$ and $y$.
%% (Scores {tree_length = 76, tree_depth = 13, characters = 160, tokens = 59, subsequent_dollars = 0, initial_dollars = 0, parses = 1},309)

Thm20a. Assume that $p$ is a prime natural number. Let $k \in N$. Assume that $$p = 4 k + 1.$$ then $$p = x ^{ 2}+ y ^{ 2}$$ for some natural numbers $x$ and $y$.
%% (Scores {tree_length = 76, tree_depth = 13, characters = 162, tokens = 59, subsequent_dollars = 0, initial_dollars = 0, parses = 1},311)

Thm20a. Assume that $p$ is a prime natural number. Then if $p = 4 k + 1$, then $p = x ^{ 2}+ y ^{ 2}$ for some natural numbers $x$ and $y$ for every natural number $k$.
%% (Scores {tree_length = 72, tree_depth = 12, characters = 168, tokens = 59, subsequent_dollars = 0, initial_dollars = 0, parses = 1},312)

Thm20a. Assume that $p$ is a prime natural number. Then if $p = 4 k + 1$, then $p = x ^{ 2}+ y ^{ 2}$ for some natural numbers $x$ and $y$ for all natural numbers $k$.
%% (Scores {tree_length = 73, tree_depth = 12, characters = 167, tokens = 59, subsequent_dollars = 0, initial_dollars = 0, parses = 1},312)

Thm20a. Assume that $p$ is a prime natural number. Let $k$ be a natural number. Then if $p = 4 k + 1$, then $p = x ^{ 2}+ y ^{ 2}$ for some natural numbers $x$ and $y$.
%% (Scores {tree_length = 73, tree_depth = 11, characters = 168, tokens = 61, subsequent_dollars = 0, initial_dollars = 0, parses = 1},314)

Thm20a. Assume that $p$ is a prime natural number. Let $k$ be a natural number. Then $p = 4 k + 1$, only if $p = x ^{ 2}+ y ^{ 2}$ for some natural numbers $x$ and $y$.
%% (Scores {tree_length = 73, tree_depth = 11, characters = 168, tokens = 61, subsequent_dollars = 0, initial_dollars = 0, parses = 1},314)

Thm20a. Assume that $p$ is a prime natural number. Then for all natural numbers $k$, if $p = 4 k + 1$, then $p = x ^{ 2}+ y ^{ 2}$ for some natural numbers $x$ and $y$.
%% (Scores {tree_length = 74, tree_depth = 12, characters = 168, tokens = 60, subsequent_dollars = 0, initial_dollars = 0, parses = 1},315)

Thm20a. Assume that $p$ is a prime natural number. Let $k \in N$. Then if $p = 4 k + 1$, then there exist natural numbers $x$ and $y$, such that $p = x ^{ 2}+ y ^{ 2}$.
%% (Scores {tree_length = 76, tree_depth = 11, characters = 168, tokens = 62, subsequent_dollars = 0, initial_dollars = 0, parses = 1},318)

Thm20a. Assume that $p$ is a prime natural number. Let $k \in N$. Then $p = 4 k + 1$, only if there exist natural numbers $x$ and $y$, such that $p = x ^{ 2}+ y ^{ 2}$.
%% (Scores {tree_length = 76, tree_depth = 11, characters = 168, tokens = 62, subsequent_dollars = 0, initial_dollars = 0, parses = 1},318)

Thm20a. If $p$ is a prime natural number, then for all natural numbers $k$, if $p = 4 k + 1$, then there exist natural numbers $x$ and $y$, such that $p = x ^{ 2}+ y ^{ 2}$.
%% (Scores {tree_length = 74, tree_depth = 13, characters = 173, tokens = 62, subsequent_dollars = 0, initial_dollars = 0, parses = 1},323)

Thm20a. Assume that $p$ is a prime natural number. Let $k$ be a natural number. Assume that $$p = 4 k + 1.$$ then $p = x ^{ 2}+ y ^{ 2}$ for some natural numbers $x$ and $y$.
%% (Scores {tree_length = 74, tree_depth = 13, characters = 174, tokens = 61, subsequent_dollars = 0, initial_dollars = 0, parses = 1},323)

Thm20a. $p$ is a prime natural number, only if for all natural numbers $k$, if $p = 4 k + 1$, then there exist natural numbers $x$ and $y$, such that $p = x ^{ 2}+ y ^{ 2}$.
%% (Scores {tree_length = 74, tree_depth = 13, characters = 173, tokens = 62, subsequent_dollars = 0, initial_dollars = 1, parses = 1},324)

Thm20a. Assume that $p$ is a prime natural number. Let $k$ be a natural number. Assume that $$p = 4 k + 1.$$ then $$p = x ^{ 2}+ y ^{ 2}$$ for some natural numbers $x$ and $y$.
%% (Scores {tree_length = 74, tree_depth = 13, characters = 176, tokens = 61, subsequent_dollars = 0, initial_dollars = 0, parses = 1},325)

Thm20a. Assume that $p$ is a prime natural number. Let $k \in N$. Assume that $$p = 4 k + 1.$$ then there exist natural numbers $x$ and $y$, such that $p = x ^{ 2}+ y ^{ 2}$.
%% (Scores {tree_length = 77, tree_depth = 13, characters = 174, tokens = 62, subsequent_dollars = 0, initial_dollars = 0, parses = 1},327)

Thm20a. Assume that $p$ is a prime natural number. Then if $p = 4 k + 1$, then there exist natural numbers $x$ and $y$, such that $p = x ^{ 2}+ y ^{ 2}$ for every natural number $k$.
%% (Scores {tree_length = 73, tree_depth = 12, characters = 182, tokens = 62, subsequent_dollars = 0, initial_dollars = 0, parses = 1},330)

Thm20a. Assume that $p$ is a prime natural number. Then if $p = 4 k + 1$, then there exist natural numbers $x$ and $y$, such that $p = x ^{ 2}+ y ^{ 2}$ for all natural numbers $k$.
%% (Scores {tree_length = 74, tree_depth = 12, characters = 181, tokens = 62, subsequent_dollars = 0, initial_dollars = 0, parses = 1},330)

Thm20a. Assume that $p$ is a prime natural number. Let $k$ be a natural number. Then if $p = 4 k + 1$, then there exist natural numbers $x$ and $y$, such that $p = x ^{ 2}+ y ^{ 2}$.
%% (Scores {tree_length = 74, tree_depth = 11, characters = 182, tokens = 64, subsequent_dollars = 0, initial_dollars = 0, parses = 1},332)

Thm20a. Assume that $p$ is a prime natural number. Let $k$ be a natural number. Then $p = 4 k + 1$, only if there exist natural numbers $x$ and $y$, such that $p = x ^{ 2}+ y ^{ 2}$.
%% (Scores {tree_length = 74, tree_depth = 11, characters = 182, tokens = 64, subsequent_dollars = 0, initial_dollars = 0, parses = 1},332)

Thm20a. Assume that $p$ is a prime natural number. Then for all natural numbers $k$, if $p = 4 k + 1$, then there exist natural numbers $x$ and $y$, such that $p = x ^{ 2}+ y ^{ 2}$.
%% (Scores {tree_length = 75, tree_depth = 12, characters = 182, tokens = 63, subsequent_dollars = 0, initial_dollars = 0, parses = 1},333)

Thm20a. Let $p \in N$. Assume that $p$ is prime. Let $k \in N$. Assume that $$p = 4 k + 1.$$ then $p = x ^{ 2}+ y ^{ 2}$ for a natural number $y$ for a natural number $x$.
%% (Scores {tree_length = 84, tree_depth = 14, characters = 171, tokens = 66, subsequent_dollars = 0, initial_dollars = 0, parses = 1},336)

Thm20a. Assume that $p$ is a prime natural number. Let $k$ be a natural number. Assume that $p = 4 k + 1$. Then there exist natural numbers $x$ and $y$, such that $p = x ^{ 2}+ y ^{ 2}$.
%% (Scores {tree_length = 75, tree_depth = 13, characters = 186, tokens = 64, subsequent_dollars = 0, initial_dollars = 0, parses = 1},339)

Thm20a. Let $p \in N$. Assume that $p$ is prime. Let $k \in N$. Assume that $$p = 4 k + 1.$$ then $p = x ^{ 2}+ y ^{ 2}$ for some natural number $y$ for a natural number $x$.
%% (Scores {tree_length = 85, tree_depth = 14, characters = 174, tokens = 66, subsequent_dollars = 0, initial_dollars = 0, parses = 1},340)

Thm20a. Let $p \in N$. Assume that $p$ is prime. Let $k \in N$. Assume that $$p = 4 k + 1.$$ then $p = x ^{ 2}+ y ^{ 2}$ for a natural number $y$ for some natural number $x$.
%% (Scores {tree_length = 85, tree_depth = 14, characters = 174, tokens = 66, subsequent_dollars = 0, initial_dollars = 0, parses = 1},340)

Thm20a. Assume that $p$ is a prime natural number. Let $k$ be a natural number. Assume that $$p = 4 k + 1.$$ then there exist natural numbers $x$ and $y$, such that $p = x ^{ 2}+ y ^{ 2}$.
%% (Scores {tree_length = 75, tree_depth = 13, characters = 188, tokens = 64, subsequent_dollars = 0, initial_dollars = 0, parses = 1},341)

Thm20a. Let $p \in N$. Assume that $p$ is prime. Let $k \in N$. Assume that $$p = 4 k + 1.$$ then $p = x ^{ 2}+ y ^{ 2}$ for some natural number $y$ for some natural number $x$.
%% (Scores {tree_length = 86, tree_depth = 14, characters = 177, tokens = 66, subsequent_dollars = 0, initial_dollars = 0, parses = 1},344)

Thm20a. Let $p$ be a natural number. Assume that $p$ is prime. Let $k \in N$. Assume that $$p = 4 k + 1.$$ then $p = x ^{ 2}+ y ^{ 2}$ for a natural number $y$ for a natural number $x$.
%% (Scores {tree_length = 82, tree_depth = 14, characters = 185, tokens = 68, subsequent_dollars = 0, initial_dollars = 0, parses = 1},350)

Thm20a. Let $p \in N$. Assume that $p$ is prime. Let $k$ be a natural number. Assume that $$p = 4 k + 1.$$ then $p = x ^{ 2}+ y ^{ 2}$ for a natural number $y$ for a natural number $x$.
%% (Scores {tree_length = 82, tree_depth = 14, characters = 185, tokens = 68, subsequent_dollars = 0, initial_dollars = 0, parses = 1},350)

Thm20a. Let $p \in N$. Assume that $p$ is prime. Let $k \in N$. Then if $p = 4 k + 1$, then there exists a natural number $y$, such that $p = x ^{ 2}+ y ^{ 2}$ for a natural number $x$.
%% (Scores {tree_length = 85, tree_depth = 12, characters = 185, tokens = 70, subsequent_dollars = 0, initial_dollars = 0, parses = 1},353)

Thm20a. Let $p \in N$. Assume that $p$ is prime. Let $k \in N$. Then $p = 4 k + 1$, only if there exists a natural number $y$, such that $p = x ^{ 2}+ y ^{ 2}$ for a natural number $x$.
%% (Scores {tree_length = 85, tree_depth = 12, characters = 185, tokens = 70, subsequent_dollars = 0, initial_dollars = 0, parses = 1},353)

Thm20a. Let $p$ be a natural number. Assume that $p$ is prime. Let $k \in N$. Assume that $$p = 4 k + 1.$$ then $p = x ^{ 2}+ y ^{ 2}$ for some natural number $y$ for a natural number $x$.
%% (Scores {tree_length = 83, tree_depth = 14, characters = 188, tokens = 68, subsequent_dollars = 0, initial_dollars = 0, parses = 1},354)

Thm20a. Let $p$ be a natural number. Assume that $p$ is prime. Let $k \in N$. Assume that $$p = 4 k + 1.$$ then $p = x ^{ 2}+ y ^{ 2}$ for a natural number $y$ for some natural number $x$.
%% (Scores {tree_length = 83, tree_depth = 14, characters = 188, tokens = 68, subsequent_dollars = 0, initial_dollars = 0, parses = 1},354)

Thm20a. Let $p \in N$. Assume that $p$ is prime. Let $k$ be a natural number. Assume that $$p = 4 k + 1.$$ then $p = x ^{ 2}+ y ^{ 2}$ for some natural number $y$ for a natural number $x$.
%% (Scores {tree_length = 83, tree_depth = 14, characters = 188, tokens = 68, subsequent_dollars = 0, initial_dollars = 0, parses = 1},354)

Thm20a. Let $p \in N$. Assume that $p$ is prime. Let $k$ be a natural number. Assume that $$p = 4 k + 1.$$ then $p = x ^{ 2}+ y ^{ 2}$ for a natural number $y$ for some natural number $x$.
%% (Scores {tree_length = 83, tree_depth = 14, characters = 188, tokens = 68, subsequent_dollars = 0, initial_dollars = 0, parses = 1},354)

Thm20a. Let $p \in N$. Assume that $p$ is prime. Let $k \in N$. Then if $p = 4 k + 1$, then there exists a natural number $y$, such that $p = x ^{ 2}+ y ^{ 2}$ for some natural number $x$.
%% (Scores {tree_length = 86, tree_depth = 12, characters = 188, tokens = 70, subsequent_dollars = 0, initial_dollars = 0, parses = 1},357)

Thm20a. Let $p \in N$. Assume that $p$ is prime. Let $k \in N$. Then $p = 4 k + 1$, only if there exists a natural number $y$, such that $p = x ^{ 2}+ y ^{ 2}$ for some natural number $x$.
%% (Scores {tree_length = 86, tree_depth = 12, characters = 188, tokens = 70, subsequent_dollars = 0, initial_dollars = 0, parses = 1},357)

Thm20a. Let $p$ be a natural number. Assume that $p$ is prime. Let $k \in N$. Assume that $$p = 4 k + 1.$$ then $p = x ^{ 2}+ y ^{ 2}$ for some natural number $y$ for some natural number $x$.
%% (Scores {tree_length = 84, tree_depth = 14, characters = 191, tokens = 68, subsequent_dollars = 0, initial_dollars = 0, parses = 1},358)

Thm20a. Let $p \in N$. Assume that $p$ is prime. Let $k$ be a natural number. Assume that $$p = 4 k + 1.$$ then $p = x ^{ 2}+ y ^{ 2}$ for some natural number $y$ for some natural number $x$.
%% (Scores {tree_length = 84, tree_depth = 14, characters = 191, tokens = 68, subsequent_dollars = 0, initial_dollars = 0, parses = 1},358)

Thm20a. Let $p \in N$. Assume that $p$ is prime. Let $k \in N$. Assume that $$p = 4 k + 1.$$ then there exists a natural number $y$, such that $p = x ^{ 2}+ y ^{ 2}$ for a natural number $x$.
%% (Scores {tree_length = 86, tree_depth = 14, characters = 191, tokens = 70, subsequent_dollars = 0, initial_dollars = 0, parses = 1},362)

Thm20a. Let $p \in N$. Then if $p$ is prime, then for all natural numbers $k$, if $p = 4 k + 1$, then there exists a natural number $y$, such that $p = x ^{ 2}+ y ^{ 2}$ for a natural number $x$.
%% (Scores {tree_length = 83, tree_depth = 14, characters = 195, tokens = 71, subsequent_dollars = 0, initial_dollars = 0, parses = 1},364)

Thm20a. Let $p \in N$. Then $p$ is prime, only if for all natural numbers $k$, if $p = 4 k + 1$, then there exists a natural number $y$, such that $p = x ^{ 2}+ y ^{ 2}$ for a natural number $x$.
%% (Scores {tree_length = 83, tree_depth = 14, characters = 195, tokens = 71, subsequent_dollars = 0, initial_dollars = 0, parses = 1},364)

Thm20a. Let $p$ be a natural number. Assume that $p$ is prime. Let $k$ be a natural number. Assume that $$p = 4 k + 1.$$ then $p = x ^{ 2}+ y ^{ 2}$ for a natural number $y$ for a natural number $x$.
%% (Scores {tree_length = 80, tree_depth = 14, characters = 199, tokens = 70, subsequent_dollars = 0, initial_dollars = 0, parses = 1},364)

Thm20a. Let $p \in N$. Assume that $p$ is prime. Then if $p = 4 k + 1$, then there exists a natural number $y$, such that $p = x ^{ 2}+ y ^{ 2}$ for a natural number $x$ for every natural number $k$.
%% (Scores {tree_length = 82, tree_depth = 13, characters = 199, tokens = 70, subsequent_dollars = 0, initial_dollars = 0, parses = 1},365)

Thm20a. Let $p \in N$. Assume that $p$ is prime. Then if $p = 4 k + 1$, then there exists a natural number $y$, such that $p = x ^{ 2}+ y ^{ 2}$ for a natural number $x$ for all natural numbers $k$.
%% (Scores {tree_length = 83, tree_depth = 13, characters = 198, tokens = 70, subsequent_dollars = 0, initial_dollars = 0, parses = 1},365)

Thm20a. Let $p \in N$. Assume that $p$ is prime. Let $k \in N$. Assume that $$p = 4 k + 1.$$ then there exists a natural number $y$, such that $p = x ^{ 2}+ y ^{ 2}$ for some natural number $x$.
%% (Scores {tree_length = 87, tree_depth = 14, characters = 194, tokens = 70, subsequent_dollars = 0, initial_dollars = 0, parses = 1},366)

Thm20a. Let $p$ be a natural number. Assume that $p$ is prime. Let $k \in N$. Then if $p = 4 k + 1$, then there exists a natural number $y$, such that $p = x ^{ 2}+ y ^{ 2}$ for a natural number $x$.
%% (Scores {tree_length = 83, tree_depth = 12, characters = 199, tokens = 72, subsequent_dollars = 0, initial_dollars = 0, parses = 1},367)

Thm20a. Let $p$ be a natural number. Assume that $p$ is prime. Let $k \in N$. Then $p = 4 k + 1$, only if there exists a natural number $y$, such that $p = x ^{ 2}+ y ^{ 2}$ for a natural number $x$.
%% (Scores {tree_length = 83, tree_depth = 12, characters = 199, tokens = 72, subsequent_dollars = 0, initial_dollars = 0, parses = 1},367)

Thm20a. Let $p \in N$. Assume that $p$ is prime. Let $k$ be a natural number. Then if $p = 4 k + 1$, then there exists a natural number $y$, such that $p = x ^{ 2}+ y ^{ 2}$ for a natural number $x$.
%% (Scores {tree_length = 83, tree_depth = 12, characters = 199, tokens = 72, subsequent_dollars = 0, initial_dollars = 0, parses = 1},367)

Thm20a. Let $p \in N$. Assume that $p$ is prime. Let $k$ be a natural number. Then $p = 4 k + 1$, only if there exists a natural number $y$, such that $p = x ^{ 2}+ y ^{ 2}$ for a natural number $x$.
%% (Scores {tree_length = 83, tree_depth = 12, characters = 199, tokens = 72, subsequent_dollars = 0, initial_dollars = 0, parses = 1},367)

Thm20a. Let $p \in N$. Then if $p$ is prime, then for all natural numbers $k$, if $p = 4 k + 1$, then there exists a natural number $y$, such that $p = x ^{ 2}+ y ^{ 2}$ for some natural number $x$.
%% (Scores {tree_length = 84, tree_depth = 14, characters = 198, tokens = 71, subsequent_dollars = 0, initial_dollars = 0, parses = 1},368)

Thm20a. Let $p \in N$. Then $p$ is prime, only if for all natural numbers $k$, if $p = 4 k + 1$, then there exists a natural number $y$, such that $p = x ^{ 2}+ y ^{ 2}$ for some natural number $x$.
%% (Scores {tree_length = 84, tree_depth = 14, characters = 198, tokens = 71, subsequent_dollars = 0, initial_dollars = 0, parses = 1},368)

Thm20a. Let $p \in N$. Assume that $p$ is prime. Then for all natural numbers $k$, if $p = 4 k + 1$, then there exists a natural number $y$, such that $p = x ^{ 2}+ y ^{ 2}$ for a natural number $x$.
%% (Scores {tree_length = 84, tree_depth = 13, characters = 199, tokens = 71, subsequent_dollars = 0, initial_dollars = 0, parses = 1},368)

Thm20a. Let $p$ be a natural number. Assume that $p$ is prime. Let $k$ be a natural number. Assume that $$p = 4 k + 1.$$ then $p = x ^{ 2}+ y ^{ 2}$ for some natural number $y$ for a natural number $x$.
%% (Scores {tree_length = 81, tree_depth = 14, characters = 202, tokens = 70, subsequent_dollars = 0, initial_dollars = 0, parses = 1},368)

Thm20a. Let $p$ be a natural number. Assume that $p$ is prime. Let $k$ be a natural number. Assume that $$p = 4 k + 1.$$ then $p = x ^{ 2}+ y ^{ 2}$ for a natural number $y$ for some natural number $x$.
%% (Scores {tree_length = 81, tree_depth = 14, characters = 202, tokens = 70, subsequent_dollars = 0, initial_dollars = 0, parses = 1},368)

Thm20a. Let $p \in N$. Assume that $p$ is prime. Then if $p = 4 k + 1$, then there exists a natural number $y$, such that $p = x ^{ 2}+ y ^{ 2}$ for some natural number $x$ for every natural number $k$.
%% (Scores {tree_length = 83, tree_depth = 13, characters = 202, tokens = 70, subsequent_dollars = 0, initial_dollars = 0, parses = 1},369)

Thm20a. Let $p \in N$. Assume that $p$ is prime. Then if $p = 4 k + 1$, then there exists a natural number $y$, such that $p = x ^{ 2}+ y ^{ 2}$ for some natural number $x$ for all natural numbers $k$.
%% (Scores {tree_length = 84, tree_depth = 13, characters = 201, tokens = 70, subsequent_dollars = 0, initial_dollars = 0, parses = 1},369)

Thm20a. If $p$ is prime, then for all natural numbers $k$, if $p = 4 k + 1$, then there exists a natural number $y$, such that $p = x ^{ 2}+ y ^{ 2}$ for a natural number $x$ for every natural number $p$.
%% (Scores {tree_length = 80, tree_depth = 15, characters = 204, tokens = 70, subsequent_dollars = 0, initial_dollars = 0, parses = 1},370)

Thm20a. If $p$ is prime, then for all natural numbers $k$, if $p = 4 k + 1$, then there exists a natural number $y$, such that $p = x ^{ 2}+ y ^{ 2}$ for a natural number $x$ for all natural numbers $p$.
%% (Scores {tree_length = 81, tree_depth = 15, characters = 203, tokens = 70, subsequent_dollars = 0, initial_dollars = 0, parses = 1},370)

Thm20a. Let $p$ be a natural number. Assume that $p$ is prime. Let $k \in N$. Then if $p = 4 k + 1$, then there exists a natural number $y$, such that $p = x ^{ 2}+ y ^{ 2}$ for some natural number $x$.
%% (Scores {tree_length = 84, tree_depth = 12, characters = 202, tokens = 72, subsequent_dollars = 0, initial_dollars = 0, parses = 1},371)

Thm20a. Let $p$ be a natural number. Assume that $p$ is prime. Let $k \in N$. Then $p = 4 k + 1$, only if there exists a natural number $y$, such that $p = x ^{ 2}+ y ^{ 2}$ for some natural number $x$.
%% (Scores {tree_length = 84, tree_depth = 12, characters = 202, tokens = 72, subsequent_dollars = 0, initial_dollars = 0, parses = 1},371)

Thm20a. Let $p \in N$. Assume that $p$ is prime. Let $k$ be a natural number. Then if $p = 4 k + 1$, then there exists a natural number $y$, such that $p = x ^{ 2}+ y ^{ 2}$ for some natural number $x$.
%% (Scores {tree_length = 84, tree_depth = 12, characters = 202, tokens = 72, subsequent_dollars = 0, initial_dollars = 0, parses = 1},371)

Thm20a. Let $p \in N$. Assume that $p$ is prime. Let $k$ be a natural number. Then $p = 4 k + 1$, only if there exists a natural number $y$, such that $p = x ^{ 2}+ y ^{ 2}$ for some natural number $x$.
%% (Scores {tree_length = 84, tree_depth = 12, characters = 202, tokens = 72, subsequent_dollars = 0, initial_dollars = 0, parses = 1},371)

Thm20a. Let $p \in N$. Assume that $p$ is prime. Then for all natural numbers $k$, if $p = 4 k + 1$, then there exists a natural number $y$, such that $p = x ^{ 2}+ y ^{ 2}$ for some natural number $x$.
%% (Scores {tree_length = 85, tree_depth = 13, characters = 202, tokens = 71, subsequent_dollars = 0, initial_dollars = 0, parses = 1},372)

Thm20a. Let $p$ be a natural number. Assume that $p$ is prime. Let $k$ be a natural number. Assume that $$p = 4 k + 1.$$ then $p = x ^{ 2}+ y ^{ 2}$ for some natural number $y$ for some natural number $x$.
%% (Scores {tree_length = 82, tree_depth = 14, characters = 205, tokens = 70, subsequent_dollars = 0, initial_dollars = 0, parses = 1},372)

Thm20a. For all natural numbers $p$, if $p$ is prime, then for all natural numbers $k$, if $p = 4 k + 1$, then there exists a natural number $y$, such that $p = x ^{ 2}+ y ^{ 2}$ for a natural number $x$.
%% (Scores {tree_length = 82, tree_depth = 15, characters = 204, tokens = 71, subsequent_dollars = 0, initial_dollars = 0, parses = 1},373)

Thm20a. If $p$ is prime, then for all natural numbers $k$, if $p = 4 k + 1$, then there exists a natural number $y$, such that $p = x ^{ 2}+ y ^{ 2}$ for some natural number $x$ for every natural number $p$.
%% (Scores {tree_length = 81, tree_depth = 15, characters = 207, tokens = 70, subsequent_dollars = 0, initial_dollars = 0, parses = 1},374)

Thm20a. If $p$ is prime, then for all natural numbers $k$, if $p = 4 k + 1$, then there exists a natural number $y$, such that $p = x ^{ 2}+ y ^{ 2}$ for some natural number $x$ for all natural numbers $p$.
%% (Scores {tree_length = 82, tree_depth = 15, characters = 206, tokens = 70, subsequent_dollars = 0, initial_dollars = 0, parses = 1},374)

Thm20a. Let $p$ be a natural number. Assume that $p$ is prime. Let $k \in N$. Assume that $$p = 4 k + 1.$$ then there exists a natural number $y$, such that $p = x ^{ 2}+ y ^{ 2}$ for a natural number $x$.
%% (Scores {tree_length = 84, tree_depth = 14, characters = 205, tokens = 72, subsequent_dollars = 0, initial_dollars = 0, parses = 1},376)

Thm20a. Let $p \in N$. Assume that $p$ is prime. Let $k$ be a natural number. Assume that $$p = 4 k + 1.$$ then there exists a natural number $y$, such that $p = x ^{ 2}+ y ^{ 2}$ for a natural number $x$.
%% (Scores {tree_length = 84, tree_depth = 14, characters = 205, tokens = 72, subsequent_dollars = 0, initial_dollars = 0, parses = 1},376)

Thm20a. For all natural numbers $p$, if $p$ is prime, then for all natural numbers $k$, if $p = 4 k + 1$, then there exists a natural number $y$, such that $p = x ^{ 2}+ y ^{ 2}$ for some natural number $x$.
%% (Scores {tree_length = 83, tree_depth = 15, characters = 207, tokens = 71, subsequent_dollars = 0, initial_dollars = 0, parses = 1},377)

Thm20a. Let $p$ be a natural number. Then if $p$ is prime, then for all natural numbers $k$, if $p = 4 k + 1$, then there exists a natural number $y$, such that $p = x ^{ 2}+ y ^{ 2}$ for a natural number $x$.
%% (Scores {tree_length = 81, tree_depth = 14, characters = 209, tokens = 73, subsequent_dollars = 0, initial_dollars = 0, parses = 1},378)

Thm20a. Let $p$ be a natural number. Then $p$ is prime, only if for all natural numbers $k$, if $p = 4 k + 1$, then there exists a natural number $y$, such that $p = x ^{ 2}+ y ^{ 2}$ for a natural number $x$.
%% (Scores {tree_length = 81, tree_depth = 14, characters = 209, tokens = 73, subsequent_dollars = 0, initial_dollars = 0, parses = 1},378)

Thm20a. Let $p$ be a natural number. Assume that $p$ is prime. Then if $p = 4 k + 1$, then there exists a natural number $y$, such that $p = x ^{ 2}+ y ^{ 2}$ for a natural number $x$ for every natural number $k$.
%% (Scores {tree_length = 80, tree_depth = 13, characters = 213, tokens = 72, subsequent_dollars = 0, initial_dollars = 0, parses = 1},379)

Thm20a. Let $p$ be a natural number. Assume that $p$ is prime. Then if $p = 4 k + 1$, then there exists a natural number $y$, such that $p = x ^{ 2}+ y ^{ 2}$ for a natural number $x$ for all natural numbers $k$.
%% (Scores {tree_length = 81, tree_depth = 13, characters = 212, tokens = 72, subsequent_dollars = 0, initial_dollars = 0, parses = 1},379)

Thm20a. Let $p \in N$. Assume that $p$ is prime. Let $k \in N$. Then if $p = 4 k + 1$, then there exists a natural number $x$, such that there exists a natural number $y$, such that $p = x ^{ 2}+ y ^{ 2}$.
%% (Scores {tree_length = 87, tree_depth = 12, characters = 205, tokens = 74, subsequent_dollars = 0, initial_dollars = 0, parses = 1},379)

Thm20a. Let $p \in N$. Assume that $p$ is prime. Let $k \in N$. Then $p = 4 k + 1$, only if there exists a natural number $x$, such that there exists a natural number $y$, such that $p = x ^{ 2}+ y ^{ 2}$.
%% (Scores {tree_length = 87, tree_depth = 12, characters = 205, tokens = 74, subsequent_dollars = 0, initial_dollars = 0, parses = 1},379)

Thm20a. Let $p$ be a natural number. Assume that $p$ is prime. Let $k \in N$. Assume that $$p = 4 k + 1.$$ then there exists a natural number $y$, such that $p = x ^{ 2}+ y ^{ 2}$ for some natural number $x$.
%% (Scores {tree_length = 85, tree_depth = 14, characters = 208, tokens = 72, subsequent_dollars = 0, initial_dollars = 0, parses = 1},380)

Thm20a. Let $p \in N$. Assume that $p$ is prime. Let $k$ be a natural number. Assume that $$p = 4 k + 1.$$ then there exists a natural number $y$, such that $p = x ^{ 2}+ y ^{ 2}$ for some natural number $x$.
%% (Scores {tree_length = 85, tree_depth = 14, characters = 208, tokens = 72, subsequent_dollars = 0, initial_dollars = 0, parses = 1},380)

Thm20a. Let $p$ be a natural number. Assume that $p$ is prime. Let $k$ be a natural number. Then if $p = 4 k + 1$, then there exists a natural number $y$, such that $p = x ^{ 2}+ y ^{ 2}$ for a natural number $x$.
%% (Scores {tree_length = 81, tree_depth = 12, characters = 213, tokens = 74, subsequent_dollars = 0, initial_dollars = 0, parses = 1},381)

Thm20a. Let $p$ be a natural number. Assume that $p$ is prime. Let $k$ be a natural number. Then $p = 4 k + 1$, only if there exists a natural number $y$, such that $p = x ^{ 2}+ y ^{ 2}$ for a natural number $x$.
%% (Scores {tree_length = 81, tree_depth = 12, characters = 213, tokens = 74, subsequent_dollars = 0, initial_dollars = 0, parses = 1},381)

Thm20a. Let $p$ be a natural number. Then if $p$ is prime, then for all natural numbers $k$, if $p = 4 k + 1$, then there exists a natural number $y$, such that $p = x ^{ 2}+ y ^{ 2}$ for some natural number $x$.
%% (Scores {tree_length = 82, tree_depth = 14, characters = 212, tokens = 73, subsequent_dollars = 0, initial_dollars = 0, parses = 1},382)

Thm20a. Let $p$ be a natural number. Then $p$ is prime, only if for all natural numbers $k$, if $p = 4 k + 1$, then there exists a natural number $y$, such that $p = x ^{ 2}+ y ^{ 2}$ for some natural number $x$.
%% (Scores {tree_length = 82, tree_depth = 14, characters = 212, tokens = 73, subsequent_dollars = 0, initial_dollars = 0, parses = 1},382)

Thm20a. Let $p$ be a natural number. Assume that $p$ is prime. Then for all natural numbers $k$, if $p = 4 k + 1$, then there exists a natural number $y$, such that $p = x ^{ 2}+ y ^{ 2}$ for a natural number $x$.
%% (Scores {tree_length = 82, tree_depth = 13, characters = 213, tokens = 73, subsequent_dollars = 0, initial_dollars = 0, parses = 1},382)

Thm20a. Let $p$ be a natural number. Assume that $p$ is prime. Then if $p = 4 k + 1$, then there exists a natural number $y$, such that $p = x ^{ 2}+ y ^{ 2}$ for some natural number $x$ for every natural number $k$.
%% (Scores {tree_length = 81, tree_depth = 13, characters = 216, tokens = 72, subsequent_dollars = 0, initial_dollars = 0, parses = 1},383)

Thm20a. Let $p$ be a natural number. Assume that $p$ is prime. Then if $p = 4 k + 1$, then there exists a natural number $y$, such that $p = x ^{ 2}+ y ^{ 2}$ for some natural number $x$ for all natural numbers $k$.
%% (Scores {tree_length = 82, tree_depth = 13, characters = 215, tokens = 72, subsequent_dollars = 0, initial_dollars = 0, parses = 1},383)

Thm20a. Let $p$ be a natural number. Assume that $p$ is prime. Let $k$ be a natural number. Then if $p = 4 k + 1$, then there exists a natural number $y$, such that $p = x ^{ 2}+ y ^{ 2}$ for some natural number $x$.
%% (Scores {tree_length = 82, tree_depth = 12, characters = 216, tokens = 74, subsequent_dollars = 0, initial_dollars = 0, parses = 1},385)

Thm20a. Let $p$ be a natural number. Assume that $p$ is prime. Let $k$ be a natural number. Then $p = 4 k + 1$, only if there exists a natural number $y$, such that $p = x ^{ 2}+ y ^{ 2}$ for some natural number $x$.
%% (Scores {tree_length = 82, tree_depth = 12, characters = 216, tokens = 74, subsequent_dollars = 0, initial_dollars = 0, parses = 1},385)

Thm20a. Let $p$ be a natural number. Assume that $p$ is prime. Then for all natural numbers $k$, if $p = 4 k + 1$, then there exists a natural number $y$, such that $p = x ^{ 2}+ y ^{ 2}$ for some natural number $x$.
%% (Scores {tree_length = 83, tree_depth = 13, characters = 216, tokens = 73, subsequent_dollars = 0, initial_dollars = 0, parses = 1},386)

Thm20a. Let $p \in N$. Assume that $p$ is prime. Let $k \in N$. Assume that $$p = 4 k + 1.$$ then there exists a natural number $x$, such that there exists a natural number $y$, such that $p = x ^{ 2}+ y ^{ 2}$.
%% (Scores {tree_length = 88, tree_depth = 14, characters = 211, tokens = 74, subsequent_dollars = 0, initial_dollars = 0, parses = 1},388)

Thm20a. Let $p \in N$. Then if $p$ is prime, then for all natural numbers $k$, if $p = 4 k + 1$, then there exists a natural number $x$, such that there exists a natural number $y$, such that $p = x ^{ 2}+ y ^{ 2}$.
%% (Scores {tree_length = 85, tree_depth = 14, characters = 215, tokens = 75, subsequent_dollars = 0, initial_dollars = 0, parses = 1},390)

Thm20a. Let $p \in N$. Then $p$ is prime, only if for all natural numbers $k$, if $p = 4 k + 1$, then there exists a natural number $x$, such that there exists a natural number $y$, such that $p = x ^{ 2}+ y ^{ 2}$.
%% (Scores {tree_length = 85, tree_depth = 14, characters = 215, tokens = 75, subsequent_dollars = 0, initial_dollars = 0, parses = 1},390)

Thm20a. Let $p$ be a natural number. Assume that $p$ is prime. Let $k$ be a natural number. Assume that $$p = 4 k + 1.$$ then there exists a natural number $y$, such that $p = x ^{ 2}+ y ^{ 2}$ for a natural number $x$.
%% (Scores {tree_length = 82, tree_depth = 14, characters = 219, tokens = 74, subsequent_dollars = 0, initial_dollars = 0, parses = 1},390)

Thm20a. Let $p \in N$. Assume that $p$ is prime. Then if $p = 4 k + 1$, then there exists a natural number $x$, such that there exists a natural number $y$, such that $p = x ^{ 2}+ y ^{ 2}$ for every natural number $k$.
%% (Scores {tree_length = 84, tree_depth = 13, characters = 219, tokens = 74, subsequent_dollars = 0, initial_dollars = 0, parses = 1},391)

Thm20a. Let $p \in N$. Assume that $p$ is prime. Then if $p = 4 k + 1$, then there exists a natural number $x$, such that there exists a natural number $y$, such that $p = x ^{ 2}+ y ^{ 2}$ for all natural numbers $k$.
%% (Scores {tree_length = 85, tree_depth = 13, characters = 218, tokens = 74, subsequent_dollars = 0, initial_dollars = 0, parses = 1},391)

Thm20a. Let $p$ be a natural number. Assume that $p$ is prime. Let $k \in N$. Then if $p = 4 k + 1$, then there exists a natural number $x$, such that there exists a natural number $y$, such that $p = x ^{ 2}+ y ^{ 2}$.
%% (Scores {tree_length = 85, tree_depth = 12, characters = 219, tokens = 76, subsequent_dollars = 0, initial_dollars = 0, parses = 1},393)

Thm20a. Let $p$ be a natural number. Assume that $p$ is prime. Let $k \in N$. Then $p = 4 k + 1$, only if there exists a natural number $x$, such that there exists a natural number $y$, such that $p = x ^{ 2}+ y ^{ 2}$.
%% (Scores {tree_length = 85, tree_depth = 12, characters = 219, tokens = 76, subsequent_dollars = 0, initial_dollars = 0, parses = 1},393)

Thm20a. Let $p \in N$. Assume that $p$ is prime. Let $k$ be a natural number. Then if $p = 4 k + 1$, then there exists a natural number $x$, such that there exists a natural number $y$, such that $p = x ^{ 2}+ y ^{ 2}$.
%% (Scores {tree_length = 85, tree_depth = 12, characters = 219, tokens = 76, subsequent_dollars = 0, initial_dollars = 0, parses = 1},393)

Thm20a. Let $p \in N$. Assume that $p$ is prime. Let $k$ be a natural number. Then $p = 4 k + 1$, only if there exists a natural number $x$, such that there exists a natural number $y$, such that $p = x ^{ 2}+ y ^{ 2}$.
%% (Scores {tree_length = 85, tree_depth = 12, characters = 219, tokens = 76, subsequent_dollars = 0, initial_dollars = 0, parses = 1},393)

Thm20a. Let $p \in N$. Assume that $p$ is prime. Then for all natural numbers $k$, if $p = 4 k + 1$, then there exists a natural number $x$, such that there exists a natural number $y$, such that $p = x ^{ 2}+ y ^{ 2}$.
%% (Scores {tree_length = 86, tree_depth = 13, characters = 219, tokens = 75, subsequent_dollars = 0, initial_dollars = 0, parses = 1},394)

Thm20a. Let $p$ be a natural number. Assume that $p$ is prime. Let $k$ be a natural number. Assume that $$p = 4 k + 1.$$ then there exists a natural number $y$, such that $p = x ^{ 2}+ y ^{ 2}$ for some natural number $x$.
%% (Scores {tree_length = 83, tree_depth = 14, characters = 222, tokens = 74, subsequent_dollars = 0, initial_dollars = 0, parses = 1},394)

Thm20a. If $p$ is prime, then for all natural numbers $k$, if $p = 4 k + 1$, then there exists a natural number $x$, such that there exists a natural number $y$, such that $p = x ^{ 2}+ y ^{ 2}$ for every natural number $p$.
%% (Scores {tree_length = 82, tree_depth = 15, characters = 224, tokens = 74, subsequent_dollars = 0, initial_dollars = 0, parses = 1},396)

Thm20a. If $p$ is prime, then for all natural numbers $k$, if $p = 4 k + 1$, then there exists a natural number $x$, such that there exists a natural number $y$, such that $p = x ^{ 2}+ y ^{ 2}$ for all natural numbers $p$.
%% (Scores {tree_length = 83, tree_depth = 15, characters = 223, tokens = 74, subsequent_dollars = 0, initial_dollars = 0, parses = 1},396)

Thm20a. For all natural numbers $p$, if $p$ is prime, then for all natural numbers $k$, if $p = 4 k + 1$, then there exists a natural number $x$, such that there exists a natural number $y$, such that $p = x ^{ 2}+ y ^{ 2}$.
%% (Scores {tree_length = 84, tree_depth = 15, characters = 224, tokens = 75, subsequent_dollars = 0, initial_dollars = 0, parses = 1},399)

Thm20a. Let $p$ be a natural number. Assume that $p$ is prime. Let $k \in N$. Assume that $$p = 4 k + 1.$$ then there exists a natural number $x$, such that there exists a natural number $y$, such that $p = x ^{ 2}+ y ^{ 2}$.
%% (Scores {tree_length = 86, tree_depth = 14, characters = 225, tokens = 76, subsequent_dollars = 0, initial_dollars = 0, parses = 1},402)

Thm20a. Let $p \in N$. Assume that $p$ is prime. Let $k$ be a natural number. Assume that $$p = 4 k + 1.$$ then there exists a natural number $x$, such that there exists a natural number $y$, such that $p = x ^{ 2}+ y ^{ 2}$.
%% (Scores {tree_length = 86, tree_depth = 14, characters = 225, tokens = 76, subsequent_dollars = 0, initial_dollars = 0, parses = 1},402)

Thm20a. Let $p$ be a natural number. Then if $p$ is prime, then for all natural numbers $k$, if $p = 4 k + 1$, then there exists a natural number $x$, such that there exists a natural number $y$, such that $p = x ^{ 2}+ y ^{ 2}$.
%% (Scores {tree_length = 83, tree_depth = 14, characters = 229, tokens = 77, subsequent_dollars = 0, initial_dollars = 0, parses = 1},404)

Thm20a. Let $p$ be a natural number. Then $p$ is prime, only if for all natural numbers $k$, if $p = 4 k + 1$, then there exists a natural number $x$, such that there exists a natural number $y$, such that $p = x ^{ 2}+ y ^{ 2}$.
%% (Scores {tree_length = 83, tree_depth = 14, characters = 229, tokens = 77, subsequent_dollars = 0, initial_dollars = 0, parses = 1},404)

Thm20a. Let $p$ be a natural number. Assume that $p$ is prime. Then if $p = 4 k + 1$, then there exists a natural number $x$, such that there exists a natural number $y$, such that $p = x ^{ 2}+ y ^{ 2}$ for every natural number $k$.
%% (Scores {tree_length = 82, tree_depth = 13, characters = 233, tokens = 76, subsequent_dollars = 0, initial_dollars = 0, parses = 1},405)

Thm20a. Let $p$ be a natural number. Assume that $p$ is prime. Then if $p = 4 k + 1$, then there exists a natural number $x$, such that there exists a natural number $y$, such that $p = x ^{ 2}+ y ^{ 2}$ for all natural numbers $k$.
%% (Scores {tree_length = 83, tree_depth = 13, characters = 232, tokens = 76, subsequent_dollars = 0, initial_dollars = 0, parses = 1},405)

Thm20a. Let $p$ be a natural number. Assume that $p$ is prime. Let $k$ be a natural number. Then if $p = 4 k + 1$, then there exists a natural number $x$, such that there exists a natural number $y$, such that $p = x ^{ 2}+ y ^{ 2}$.
%% (Scores {tree_length = 83, tree_depth = 12, characters = 233, tokens = 78, subsequent_dollars = 0, initial_dollars = 0, parses = 1},407)

Thm20a. Let $p$ be a natural number. Assume that $p$ is prime. Let $k$ be a natural number. Then $p = 4 k + 1$, only if there exists a natural number $x$, such that there exists a natural number $y$, such that $p = x ^{ 2}+ y ^{ 2}$.
%% (Scores {tree_length = 83, tree_depth = 12, characters = 233, tokens = 78, subsequent_dollars = 0, initial_dollars = 0, parses = 1},407)

Thm20a. Let $p$ be a natural number. Assume that $p$ is prime. Then for all natural numbers $k$, if $p = 4 k + 1$, then there exists a natural number $x$, such that there exists a natural number $y$, such that $p = x ^{ 2}+ y ^{ 2}$.
%% (Scores {tree_length = 84, tree_depth = 13, characters = 233, tokens = 77, subsequent_dollars = 0, initial_dollars = 0, parses = 1},408)

Thm20a. Let $p$ be a natural number. Assume that $p$ is prime. Let $k$ be a natural number. Assume that $p = 4 k + 1$. Then there exists a natural number $x$, such that there exists a natural number $y$, such that $p = x ^{ 2}+ y ^{ 2}$.
%% (Scores {tree_length = 84, tree_depth = 14, characters = 237, tokens = 78, subsequent_dollars = 0, initial_dollars = 0, parses = 1},414)

Thm20a. Let $p$ be a natural number. Assume that $p$ is prime. Let $k$ be a natural number. Assume that $$p = 4 k + 1.$$ then there exists a natural number $x$, such that there exists a natural number $y$, such that $p = x ^{ 2}+ y ^{ 2}$.
%% (Scores {tree_length = 84, tree_depth = 14, characters = 239, tokens = 78, subsequent_dollars = 0, initial_dollars = 0, parses = 1},416)

Thm20a. Let $p \in N$. Assume that $p$ is prime. Let $k \in N$. Assume that $p$ is equal to the sum of the product of $4$ and $k$ and $1$. Then $p$ is equal to the sum of the square of $x$ and the square of $y$ for a natural number $y$ for a natural number $x$.
%% (Scores {tree_length = 96, tree_depth = 17, characters = 261, tokens = 88, subsequent_dollars = 0, initial_dollars = 0, parses = 1},463)

Thm20a. Let $p \in N$. Assume that $p$ is prime. Let $k \in N$. Assume that $p$ is equal to the sum of the product of $4$ and $k$ and $1$. Then $p$ is equal to the sum of the square of $x$ and the square of $y$ for some natural number $y$ for a natural number $x$.
%% (Scores {tree_length = 97, tree_depth = 17, characters = 264, tokens = 88, subsequent_dollars = 0, initial_dollars = 0, parses = 1},467)

Thm20a. Let $p \in N$. Assume that $p$ is prime. Let $k \in N$. Assume that $p$ is equal to the sum of the product of $4$ and $k$ and $1$. Then $p$ is equal to the sum of the square of $x$ and the square of $y$ for a natural number $y$ for some natural number $x$.
%% (Scores {tree_length = 97, tree_depth = 17, characters = 264, tokens = 88, subsequent_dollars = 0, initial_dollars = 0, parses = 1},467)

Thm20a. Let $p \in N$. Assume that $p$ is prime. Let $k \in N$. Assume that $p$ is equal to the sum of the product of $4$ and $k$ and $1$. Then $p$ is equal to the sum of the square of $x$ and the square of $y$ for some natural number $y$ for some natural number $x$.
%% (Scores {tree_length = 98, tree_depth = 17, characters = 267, tokens = 88, subsequent_dollars = 0, initial_dollars = 0, parses = 1},471)

Thm20a. Let $p$ be a natural number. Assume that $p$ is prime. Let $k \in N$. Assume that $p$ is equal to the sum of the product of $4$ and $k$ and $1$. Then $p$ is equal to the sum of the square of $x$ and the square of $y$ for a natural number $y$ for a natural number $x$.
%% (Scores {tree_length = 94, tree_depth = 17, characters = 275, tokens = 90, subsequent_dollars = 0, initial_dollars = 0, parses = 1},477)

Thm20a. Let $p \in N$. Assume that $p$ is prime. Let $k$ be a natural number. Assume that $p$ is equal to the sum of the product of $4$ and $k$ and $1$. Then $p$ is equal to the sum of the square of $x$ and the square of $y$ for a natural number $y$ for a natural number $x$.
%% (Scores {tree_length = 94, tree_depth = 17, characters = 275, tokens = 90, subsequent_dollars = 0, initial_dollars = 0, parses = 1},477)

Thm20a. Let $p$ be a natural number. Assume that $p$ is prime. Let $k \in N$. Assume that $p$ is equal to the sum of the product of $4$ and $k$ and $1$. Then $p$ is equal to the sum of the square of $x$ and the square of $y$ for some natural number $y$ for a natural number $x$.
%% (Scores {tree_length = 95, tree_depth = 17, characters = 278, tokens = 90, subsequent_dollars = 0, initial_dollars = 0, parses = 1},481)

Thm20a. Let $p$ be a natural number. Assume that $p$ is prime. Let $k \in N$. Assume that $p$ is equal to the sum of the product of $4$ and $k$ and $1$. Then $p$ is equal to the sum of the square of $x$ and the square of $y$ for a natural number $y$ for some natural number $x$.
%% (Scores {tree_length = 95, tree_depth = 17, characters = 278, tokens = 90, subsequent_dollars = 0, initial_dollars = 0, parses = 1},481)

Thm20a. Let $p \in N$. Assume that $p$ is prime. Let $k$ be a natural number. Assume that $p$ is equal to the sum of the product of $4$ and $k$ and $1$. Then $p$ is equal to the sum of the square of $x$ and the square of $y$ for some natural number $y$ for a natural number $x$.
%% (Scores {tree_length = 95, tree_depth = 17, characters = 278, tokens = 90, subsequent_dollars = 0, initial_dollars = 0, parses = 1},481)

Thm20a. Let $p \in N$. Assume that $p$ is prime. Let $k$ be a natural number. Assume that $p$ is equal to the sum of the product of $4$ and $k$ and $1$. Then $p$ is equal to the sum of the square of $x$ and the square of $y$ for a natural number $y$ for some natural number $x$.
%% (Scores {tree_length = 95, tree_depth = 17, characters = 278, tokens = 90, subsequent_dollars = 0, initial_dollars = 0, parses = 1},481)

Thm20a. Let $p \in N$. Assume that $p$ is prime. Let $k \in N$. Then if $p$ is equal to the sum of the product of $4$ and $k$ and $1$, then there exists a natural number $y$, such that $p$ is equal to the sum of the square of $x$ and the square of $y$ for a natural number $x$.
%% (Scores {tree_length = 97, tree_depth = 15, characters = 277, tokens = 92, subsequent_dollars = 0, initial_dollars = 0, parses = 1},482)

Thm20a. Let $p \in N$. Assume that $p$ is prime. Let $k \in N$. Then $p$ is equal to the sum of the product of $4$ and $k$ and $1$, only if there exists a natural number $y$, such that $p$ is equal to the sum of the square of $x$ and the square of $y$ for a natural number $x$.
%% (Scores {tree_length = 97, tree_depth = 15, characters = 277, tokens = 92, subsequent_dollars = 0, initial_dollars = 0, parses = 1},482)

Thm20a. Let $p$ be a natural number. Assume that $p$ is prime. Let $k \in N$. Assume that $p$ is equal to the sum of the product of $4$ and $k$ and $1$. Then $p$ is equal to the sum of the square of $x$ and the square of $y$ for some natural number $y$ for some natural number $x$.
%% (Scores {tree_length = 96, tree_depth = 17, characters = 281, tokens = 90, subsequent_dollars = 0, initial_dollars = 0, parses = 1},485)

Thm20a. Let $p \in N$. Assume that $p$ is prime. Let $k$ be a natural number. Assume that $p$ is equal to the sum of the product of $4$ and $k$ and $1$. Then $p$ is equal to the sum of the square of $x$ and the square of $y$ for some natural number $y$ for some natural number $x$.
%% (Scores {tree_length = 96, tree_depth = 17, characters = 281, tokens = 90, subsequent_dollars = 0, initial_dollars = 0, parses = 1},485)

Thm20a. Let $p \in N$. Assume that $p$ is prime. Let $k \in N$. Then if $p$ is equal to the sum of the product of $4$ and $k$ and $1$, then there exists a natural number $y$, such that $p$ is equal to the sum of the square of $x$ and the square of $y$ for some natural number $x$.
%% (Scores {tree_length = 98, tree_depth = 15, characters = 280, tokens = 92, subsequent_dollars = 0, initial_dollars = 0, parses = 1},486)

Thm20a. Let $p \in N$. Assume that $p$ is prime. Let $k \in N$. Then $p$ is equal to the sum of the product of $4$ and $k$ and $1$, only if there exists a natural number $y$, such that $p$ is equal to the sum of the square of $x$ and the square of $y$ for some natural number $x$.
%% (Scores {tree_length = 98, tree_depth = 15, characters = 280, tokens = 92, subsequent_dollars = 0, initial_dollars = 0, parses = 1},486)

Thm20a. Let $p \in N$. Assume that $p$ is prime. Let $k \in N$. Assume that $p$ is equal to the sum of the product of $4$ and $k$ and $1$. Then there exists a natural number $y$, such that $p$ is equal to the sum of the square of $x$ and the square of $y$ for a natural number $x$.
%% (Scores {tree_length = 98, tree_depth = 17, characters = 281, tokens = 92, subsequent_dollars = 0, initial_dollars = 0, parses = 1},489)

Thm20a. Let $p$ be a natural number. Assume that $p$ is prime. Let $k$ be a natural number. Assume that $p$ is equal to the sum of the product of $4$ and $k$ and $1$. Then $p$ is equal to the sum of the square of $x$ and the square of $y$ for a natural number $y$ for a natural number $x$.
%% (Scores {tree_length = 92, tree_depth = 17, characters = 289, tokens = 92, subsequent_dollars = 0, initial_dollars = 0, parses = 1},491)

Thm20a. Let $p \in N$. Then if $p$ is prime, then for all natural numbers $k$, if $p$ is equal to the sum of the product of $4$ and $k$ and $1$, then there exists a natural number $y$, such that $p$ is equal to the sum of the square of $x$ and the square of $y$ for a natural number $x$.
%% (Scores {tree_length = 95, tree_depth = 17, characters = 287, tokens = 93, subsequent_dollars = 0, initial_dollars = 0, parses = 1},493)

Thm20a. Let $p \in N$. Then $p$ is prime, only if for all natural numbers $k$, if $p$ is equal to the sum of the product of $4$ and $k$ and $1$, then there exists a natural number $y$, such that $p$ is equal to the sum of the square of $x$ and the square of $y$ for a natural number $x$.
%% (Scores {tree_length = 95, tree_depth = 17, characters = 287, tokens = 93, subsequent_dollars = 0, initial_dollars = 0, parses = 1},493)

Thm20a. Let $p \in N$. Assume that $p$ is prime. Let $k \in N$. Assume that $p$ is equal to the sum of the product of $4$ and $k$ and $1$. Then there exists a natural number $y$, such that $p$ is equal to the sum of the square of $x$ and the square of $y$ for some natural number $x$.
%% (Scores {tree_length = 99, tree_depth = 17, characters = 284, tokens = 92, subsequent_dollars = 0, initial_dollars = 0, parses = 1},493)

Thm20a. Let $p \in N$. Assume that $p$ is prime. Then if $p$ is equal to the sum of the product of $4$ and $k$ and $1$, then there exists a natural number $y$, such that $p$ is equal to the sum of the square of $x$ and the square of $y$ for a natural number $x$ for every natural number $k$.
%% (Scores {tree_length = 94, tree_depth = 16, characters = 291, tokens = 92, subsequent_dollars = 0, initial_dollars = 0, parses = 1},494)

Thm20a. Let $p \in N$. Assume that $p$ is prime. Then if $p$ is equal to the sum of the product of $4$ and $k$ and $1$, then there exists a natural number $y$, such that $p$ is equal to the sum of the square of $x$ and the square of $y$ for a natural number $x$ for all natural numbers $k$.
%% (Scores {tree_length = 95, tree_depth = 16, characters = 290, tokens = 92, subsequent_dollars = 0, initial_dollars = 0, parses = 1},494)

Thm20a. Let $p$ be a natural number. Assume that $p$ is prime. Let $k$ be a natural number. Assume that $p$ is equal to the sum of the product of $4$ and $k$ and $1$. Then $p$ is equal to the sum of the square of $x$ and the square of $y$ for some natural number $y$ for a natural number $x$.
%% (Scores {tree_length = 93, tree_depth = 17, characters = 292, tokens = 92, subsequent_dollars = 0, initial_dollars = 0, parses = 1},495)

Thm20a. Let $p$ be a natural number. Assume that $p$ is prime. Let $k$ be a natural number. Assume that $p$ is equal to the sum of the product of $4$ and $k$ and $1$. Then $p$ is equal to the sum of the square of $x$ and the square of $y$ for a natural number $y$ for some natural number $x$.
%% (Scores {tree_length = 93, tree_depth = 17, characters = 292, tokens = 92, subsequent_dollars = 0, initial_dollars = 0, parses = 1},495)

Thm20a. Let $p$ be a natural number. Assume that $p$ is prime. Let $k \in N$. Then if $p$ is equal to the sum of the product of $4$ and $k$ and $1$, then there exists a natural number $y$, such that $p$ is equal to the sum of the square of $x$ and the square of $y$ for a natural number $x$.
%% (Scores {tree_length = 95, tree_depth = 15, characters = 291, tokens = 94, subsequent_dollars = 0, initial_dollars = 0, parses = 1},496)

Thm20a. Let $p$ be a natural number. Assume that $p$ is prime. Let $k \in N$. Then $p$ is equal to the sum of the product of $4$ and $k$ and $1$, only if there exists a natural number $y$, such that $p$ is equal to the sum of the square of $x$ and the square of $y$ for a natural number $x$.
%% (Scores {tree_length = 95, tree_depth = 15, characters = 291, tokens = 94, subsequent_dollars = 0, initial_dollars = 0, parses = 1},496)

Thm20a. Let $p \in N$. Assume that $p$ is prime. Let $k$ be a natural number. Then if $p$ is equal to the sum of the product of $4$ and $k$ and $1$, then there exists a natural number $y$, such that $p$ is equal to the sum of the square of $x$ and the square of $y$ for a natural number $x$.
%% (Scores {tree_length = 95, tree_depth = 15, characters = 291, tokens = 94, subsequent_dollars = 0, initial_dollars = 0, parses = 1},496)

Thm20a. Let $p \in N$. Assume that $p$ is prime. Let $k$ be a natural number. Then $p$ is equal to the sum of the product of $4$ and $k$ and $1$, only if there exists a natural number $y$, such that $p$ is equal to the sum of the square of $x$ and the square of $y$ for a natural number $x$.
%% (Scores {tree_length = 95, tree_depth = 15, characters = 291, tokens = 94, subsequent_dollars = 0, initial_dollars = 0, parses = 1},496)

Thm20a. Let $p \in N$. Then if $p$ is prime, then for all natural numbers $k$, if $p$ is equal to the sum of the product of $4$ and $k$ and $1$, then there exists a natural number $y$, such that $p$ is equal to the sum of the square of $x$ and the square of $y$ for some natural number $x$.
%% (Scores {tree_length = 96, tree_depth = 17, characters = 290, tokens = 93, subsequent_dollars = 0, initial_dollars = 0, parses = 1},497)

Thm20a. Let $p \in N$. Then $p$ is prime, only if for all natural numbers $k$, if $p$ is equal to the sum of the product of $4$ and $k$ and $1$, then there exists a natural number $y$, such that $p$ is equal to the sum of the square of $x$ and the square of $y$ for some natural number $x$.
%% (Scores {tree_length = 96, tree_depth = 17, characters = 290, tokens = 93, subsequent_dollars = 0, initial_dollars = 0, parses = 1},497)

Thm20a. Let $p \in N$. Assume that $p$ is prime. Then for all natural numbers $k$, if $p$ is equal to the sum of the product of $4$ and $k$ and $1$, then there exists a natural number $y$, such that $p$ is equal to the sum of the square of $x$ and the square of $y$ for a natural number $x$.
%% (Scores {tree_length = 96, tree_depth = 16, characters = 291, tokens = 93, subsequent_dollars = 0, initial_dollars = 0, parses = 1},497)

Thm20a. Let $p \in N$. Assume that $p$ is prime. Then if $p$ is equal to the sum of the product of $4$ and $k$ and $1$, then there exists a natural number $y$, such that $p$ is equal to the sum of the square of $x$ and the square of $y$ for some natural number $x$ for every natural number $k$.
%% (Scores {tree_length = 95, tree_depth = 16, characters = 294, tokens = 92, subsequent_dollars = 0, initial_dollars = 0, parses = 1},498)

Thm20a. Let $p \in N$. Assume that $p$ is prime. Then if $p$ is equal to the sum of the product of $4$ and $k$ and $1$, then there exists a natural number $y$, such that $p$ is equal to the sum of the square of $x$ and the square of $y$ for some natural number $x$ for all natural numbers $k$.
%% (Scores {tree_length = 96, tree_depth = 16, characters = 293, tokens = 92, subsequent_dollars = 0, initial_dollars = 0, parses = 1},498)

Thm20a. If $p$ is prime, then for all natural numbers $k$, if $p$ is equal to the sum of the product of $4$ and $k$ and $1$, then there exists a natural number $y$, such that $p$ is equal to the sum of the square of $x$ and the square of $y$ for a natural number $x$ for every natural number $p$.
%% (Scores {tree_length = 92, tree_depth = 18, characters = 296, tokens = 92, subsequent_dollars = 0, initial_dollars = 0, parses = 1},499)

Thm20a. If $p$ is prime, then for all natural numbers $k$, if $p$ is equal to the sum of the product of $4$ and $k$ and $1$, then there exists a natural number $y$, such that $p$ is equal to the sum of the square of $x$ and the square of $y$ for a natural number $x$ for all natural numbers $p$.
%% (Scores {tree_length = 93, tree_depth = 18, characters = 295, tokens = 92, subsequent_dollars = 0, initial_dollars = 0, parses = 1},499)

Thm20a. Let $p$ be a natural number. Assume that $p$ is prime. Let $k$ be a natural number. Assume that $p$ is equal to the sum of the product of $4$ and $k$ and $1$. Then $p$ is equal to the sum of the square of $x$ and the square of $y$ for some natural number $y$ for some natural number $x$.
%% (Scores {tree_length = 94, tree_depth = 17, characters = 295, tokens = 92, subsequent_dollars = 0, initial_dollars = 0, parses = 1},499)

Thm20a. Let $p$ be a natural number. Assume that $p$ is prime. Let $k \in N$. Then if $p$ is equal to the sum of the product of $4$ and $k$ and $1$, then there exists a natural number $y$, such that $p$ is equal to the sum of the square of $x$ and the square of $y$ for some natural number $x$.
%% (Scores {tree_length = 96, tree_depth = 15, characters = 294, tokens = 94, subsequent_dollars = 0, initial_dollars = 0, parses = 1},500)

Thm20a. Let $p$ be a natural number. Assume that $p$ is prime. Let $k \in N$. Then $p$ is equal to the sum of the product of $4$ and $k$ and $1$, only if there exists a natural number $y$, such that $p$ is equal to the sum of the square of $x$ and the square of $y$ for some natural number $x$.
%% (Scores {tree_length = 96, tree_depth = 15, characters = 294, tokens = 94, subsequent_dollars = 0, initial_dollars = 0, parses = 1},500)

Thm20a. Let $p \in N$. Assume that $p$ is prime. Let $k$ be a natural number. Then if $p$ is equal to the sum of the product of $4$ and $k$ and $1$, then there exists a natural number $y$, such that $p$ is equal to the sum of the square of $x$ and the square of $y$ for some natural number $x$.
%% (Scores {tree_length = 96, tree_depth = 15, characters = 294, tokens = 94, subsequent_dollars = 0, initial_dollars = 0, parses = 1},500)

Thm20a. Let $p \in N$. Assume that $p$ is prime. Let $k$ be a natural number. Then $p$ is equal to the sum of the product of $4$ and $k$ and $1$, only if there exists a natural number $y$, such that $p$ is equal to the sum of the square of $x$ and the square of $y$ for some natural number $x$.
%% (Scores {tree_length = 96, tree_depth = 15, characters = 294, tokens = 94, subsequent_dollars = 0, initial_dollars = 0, parses = 1},500)

Thm20a. Let $p \in N$. Assume that $p$ is prime. Then for all natural numbers $k$, if $p$ is equal to the sum of the product of $4$ and $k$ and $1$, then there exists a natural number $y$, such that $p$ is equal to the sum of the square of $x$ and the square of $y$ for some natural number $x$.
%% (Scores {tree_length = 97, tree_depth = 16, characters = 294, tokens = 93, subsequent_dollars = 0, initial_dollars = 0, parses = 1},501)

Thm20a. For all natural numbers $p$, if $p$ is prime, then for all natural numbers $k$, if $p$ is equal to the sum of the product of $4$ and $k$ and $1$, then there exists a natural number $y$, such that $p$ is equal to the sum of the square of $x$ and the square of $y$ for a natural number $x$.
%% (Scores {tree_length = 94, tree_depth = 18, characters = 296, tokens = 93, subsequent_dollars = 0, initial_dollars = 0, parses = 1},502)

Thm20a. If $p$ is prime, then for all natural numbers $k$, if $p$ is equal to the sum of the product of $4$ and $k$ and $1$, then there exists a natural number $y$, such that $p$ is equal to the sum of the square of $x$ and the square of $y$ for some natural number $x$ for every natural number $p$.
%% (Scores {tree_length = 93, tree_depth = 18, characters = 299, tokens = 92, subsequent_dollars = 0, initial_dollars = 0, parses = 1},503)

Thm20a. If $p$ is prime, then for all natural numbers $k$, if $p$ is equal to the sum of the product of $4$ and $k$ and $1$, then there exists a natural number $y$, such that $p$ is equal to the sum of the square of $x$ and the square of $y$ for some natural number $x$ for all natural numbers $p$.
%% (Scores {tree_length = 94, tree_depth = 18, characters = 298, tokens = 92, subsequent_dollars = 0, initial_dollars = 0, parses = 1},503)

Thm20a. Let $p$ be a natural number. Assume that $p$ is prime. Let $k \in N$. Assume that $p$ is equal to the sum of the product of $4$ and $k$ and $1$. Then there exists a natural number $y$, such that $p$ is equal to the sum of the square of $x$ and the square of $y$ for a natural number $x$.
%% (Scores {tree_length = 96, tree_depth = 17, characters = 295, tokens = 94, subsequent_dollars = 0, initial_dollars = 0, parses = 1},503)

Thm20a. Let $p \in N$. Assume that $p$ is prime. Let $k$ be a natural number. Assume that $p$ is equal to the sum of the product of $4$ and $k$ and $1$. Then there exists a natural number $y$, such that $p$ is equal to the sum of the square of $x$ and the square of $y$ for a natural number $x$.
%% (Scores {tree_length = 96, tree_depth = 17, characters = 295, tokens = 94, subsequent_dollars = 0, initial_dollars = 0, parses = 1},503)

Thm20a. For all natural numbers $p$, if $p$ is prime, then for all natural numbers $k$, if $p$ is equal to the sum of the product of $4$ and $k$ and $1$, then there exists a natural number $y$, such that $p$ is equal to the sum of the square of $x$ and the square of $y$ for some natural number $x$.
%% (Scores {tree_length = 95, tree_depth = 18, characters = 299, tokens = 93, subsequent_dollars = 0, initial_dollars = 0, parses = 1},506)

Thm20a. Let $p$ be a natural number. Then if $p$ is prime, then for all natural numbers $k$, if $p$ is equal to the sum of the product of $4$ and $k$ and $1$, then there exists a natural number $y$, such that $p$ is equal to the sum of the square of $x$ and the square of $y$ for a natural number $x$.
%% (Scores {tree_length = 93, tree_depth = 17, characters = 301, tokens = 95, subsequent_dollars = 0, initial_dollars = 0, parses = 1},507)

Thm20a. Let $p$ be a natural number. Then $p$ is prime, only if for all natural numbers $k$, if $p$ is equal to the sum of the product of $4$ and $k$ and $1$, then there exists a natural number $y$, such that $p$ is equal to the sum of the square of $x$ and the square of $y$ for a natural number $x$.
%% (Scores {tree_length = 93, tree_depth = 17, characters = 301, tokens = 95, subsequent_dollars = 0, initial_dollars = 0, parses = 1},507)

Thm20a. Let $p$ be a natural number. Assume that $p$ is prime. Let $k \in N$. Assume that $p$ is equal to the sum of the product of $4$ and $k$ and $1$. Then there exists a natural number $y$, such that $p$ is equal to the sum of the square of $x$ and the square of $y$ for some natural number $x$.
%% (Scores {tree_length = 97, tree_depth = 17, characters = 298, tokens = 94, subsequent_dollars = 0, initial_dollars = 0, parses = 1},507)

Thm20a. Let $p \in N$. Assume that $p$ is prime. Let $k$ be a natural number. Assume that $p$ is equal to the sum of the product of $4$ and $k$ and $1$. Then there exists a natural number $y$, such that $p$ is equal to the sum of the square of $x$ and the square of $y$ for some natural number $x$.
%% (Scores {tree_length = 97, tree_depth = 17, characters = 298, tokens = 94, subsequent_dollars = 0, initial_dollars = 0, parses = 1},507)

Thm20a. Let $p$ be a natural number. Assume that $p$ is prime. Then if $p$ is equal to the sum of the product of $4$ and $k$ and $1$, then there exists a natural number $y$, such that $p$ is equal to the sum of the square of $x$ and the square of $y$ for a natural number $x$ for every natural number $k$.
%% (Scores {tree_length = 92, tree_depth = 16, characters = 305, tokens = 94, subsequent_dollars = 0, initial_dollars = 0, parses = 1},508)

Thm20a. Let $p$ be a natural number. Assume that $p$ is prime. Then if $p$ is equal to the sum of the product of $4$ and $k$ and $1$, then there exists a natural number $y$, such that $p$ is equal to the sum of the square of $x$ and the square of $y$ for a natural number $x$ for all natural numbers $k$.
%% (Scores {tree_length = 93, tree_depth = 16, characters = 304, tokens = 94, subsequent_dollars = 0, initial_dollars = 0, parses = 1},508)

Thm20a. Let $p \in N$. Assume that $p$ is prime. Let $k \in N$. Then if $p$ is equal to the sum of the product of $4$ and $k$ and $1$, then there exists a natural number $x$, such that there exists a natural number $y$, such that $p$ is equal to the sum of the square of $x$ and the square of $y$.
%% (Scores {tree_length = 99, tree_depth = 15, characters = 297, tokens = 96, subsequent_dollars = 0, initial_dollars = 0, parses = 1},508)

Thm20a. Let $p \in N$. Assume that $p$ is prime. Let $k \in N$. Then $p$ is equal to the sum of the product of $4$ and $k$ and $1$, only if there exists a natural number $x$, such that there exists a natural number $y$, such that $p$ is equal to the sum of the square of $x$ and the square of $y$.
%% (Scores {tree_length = 99, tree_depth = 15, characters = 297, tokens = 96, subsequent_dollars = 0, initial_dollars = 0, parses = 1},508)

Thm20a. Let $p$ be a natural number. Assume that $p$ is prime. Let $k$ be a natural number. Then if $p$ is equal to the sum of the product of $4$ and $k$ and $1$, then there exists a natural number $y$, such that $p$ is equal to the sum of the square of $x$ and the square of $y$ for a natural number $x$.
%% (Scores {tree_length = 93, tree_depth = 15, characters = 305, tokens = 96, subsequent_dollars = 0, initial_dollars = 0, parses = 1},510)

Thm20a. Let $p$ be a natural number. Assume that $p$ is prime. Let $k$ be a natural number. Then $p$ is equal to the sum of the product of $4$ and $k$ and $1$, only if there exists a natural number $y$, such that $p$ is equal to the sum of the square of $x$ and the square of $y$ for a natural number $x$.
%% (Scores {tree_length = 93, tree_depth = 15, characters = 305, tokens = 96, subsequent_dollars = 0, initial_dollars = 0, parses = 1},510)

Thm20a. Let $p$ be a natural number. Then if $p$ is prime, then for all natural numbers $k$, if $p$ is equal to the sum of the product of $4$ and $k$ and $1$, then there exists a natural number $y$, such that $p$ is equal to the sum of the square of $x$ and the square of $y$ for some natural number $x$.
%% (Scores {tree_length = 94, tree_depth = 17, characters = 304, tokens = 95, subsequent_dollars = 0, initial_dollars = 0, parses = 1},511)

Thm20a. Let $p$ be a natural number. Then $p$ is prime, only if for all natural numbers $k$, if $p$ is equal to the sum of the product of $4$ and $k$ and $1$, then there exists a natural number $y$, such that $p$ is equal to the sum of the square of $x$ and the square of $y$ for some natural number $x$.
%% (Scores {tree_length = 94, tree_depth = 17, characters = 304, tokens = 95, subsequent_dollars = 0, initial_dollars = 0, parses = 1},511)

Thm20a. Let $p$ be a natural number. Assume that $p$ is prime. Then for all natural numbers $k$, if $p$ is equal to the sum of the product of $4$ and $k$ and $1$, then there exists a natural number $y$, such that $p$ is equal to the sum of the square of $x$ and the square of $y$ for a natural number $x$.
%% (Scores {tree_length = 94, tree_depth = 16, characters = 305, tokens = 95, subsequent_dollars = 0, initial_dollars = 0, parses = 1},511)

Thm20a. Let $p$ be a natural number. Assume that $p$ is prime. Then if $p$ is equal to the sum of the product of $4$ and $k$ and $1$, then there exists a natural number $y$, such that $p$ is equal to the sum of the square of $x$ and the square of $y$ for some natural number $x$ for every natural number $k$.
%% (Scores {tree_length = 93, tree_depth = 16, characters = 308, tokens = 94, subsequent_dollars = 0, initial_dollars = 0, parses = 1},512)

Thm20a. Let $p$ be a natural number. Assume that $p$ is prime. Then if $p$ is equal to the sum of the product of $4$ and $k$ and $1$, then there exists a natural number $y$, such that $p$ is equal to the sum of the square of $x$ and the square of $y$ for some natural number $x$ for all natural numbers $k$.
%% (Scores {tree_length = 94, tree_depth = 16, characters = 307, tokens = 94, subsequent_dollars = 0, initial_dollars = 0, parses = 1},512)

Thm20a. Let $p$ be a natural number. Assume that $p$ is prime. Let $k$ be a natural number. Then if $p$ is equal to the sum of the product of $4$ and $k$ and $1$, then there exists a natural number $y$, such that $p$ is equal to the sum of the square of $x$ and the square of $y$ for some natural number $x$.
%% (Scores {tree_length = 94, tree_depth = 15, characters = 308, tokens = 96, subsequent_dollars = 0, initial_dollars = 0, parses = 1},514)

Thm20a. Let $p$ be a natural number. Assume that $p$ is prime. Let $k$ be a natural number. Then $p$ is equal to the sum of the product of $4$ and $k$ and $1$, only if there exists a natural number $y$, such that $p$ is equal to the sum of the square of $x$ and the square of $y$ for some natural number $x$.
%% (Scores {tree_length = 94, tree_depth = 15, characters = 308, tokens = 96, subsequent_dollars = 0, initial_dollars = 0, parses = 1},514)

Thm20a. Let $p$ be a natural number. Assume that $p$ is prime. Then for all natural numbers $k$, if $p$ is equal to the sum of the product of $4$ and $k$ and $1$, then there exists a natural number $y$, such that $p$ is equal to the sum of the square of $x$ and the square of $y$ for some natural number $x$.
%% (Scores {tree_length = 95, tree_depth = 16, characters = 308, tokens = 95, subsequent_dollars = 0, initial_dollars = 0, parses = 1},515)

Thm20a. Let $p \in N$. Assume that $p$ is prime. Let $k \in N$. Assume that $p$ is equal to the sum of the product of $4$ and $k$ and $1$. Then there exists a natural number $x$, such that there exists a natural number $y$, such that $p$ is equal to the sum of the square of $x$ and the square of $y$.
%% (Scores {tree_length = 100, tree_depth = 17, characters = 301, tokens = 96, subsequent_dollars = 0, initial_dollars = 0, parses = 1},515)

Thm20a. Let $p$ be a natural number. Assume that $p$ is prime. Let $k$ be a natural number. Assume that $p$ is equal to the sum of the product of $4$ and $k$ and $1$. Then there exists a natural number $y$, such that $p$ is equal to the sum of the square of $x$ and the square of $y$ for a natural number $x$.
%% (Scores {tree_length = 94, tree_depth = 17, characters = 309, tokens = 96, subsequent_dollars = 0, initial_dollars = 0, parses = 1},517)

Thm20a. Let $p \in N$. Then if $p$ is prime, then for all natural numbers $k$, if $p$ is equal to the sum of the product of $4$ and $k$ and $1$, then there exists a natural number $x$, such that there exists a natural number $y$, such that $p$ is equal to the sum of the square of $x$ and the square of $y$.
%% (Scores {tree_length = 97, tree_depth = 17, characters = 307, tokens = 97, subsequent_dollars = 0, initial_dollars = 0, parses = 1},519)

Thm20a. Let $p \in N$. Then $p$ is prime, only if for all natural numbers $k$, if $p$ is equal to the sum of the product of $4$ and $k$ and $1$, then there exists a natural number $x$, such that there exists a natural number $y$, such that $p$ is equal to the sum of the square of $x$ and the square of $y$.
%% (Scores {tree_length = 97, tree_depth = 17, characters = 307, tokens = 97, subsequent_dollars = 0, initial_dollars = 0, parses = 1},519)

Thm20a. Let $p \in N$. Assume that $p$ is prime. Then if $p$ is equal to the sum of the product of $4$ and $k$ and $1$, then there exists a natural number $x$, such that there exists a natural number $y$, such that $p$ is equal to the sum of the square of $x$ and the square of $y$ for every natural number $k$.
%% (Scores {tree_length = 96, tree_depth = 16, characters = 311, tokens = 96, subsequent_dollars = 0, initial_dollars = 0, parses = 1},520)

Thm20a. Let $p \in N$. Assume that $p$ is prime. Then if $p$ is equal to the sum of the product of $4$ and $k$ and $1$, then there exists a natural number $x$, such that there exists a natural number $y$, such that $p$ is equal to the sum of the square of $x$ and the square of $y$ for all natural numbers $k$.
%% (Scores {tree_length = 97, tree_depth = 16, characters = 310, tokens = 96, subsequent_dollars = 0, initial_dollars = 0, parses = 1},520)

Thm20a. Let $p$ be a natural number. Assume that $p$ is prime. Let $k$ be a natural number. Assume that $p$ is equal to the sum of the product of $4$ and $k$ and $1$. Then there exists a natural number $y$, such that $p$ is equal to the sum of the square of $x$ and the square of $y$ for some natural number $x$.
%% (Scores {tree_length = 95, tree_depth = 17, characters = 312, tokens = 96, subsequent_dollars = 0, initial_dollars = 0, parses = 1},521)

Thm20a. Let $p$ be a natural number. Assume that $p$ is prime. Let $k \in N$. Then if $p$ is equal to the sum of the product of $4$ and $k$ and $1$, then there exists a natural number $x$, such that there exists a natural number $y$, such that $p$ is equal to the sum of the square of $x$ and the square of $y$.
%% (Scores {tree_length = 97, tree_depth = 15, characters = 311, tokens = 98, subsequent_dollars = 0, initial_dollars = 0, parses = 1},522)

Thm20a. Let $p$ be a natural number. Assume that $p$ is prime. Let $k \in N$. Then $p$ is equal to the sum of the product of $4$ and $k$ and $1$, only if there exists a natural number $x$, such that there exists a natural number $y$, such that $p$ is equal to the sum of the square of $x$ and the square of $y$.
%% (Scores {tree_length = 97, tree_depth = 15, characters = 311, tokens = 98, subsequent_dollars = 0, initial_dollars = 0, parses = 1},522)

Thm20a. Let $p \in N$. Assume that $p$ is prime. Let $k$ be a natural number. Then if $p$ is equal to the sum of the product of $4$ and $k$ and $1$, then there exists a natural number $x$, such that there exists a natural number $y$, such that $p$ is equal to the sum of the square of $x$ and the square of $y$.
%% (Scores {tree_length = 97, tree_depth = 15, characters = 311, tokens = 98, subsequent_dollars = 0, initial_dollars = 0, parses = 1},522)

Thm20a. Let $p \in N$. Assume that $p$ is prime. Let $k$ be a natural number. Then $p$ is equal to the sum of the product of $4$ and $k$ and $1$, only if there exists a natural number $x$, such that there exists a natural number $y$, such that $p$ is equal to the sum of the square of $x$ and the square of $y$.
%% (Scores {tree_length = 97, tree_depth = 15, characters = 311, tokens = 98, subsequent_dollars = 0, initial_dollars = 0, parses = 1},522)

Thm20a. Let $p \in N$. Assume that $p$ is prime. Then for all natural numbers $k$, if $p$ is equal to the sum of the product of $4$ and $k$ and $1$, then there exists a natural number $x$, such that there exists a natural number $y$, such that $p$ is equal to the sum of the square of $x$ and the square of $y$.
%% (Scores {tree_length = 98, tree_depth = 16, characters = 311, tokens = 97, subsequent_dollars = 0, initial_dollars = 0, parses = 1},523)

Thm20a. If $p$ is prime, then for all natural numbers $k$, if $p$ is equal to the sum of the product of $4$ and $k$ and $1$, then there exists a natural number $x$, such that there exists a natural number $y$, such that $p$ is equal to the sum of the square of $x$ and the square of $y$ for every natural number $p$.
%% (Scores {tree_length = 94, tree_depth = 18, characters = 316, tokens = 96, subsequent_dollars = 0, initial_dollars = 0, parses = 1},525)

Thm20a. If $p$ is prime, then for all natural numbers $k$, if $p$ is equal to the sum of the product of $4$ and $k$ and $1$, then there exists a natural number $x$, such that there exists a natural number $y$, such that $p$ is equal to the sum of the square of $x$ and the square of $y$ for all natural numbers $p$.
%% (Scores {tree_length = 95, tree_depth = 18, characters = 315, tokens = 96, subsequent_dollars = 0, initial_dollars = 0, parses = 1},525)

Thm20a. For all natural numbers $p$, if $p$ is prime, then for all natural numbers $k$, if $p$ is equal to the sum of the product of $4$ and $k$ and $1$, then there exists a natural number $x$, such that there exists a natural number $y$, such that $p$ is equal to the sum of the square of $x$ and the square of $y$.
%% (Scores {tree_length = 96, tree_depth = 18, characters = 316, tokens = 97, subsequent_dollars = 0, initial_dollars = 0, parses = 1},528)

Thm20a. Let $p$ be a natural number. Assume that $p$ is prime. Let $k \in N$. Assume that $p$ is equal to the sum of the product of $4$ and $k$ and $1$. Then there exists a natural number $x$, such that there exists a natural number $y$, such that $p$ is equal to the sum of the square of $x$ and the square of $y$.
%% (Scores {tree_length = 98, tree_depth = 17, characters = 315, tokens = 98, subsequent_dollars = 0, initial_dollars = 0, parses = 1},529)

Thm20a. Let $p \in N$. Assume that $p$ is prime. Let $k$ be a natural number. Assume that $p$ is equal to the sum of the product of $4$ and $k$ and $1$. Then there exists a natural number $x$, such that there exists a natural number $y$, such that $p$ is equal to the sum of the square of $x$ and the square of $y$.
%% (Scores {tree_length = 98, tree_depth = 17, characters = 315, tokens = 98, subsequent_dollars = 0, initial_dollars = 0, parses = 1},529)

Thm20a. Let $p$ be a natural number. Then if $p$ is prime, then for all natural numbers $k$, if $p$ is equal to the sum of the product of $4$ and $k$ and $1$, then there exists a natural number $x$, such that there exists a natural number $y$, such that $p$ is equal to the sum of the square of $x$ and the square of $y$.
%% (Scores {tree_length = 95, tree_depth = 17, characters = 321, tokens = 99, subsequent_dollars = 0, initial_dollars = 0, parses = 1},533)

Thm20a. Let $p$ be a natural number. Then $p$ is prime, only if for all natural numbers $k$, if $p$ is equal to the sum of the product of $4$ and $k$ and $1$, then there exists a natural number $x$, such that there exists a natural number $y$, such that $p$ is equal to the sum of the square of $x$ and the square of $y$.
%% (Scores {tree_length = 95, tree_depth = 17, characters = 321, tokens = 99, subsequent_dollars = 0, initial_dollars = 0, parses = 1},533)

Thm20a. Let $p$ be a natural number. Assume that $p$ is prime. Then if $p$ is equal to the sum of the product of $4$ and $k$ and $1$, then there exists a natural number $x$, such that there exists a natural number $y$, such that $p$ is equal to the sum of the square of $x$ and the square of $y$ for every natural number $k$.
%% (Scores {tree_length = 94, tree_depth = 16, characters = 325, tokens = 98, subsequent_dollars = 0, initial_dollars = 0, parses = 1},534)

Thm20a. Let $p$ be a natural number. Assume that $p$ is prime. Then if $p$ is equal to the sum of the product of $4$ and $k$ and $1$, then there exists a natural number $x$, such that there exists a natural number $y$, such that $p$ is equal to the sum of the square of $x$ and the square of $y$ for all natural numbers $k$.
%% (Scores {tree_length = 95, tree_depth = 16, characters = 324, tokens = 98, subsequent_dollars = 0, initial_dollars = 0, parses = 1},534)

Thm20a. Let $p$ be a natural number. Assume that $p$ is prime. Let $k$ be a natural number. Then if $p$ is equal to the sum of the product of $4$ and $k$ and $1$, then there exists a natural number $x$, such that there exists a natural number $y$, such that $p$ is equal to the sum of the square of $x$ and the square of $y$.
%% (Scores {tree_length = 95, tree_depth = 15, characters = 325, tokens = 100, subsequent_dollars = 0, initial_dollars = 0, parses = 1},536)

Thm20a. Let $p$ be a natural number. Assume that $p$ is prime. Let $k$ be a natural number. Then $p$ is equal to the sum of the product of $4$ and $k$ and $1$, only if there exists a natural number $x$, such that there exists a natural number $y$, such that $p$ is equal to the sum of the square of $x$ and the square of $y$.
%% (Scores {tree_length = 95, tree_depth = 15, characters = 325, tokens = 100, subsequent_dollars = 0, initial_dollars = 0, parses = 1},536)

Thm20a. Let $p$ be a natural number. Assume that $p$ is prime. Then for all natural numbers $k$, if $p$ is equal to the sum of the product of $4$ and $k$ and $1$, then there exists a natural number $x$, such that there exists a natural number $y$, such that $p$ is equal to the sum of the square of $x$ and the square of $y$.
%% (Scores {tree_length = 96, tree_depth = 16, characters = 325, tokens = 99, subsequent_dollars = 0, initial_dollars = 0, parses = 1},537)

Thm20a. Let $p$ be a natural number. Assume that $p$ is prime. Let $k$ be a natural number. Assume that $p$ is equal to the sum of the product of $4$ and $k$ and $1$. Then there exists a natural number $x$, such that there exists a natural number $y$, such that $p$ is equal to the sum of the square of $x$ and the square of $y$.
%% (Scores {tree_length = 96, tree_depth = 17, characters = 329, tokens = 100, subsequent_dollars = 0, initial_dollars = 0, parses = 1},543)

Thm20a. Let $p$ be an instance of natural numbers. Assume that we can prove that $p$ is prime. Let $k$ be an instance of natural numbers. Assume that we can prove that $p$ is equal to the sum of the product of $4$ and $k$ and $1$. Then we can prove that $p$ is equal to the sum of the square of $x$ and the square of $y$ for a natural number $y$ for a natural number $x$.
%% (Scores {tree_length = 97, tree_depth = 18, characters = 371, tokens = 108, subsequent_dollars = 0, initial_dollars = 0, parses = 1},595)

Thm20a. Let $p$ be an instance of natural numbers. Assume that we can prove that $p$ is prime. Let $k$ be an instance of natural numbers. Assume that we can prove that $p$ is equal to the sum of the product of $4$ and $k$ and $1$. Then we can prove that $p$ is equal to the sum of the square of $x$ and the square of $y$ for some natural number $y$ for a natural number $x$.
%% (Scores {tree_length = 98, tree_depth = 18, characters = 374, tokens = 108, subsequent_dollars = 0, initial_dollars = 0, parses = 1},599)

Thm20a. Let $p$ be an instance of natural numbers. Assume that we can prove that $p$ is prime. Let $k$ be an instance of natural numbers. Assume that we can prove that $p$ is equal to the sum of the product of $4$ and $k$ and $1$. Then we can prove that $p$ is equal to the sum of the square of $x$ and the square of $y$ for a natural number $y$ for some natural number $x$.
%% (Scores {tree_length = 98, tree_depth = 18, characters = 374, tokens = 108, subsequent_dollars = 0, initial_dollars = 0, parses = 1},599)

Thm20a. If we can prove that $p$ is prime, then for all instances $k$ of natural numbers, if we can prove that $p$ is equal to the sum of the product of $4$ and $k$ and $1$, then we can prove that there exists a natural number $y$, such that $p$ is equal to the sum of the square of $x$ and the square of $y$ for a natural number $x$ for every instance $p$ of natural numbers.
%% (Scores {tree_length = 97, tree_depth = 19, characters = 376, tokens = 108, subsequent_dollars = 0, initial_dollars = 0, parses = 1},601)

Thm20a. If we can prove that $p$ is prime, then for all instances $k$ of natural numbers, if we can prove that $p$ is equal to the sum of the product of $4$ and $k$ and $1$, then we can prove that there exists a natural number $y$, such that $p$ is equal to the sum of the square of $x$ and the square of $y$ for a natural number $x$ for all instances $p$ of natural numbers.
%% (Scores {tree_length = 98, tree_depth = 19, characters = 375, tokens = 108, subsequent_dollars = 0, initial_dollars = 0, parses = 1},601)

Thm20a. Let $p$ be an instance of natural numbers. Assume that we can prove that $p$ is prime. Let $k$ be an instance of natural numbers. Assume that we can prove that $p$ is equal to the sum of the product of $4$ and $k$ and $1$. Then we can prove that $p$ is equal to the sum of the square of $x$ and the square of $y$ for some natural number $y$ for some natural number $x$.
%% (Scores {tree_length = 99, tree_depth = 18, characters = 377, tokens = 108, subsequent_dollars = 0, initial_dollars = 0, parses = 1},603)

Thm20a. For all instances $p$ of natural numbers, if we can prove that $p$ is prime, then for all instances $k$ of natural numbers, if we can prove that $p$ is equal to the sum of the product of $4$ and $k$ and $1$, then we can prove that there exists a natural number $y$, such that $p$ is equal to the sum of the square of $x$ and the square of $y$ for a natural number $x$.
%% (Scores {tree_length = 99, tree_depth = 19, characters = 376, tokens = 109, subsequent_dollars = 0, initial_dollars = 0, parses = 1},604)

Thm20a. If we can prove that $p$ is prime, then for all instances $k$ of natural numbers, if we can prove that $p$ is equal to the sum of the product of $4$ and $k$ and $1$, then we can prove that there exists a natural number $y$, such that $p$ is equal to the sum of the square of $x$ and the square of $y$ for some natural number $x$ for every instance $p$ of natural numbers.
%% (Scores {tree_length = 98, tree_depth = 19, characters = 379, tokens = 108, subsequent_dollars = 0, initial_dollars = 0, parses = 1},605)

Thm20a. If we can prove that $p$ is prime, then for all instances $k$ of natural numbers, if we can prove that $p$ is equal to the sum of the product of $4$ and $k$ and $1$, then we can prove that there exists a natural number $y$, such that $p$ is equal to the sum of the square of $x$ and the square of $y$ for some natural number $x$ for all instances $p$ of natural numbers.
%% (Scores {tree_length = 99, tree_depth = 19, characters = 378, tokens = 108, subsequent_dollars = 0, initial_dollars = 0, parses = 1},605)

Thm20a. For all instances $p$ of natural numbers, if we can prove that $p$ is prime, then for all instances $k$ of natural numbers, if we can prove that $p$ is equal to the sum of the product of $4$ and $k$ and $1$, then we can prove that there exists a natural number $y$, such that $p$ is equal to the sum of the square of $x$ and the square of $y$ for some natural number $x$.
%% (Scores {tree_length = 100, tree_depth = 19, characters = 379, tokens = 109, subsequent_dollars = 0, initial_dollars = 0, parses = 1},608)

Thm20a. Let $p$ be an instance of natural numbers. Then if we can prove that $p$ is prime, then for all instances $k$ of natural numbers, if we can prove that $p$ is equal to the sum of the product of $4$ and $k$ and $1$, then we can prove that there exists a natural number $y$, such that $p$ is equal to the sum of the square of $x$ and the square of $y$ for a natural number $x$.
%% (Scores {tree_length = 98, tree_depth = 18, characters = 382, tokens = 111, subsequent_dollars = 0, initial_dollars = 0, parses = 1},610)

Thm20a. Let $p$ be an instance of natural numbers. Then we can prove that $p$ is prime, only if for all instances $k$ of natural numbers, if we can prove that $p$ is equal to the sum of the product of $4$ and $k$ and $1$, then we can prove that there exists a natural number $y$, such that $p$ is equal to the sum of the square of $x$ and the square of $y$ for a natural number $x$.
%% (Scores {tree_length = 98, tree_depth = 18, characters = 382, tokens = 111, subsequent_dollars = 0, initial_dollars = 0, parses = 1},610)

Thm20a. Let $p$ be an instance of natural numbers. Assume that we can prove that $p$ is prime. Then if we can prove that $p$ is equal to the sum of the product of $4$ and $k$ and $1$, then we can prove that there exists a natural number $y$, such that $p$ is equal to the sum of the square of $x$ and the square of $y$ for a natural number $x$ for every instance $k$ of natural numbers.
%% (Scores {tree_length = 97, tree_depth = 17, characters = 386, tokens = 110, subsequent_dollars = 0, initial_dollars = 0, parses = 1},611)

Thm20a. Let $p$ be an instance of natural numbers. Assume that we can prove that $p$ is prime. Then if we can prove that $p$ is equal to the sum of the product of $4$ and $k$ and $1$, then we can prove that there exists a natural number $y$, such that $p$ is equal to the sum of the square of $x$ and the square of $y$ for a natural number $x$ for all instances $k$ of natural numbers.
%% (Scores {tree_length = 98, tree_depth = 17, characters = 385, tokens = 110, subsequent_dollars = 0, initial_dollars = 0, parses = 1},611)

Thm20a. Let $p$ be an instance of natural numbers. Then if we can prove that $p$ is prime, then for all instances $k$ of natural numbers, if we can prove that $p$ is equal to the sum of the product of $4$ and $k$ and $1$, then we can prove that there exists a natural number $y$, such that $p$ is equal to the sum of the square of $x$ and the square of $y$ for some natural number $x$.
%% (Scores {tree_length = 99, tree_depth = 18, characters = 385, tokens = 111, subsequent_dollars = 0, initial_dollars = 0, parses = 1},614)

Thm20a. Let $p$ be an instance of natural numbers. Then we can prove that $p$ is prime, only if for all instances $k$ of natural numbers, if we can prove that $p$ is equal to the sum of the product of $4$ and $k$ and $1$, then we can prove that there exists a natural number $y$, such that $p$ is equal to the sum of the square of $x$ and the square of $y$ for some natural number $x$.
%% (Scores {tree_length = 99, tree_depth = 18, characters = 385, tokens = 111, subsequent_dollars = 0, initial_dollars = 0, parses = 1},614)

Thm20a. Let $p$ be an instance of natural numbers. Assume that we can prove that $p$ is prime. Then for all instances $k$ of natural numbers, if we can prove that $p$ is equal to the sum of the product of $4$ and $k$ and $1$, then we can prove that there exists a natural number $y$, such that $p$ is equal to the sum of the square of $x$ and the square of $y$ for a natural number $x$.
%% (Scores {tree_length = 99, tree_depth = 17, characters = 386, tokens = 111, subsequent_dollars = 0, initial_dollars = 0, parses = 1},614)

Thm20a. Let $p$ be an instance of natural numbers. Assume that we can prove that $p$ is prime. Let $k$ be an instance of natural numbers. Then if we can prove that $p$ is equal to the sum of the product of $4$ and $k$ and $1$, then we can prove that there exists a natural number $y$, such that $p$ is equal to the sum of the square of $x$ and the square of $y$ for a natural number $x$.
%% (Scores {tree_length = 98, tree_depth = 16, characters = 387, tokens = 112, subsequent_dollars = 0, initial_dollars = 0, parses = 1},614)

Thm20a. Let $p$ be an instance of natural numbers. Assume that we can prove that $p$ is prime. Let $k$ be an instance of natural numbers. Then we can prove that $p$ is equal to the sum of the product of $4$ and $k$ and $1$, only if we can prove that there exists a natural number $y$, such that $p$ is equal to the sum of the square of $x$ and the square of $y$ for a natural number $x$.
%% (Scores {tree_length = 98, tree_depth = 16, characters = 387, tokens = 112, subsequent_dollars = 0, initial_dollars = 0, parses = 1},614)

Thm20a. Let $p$ be an instance of natural numbers. Assume that we can prove that $p$ is prime. Then if we can prove that $p$ is equal to the sum of the product of $4$ and $k$ and $1$, then we can prove that there exists a natural number $y$, such that $p$ is equal to the sum of the square of $x$ and the square of $y$ for some natural number $x$ for every instance $k$ of natural numbers.
%% (Scores {tree_length = 98, tree_depth = 17, characters = 389, tokens = 110, subsequent_dollars = 0, initial_dollars = 0, parses = 1},615)

Thm20a. Let $p$ be an instance of natural numbers. Assume that we can prove that $p$ is prime. Then if we can prove that $p$ is equal to the sum of the product of $4$ and $k$ and $1$, then we can prove that there exists a natural number $y$, such that $p$ is equal to the sum of the square of $x$ and the square of $y$ for some natural number $x$ for all instances $k$ of natural numbers.
%% (Scores {tree_length = 99, tree_depth = 17, characters = 388, tokens = 110, subsequent_dollars = 0, initial_dollars = 0, parses = 1},615)

Thm20a. Let $p$ be an instance of natural numbers. Assume that we can prove that $p$ is prime. Then for all instances $k$ of natural numbers, if we can prove that $p$ is equal to the sum of the product of $4$ and $k$ and $1$, then we can prove that there exists a natural number $y$, such that $p$ is equal to the sum of the square of $x$ and the square of $y$ for some natural number $x$.
%% (Scores {tree_length = 100, tree_depth = 17, characters = 389, tokens = 111, subsequent_dollars = 0, initial_dollars = 0, parses = 1},618)

Thm20a. Let $p$ be an instance of natural numbers. Assume that we can prove that $p$ is prime. Let $k$ be an instance of natural numbers. Then if we can prove that $p$ is equal to the sum of the product of $4$ and $k$ and $1$, then we can prove that there exists a natural number $y$, such that $p$ is equal to the sum of the square of $x$ and the square of $y$ for some natural number $x$.
%% (Scores {tree_length = 99, tree_depth = 16, characters = 390, tokens = 112, subsequent_dollars = 0, initial_dollars = 0, parses = 1},618)

Thm20a. Let $p$ be an instance of natural numbers. Assume that we can prove that $p$ is prime. Let $k$ be an instance of natural numbers. Then we can prove that $p$ is equal to the sum of the product of $4$ and $k$ and $1$, only if we can prove that there exists a natural number $y$, such that $p$ is equal to the sum of the square of $x$ and the square of $y$ for some natural number $x$.
%% (Scores {tree_length = 99, tree_depth = 16, characters = 390, tokens = 112, subsequent_dollars = 0, initial_dollars = 0, parses = 1},618)

Thm20a. Let $p$ be an instance of natural numbers. Assume that we can prove that $p$ is prime. Let $k$ be an instance of natural numbers. Assume that we can prove that $p$ is equal to the sum of the product of $4$ and $k$ and $1$. Then we can prove that there exists a natural number $y$, such that $p$ is equal to the sum of the square of $x$ and the square of $y$ for a natural number $x$.
%% (Scores {tree_length = 99, tree_depth = 18, characters = 391, tokens = 112, subsequent_dollars = 0, initial_dollars = 0, parses = 1},621)

Thm20a. Let $p$ be an instance of natural numbers. Assume that we can prove that $p$ is prime. Let $k$ be an instance of natural numbers. Assume that we can prove that $p$ is equal to the sum of the product of $4$ and $k$ and $1$. Then we can prove that there exists a natural number $y$, such that $p$ is equal to the sum of the square of $x$ and the square of $y$ for some natural number $x$.
%% (Scores {tree_length = 100, tree_depth = 18, characters = 394, tokens = 112, subsequent_dollars = 0, initial_dollars = 0, parses = 1},625)

Thm20a. If we can prove that $p$ is prime, then for all instances $k$ of natural numbers, if we can prove that $p$ is equal to the sum of the product of $4$ and $k$ and $1$, then we can prove that there exists a natural number $x$, such that there exists a natural number $y$, such that $p$ is equal to the sum of the square of $x$ and the square of $y$ for every instance $p$ of natural numbers.
%% (Scores {tree_length = 99, tree_depth = 19, characters = 396, tokens = 112, subsequent_dollars = 0, initial_dollars = 0, parses = 1},627)

Thm20a. If we can prove that $p$ is prime, then for all instances $k$ of natural numbers, if we can prove that $p$ is equal to the sum of the product of $4$ and $k$ and $1$, then we can prove that there exists a natural number $x$, such that there exists a natural number $y$, such that $p$ is equal to the sum of the square of $x$ and the square of $y$ for all instances $p$ of natural numbers.
%% (Scores {tree_length = 100, tree_depth = 19, characters = 395, tokens = 112, subsequent_dollars = 0, initial_dollars = 0, parses = 1},627)

Thm20a. For all instances $p$ of natural numbers, if we can prove that $p$ is prime, then for all instances $k$ of natural numbers, if we can prove that $p$ is equal to the sum of the product of $4$ and $k$ and $1$, then we can prove that there exists a natural number $x$, such that there exists a natural number $y$, such that $p$ is equal to the sum of the square of $x$ and the square of $y$.
%% (Scores {tree_length = 101, tree_depth = 19, characters = 396, tokens = 113, subsequent_dollars = 0, initial_dollars = 0, parses = 1},630)

Thm20a. Let $p$ be an instance of natural numbers. Then if we can prove that $p$ is prime, then for all instances $k$ of natural numbers, if we can prove that $p$ is equal to the sum of the product of $4$ and $k$ and $1$, then we can prove that there exists a natural number $x$, such that there exists a natural number $y$, such that $p$ is equal to the sum of the square of $x$ and the square of $y$.
%% (Scores {tree_length = 100, tree_depth = 18, characters = 402, tokens = 115, subsequent_dollars = 0, initial_dollars = 0, parses = 1},636)

Thm20a. Let $p$ be an instance of natural numbers. Then we can prove that $p$ is prime, only if for all instances $k$ of natural numbers, if we can prove that $p$ is equal to the sum of the product of $4$ and $k$ and $1$, then we can prove that there exists a natural number $x$, such that there exists a natural number $y$, such that $p$ is equal to the sum of the square of $x$ and the square of $y$.
%% (Scores {tree_length = 100, tree_depth = 18, characters = 402, tokens = 115, subsequent_dollars = 0, initial_dollars = 0, parses = 1},636)

Thm20a. Let $p$ be an instance of natural numbers. Assume that we can prove that $p$ is prime. Then if we can prove that $p$ is equal to the sum of the product of $4$ and $k$ and $1$, then we can prove that there exists a natural number $x$, such that there exists a natural number $y$, such that $p$ is equal to the sum of the square of $x$ and the square of $y$ for every instance $k$ of natural numbers.
%% (Scores {tree_length = 99, tree_depth = 17, characters = 406, tokens = 114, subsequent_dollars = 0, initial_dollars = 0, parses = 1},637)

Thm20a. Let $p$ be an instance of natural numbers. Assume that we can prove that $p$ is prime. Then if we can prove that $p$ is equal to the sum of the product of $4$ and $k$ and $1$, then we can prove that there exists a natural number $x$, such that there exists a natural number $y$, such that $p$ is equal to the sum of the square of $x$ and the square of $y$ for all instances $k$ of natural numbers.
%% (Scores {tree_length = 100, tree_depth = 17, characters = 405, tokens = 114, subsequent_dollars = 0, initial_dollars = 0, parses = 1},637)

Thm20a. Let $p$ be an instance of natural numbers. Assume that we can prove that $p$ is prime. Then for all instances $k$ of natural numbers, if we can prove that $p$ is equal to the sum of the product of $4$ and $k$ and $1$, then we can prove that there exists a natural number $x$, such that there exists a natural number $y$, such that $p$ is equal to the sum of the square of $x$ and the square of $y$.
%% (Scores {tree_length = 101, tree_depth = 17, characters = 406, tokens = 115, subsequent_dollars = 0, initial_dollars = 0, parses = 1},640)

Thm20a. Let $p$ be an instance of natural numbers. Assume that we can prove that $p$ is prime. Let $k$ be an instance of natural numbers. Then if we can prove that $p$ is equal to the sum of the product of $4$ and $k$ and $1$, then we can prove that there exists a natural number $x$, such that there exists a natural number $y$, such that $p$ is equal to the sum of the square of $x$ and the square of $y$.
%% (Scores {tree_length = 100, tree_depth = 16, characters = 407, tokens = 116, subsequent_dollars = 0, initial_dollars = 0, parses = 1},640)

Thm20a. Let $p$ be an instance of natural numbers. Assume that we can prove that $p$ is prime. Let $k$ be an instance of natural numbers. Then we can prove that $p$ is equal to the sum of the product of $4$ and $k$ and $1$, only if we can prove that there exists a natural number $x$, such that there exists a natural number $y$, such that $p$ is equal to the sum of the square of $x$ and the square of $y$.
%% (Scores {tree_length = 100, tree_depth = 16, characters = 407, tokens = 116, subsequent_dollars = 0, initial_dollars = 0, parses = 1},640)

Thm20a. Let $p$ be an instance of natural numbers. Assume that we can prove that $p$ is prime. Let $k$ be an instance of natural numbers. Assume that we can prove that $p$ is equal to the sum of the product of $4$ and $k$ and $1$. Then we can prove that there exists a natural number $x$, such that there exists a natural number $y$, such that $p$ is equal to the sum of the square of $x$ and the square of $y$.
%% (Scores {tree_length = 101, tree_depth = 18, characters = 411, tokens = 116, subsequent_dollars = 0, initial_dollars = 0, parses = 1},647)

Thm20b. If $p$ is a prime natural number, then if $p \equiv 1 \pmod{ 4}$, then $p = x ^{ 2}+ y ^{ 2}$ for some natural numbers $x$ and $y$.
%% (Scores {tree_length = 56, tree_depth = 12, characters = 139, tokens = 52, subsequent_dollars = 0, initial_dollars = 0, parses = 1},260)

Thm20b. $p$ is a prime natural number, only if if $p \equiv 1 \pmod{ 4}$, then $p = x ^{ 2}+ y ^{ 2}$ for some natural numbers $x$ and $y$.
%% (Scores {tree_length = 56, tree_depth = 12, characters = 139, tokens = 52, subsequent_dollars = 0, initial_dollars = 1, parses = 1},261)

Thm20b. Assume that $p$ is a prime natural number. Then if $p \equiv 1 \pmod{ 4}$, then $p = x ^{ 2}+ y ^{ 2}$ for some natural numbers $x$ and $y$.
%% (Scores {tree_length = 57, tree_depth = 11, characters = 148, tokens = 53, subsequent_dollars = 0, initial_dollars = 0, parses = 1},270)

Thm20b. Assume that $p$ is a prime natural number. Then $p \equiv 1 \pmod{ 4}$, only if $p = x ^{ 2}+ y ^{ 2}$ for some natural numbers $x$ and $y$.
%% (Scores {tree_length = 57, tree_depth = 11, characters = 148, tokens = 53, subsequent_dollars = 0, initial_dollars = 0, parses = 1},270)

Thm20b. Assume that $p$ is a prime natural number. Assume that $$p \equiv 1 \pmod{ 4}.$$ then $p = x ^{ 2}+ y ^{ 2}$ for some natural numbers $x$ and $y$.
%% (Scores {tree_length = 58, tree_depth = 10, characters = 154, tokens = 53, subsequent_dollars = 0, initial_dollars = 0, parses = 1},276)

Thm20b. If $p$ is a prime natural number, then if $p \equiv 1 \pmod{ 4}$, then there exist natural numbers $x$ and $y$, such that $p = x ^{ 2}+ y ^{ 2}$.
%% (Scores {tree_length = 57, tree_depth = 12, characters = 153, tokens = 55, subsequent_dollars = 0, initial_dollars = 0, parses = 1},278)

Thm20b. Assume that $p$ is a prime natural number. Assume that $$p \equiv 1 \pmod{ 4}.$$ then $$p = x ^{ 2}+ y ^{ 2}$$ for some natural numbers $x$ and $y$.
%% (Scores {tree_length = 58, tree_depth = 10, characters = 156, tokens = 53, subsequent_dollars = 0, initial_dollars = 0, parses = 1},278)

Thm20b. $p$ is a prime natural number, only if if $p \equiv 1 \pmod{ 4}$, then there exist natural numbers $x$ and $y$, such that $p = x ^{ 2}+ y ^{ 2}$.
%% (Scores {tree_length = 57, tree_depth = 12, characters = 153, tokens = 55, subsequent_dollars = 0, initial_dollars = 1, parses = 1},279)

Thm20b. Assume that $p$ is a prime natural number. Then if $p \equiv 1 \pmod{ 4}$, then there exist natural numbers $x$ and $y$, such that $p = x ^{ 2}+ y ^{ 2}$.
%% (Scores {tree_length = 58, tree_depth = 11, characters = 162, tokens = 56, subsequent_dollars = 0, initial_dollars = 0, parses = 1},288)

Thm20b. Assume that $p$ is a prime natural number. Then $p \equiv 1 \pmod{ 4}$, only if there exist natural numbers $x$ and $y$, such that $p = x ^{ 2}+ y ^{ 2}$.
%% (Scores {tree_length = 58, tree_depth = 11, characters = 162, tokens = 56, subsequent_dollars = 0, initial_dollars = 0, parses = 1},288)

Thm20b. Assume that $p$ is a prime natural number. Assume that $p \equiv 1 \pmod{ 4}$. Then there exist natural numbers $x$ and $y$, such that $p = x ^{ 2}+ y ^{ 2}$.
%% (Scores {tree_length = 59, tree_depth = 10, characters = 166, tokens = 56, subsequent_dollars = 0, initial_dollars = 0, parses = 1},292)

Thm20b. Assume that $p$ is a prime natural number. Assume that $$p \equiv 1 \pmod{ 4}.$$ then there exist natural numbers $x$ and $y$, such that $p = x ^{ 2}+ y ^{ 2}$.
%% (Scores {tree_length = 59, tree_depth = 10, characters = 168, tokens = 56, subsequent_dollars = 0, initial_dollars = 0, parses = 1},294)

Thm20b. Let $p \in N$. Assume that $p$ is prime. Assume that $$p \equiv 1 \pmod{ 4}.$$ then $p = x ^{ 2}+ y ^{ 2}$ for a natural number $y$ for a natural number $x$.
%% (Scores {tree_length = 66, tree_depth = 11, characters = 165, tokens = 60, subsequent_dollars = 0, initial_dollars = 0, parses = 1},303)

Thm20b. Let $p \in N$. Assume that $p$ is prime. Assume that $$p \equiv 1 \pmod{ 4}.$$ then $p = x ^{ 2}+ y ^{ 2}$ for some natural number $y$ for a natural number $x$.
%% (Scores {tree_length = 67, tree_depth = 11, characters = 168, tokens = 60, subsequent_dollars = 0, initial_dollars = 0, parses = 1},307)

Thm20b. Let $p \in N$. Assume that $p$ is prime. Assume that $$p \equiv 1 \pmod{ 4}.$$ then $p = x ^{ 2}+ y ^{ 2}$ for a natural number $y$ for some natural number $x$.
%% (Scores {tree_length = 67, tree_depth = 11, characters = 168, tokens = 60, subsequent_dollars = 0, initial_dollars = 0, parses = 1},307)

Thm20b. Let $p \in N$. Assume that $p$ is prime. Assume that $$p \equiv 1 \pmod{ 4}.$$ then $p = x ^{ 2}+ y ^{ 2}$ for some natural number $y$ for some natural number $x$.
%% (Scores {tree_length = 68, tree_depth = 11, characters = 171, tokens = 60, subsequent_dollars = 0, initial_dollars = 0, parses = 1},311)

Thm20b. Let $p$ be a natural number. Assume that $p$ is prime. Assume that $$p \equiv 1 \pmod{ 4}.$$ then $p = x ^{ 2}+ y ^{ 2}$ for a natural number $y$ for a natural number $x$.
%% (Scores {tree_length = 64, tree_depth = 11, characters = 179, tokens = 62, subsequent_dollars = 0, initial_dollars = 0, parses = 1},317)

Thm20b. Let $p \in N$. Then if $p$ is prime, then if $p \equiv 1 \pmod{ 4}$, then there exists a natural number $y$, such that $p = x ^{ 2}+ y ^{ 2}$ for a natural number $x$.
%% (Scores {tree_length = 66, tree_depth = 13, characters = 175, tokens = 64, subsequent_dollars = 0, initial_dollars = 0, parses = 1},319)

Thm20b. Let $p \in N$. Then $p$ is prime, only if if $p \equiv 1 \pmod{ 4}$, then there exists a natural number $y$, such that $p = x ^{ 2}+ y ^{ 2}$ for a natural number $x$.
%% (Scores {tree_length = 66, tree_depth = 13, characters = 175, tokens = 64, subsequent_dollars = 0, initial_dollars = 0, parses = 1},319)

Thm20b. Let $p$ be a natural number. Assume that $p$ is prime. Assume that $$p \equiv 1 \pmod{ 4}.$$ then $p = x ^{ 2}+ y ^{ 2}$ for some natural number $y$ for a natural number $x$.
%% (Scores {tree_length = 65, tree_depth = 11, characters = 182, tokens = 62, subsequent_dollars = 0, initial_dollars = 0, parses = 1},321)

Thm20b. Let $p$ be a natural number. Assume that $p$ is prime. Assume that $$p \equiv 1 \pmod{ 4}.$$ then $p = x ^{ 2}+ y ^{ 2}$ for a natural number $y$ for some natural number $x$.
%% (Scores {tree_length = 65, tree_depth = 11, characters = 182, tokens = 62, subsequent_dollars = 0, initial_dollars = 0, parses = 1},321)

Thm20b. Let $p \in N$. Then if $p$ is prime, then if $p \equiv 1 \pmod{ 4}$, then there exists a natural number $y$, such that $p = x ^{ 2}+ y ^{ 2}$ for some natural number $x$.
%% (Scores {tree_length = 67, tree_depth = 13, characters = 178, tokens = 64, subsequent_dollars = 0, initial_dollars = 0, parses = 1},323)

Thm20b. Let $p \in N$. Then $p$ is prime, only if if $p \equiv 1 \pmod{ 4}$, then there exists a natural number $y$, such that $p = x ^{ 2}+ y ^{ 2}$ for some natural number $x$.
%% (Scores {tree_length = 67, tree_depth = 13, characters = 178, tokens = 64, subsequent_dollars = 0, initial_dollars = 0, parses = 1},323)

Thm20b. Let $p \in N$. Assume that $p$ is prime. Then if $p \equiv 1 \pmod{ 4}$, then there exists a natural number $y$, such that $p = x ^{ 2}+ y ^{ 2}$ for a natural number $x$.
%% (Scores {tree_length = 67, tree_depth = 12, characters = 179, tokens = 64, subsequent_dollars = 0, initial_dollars = 0, parses = 1},323)

Thm20b. Let $p \in N$. Assume that $p$ is prime. Then $p \equiv 1 \pmod{ 4}$, only if there exists a natural number $y$, such that $p = x ^{ 2}+ y ^{ 2}$ for a natural number $x$.
%% (Scores {tree_length = 67, tree_depth = 12, characters = 179, tokens = 64, subsequent_dollars = 0, initial_dollars = 0, parses = 1},323)

Thm20b. If $p$ is prime, then if $p \equiv 1 \pmod{ 4}$, then there exists a natural number $y$, such that $p = x ^{ 2}+ y ^{ 2}$ for a natural number $x$ for every natural number $p$.
%% (Scores {tree_length = 63, tree_depth = 14, characters = 184, tokens = 63, subsequent_dollars = 0, initial_dollars = 0, parses = 1},325)

Thm20b. If $p$ is prime, then if $p \equiv 1 \pmod{ 4}$, then there exists a natural number $y$, such that $p = x ^{ 2}+ y ^{ 2}$ for a natural number $x$ for all natural numbers $p$.
%% (Scores {tree_length = 64, tree_depth = 14, characters = 183, tokens = 63, subsequent_dollars = 0, initial_dollars = 0, parses = 1},325)

Thm20b. Let $p$ be a natural number. Assume that $p$ is prime. Assume that $$p \equiv 1 \pmod{ 4}.$$ then $p = x ^{ 2}+ y ^{ 2}$ for some natural number $y$ for some natural number $x$.
%% (Scores {tree_length = 66, tree_depth = 11, characters = 185, tokens = 62, subsequent_dollars = 0, initial_dollars = 0, parses = 1},325)

Thm20b. Let $p \in N$. Assume that $p$ is prime. Then if $p \equiv 1 \pmod{ 4}$, then there exists a natural number $y$, such that $p = x ^{ 2}+ y ^{ 2}$ for some natural number $x$.
%% (Scores {tree_length = 68, tree_depth = 12, characters = 182, tokens = 64, subsequent_dollars = 0, initial_dollars = 0, parses = 1},327)

Thm20b. Let $p \in N$. Assume that $p$ is prime. Then $p \equiv 1 \pmod{ 4}$, only if there exists a natural number $y$, such that $p = x ^{ 2}+ y ^{ 2}$ for some natural number $x$.
%% (Scores {tree_length = 68, tree_depth = 12, characters = 182, tokens = 64, subsequent_dollars = 0, initial_dollars = 0, parses = 1},327)

Thm20b. For all natural numbers $p$, if $p$ is prime, then if $p \equiv 1 \pmod{ 4}$, then there exists a natural number $y$, such that $p = x ^{ 2}+ y ^{ 2}$ for a natural number $x$.
%% (Scores {tree_length = 65, tree_depth = 14, characters = 184, tokens = 64, subsequent_dollars = 0, initial_dollars = 0, parses = 1},328)

Thm20b. If $p$ is prime, then if $p \equiv 1 \pmod{ 4}$, then there exists a natural number $y$, such that $p = x ^{ 2}+ y ^{ 2}$ for some natural number $x$ for every natural number $p$.
%% (Scores {tree_length = 64, tree_depth = 14, characters = 187, tokens = 63, subsequent_dollars = 0, initial_dollars = 0, parses = 1},329)

Thm20b. If $p$ is prime, then if $p \equiv 1 \pmod{ 4}$, then there exists a natural number $y$, such that $p = x ^{ 2}+ y ^{ 2}$ for some natural number $x$ for all natural numbers $p$.
%% (Scores {tree_length = 65, tree_depth = 14, characters = 186, tokens = 63, subsequent_dollars = 0, initial_dollars = 0, parses = 1},329)

Thm20b. Let $p \in N$. Assume that $p$ is prime. Assume that $$p \equiv 1 \pmod{ 4}.$$ then there exists a natural number $y$, such that $p = x ^{ 2}+ y ^{ 2}$ for a natural number $x$.
%% (Scores {tree_length = 68, tree_depth = 11, characters = 185, tokens = 64, subsequent_dollars = 0, initial_dollars = 0, parses = 1},329)

Thm20b. For all natural numbers $p$, if $p$ is prime, then if $p \equiv 1 \pmod{ 4}$, then there exists a natural number $y$, such that $p = x ^{ 2}+ y ^{ 2}$ for some natural number $x$.
%% (Scores {tree_length = 66, tree_depth = 14, characters = 187, tokens = 64, subsequent_dollars = 0, initial_dollars = 0, parses = 1},332)

Thm20b. Let $p$ be a natural number. Then if $p$ is prime, then if $p \equiv 1 \pmod{ 4}$, then there exists a natural number $y$, such that $p = x ^{ 2}+ y ^{ 2}$ for a natural number $x$.
%% (Scores {tree_length = 64, tree_depth = 13, characters = 189, tokens = 66, subsequent_dollars = 0, initial_dollars = 0, parses = 1},333)

Thm20b. Let $p$ be a natural number. Then $p$ is prime, only if if $p \equiv 1 \pmod{ 4}$, then there exists a natural number $y$, such that $p = x ^{ 2}+ y ^{ 2}$ for a natural number $x$.
%% (Scores {tree_length = 64, tree_depth = 13, characters = 189, tokens = 66, subsequent_dollars = 0, initial_dollars = 0, parses = 1},333)

Thm20b. Let $p \in N$. Assume that $p$ is prime. Assume that $$p \equiv 1 \pmod{ 4}.$$ then there exists a natural number $y$, such that $p = x ^{ 2}+ y ^{ 2}$ for some natural number $x$.
%% (Scores {tree_length = 69, tree_depth = 11, characters = 188, tokens = 64, subsequent_dollars = 0, initial_dollars = 0, parses = 1},333)

Thm20b. Let $p$ be a natural number. Then if $p$ is prime, then if $p \equiv 1 \pmod{ 4}$, then there exists a natural number $y$, such that $p = x ^{ 2}+ y ^{ 2}$ for some natural number $x$.
%% (Scores {tree_length = 65, tree_depth = 13, characters = 192, tokens = 66, subsequent_dollars = 0, initial_dollars = 0, parses = 1},337)

Thm20b. Let $p$ be a natural number. Then $p$ is prime, only if if $p \equiv 1 \pmod{ 4}$, then there exists a natural number $y$, such that $p = x ^{ 2}+ y ^{ 2}$ for some natural number $x$.
%% (Scores {tree_length = 65, tree_depth = 13, characters = 192, tokens = 66, subsequent_dollars = 0, initial_dollars = 0, parses = 1},337)

Thm20b. Let $p$ be a natural number. Assume that $p$ is prime. Then if $p \equiv 1 \pmod{ 4}$, then there exists a natural number $y$, such that $p = x ^{ 2}+ y ^{ 2}$ for a natural number $x$.
%% (Scores {tree_length = 65, tree_depth = 12, characters = 193, tokens = 66, subsequent_dollars = 0, initial_dollars = 0, parses = 1},337)

Thm20b. Let $p$ be a natural number. Assume that $p$ is prime. Then $p \equiv 1 \pmod{ 4}$, only if there exists a natural number $y$, such that $p = x ^{ 2}+ y ^{ 2}$ for a natural number $x$.
%% (Scores {tree_length = 65, tree_depth = 12, characters = 193, tokens = 66, subsequent_dollars = 0, initial_dollars = 0, parses = 1},337)

Thm20b. Let $p$ be a natural number. Assume that $p$ is prime. Then if $p \equiv 1 \pmod{ 4}$, then there exists a natural number $y$, such that $p = x ^{ 2}+ y ^{ 2}$ for some natural number $x$.
%% (Scores {tree_length = 66, tree_depth = 12, characters = 196, tokens = 66, subsequent_dollars = 0, initial_dollars = 0, parses = 1},341)

Thm20b. Let $p$ be a natural number. Assume that $p$ is prime. Then $p \equiv 1 \pmod{ 4}$, only if there exists a natural number $y$, such that $p = x ^{ 2}+ y ^{ 2}$ for some natural number $x$.
%% (Scores {tree_length = 66, tree_depth = 12, characters = 196, tokens = 66, subsequent_dollars = 0, initial_dollars = 0, parses = 1},341)

Thm20b. Let $p$ be a natural number. Assume that $p$ is prime. Assume that $$p \equiv 1 \pmod{ 4}.$$ then there exists a natural number $y$, such that $p = x ^{ 2}+ y ^{ 2}$ for a natural number $x$.
%% (Scores {tree_length = 66, tree_depth = 11, characters = 199, tokens = 66, subsequent_dollars = 0, initial_dollars = 0, parses = 1},343)

Thm20b. Let $p \in N$. Then if $p$ is prime, then if $p \equiv 1 \pmod{ 4}$, then there exists a natural number $x$, such that there exists a natural number $y$, such that $p = x ^{ 2}+ y ^{ 2}$.
%% (Scores {tree_length = 68, tree_depth = 13, characters = 195, tokens = 68, subsequent_dollars = 0, initial_dollars = 0, parses = 1},345)

Thm20b. Let $p \in N$. Then $p$ is prime, only if if $p \equiv 1 \pmod{ 4}$, then there exists a natural number $x$, such that there exists a natural number $y$, such that $p = x ^{ 2}+ y ^{ 2}$.
%% (Scores {tree_length = 68, tree_depth = 13, characters = 195, tokens = 68, subsequent_dollars = 0, initial_dollars = 0, parses = 1},345)

Thm20b. Let $p$ be a natural number. Assume that $p$ is prime. Assume that $$p \equiv 1 \pmod{ 4}.$$ then there exists a natural number $y$, such that $p = x ^{ 2}+ y ^{ 2}$ for some natural number $x$.
%% (Scores {tree_length = 67, tree_depth = 11, characters = 202, tokens = 66, subsequent_dollars = 0, initial_dollars = 0, parses = 1},347)

Thm20b. Let $p \in N$. Assume that $p$ is prime. Then if $p \equiv 1 \pmod{ 4}$, then there exists a natural number $x$, such that there exists a natural number $y$, such that $p = x ^{ 2}+ y ^{ 2}$.
%% (Scores {tree_length = 69, tree_depth = 12, characters = 199, tokens = 68, subsequent_dollars = 0, initial_dollars = 0, parses = 1},349)

Thm20b. Let $p \in N$. Assume that $p$ is prime. Then $p \equiv 1 \pmod{ 4}$, only if there exists a natural number $x$, such that there exists a natural number $y$, such that $p = x ^{ 2}+ y ^{ 2}$.
%% (Scores {tree_length = 69, tree_depth = 12, characters = 199, tokens = 68, subsequent_dollars = 0, initial_dollars = 0, parses = 1},349)

Thm20b. If $p$ is prime, then if $p \equiv 1 \pmod{ 4}$, then there exists a natural number $x$, such that there exists a natural number $y$, such that $p = x ^{ 2}+ y ^{ 2}$ for every natural number $p$.
%% (Scores {tree_length = 65, tree_depth = 14, characters = 204, tokens = 67, subsequent_dollars = 0, initial_dollars = 0, parses = 1},351)

Thm20b. If $p$ is prime, then if $p \equiv 1 \pmod{ 4}$, then there exists a natural number $x$, such that there exists a natural number $y$, such that $p = x ^{ 2}+ y ^{ 2}$ for all natural numbers $p$.
%% (Scores {tree_length = 66, tree_depth = 14, characters = 203, tokens = 67, subsequent_dollars = 0, initial_dollars = 0, parses = 1},351)

Thm20b. For all natural numbers $p$, if $p$ is prime, then if $p \equiv 1 \pmod{ 4}$, then there exists a natural number $x$, such that there exists a natural number $y$, such that $p = x ^{ 2}+ y ^{ 2}$.
%% (Scores {tree_length = 67, tree_depth = 14, characters = 204, tokens = 68, subsequent_dollars = 0, initial_dollars = 0, parses = 1},354)

Thm20b. Let $p \in N$. Assume that $p$ is prime. Assume that $$p \equiv 1 \pmod{ 4}.$$ then there exists a natural number $x$, such that there exists a natural number $y$, such that $p = x ^{ 2}+ y ^{ 2}$.
%% (Scores {tree_length = 70, tree_depth = 11, characters = 205, tokens = 68, subsequent_dollars = 0, initial_dollars = 0, parses = 1},355)

Thm20b. Let $p$ be a natural number. Then if $p$ is prime, then if $p \equiv 1 \pmod{ 4}$, then there exists a natural number $x$, such that there exists a natural number $y$, such that $p = x ^{ 2}+ y ^{ 2}$.
%% (Scores {tree_length = 66, tree_depth = 13, characters = 209, tokens = 70, subsequent_dollars = 0, initial_dollars = 0, parses = 1},359)

Thm20b. Let $p$ be a natural number. Then $p$ is prime, only if if $p \equiv 1 \pmod{ 4}$, then there exists a natural number $x$, such that there exists a natural number $y$, such that $p = x ^{ 2}+ y ^{ 2}$.
%% (Scores {tree_length = 66, tree_depth = 13, characters = 209, tokens = 70, subsequent_dollars = 0, initial_dollars = 0, parses = 1},359)

Thm20b. Let $p$ be a natural number. Assume that $p$ is prime. Then if $p \equiv 1 \pmod{ 4}$, then there exists a natural number $x$, such that there exists a natural number $y$, such that $p = x ^{ 2}+ y ^{ 2}$.
%% (Scores {tree_length = 67, tree_depth = 12, characters = 213, tokens = 70, subsequent_dollars = 0, initial_dollars = 0, parses = 1},363)

Thm20b. Let $p$ be a natural number. Assume that $p$ is prime. Then $p \equiv 1 \pmod{ 4}$, only if there exists a natural number $x$, such that there exists a natural number $y$, such that $p = x ^{ 2}+ y ^{ 2}$.
%% (Scores {tree_length = 67, tree_depth = 12, characters = 213, tokens = 70, subsequent_dollars = 0, initial_dollars = 0, parses = 1},363)

Thm20b. Let $p$ be a natural number. Assume that $p$ is prime. Assume that $p \equiv 1 \pmod{ 4}$. Then there exists a natural number $x$, such that there exists a natural number $y$, such that $p = x ^{ 2}+ y ^{ 2}$.
%% (Scores {tree_length = 68, tree_depth = 11, characters = 217, tokens = 70, subsequent_dollars = 0, initial_dollars = 0, parses = 1},367)

Thm20b. Let $p$ be a natural number. Assume that $p$ is prime. Assume that $$p \equiv 1 \pmod{ 4}.$$ then there exists a natural number $x$, such that there exists a natural number $y$, such that $p = x ^{ 2}+ y ^{ 2}$.
%% (Scores {tree_length = 68, tree_depth = 11, characters = 219, tokens = 70, subsequent_dollars = 0, initial_dollars = 0, parses = 1},369)

Thm20b. Let $p \in N$. Assume that $p$ is prime. Assume that $p$ is congruent to $1$ modulo $4$. Then $p$ is equal to the sum of the square of $x$ and the square of $y$ for a natural number $y$ for a natural number $x$.
%% (Scores {tree_length = 75, tree_depth = 14, characters = 219, tokens = 71, subsequent_dollars = 0, initial_dollars = 0, parses = 1},380)

Thm20b. Let $p \in N$. Assume that $p$ is prime. Assume that $p$ is congruent to $1$ modulo $4$. Then $p$ is equal to the sum of the square of $x$ and the square of $y$ for some natural number $y$ for a natural number $x$.
%% (Scores {tree_length = 76, tree_depth = 14, characters = 222, tokens = 71, subsequent_dollars = 0, initial_dollars = 0, parses = 1},384)

Thm20b. Let $p \in N$. Assume that $p$ is prime. Assume that $p$ is congruent to $1$ modulo $4$. Then $p$ is equal to the sum of the square of $x$ and the square of $y$ for a natural number $y$ for some natural number $x$.
%% (Scores {tree_length = 76, tree_depth = 14, characters = 222, tokens = 71, subsequent_dollars = 0, initial_dollars = 0, parses = 1},384)

Thm20b. Let $p \in N$. Assume that $p$ is prime. Assume that $p$ is congruent to $1$ modulo $4$. Then $p$ is equal to the sum of the square of $x$ and the square of $y$ for some natural number $y$ for some natural number $x$.
%% (Scores {tree_length = 77, tree_depth = 14, characters = 225, tokens = 71, subsequent_dollars = 0, initial_dollars = 0, parses = 1},388)

Thm20b. Let $p$ be a natural number. Assume that $p$ is prime. Assume that $p$ is congruent to $1$ modulo $4$. Then $p$ is equal to the sum of the square of $x$ and the square of $y$ for a natural number $y$ for a natural number $x$.
%% (Scores {tree_length = 73, tree_depth = 14, characters = 233, tokens = 73, subsequent_dollars = 0, initial_dollars = 0, parses = 1},394)

Thm20b. Let $p \in N$. Then if $p$ is prime, then if $p$ is congruent to $1$ modulo $4$, then there exists a natural number $y$, such that $p$ is equal to the sum of the square of $x$ and the square of $y$ for a natural number $x$.
%% (Scores {tree_length = 75, tree_depth = 16, characters = 231, tokens = 75, subsequent_dollars = 0, initial_dollars = 0, parses = 1},398)

Thm20b. Let $p \in N$. Then $p$ is prime, only if if $p$ is congruent to $1$ modulo $4$, then there exists a natural number $y$, such that $p$ is equal to the sum of the square of $x$ and the square of $y$ for a natural number $x$.
%% (Scores {tree_length = 75, tree_depth = 16, characters = 231, tokens = 75, subsequent_dollars = 0, initial_dollars = 0, parses = 1},398)

Thm20b. Let $p$ be a natural number. Assume that $p$ is prime. Assume that $p$ is congruent to $1$ modulo $4$. Then $p$ is equal to the sum of the square of $x$ and the square of $y$ for some natural number $y$ for a natural number $x$.
%% (Scores {tree_length = 74, tree_depth = 14, characters = 236, tokens = 73, subsequent_dollars = 0, initial_dollars = 0, parses = 1},398)

Thm20b. Let $p$ be a natural number. Assume that $p$ is prime. Assume that $p$ is congruent to $1$ modulo $4$. Then $p$ is equal to the sum of the square of $x$ and the square of $y$ for a natural number $y$ for some natural number $x$.
%% (Scores {tree_length = 74, tree_depth = 14, characters = 236, tokens = 73, subsequent_dollars = 0, initial_dollars = 0, parses = 1},398)

Thm20b. Let $p \in N$. Then if $p$ is prime, then if $p$ is congruent to $1$ modulo $4$, then there exists a natural number $y$, such that $p$ is equal to the sum of the square of $x$ and the square of $y$ for some natural number $x$.
%% (Scores {tree_length = 76, tree_depth = 16, characters = 234, tokens = 75, subsequent_dollars = 0, initial_dollars = 0, parses = 1},402)

Thm20b. Let $p \in N$. Then $p$ is prime, only if if $p$ is congruent to $1$ modulo $4$, then there exists a natural number $y$, such that $p$ is equal to the sum of the square of $x$ and the square of $y$ for some natural number $x$.
%% (Scores {tree_length = 76, tree_depth = 16, characters = 234, tokens = 75, subsequent_dollars = 0, initial_dollars = 0, parses = 1},402)

Thm20b. Let $p \in N$. Assume that $p$ is prime. Then if $p$ is congruent to $1$ modulo $4$, then there exists a natural number $y$, such that $p$ is equal to the sum of the square of $x$ and the square of $y$ for a natural number $x$.
%% (Scores {tree_length = 76, tree_depth = 15, characters = 235, tokens = 75, subsequent_dollars = 0, initial_dollars = 0, parses = 1},402)

Thm20b. Let $p \in N$. Assume that $p$ is prime. Then $p$ is congruent to $1$ modulo $4$, only if there exists a natural number $y$, such that $p$ is equal to the sum of the square of $x$ and the square of $y$ for a natural number $x$.
%% (Scores {tree_length = 76, tree_depth = 15, characters = 235, tokens = 75, subsequent_dollars = 0, initial_dollars = 0, parses = 1},402)

Thm20b. Let $p$ be a natural number. Assume that $p$ is prime. Assume that $p$ is congruent to $1$ modulo $4$. Then $p$ is equal to the sum of the square of $x$ and the square of $y$ for some natural number $y$ for some natural number $x$.
%% (Scores {tree_length = 75, tree_depth = 14, characters = 239, tokens = 73, subsequent_dollars = 0, initial_dollars = 0, parses = 1},402)

Thm20b. If $p$ is prime, then if $p$ is congruent to $1$ modulo $4$, then there exists a natural number $y$, such that $p$ is equal to the sum of the square of $x$ and the square of $y$ for a natural number $x$ for every natural number $p$.
%% (Scores {tree_length = 72, tree_depth = 17, characters = 240, tokens = 74, subsequent_dollars = 0, initial_dollars = 0, parses = 1},404)

Thm20b. If $p$ is prime, then if $p$ is congruent to $1$ modulo $4$, then there exists a natural number $y$, such that $p$ is equal to the sum of the square of $x$ and the square of $y$ for a natural number $x$ for all natural numbers $p$.
%% (Scores {tree_length = 73, tree_depth = 17, characters = 239, tokens = 74, subsequent_dollars = 0, initial_dollars = 0, parses = 1},404)

Thm20b. Let $p \in N$. Assume that $p$ is prime. Then if $p$ is congruent to $1$ modulo $4$, then there exists a natural number $y$, such that $p$ is equal to the sum of the square of $x$ and the square of $y$ for some natural number $x$.
%% (Scores {tree_length = 77, tree_depth = 15, characters = 238, tokens = 75, subsequent_dollars = 0, initial_dollars = 0, parses = 1},406)

Thm20b. Let $p \in N$. Assume that $p$ is prime. Then $p$ is congruent to $1$ modulo $4$, only if there exists a natural number $y$, such that $p$ is equal to the sum of the square of $x$ and the square of $y$ for some natural number $x$.
%% (Scores {tree_length = 77, tree_depth = 15, characters = 238, tokens = 75, subsequent_dollars = 0, initial_dollars = 0, parses = 1},406)

Thm20b. Let $p \in N$. Assume that $p$ is prime. Assume that $p$ is congruent to $1$ modulo $4$. Then there exists a natural number $y$, such that $p$ is equal to the sum of the square of $x$ and the square of $y$ for a natural number $x$.
%% (Scores {tree_length = 77, tree_depth = 14, characters = 239, tokens = 75, subsequent_dollars = 0, initial_dollars = 0, parses = 1},406)

Thm20b. For all natural numbers $p$, if $p$ is prime, then if $p$ is congruent to $1$ modulo $4$, then there exists a natural number $y$, such that $p$ is equal to the sum of the square of $x$ and the square of $y$ for a natural number $x$.
%% (Scores {tree_length = 74, tree_depth = 17, characters = 240, tokens = 75, subsequent_dollars = 0, initial_dollars = 0, parses = 1},407)

Thm20b. If $p$ is prime, then if $p$ is congruent to $1$ modulo $4$, then there exists a natural number $y$, such that $p$ is equal to the sum of the square of $x$ and the square of $y$ for some natural number $x$ for every natural number $p$.
%% (Scores {tree_length = 73, tree_depth = 17, characters = 243, tokens = 74, subsequent_dollars = 0, initial_dollars = 0, parses = 1},408)

Thm20b. If $p$ is prime, then if $p$ is congruent to $1$ modulo $4$, then there exists a natural number $y$, such that $p$ is equal to the sum of the square of $x$ and the square of $y$ for some natural number $x$ for all natural numbers $p$.
%% (Scores {tree_length = 74, tree_depth = 17, characters = 242, tokens = 74, subsequent_dollars = 0, initial_dollars = 0, parses = 1},408)

Thm20b. Let $p \in N$. Assume that $p$ is prime. Assume that $p$ is congruent to $1$ modulo $4$. Then there exists a natural number $y$, such that $p$ is equal to the sum of the square of $x$ and the square of $y$ for some natural number $x$.
%% (Scores {tree_length = 78, tree_depth = 14, characters = 242, tokens = 75, subsequent_dollars = 0, initial_dollars = 0, parses = 1},410)

Thm20b. For all natural numbers $p$, if $p$ is prime, then if $p$ is congruent to $1$ modulo $4$, then there exists a natural number $y$, such that $p$ is equal to the sum of the square of $x$ and the square of $y$ for some natural number $x$.
%% (Scores {tree_length = 75, tree_depth = 17, characters = 243, tokens = 75, subsequent_dollars = 0, initial_dollars = 0, parses = 1},411)

Thm20b. Let $p$ be a natural number. Then if $p$ is prime, then if $p$ is congruent to $1$ modulo $4$, then there exists a natural number $y$, such that $p$ is equal to the sum of the square of $x$ and the square of $y$ for a natural number $x$.
%% (Scores {tree_length = 73, tree_depth = 16, characters = 245, tokens = 77, subsequent_dollars = 0, initial_dollars = 0, parses = 1},412)

Thm20b. Let $p$ be a natural number. Then $p$ is prime, only if if $p$ is congruent to $1$ modulo $4$, then there exists a natural number $y$, such that $p$ is equal to the sum of the square of $x$ and the square of $y$ for a natural number $x$.
%% (Scores {tree_length = 73, tree_depth = 16, characters = 245, tokens = 77, subsequent_dollars = 0, initial_dollars = 0, parses = 1},412)

Thm20b. Let $p$ be a natural number. Then if $p$ is prime, then if $p$ is congruent to $1$ modulo $4$, then there exists a natural number $y$, such that $p$ is equal to the sum of the square of $x$ and the square of $y$ for some natural number $x$.
%% (Scores {tree_length = 74, tree_depth = 16, characters = 248, tokens = 77, subsequent_dollars = 0, initial_dollars = 0, parses = 1},416)

Thm20b. Let $p$ be a natural number. Then $p$ is prime, only if if $p$ is congruent to $1$ modulo $4$, then there exists a natural number $y$, such that $p$ is equal to the sum of the square of $x$ and the square of $y$ for some natural number $x$.
%% (Scores {tree_length = 74, tree_depth = 16, characters = 248, tokens = 77, subsequent_dollars = 0, initial_dollars = 0, parses = 1},416)

Thm20b. Let $p$ be a natural number. Assume that $p$ is prime. Then if $p$ is congruent to $1$ modulo $4$, then there exists a natural number $y$, such that $p$ is equal to the sum of the square of $x$ and the square of $y$ for a natural number $x$.
%% (Scores {tree_length = 74, tree_depth = 15, characters = 249, tokens = 77, subsequent_dollars = 0, initial_dollars = 0, parses = 1},416)

Thm20b. Let $p$ be a natural number. Assume that $p$ is prime. Then $p$ is congruent to $1$ modulo $4$, only if there exists a natural number $y$, such that $p$ is equal to the sum of the square of $x$ and the square of $y$ for a natural number $x$.
%% (Scores {tree_length = 74, tree_depth = 15, characters = 249, tokens = 77, subsequent_dollars = 0, initial_dollars = 0, parses = 1},416)

Thm20b. Let $p$ be a natural number. Assume that $p$ is prime. Then if $p$ is congruent to $1$ modulo $4$, then there exists a natural number $y$, such that $p$ is equal to the sum of the square of $x$ and the square of $y$ for some natural number $x$.
%% (Scores {tree_length = 75, tree_depth = 15, characters = 252, tokens = 77, subsequent_dollars = 0, initial_dollars = 0, parses = 1},420)

Thm20b. Let $p$ be a natural number. Assume that $p$ is prime. Then $p$ is congruent to $1$ modulo $4$, only if there exists a natural number $y$, such that $p$ is equal to the sum of the square of $x$ and the square of $y$ for some natural number $x$.
%% (Scores {tree_length = 75, tree_depth = 15, characters = 252, tokens = 77, subsequent_dollars = 0, initial_dollars = 0, parses = 1},420)

Thm20b. Let $p$ be a natural number. Assume that $p$ is prime. Assume that $p$ is congruent to $1$ modulo $4$. Then there exists a natural number $y$, such that $p$ is equal to the sum of the square of $x$ and the square of $y$ for a natural number $x$.
%% (Scores {tree_length = 75, tree_depth = 14, characters = 253, tokens = 77, subsequent_dollars = 0, initial_dollars = 0, parses = 1},420)

Thm20b. Let $p \in N$. Then if $p$ is prime, then if $p$ is congruent to $1$ modulo $4$, then there exists a natural number $x$, such that there exists a natural number $y$, such that $p$ is equal to the sum of the square of $x$ and the square of $y$.
%% (Scores {tree_length = 77, tree_depth = 16, characters = 251, tokens = 79, subsequent_dollars = 0, initial_dollars = 0, parses = 1},424)

Thm20b. Let $p \in N$. Then $p$ is prime, only if if $p$ is congruent to $1$ modulo $4$, then there exists a natural number $x$, such that there exists a natural number $y$, such that $p$ is equal to the sum of the square of $x$ and the square of $y$.
%% (Scores {tree_length = 77, tree_depth = 16, characters = 251, tokens = 79, subsequent_dollars = 0, initial_dollars = 0, parses = 1},424)

Thm20b. Let $p$ be a natural number. Assume that $p$ is prime. Assume that $p$ is congruent to $1$ modulo $4$. Then there exists a natural number $y$, such that $p$ is equal to the sum of the square of $x$ and the square of $y$ for some natural number $x$.
%% (Scores {tree_length = 76, tree_depth = 14, characters = 256, tokens = 77, subsequent_dollars = 0, initial_dollars = 0, parses = 1},424)

Thm20b. Let $p \in N$. Assume that $p$ is prime. Then if $p$ is congruent to $1$ modulo $4$, then there exists a natural number $x$, such that there exists a natural number $y$, such that $p$ is equal to the sum of the square of $x$ and the square of $y$.
%% (Scores {tree_length = 78, tree_depth = 15, characters = 255, tokens = 79, subsequent_dollars = 0, initial_dollars = 0, parses = 1},428)

Thm20b. Let $p \in N$. Assume that $p$ is prime. Then $p$ is congruent to $1$ modulo $4$, only if there exists a natural number $x$, such that there exists a natural number $y$, such that $p$ is equal to the sum of the square of $x$ and the square of $y$.
%% (Scores {tree_length = 78, tree_depth = 15, characters = 255, tokens = 79, subsequent_dollars = 0, initial_dollars = 0, parses = 1},428)

Thm20b. If $p$ is prime, then if $p$ is congruent to $1$ modulo $4$, then there exists a natural number $x$, such that there exists a natural number $y$, such that $p$ is equal to the sum of the square of $x$ and the square of $y$ for every natural number $p$.
%% (Scores {tree_length = 74, tree_depth = 17, characters = 260, tokens = 78, subsequent_dollars = 0, initial_dollars = 0, parses = 1},430)

Thm20b. If $p$ is prime, then if $p$ is congruent to $1$ modulo $4$, then there exists a natural number $x$, such that there exists a natural number $y$, such that $p$ is equal to the sum of the square of $x$ and the square of $y$ for all natural numbers $p$.
%% (Scores {tree_length = 75, tree_depth = 17, characters = 259, tokens = 78, subsequent_dollars = 0, initial_dollars = 0, parses = 1},430)

Thm20b. Let $p \in N$. Assume that $p$ is prime. Assume that $p$ is congruent to $1$ modulo $4$. Then there exists a natural number $x$, such that there exists a natural number $y$, such that $p$ is equal to the sum of the square of $x$ and the square of $y$.
%% (Scores {tree_length = 79, tree_depth = 14, characters = 259, tokens = 79, subsequent_dollars = 0, initial_dollars = 0, parses = 1},432)

Thm20b. For all natural numbers $p$, if $p$ is prime, then if $p$ is congruent to $1$ modulo $4$, then there exists a natural number $x$, such that there exists a natural number $y$, such that $p$ is equal to the sum of the square of $x$ and the square of $y$.
%% (Scores {tree_length = 76, tree_depth = 17, characters = 260, tokens = 79, subsequent_dollars = 0, initial_dollars = 0, parses = 1},433)

Thm20b. Let $p$ be a natural number. Then if $p$ is prime, then if $p$ is congruent to $1$ modulo $4$, then there exists a natural number $x$, such that there exists a natural number $y$, such that $p$ is equal to the sum of the square of $x$ and the square of $y$.
%% (Scores {tree_length = 75, tree_depth = 16, characters = 265, tokens = 81, subsequent_dollars = 0, initial_dollars = 0, parses = 1},438)

Thm20b. Let $p$ be a natural number. Then $p$ is prime, only if if $p$ is congruent to $1$ modulo $4$, then there exists a natural number $x$, such that there exists a natural number $y$, such that $p$ is equal to the sum of the square of $x$ and the square of $y$.
%% (Scores {tree_length = 75, tree_depth = 16, characters = 265, tokens = 81, subsequent_dollars = 0, initial_dollars = 0, parses = 1},438)

Thm20b. Let $p$ be a natural number. Assume that $p$ is prime. Then if $p$ is congruent to $1$ modulo $4$, then there exists a natural number $x$, such that there exists a natural number $y$, such that $p$ is equal to the sum of the square of $x$ and the square of $y$.
%% (Scores {tree_length = 76, tree_depth = 15, characters = 269, tokens = 81, subsequent_dollars = 0, initial_dollars = 0, parses = 1},442)

Thm20b. Let $p$ be a natural number. Assume that $p$ is prime. Then $p$ is congruent to $1$ modulo $4$, only if there exists a natural number $x$, such that there exists a natural number $y$, such that $p$ is equal to the sum of the square of $x$ and the square of $y$.
%% (Scores {tree_length = 76, tree_depth = 15, characters = 269, tokens = 81, subsequent_dollars = 0, initial_dollars = 0, parses = 1},442)

Thm20b. Let $p$ be a natural number. Assume that $p$ is prime. Assume that $p$ is congruent to $1$ modulo $4$. Then there exists a natural number $x$, such that there exists a natural number $y$, such that $p$ is equal to the sum of the square of $x$ and the square of $y$.
%% (Scores {tree_length = 77, tree_depth = 14, characters = 273, tokens = 81, subsequent_dollars = 0, initial_dollars = 0, parses = 1},446)

Thm20b. Let $p$ be an instance of natural numbers. Assume that we can prove that $p$ is prime. Assume that we can prove that $p$ is congruent to $1$ modulo $4$. Then we can prove that $p$ is equal to the sum of the square of $x$ and the square of $y$ for a natural number $y$ for a natural number $x$.
%% (Scores {tree_length = 77, tree_depth = 15, characters = 301, tokens = 87, subsequent_dollars = 0, initial_dollars = 0, parses = 1},481)

Thm20b. Let $p$ be an instance of natural numbers. Assume that we can prove that $p$ is prime. Assume that we can prove that $p$ is congruent to $1$ modulo $4$. Then we can prove that $p$ is equal to the sum of the square of $x$ and the square of $y$ for some natural number $y$ for a natural number $x$.
%% (Scores {tree_length = 78, tree_depth = 15, characters = 304, tokens = 87, subsequent_dollars = 0, initial_dollars = 0, parses = 1},485)

Thm20b. Let $p$ be an instance of natural numbers. Assume that we can prove that $p$ is prime. Assume that we can prove that $p$ is congruent to $1$ modulo $4$. Then we can prove that $p$ is equal to the sum of the square of $x$ and the square of $y$ for a natural number $y$ for some natural number $x$.
%% (Scores {tree_length = 78, tree_depth = 15, characters = 304, tokens = 87, subsequent_dollars = 0, initial_dollars = 0, parses = 1},485)

Thm20b. Let $p$ be an instance of natural numbers. Assume that we can prove that $p$ is prime. Assume that we can prove that $p$ is congruent to $1$ modulo $4$. Then we can prove that $p$ is equal to the sum of the square of $x$ and the square of $y$ for some natural number $y$ for some natural number $x$.
%% (Scores {tree_length = 79, tree_depth = 15, characters = 307, tokens = 87, subsequent_dollars = 0, initial_dollars = 0, parses = 1},489)

Thm20b. If we can prove that $p$ is prime, then if we can prove that $p$ is congruent to $1$ modulo $4$, then we can prove that there exists a natural number $y$, such that $p$ is equal to the sum of the square of $x$ and the square of $y$ for a natural number $x$ for every instance $p$ of natural numbers.
%% (Scores {tree_length = 76, tree_depth = 18, characters = 307, tokens = 88, subsequent_dollars = 0, initial_dollars = 0, parses = 1},490)

Thm20b. If we can prove that $p$ is prime, then if we can prove that $p$ is congruent to $1$ modulo $4$, then we can prove that there exists a natural number $y$, such that $p$ is equal to the sum of the square of $x$ and the square of $y$ for a natural number $x$ for all instances $p$ of natural numbers.
%% (Scores {tree_length = 77, tree_depth = 18, characters = 306, tokens = 88, subsequent_dollars = 0, initial_dollars = 0, parses = 1},490)

Thm20b. For all instances $p$ of natural numbers, if we can prove that $p$ is prime, then if we can prove that $p$ is congruent to $1$ modulo $4$, then we can prove that there exists a natural number $y$, such that $p$ is equal to the sum of the square of $x$ and the square of $y$ for a natural number $x$.
%% (Scores {tree_length = 78, tree_depth = 18, characters = 307, tokens = 89, subsequent_dollars = 0, initial_dollars = 0, parses = 1},493)

Thm20b. If we can prove that $p$ is prime, then if we can prove that $p$ is congruent to $1$ modulo $4$, then we can prove that there exists a natural number $y$, such that $p$ is equal to the sum of the square of $x$ and the square of $y$ for some natural number $x$ for every instance $p$ of natural numbers.
%% (Scores {tree_length = 77, tree_depth = 18, characters = 310, tokens = 88, subsequent_dollars = 0, initial_dollars = 0, parses = 1},494)

Thm20b. If we can prove that $p$ is prime, then if we can prove that $p$ is congruent to $1$ modulo $4$, then we can prove that there exists a natural number $y$, such that $p$ is equal to the sum of the square of $x$ and the square of $y$ for some natural number $x$ for all instances $p$ of natural numbers.
%% (Scores {tree_length = 78, tree_depth = 18, characters = 309, tokens = 88, subsequent_dollars = 0, initial_dollars = 0, parses = 1},494)

Thm20b. For all instances $p$ of natural numbers, if we can prove that $p$ is prime, then if we can prove that $p$ is congruent to $1$ modulo $4$, then we can prove that there exists a natural number $y$, such that $p$ is equal to the sum of the square of $x$ and the square of $y$ for some natural number $x$.
%% (Scores {tree_length = 79, tree_depth = 18, characters = 310, tokens = 89, subsequent_dollars = 0, initial_dollars = 0, parses = 1},497)

Thm20b. Let $p$ be an instance of natural numbers. Then if we can prove that $p$ is prime, then if we can prove that $p$ is congruent to $1$ modulo $4$, then we can prove that there exists a natural number $y$, such that $p$ is equal to the sum of the square of $x$ and the square of $y$ for a natural number $x$.
%% (Scores {tree_length = 77, tree_depth = 17, characters = 313, tokens = 91, subsequent_dollars = 0, initial_dollars = 0, parses = 1},499)

Thm20b. Let $p$ be an instance of natural numbers. Then we can prove that $p$ is prime, only if if we can prove that $p$ is congruent to $1$ modulo $4$, then we can prove that there exists a natural number $y$, such that $p$ is equal to the sum of the square of $x$ and the square of $y$ for a natural number $x$.
%% (Scores {tree_length = 77, tree_depth = 17, characters = 313, tokens = 91, subsequent_dollars = 0, initial_dollars = 0, parses = 1},499)

Thm20b. Let $p$ be an instance of natural numbers. Then if we can prove that $p$ is prime, then if we can prove that $p$ is congruent to $1$ modulo $4$, then we can prove that there exists a natural number $y$, such that $p$ is equal to the sum of the square of $x$ and the square of $y$ for some natural number $x$.
%% (Scores {tree_length = 78, tree_depth = 17, characters = 316, tokens = 91, subsequent_dollars = 0, initial_dollars = 0, parses = 1},503)

Thm20b. Let $p$ be an instance of natural numbers. Then we can prove that $p$ is prime, only if if we can prove that $p$ is congruent to $1$ modulo $4$, then we can prove that there exists a natural number $y$, such that $p$ is equal to the sum of the square of $x$ and the square of $y$ for some natural number $x$.
%% (Scores {tree_length = 78, tree_depth = 17, characters = 316, tokens = 91, subsequent_dollars = 0, initial_dollars = 0, parses = 1},503)

Thm20b. Let $p$ be an instance of natural numbers. Assume that we can prove that $p$ is prime. Then if we can prove that $p$ is congruent to $1$ modulo $4$, then we can prove that there exists a natural number $y$, such that $p$ is equal to the sum of the square of $x$ and the square of $y$ for a natural number $x$.
%% (Scores {tree_length = 78, tree_depth = 16, characters = 317, tokens = 91, subsequent_dollars = 0, initial_dollars = 0, parses = 1},503)

Thm20b. Let $p$ be an instance of natural numbers. Assume that we can prove that $p$ is prime. Then we can prove that $p$ is congruent to $1$ modulo $4$, only if we can prove that there exists a natural number $y$, such that $p$ is equal to the sum of the square of $x$ and the square of $y$ for a natural number $x$.
%% (Scores {tree_length = 78, tree_depth = 16, characters = 317, tokens = 91, subsequent_dollars = 0, initial_dollars = 0, parses = 1},503)

Thm20b. Let $p$ be an instance of natural numbers. Assume that we can prove that $p$ is prime. Then if we can prove that $p$ is congruent to $1$ modulo $4$, then we can prove that there exists a natural number $y$, such that $p$ is equal to the sum of the square of $x$ and the square of $y$ for some natural number $x$.
%% (Scores {tree_length = 79, tree_depth = 16, characters = 320, tokens = 91, subsequent_dollars = 0, initial_dollars = 0, parses = 1},507)

Thm20b. Let $p$ be an instance of natural numbers. Assume that we can prove that $p$ is prime. Then we can prove that $p$ is congruent to $1$ modulo $4$, only if we can prove that there exists a natural number $y$, such that $p$ is equal to the sum of the square of $x$ and the square of $y$ for some natural number $x$.
%% (Scores {tree_length = 79, tree_depth = 16, characters = 320, tokens = 91, subsequent_dollars = 0, initial_dollars = 0, parses = 1},507)

Thm20b. Let $p$ be an instance of natural numbers. Assume that we can prove that $p$ is prime. Assume that we can prove that $p$ is congruent to $1$ modulo $4$. Then we can prove that there exists a natural number $y$, such that $p$ is equal to the sum of the square of $x$ and the square of $y$ for a natural number $x$.
%% (Scores {tree_length = 79, tree_depth = 15, characters = 321, tokens = 91, subsequent_dollars = 0, initial_dollars = 0, parses = 1},507)

Thm20b. Let $p$ be an instance of natural numbers. Assume that we can prove that $p$ is prime. Assume that we can prove that $p$ is congruent to $1$ modulo $4$. Then we can prove that there exists a natural number $y$, such that $p$ is equal to the sum of the square of $x$ and the square of $y$ for some natural number $x$.
%% (Scores {tree_length = 80, tree_depth = 15, characters = 324, tokens = 91, subsequent_dollars = 0, initial_dollars = 0, parses = 1},511)

Thm20b. If we can prove that $p$ is prime, then if we can prove that $p$ is congruent to $1$ modulo $4$, then we can prove that there exists a natural number $x$, such that there exists a natural number $y$, such that $p$ is equal to the sum of the square of $x$ and the square of $y$ for every instance $p$ of natural numbers.
%% (Scores {tree_length = 78, tree_depth = 18, characters = 327, tokens = 92, subsequent_dollars = 0, initial_dollars = 0, parses = 1},516)

Thm20b. If we can prove that $p$ is prime, then if we can prove that $p$ is congruent to $1$ modulo $4$, then we can prove that there exists a natural number $x$, such that there exists a natural number $y$, such that $p$ is equal to the sum of the square of $x$ and the square of $y$ for all instances $p$ of natural numbers.
%% (Scores {tree_length = 79, tree_depth = 18, characters = 326, tokens = 92, subsequent_dollars = 0, initial_dollars = 0, parses = 1},516)

Thm20b. For all instances $p$ of natural numbers, if we can prove that $p$ is prime, then if we can prove that $p$ is congruent to $1$ modulo $4$, then we can prove that there exists a natural number $x$, such that there exists a natural number $y$, such that $p$ is equal to the sum of the square of $x$ and the square of $y$.
%% (Scores {tree_length = 80, tree_depth = 18, characters = 327, tokens = 93, subsequent_dollars = 0, initial_dollars = 0, parses = 1},519)

Thm20b. Let $p$ be an instance of natural numbers. Then if we can prove that $p$ is prime, then if we can prove that $p$ is congruent to $1$ modulo $4$, then we can prove that there exists a natural number $x$, such that there exists a natural number $y$, such that $p$ is equal to the sum of the square of $x$ and the square of $y$.
%% (Scores {tree_length = 79, tree_depth = 17, characters = 333, tokens = 95, subsequent_dollars = 0, initial_dollars = 0, parses = 1},525)

Thm20b. Let $p$ be an instance of natural numbers. Then we can prove that $p$ is prime, only if if we can prove that $p$ is congruent to $1$ modulo $4$, then we can prove that there exists a natural number $x$, such that there exists a natural number $y$, such that $p$ is equal to the sum of the square of $x$ and the square of $y$.
%% (Scores {tree_length = 79, tree_depth = 17, characters = 333, tokens = 95, subsequent_dollars = 0, initial_dollars = 0, parses = 1},525)

Thm20b. Let $p$ be an instance of natural numbers. Assume that we can prove that $p$ is prime. Then if we can prove that $p$ is congruent to $1$ modulo $4$, then we can prove that there exists a natural number $x$, such that there exists a natural number $y$, such that $p$ is equal to the sum of the square of $x$ and the square of $y$.
%% (Scores {tree_length = 80, tree_depth = 16, characters = 337, tokens = 95, subsequent_dollars = 0, initial_dollars = 0, parses = 1},529)

Thm20b. Let $p$ be an instance of natural numbers. Assume that we can prove that $p$ is prime. Then we can prove that $p$ is congruent to $1$ modulo $4$, only if we can prove that there exists a natural number $x$, such that there exists a natural number $y$, such that $p$ is equal to the sum of the square of $x$ and the square of $y$.
%% (Scores {tree_length = 80, tree_depth = 16, characters = 337, tokens = 95, subsequent_dollars = 0, initial_dollars = 0, parses = 1},529)

Thm20b. Let $p$ be an instance of natural numbers. Assume that we can prove that $p$ is prime. Assume that we can prove that $p$ is congruent to $1$ modulo $4$. Then we can prove that there exists a natural number $x$, such that there exists a natural number $y$, such that $p$ is equal to the sum of the square of $x$ and the square of $y$.
%% (Scores {tree_length = 81, tree_depth = 15, characters = 341, tokens = 95, subsequent_dollars = 0, initial_dollars = 0, parses = 1},533)

Thm22. $Real$ is not denumerable.
%% (Scores {tree_length = 10, tree_depth = 6, characters = 33, tokens = 9, subsequent_dollars = 0, initial_dollars = 1, parses = 1},60)

Thm22. We can prove that $Real$ is not denumerable.
%% (Scores {tree_length = 11, tree_depth = 7, characters = 51, tokens = 13, subsequent_dollars = 0, initial_dollars = 0, parses = 1},83)

Thm51wilson. Let $n \in N$. Then $n$ is prime, if and only if $$(n - 1)! \equiv - 1 \pmod{ n}.$$
%% (Scores {tree_length = 39, tree_depth = 9, characters = 96, tokens = 36, subsequent_dollars = 0, initial_dollars = 0, parses = 1},181)

Thm51wilson. $n$ is prime, if and only if $$(n - 1)! \equiv - 1 \pmod{ n}$$ for every natural number $n$.
%% (Scores {tree_length = 36, tree_depth = 10, characters = 105, tokens = 35, subsequent_dollars = 0, initial_dollars = 1, parses = 1},188)

Thm51wilson. $n$ is prime, if and only if $$(n - 1)! \equiv - 1 \pmod{ n}$$ for all natural numbers $n$.
%% (Scores {tree_length = 37, tree_depth = 10, characters = 104, tokens = 35, subsequent_dollars = 0, initial_dollars = 1, parses = 1},188)

Thm51wilson. For all natural numbers $n$, $n$ is prime, if and only if $$(n - 1)! \equiv - 1 \pmod{ n}.$$
%% (Scores {tree_length = 38, tree_depth = 10, characters = 105, tokens = 36, subsequent_dollars = 1, initial_dollars = 0, parses = 1},191)

Thm51wilson. Let $n$ be a natural number. Then $n$ is prime, if and only if $(n - 1)! \equiv - 1 \pmod{ n}$.
%% (Scores {tree_length = 37, tree_depth = 9, characters = 108, tokens = 38, subsequent_dollars = 0, initial_dollars = 0, parses = 1},193)

Thm51wilson. Let $n$ be a natural number. Then $n$ is prime, if and only if $$(n - 1)! \equiv - 1 \pmod{ n}.$$
%% (Scores {tree_length = 37, tree_depth = 9, characters = 110, tokens = 38, subsequent_dollars = 0, initial_dollars = 0, parses = 1},195)

Thm51wilson. Let $n \in N$. Then $n$ is prime, if and only if the factorial of the difference of $n$ and $1$ is congruent to the negation of $1$ modulo $n$.
%% (Scores {tree_length = 49, tree_depth = 12, characters = 156, tokens = 47, subsequent_dollars = 0, initial_dollars = 0, parses = 1},265)

Thm51wilson. $n$ is prime, if and only if the factorial of the difference of $n$ and $1$ is congruent to the negation of $1$ modulo $n$ for every natural number $n$.
%% (Scores {tree_length = 46, tree_depth = 13, characters = 165, tokens = 46, subsequent_dollars = 0, initial_dollars = 1, parses = 1},272)

Thm51wilson. $n$ is prime, if and only if the factorial of the difference of $n$ and $1$ is congruent to the negation of $1$ modulo $n$ for all natural numbers $n$.
%% (Scores {tree_length = 47, tree_depth = 13, characters = 164, tokens = 46, subsequent_dollars = 0, initial_dollars = 1, parses = 1},272)

Thm51wilson. For all natural numbers $n$, $n$ is prime, if and only if the factorial of the difference of $n$ and $1$ is congruent to the negation of $1$ modulo $n$.
%% (Scores {tree_length = 48, tree_depth = 13, characters = 165, tokens = 47, subsequent_dollars = 1, initial_dollars = 0, parses = 1},275)

Thm51wilson. Let $n$ be a natural number. Then $n$ is prime, if and only if the factorial of the difference of $n$ and $1$ is congruent to the negation of $1$ modulo $n$.
%% (Scores {tree_length = 47, tree_depth = 12, characters = 170, tokens = 49, subsequent_dollars = 0, initial_dollars = 0, parses = 1},279)

Thm51wilson. We can prove that $n$ is prime, if and only if the factorial of the difference of $n$ and $1$ is congruent to the negation of $1$ modulo $n$ for every instance $n$ of natural numbers.
%% (Scores {tree_length = 48, tree_depth = 14, characters = 196, tokens = 52, subsequent_dollars = 0, initial_dollars = 0, parses = 1},311)

Thm51wilson. We can prove that $n$ is prime, if and only if the factorial of the difference of $n$ and $1$ is congruent to the negation of $1$ modulo $n$ for all instances $n$ of natural numbers.
%% (Scores {tree_length = 49, tree_depth = 14, characters = 195, tokens = 52, subsequent_dollars = 0, initial_dollars = 0, parses = 1},311)

Thm51wilson. For all instances $n$ of natural numbers, we can prove that $n$ is prime, if and only if the factorial of the difference of $n$ and $1$ is congruent to the negation of $1$ modulo $n$.
%% (Scores {tree_length = 50, tree_depth = 14, characters = 196, tokens = 53, subsequent_dollars = 0, initial_dollars = 0, parses = 1},314)

Thm51wilson. Let $n$ be an instance of natural numbers. Then we can prove that $n$ is prime, if and only if the factorial of the difference of $n$ and $1$ is congruent to the negation of $1$ modulo $n$.
%% (Scores {tree_length = 49, tree_depth = 13, characters = 202, tokens = 55, subsequent_dollars = 0, initial_dollars = 0, parses = 1},320)

Thm51b. Let $n \in N$. Then $n$ is prime, if and only if $(n - 1)! + 1$ is divisible by $n$.
%% (Scores {tree_length = 43, tree_depth = 10, characters = 92, tokens = 37, subsequent_dollars = 0, initial_dollars = 0, parses = 1},183)

Thm51b. $n$ is prime, if and only if $(n - 1)! + 1$ is divisible by $n$ for every natural number $n$.
%% (Scores {tree_length = 40, tree_depth = 11, characters = 101, tokens = 36, subsequent_dollars = 0, initial_dollars = 1, parses = 1},190)

Thm51b. $n$ is prime, if and only if $(n - 1)! + 1$ is divisible by $n$ for all natural numbers $n$.
%% (Scores {tree_length = 41, tree_depth = 11, characters = 100, tokens = 36, subsequent_dollars = 0, initial_dollars = 1, parses = 1},190)

Thm51b. For all natural numbers $n$, $n$ is prime, if and only if $(n - 1)! + 1$ is divisible by $n$.
%% (Scores {tree_length = 42, tree_depth = 11, characters = 101, tokens = 37, subsequent_dollars = 1, initial_dollars = 0, parses = 1},193)

Thm51b. Let $n$ be a natural number. Then $n$ is prime, if and only if $(n - 1)! + 1$ is divisible by $n$.
%% (Scores {tree_length = 41, tree_depth = 10, characters = 106, tokens = 39, subsequent_dollars = 0, initial_dollars = 0, parses = 1},197)

Thm51b. Let $n \in N$. Then $n$ is prime, if and only if the sum of the factorial of the difference of $n$ and $1$ and $1$ is divisible by $n$.
%% (Scores {tree_length = 50, tree_depth = 14, characters = 143, tokens = 47, subsequent_dollars = 0, initial_dollars = 0, parses = 1},255)

Thm51b. $n$ is prime, if and only if the sum of the factorial of the difference of $n$ and $1$ and $1$ is divisible by $n$ for every natural number $n$.
%% (Scores {tree_length = 47, tree_depth = 15, characters = 152, tokens = 46, subsequent_dollars = 0, initial_dollars = 1, parses = 1},262)

Thm51b. $n$ is prime, if and only if the sum of the factorial of the difference of $n$ and $1$ and $1$ is divisible by $n$ for all natural numbers $n$.
%% (Scores {tree_length = 48, tree_depth = 15, characters = 151, tokens = 46, subsequent_dollars = 0, initial_dollars = 1, parses = 1},262)

Thm51b. For all natural numbers $n$, $n$ is prime, if and only if the sum of the factorial of the difference of $n$ and $1$ and $1$ is divisible by $n$.
%% (Scores {tree_length = 49, tree_depth = 15, characters = 152, tokens = 47, subsequent_dollars = 1, initial_dollars = 0, parses = 1},265)

Thm51b. Let $n$ be a natural number. Then $n$ is prime, if and only if the sum of the factorial of the difference of $n$ and $1$ and $1$ is divisible by $n$.
%% (Scores {tree_length = 48, tree_depth = 14, characters = 157, tokens = 49, subsequent_dollars = 0, initial_dollars = 0, parses = 1},269)

Thm51b. We can prove that $n$ is prime, if and only if the sum of the factorial of the difference of $n$ and $1$ and $1$ is divisible by $n$ for every instance $n$ of natural numbers.
%% (Scores {tree_length = 49, tree_depth = 16, characters = 183, tokens = 52, subsequent_dollars = 0, initial_dollars = 0, parses = 1},301)

Thm51b. We can prove that $n$ is prime, if and only if the sum of the factorial of the difference of $n$ and $1$ and $1$ is divisible by $n$ for all instances $n$ of natural numbers.
%% (Scores {tree_length = 50, tree_depth = 16, characters = 182, tokens = 52, subsequent_dollars = 0, initial_dollars = 0, parses = 1},301)

Thm51b. For all instances $n$ of natural numbers, we can prove that $n$ is prime, if and only if the sum of the factorial of the difference of $n$ and $1$ and $1$ is divisible by $n$.
%% (Scores {tree_length = 51, tree_depth = 16, characters = 183, tokens = 53, subsequent_dollars = 0, initial_dollars = 0, parses = 1},304)

Thm51b. Let $n$ be an instance of natural numbers. Then we can prove that $n$ is prime, if and only if the sum of the factorial of the difference of $n$ and $1$ and $1$ is divisible by $n$.
%% (Scores {tree_length = 50, tree_depth = 15, characters = 189, tokens = 55, subsequent_dollars = 0, initial_dollars = 0, parses = 1},310)

Thm52. If $A$ is a finite set, then $$| \wp A | = 2 ^ {| A |}.$$
%% (Scores {tree_length = 36, tree_depth = 10, characters = 64, tokens = 27, subsequent_dollars = 0, initial_dollars = 0, parses = 1},138)

Thm52. $A$ is a finite set, only if $$| \wp A | = 2 ^ {| A |}.$$
%% (Scores {tree_length = 36, tree_depth = 10, characters = 64, tokens = 27, subsequent_dollars = 0, initial_dollars = 1, parses = 1},139)

Thm52. Assume that $A$ is a finite set. Then $| \wp A | = 2 ^ {| A |}$.
%% (Scores {tree_length = 37, tree_depth = 9, characters = 71, tokens = 28, subsequent_dollars = 0, initial_dollars = 0, parses = 1},146)

Thm52. Assume that $A$ is a finite set. Then $$| \wp A | = 2 ^ {| A |}.$$
%% (Scores {tree_length = 37, tree_depth = 9, characters = 73, tokens = 28, subsequent_dollars = 0, initial_dollars = 0, parses = 1},148)

Thm52. If $A$ is finite, then $$| \wp A | = 2 ^ {| A |}$$ for every set $A$.
%% (Scores {tree_length = 39, tree_depth = 11, characters = 76, tokens = 31, subsequent_dollars = 0, initial_dollars = 0, parses = 1},158)

Thm52. If $A$ is finite, then $$| \wp A | = 2 ^ {| A |}$$ for all sets $A$.
%% (Scores {tree_length = 40, tree_depth = 11, characters = 75, tokens = 31, subsequent_dollars = 0, initial_dollars = 0, parses = 1},158)

Thm52. For all sets $A$, if $A$ is finite, then $$| \wp A | = 2 ^ {| A |}.$$
%% (Scores {tree_length = 41, tree_depth = 11, characters = 76, tokens = 32, subsequent_dollars = 0, initial_dollars = 0, parses = 1},161)

Thm52. Let $A$ be a set. Then if $A$ is finite, then $$| \wp A | = 2 ^ {| A |}.$$
%% (Scores {tree_length = 40, tree_depth = 10, characters = 81, tokens = 34, subsequent_dollars = 0, initial_dollars = 0, parses = 1},166)

Thm52. Let $A$ be a set. Then $A$ is finite, only if $$| \wp A | = 2 ^ {| A |}.$$
%% (Scores {tree_length = 40, tree_depth = 10, characters = 81, tokens = 34, subsequent_dollars = 0, initial_dollars = 0, parses = 1},166)

Thm52. Let $A$ be a set. Assume that $A$ is finite. Then $| \wp A | = 2 ^ {| A |}$.
%% (Scores {tree_length = 41, tree_depth = 9, characters = 83, tokens = 34, subsequent_dollars = 0, initial_dollars = 0, parses = 1},168)

Thm52. Let $A$ be a set. Assume that $A$ is finite. Then $$| \wp A | = 2 ^ {| A |}.$$
%% (Scores {tree_length = 41, tree_depth = 9, characters = 85, tokens = 34, subsequent_dollars = 0, initial_dollars = 0, parses = 1},170)

Thm52. If $A$ is finite, then the cardinality of the power set of $A$ is equal to the exponentiation of $2$ and the cardinality of $A$ for every set $A$.
%% (Scores {tree_length = 45, tree_depth = 14, characters = 153, tokens = 43, subsequent_dollars = 0, initial_dollars = 0, parses = 1},256)

Thm52. If $A$ is finite, then the cardinality of the power set of $A$ is equal to the exponentiation of $2$ and the cardinality of $A$ for all sets $A$.
%% (Scores {tree_length = 46, tree_depth = 14, characters = 152, tokens = 43, subsequent_dollars = 0, initial_dollars = 0, parses = 1},256)

Thm52. For all sets $A$, if $A$ is finite, then the cardinality of the power set of $A$ is equal to the exponentiation of $2$ and the cardinality of $A$.
%% (Scores {tree_length = 47, tree_depth = 14, characters = 153, tokens = 44, subsequent_dollars = 0, initial_dollars = 0, parses = 1},259)

Thm52. Let $A$ be a set. Then if $A$ is finite, then the cardinality of the power set of $A$ is equal to the exponentiation of $2$ and the cardinality of $A$.
%% (Scores {tree_length = 46, tree_depth = 13, characters = 158, tokens = 46, subsequent_dollars = 0, initial_dollars = 0, parses = 1},264)

Thm52. Let $A$ be a set. Then $A$ is finite, only if the cardinality of the power set of $A$ is equal to the exponentiation of $2$ and the cardinality of $A$.
%% (Scores {tree_length = 46, tree_depth = 13, characters = 158, tokens = 46, subsequent_dollars = 0, initial_dollars = 0, parses = 1},264)

Thm52. Let $A$ be a set. Assume that $A$ is finite. Then the cardinality of the power set of $A$ is equal to the exponentiation of $2$ and the cardinality of $A$.
%% (Scores {tree_length = 47, tree_depth = 12, characters = 162, tokens = 46, subsequent_dollars = 0, initial_dollars = 0, parses = 1},268)

Thm52. If we can prove that $A$ is finite, then we can prove that the cardinality of the power set of $A$ is equal to the exponentiation of $2$ and the cardinality of $A$ for every set $A$.
%% (Scores {tree_length = 47, tree_depth = 15, characters = 189, tokens = 51, subsequent_dollars = 0, initial_dollars = 0, parses = 1},303)

Thm52. If we can prove that $A$ is finite, then we can prove that the cardinality of the power set of $A$ is equal to the exponentiation of $2$ and the cardinality of $A$ for all sets $A$.
%% (Scores {tree_length = 48, tree_depth = 15, characters = 188, tokens = 51, subsequent_dollars = 0, initial_dollars = 0, parses = 1},303)

Thm52. For all sets $A$, if we can prove that $A$ is finite, then we can prove that the cardinality of the power set of $A$ is equal to the exponentiation of $2$ and the cardinality of $A$.
%% (Scores {tree_length = 49, tree_depth = 15, characters = 189, tokens = 52, subsequent_dollars = 0, initial_dollars = 0, parses = 1},306)

Thm52. Let $A$ be a set. Then if we can prove that $A$ is finite, then we can prove that the cardinality of the power set of $A$ is equal to the exponentiation of $2$ and the cardinality of $A$.
%% (Scores {tree_length = 48, tree_depth = 14, characters = 194, tokens = 54, subsequent_dollars = 0, initial_dollars = 0, parses = 1},311)

Thm52. Let $A$ be a set. Then we can prove that $A$ is finite, only if we can prove that the cardinality of the power set of $A$ is equal to the exponentiation of $2$ and the cardinality of $A$.
%% (Scores {tree_length = 48, tree_depth = 14, characters = 194, tokens = 54, subsequent_dollars = 0, initial_dollars = 0, parses = 1},311)

Thm52. Let $A$ be a set. Assume that we can prove that $A$ is finite. Then we can prove that the cardinality of the power set of $A$ is equal to the exponentiation of $2$ and the cardinality of $A$.
%% (Scores {tree_length = 49, tree_depth = 13, characters = 198, tokens = 54, subsequent_dollars = 0, initial_dollars = 0, parses = 1},315)

Thm58. Let $A$ be a set. Let $n \in N$. Assume that $$| A | = n.$$ let $k \in N$. Then if $k \leq n$, then $$| \binom{ A }{ k}| = \binom{ n }{ k}.$$
%% (Scores {tree_length = 79, tree_depth = 12, characters = 148, tokens = 63, subsequent_dollars = 0, initial_dollars = 0, parses = 1},303)

Thm58. Let $A$ be a set. Let $n \in N$. Assume that $$| A | = n.$$ let $k \in N$. Then $k \leq n$, only if $$| \binom{ A }{ k}| = \binom{ n }{ k}.$$
%% (Scores {tree_length = 79, tree_depth = 12, characters = 148, tokens = 63, subsequent_dollars = 0, initial_dollars = 0, parses = 1},303)

Thm58. Let $A$ be a set. Let $n \in N$. Then if $| A | = n$, then for all natural numbers $k$, if $k \leq n$, then $$| \binom{ A }{ k}| = \binom{ n }{ k}.$$
%% (Scores {tree_length = 77, tree_depth = 12, characters = 156, tokens = 64, subsequent_dollars = 0, initial_dollars = 0, parses = 1},310)

Thm58. Let $A$ be a set. Let $n \in N$. Then $| A | = n$, only if for all natural numbers $k$, if $k \leq n$, then $$| \binom{ A }{ k}| = \binom{ n }{ k}.$$
%% (Scores {tree_length = 77, tree_depth = 12, characters = 156, tokens = 64, subsequent_dollars = 0, initial_dollars = 0, parses = 1},310)

Thm58. Let $A$ be a set. Let $n \in N$. Assume that $$| A | = n.$$ let $k \in N$. Assume that $$k \leq n.$$ then $$| \binom{ A }{ k}| = \binom{ n }{ k}.$$
%% (Scores {tree_length = 80, tree_depth = 13, characters = 154, tokens = 63, subsequent_dollars = 0, initial_dollars = 0, parses = 1},311)

Thm58. Let $A$ be a set. Let $n \in N$. Assume that $$| A | = n.$$ then if $k \leq n$, then $$| \binom{ A }{ k}| = \binom{ n }{ k}$$ for every natural number $k$.
%% (Scores {tree_length = 76, tree_depth = 12, characters = 162, tokens = 63, subsequent_dollars = 0, initial_dollars = 0, parses = 1},314)

Thm58. Let $A$ be a set. Let $n \in N$. Assume that $$| A | = n.$$ then if $k \leq n$, then $$| \binom{ A }{ k}| = \binom{ n }{ k}$$ for all natural numbers $k$.
%% (Scores {tree_length = 77, tree_depth = 12, characters = 161, tokens = 63, subsequent_dollars = 0, initial_dollars = 0, parses = 1},314)

Thm58. For all natural numbers $n$, if $| A | = n$, then for all natural numbers $k$, if $k \leq n$, then $$| \binom{ A }{ k}| = \binom{ n }{ k}$$ for every set $A$.
%% (Scores {tree_length = 75, tree_depth = 14, characters = 165, tokens = 62, subsequent_dollars = 0, initial_dollars = 0, parses = 1},317)

Thm58. For all natural numbers $n$, if $| A | = n$, then for all natural numbers $k$, if $k \leq n$, then $$| \binom{ A }{ k}| = \binom{ n }{ k}$$ for all sets $A$.
%% (Scores {tree_length = 76, tree_depth = 14, characters = 164, tokens = 62, subsequent_dollars = 0, initial_dollars = 0, parses = 1},317)

Thm58. Let $A$ be a set. Let $n \in N$. Assume that $$| A | = n.$$ then for all natural numbers $k$, if $k \leq n$, then $$| \binom{ A }{ k}| = \binom{ n }{ k}.$$
%% (Scores {tree_length = 78, tree_depth = 12, characters = 162, tokens = 64, subsequent_dollars = 0, initial_dollars = 0, parses = 1},317)

Thm58. Let $A$ be a set. Let $n$ be a natural number. Assume that $$| A | = n.$$ let $k \in N$. Then if $k \leq n$, then $$| \binom{ A }{ k}| = \binom{ n }{ k}.$$
%% (Scores {tree_length = 77, tree_depth = 12, characters = 162, tokens = 65, subsequent_dollars = 0, initial_dollars = 0, parses = 1},317)

Thm58. Let $A$ be a set. Let $n$ be a natural number. Assume that $$| A | = n.$$ let $k \in N$. Then $k \leq n$, only if $$| \binom{ A }{ k}| = \binom{ n }{ k}.$$
%% (Scores {tree_length = 77, tree_depth = 12, characters = 162, tokens = 65, subsequent_dollars = 0, initial_dollars = 0, parses = 1},317)

Thm58. Let $A$ be a set. Let $n \in N$. Assume that $$| A | = n.$$ let $k$ be a natural number. Then if $k \leq n$, then $$| \binom{ A }{ k}| = \binom{ n }{ k}.$$
%% (Scores {tree_length = 77, tree_depth = 12, characters = 162, tokens = 65, subsequent_dollars = 0, initial_dollars = 0, parses = 1},317)

Thm58. Let $A$ be a set. Let $n \in N$. Assume that $$| A | = n.$$ let $k$ be a natural number. Then $k \leq n$, only if $$| \binom{ A }{ k}| = \binom{ n }{ k}.$$
%% (Scores {tree_length = 77, tree_depth = 12, characters = 162, tokens = 65, subsequent_dollars = 0, initial_dollars = 0, parses = 1},317)

Thm58. For all sets $A$, for all natural numbers $n$, if $| A | = n$, then for all natural numbers $k$, if $k \leq n$, then $$| \binom{ A }{ k}| = \binom{ n }{ k}.$$
%% (Scores {tree_length = 77, tree_depth = 14, characters = 165, tokens = 63, subsequent_dollars = 0, initial_dollars = 0, parses = 1},320)

Thm58. Let $A$ be a set. Then if $| A | = n$, then for all natural numbers $k$, if $k \leq n$, then $$| \binom{ A }{ k}| = \binom{ n }{ k}$$ for every natural number $n$.
%% (Scores {tree_length = 74, tree_depth = 13, characters = 170, tokens = 64, subsequent_dollars = 0, initial_dollars = 0, parses = 1},322)

Thm58. Let $A$ be a set. Then if $| A | = n$, then for all natural numbers $k$, if $k \leq n$, then $$| \binom{ A }{ k}| = \binom{ n }{ k}$$ for all natural numbers $n$.
%% (Scores {tree_length = 75, tree_depth = 13, characters = 169, tokens = 64, subsequent_dollars = 0, initial_dollars = 0, parses = 1},322)

Thm58. Let $A$ be a set. Let $n$ be a natural number. Then if $| A | = n$, then for all natural numbers $k$, if $k \leq n$, then $$| \binom{ A }{ k}| = \binom{ n }{ k}.$$
%% (Scores {tree_length = 75, tree_depth = 12, characters = 170, tokens = 66, subsequent_dollars = 0, initial_dollars = 0, parses = 1},324)

Thm58. Let $A$ be a set. Let $n$ be a natural number. Then $| A | = n$, only if for all natural numbers $k$, if $k \leq n$, then $$| \binom{ A }{ k}| = \binom{ n }{ k}.$$
%% (Scores {tree_length = 75, tree_depth = 12, characters = 170, tokens = 66, subsequent_dollars = 0, initial_dollars = 0, parses = 1},324)

Thm58. Let $A$ be a set. Then for all natural numbers $n$, if $| A | = n$, then for all natural numbers $k$, if $k \leq n$, then $$| \binom{ A }{ k}| = \binom{ n }{ k}.$$
%% (Scores {tree_length = 76, tree_depth = 13, characters = 170, tokens = 65, subsequent_dollars = 0, initial_dollars = 0, parses = 1},325)

Thm58. Let $A$ be a set. Let $n$ be a natural number. Assume that $$| A | = n.$$ let $k \in N$. Assume that $$k \leq n.$$ then $$| \binom{ A }{ k}| = \binom{ n }{ k}.$$
%% (Scores {tree_length = 78, tree_depth = 13, characters = 168, tokens = 65, subsequent_dollars = 0, initial_dollars = 0, parses = 1},325)

Thm58. Let $A$ be a set. Let $n \in N$. Assume that $$| A | = n.$$ let $k$ be a natural number. Assume that $$k \leq n.$$ then $$| \binom{ A }{ k}| = \binom{ n }{ k}.$$
%% (Scores {tree_length = 78, tree_depth = 13, characters = 168, tokens = 65, subsequent_dollars = 0, initial_dollars = 0, parses = 1},325)

Thm58. Let $A$ be a set. Let $n$ be a natural number. Assume that $$| A | = n.$$ then if $k \leq n$, then $$| \binom{ A }{ k}| = \binom{ n }{ k}$$ for every natural number $k$.
%% (Scores {tree_length = 74, tree_depth = 12, characters = 176, tokens = 65, subsequent_dollars = 0, initial_dollars = 0, parses = 1},328)

Thm58. Let $A$ be a set. Let $n$ be a natural number. Assume that $$| A | = n.$$ then if $k \leq n$, then $$| \binom{ A }{ k}| = \binom{ n }{ k}$$ for all natural numbers $k$.
%% (Scores {tree_length = 75, tree_depth = 12, characters = 175, tokens = 65, subsequent_dollars = 0, initial_dollars = 0, parses = 1},328)

Thm58. Let $A$ be a set. Let $n$ be a natural number. Assume that $$| A | = n.$$ then for all natural numbers $k$, if $k \leq n$, then $$| \binom{ A }{ k}| = \binom{ n }{ k}.$$
%% (Scores {tree_length = 76, tree_depth = 12, characters = 176, tokens = 66, subsequent_dollars = 0, initial_dollars = 0, parses = 1},331)

Thm58. Let $A$ be a set. Let $n$ be a natural number. Assume that $$| A | = n.$$ let $k$ be a natural number. Then if $k \leq n$, then $$| \binom{ A }{ k}| = \binom{ n }{ k}.$$
%% (Scores {tree_length = 75, tree_depth = 12, characters = 176, tokens = 67, subsequent_dollars = 0, initial_dollars = 0, parses = 1},331)

Thm58. Let $A$ be a set. Let $n$ be a natural number. Assume that $$| A | = n.$$ let $k$ be a natural number. Then $k \leq n$, only if $$| \binom{ A }{ k}| = \binom{ n }{ k}.$$
%% (Scores {tree_length = 75, tree_depth = 12, characters = 176, tokens = 67, subsequent_dollars = 0, initial_dollars = 0, parses = 1},331)

Thm58. Let $A$ be a set. Let $n$ be a natural number. Assume that $| A | = n$. Let $k$ be a natural number. Assume that $k \leq n$. Then $| \binom{ A }{ k}| = \binom{ n }{ k}$.
%% (Scores {tree_length = 76, tree_depth = 13, characters = 176, tokens = 67, subsequent_dollars = 0, initial_dollars = 0, parses = 1},333)

Thm58. Let $A$ be a set. Let $n$ be a natural number. Assume that $$| A | = n.$$ let $k$ be a natural number. Assume that $$k \leq n.$$ then $$| \binom{ A }{ k}| = \binom{ n }{ k}.$$
%% (Scores {tree_length = 76, tree_depth = 13, characters = 182, tokens = 67, subsequent_dollars = 0, initial_dollars = 0, parses = 1},339)

Thm58. Let $A$ be a set. Let $n \in N$. Assume that the cardinality of $A$ is equal to $n$. Let $k \in N$. Then if $k$ is less than or equal to $n$, then the cardinality of the number of combinations of $A$ and $k$ is equal to the binomial coefficient of $n$ and $k$.
%% (Scores {tree_length = 87, tree_depth = 12, characters = 267, tokens = 85, subsequent_dollars = 0, initial_dollars = 0, parses = 1},452)

Thm58. Let $A$ be a set. Let $n \in N$. Assume that the cardinality of $A$ is equal to $n$. Let $k \in N$. Then $k$ is less than or equal to $n$, only if the cardinality of the number of combinations of $A$ and $k$ is equal to the binomial coefficient of $n$ and $k$.
%% (Scores {tree_length = 87, tree_depth = 12, characters = 267, tokens = 85, subsequent_dollars = 0, initial_dollars = 0, parses = 1},452)

Thm58. Let $A$ be a set. Let $n \in N$. Assume that the cardinality of $A$ is equal to $n$. Let $k \in N$. Assume that $k$ is less than or equal to $n$. Then the cardinality of the number of combinations of $A$ and $k$ is equal to the binomial coefficient of $n$ and $k$.
%% (Scores {tree_length = 88, tree_depth = 13, characters = 271, tokens = 85, subsequent_dollars = 0, initial_dollars = 0, parses = 1},458)

Thm58. Let $A$ be a set. Let $n \in N$. Then if the cardinality of $A$ is equal to $n$, then for all natural numbers $k$, if $k$ is less than or equal to $n$, then the cardinality of the number of combinations of $A$ and $k$ is equal to the binomial coefficient of $n$ and $k$.
%% (Scores {tree_length = 85, tree_depth = 14, characters = 277, tokens = 86, subsequent_dollars = 0, initial_dollars = 0, parses = 1},463)

Thm58. Let $A$ be a set. Let $n \in N$. Then the cardinality of $A$ is equal to $n$, only if for all natural numbers $k$, if $k$ is less than or equal to $n$, then the cardinality of the number of combinations of $A$ and $k$ is equal to the binomial coefficient of $n$ and $k$.
%% (Scores {tree_length = 85, tree_depth = 14, characters = 277, tokens = 86, subsequent_dollars = 0, initial_dollars = 0, parses = 1},463)

Thm58. Let $A$ be a set. Let $n \in N$. Assume that the cardinality of $A$ is equal to $n$. Then if $k$ is less than or equal to $n$, then the cardinality of the number of combinations of $A$ and $k$ is equal to the binomial coefficient of $n$ and $k$ for every natural number $k$.
%% (Scores {tree_length = 84, tree_depth = 13, characters = 281, tokens = 85, subsequent_dollars = 0, initial_dollars = 0, parses = 1},464)

Thm58. Let $A$ be a set. Let $n \in N$. Assume that the cardinality of $A$ is equal to $n$. Then if $k$ is less than or equal to $n$, then the cardinality of the number of combinations of $A$ and $k$ is equal to the binomial coefficient of $n$ and $k$ for all natural numbers $k$.
%% (Scores {tree_length = 85, tree_depth = 13, characters = 280, tokens = 85, subsequent_dollars = 0, initial_dollars = 0, parses = 1},464)

Thm58. Let $A$ be a set. Let $n$ be a natural number. Assume that the cardinality of $A$ is equal to $n$. Let $k \in N$. Then if $k$ is less than or equal to $n$, then the cardinality of the number of combinations of $A$ and $k$ is equal to the binomial coefficient of $n$ and $k$.
%% (Scores {tree_length = 85, tree_depth = 12, characters = 281, tokens = 87, subsequent_dollars = 0, initial_dollars = 0, parses = 1},466)

Thm58. Let $A$ be a set. Let $n$ be a natural number. Assume that the cardinality of $A$ is equal to $n$. Let $k \in N$. Then $k$ is less than or equal to $n$, only if the cardinality of the number of combinations of $A$ and $k$ is equal to the binomial coefficient of $n$ and $k$.
%% (Scores {tree_length = 85, tree_depth = 12, characters = 281, tokens = 87, subsequent_dollars = 0, initial_dollars = 0, parses = 1},466)

Thm58. Let $A$ be a set. Let $n \in N$. Assume that the cardinality of $A$ is equal to $n$. Let $k$ be a natural number. Then if $k$ is less than or equal to $n$, then the cardinality of the number of combinations of $A$ and $k$ is equal to the binomial coefficient of $n$ and $k$.
%% (Scores {tree_length = 85, tree_depth = 12, characters = 281, tokens = 87, subsequent_dollars = 0, initial_dollars = 0, parses = 1},466)

Thm58. Let $A$ be a set. Let $n \in N$. Assume that the cardinality of $A$ is equal to $n$. Let $k$ be a natural number. Then $k$ is less than or equal to $n$, only if the cardinality of the number of combinations of $A$ and $k$ is equal to the binomial coefficient of $n$ and $k$.
%% (Scores {tree_length = 85, tree_depth = 12, characters = 281, tokens = 87, subsequent_dollars = 0, initial_dollars = 0, parses = 1},466)

Thm58. Let $A$ be a set. Let $n \in N$. Assume that the cardinality of $A$ is equal to $n$. Then for all natural numbers $k$, if $k$ is less than or equal to $n$, then the cardinality of the number of combinations of $A$ and $k$ is equal to the binomial coefficient of $n$ and $k$.
%% (Scores {tree_length = 86, tree_depth = 13, characters = 281, tokens = 86, subsequent_dollars = 0, initial_dollars = 0, parses = 1},467)

Thm58. For all natural numbers $n$, if the cardinality of $A$ is equal to $n$, then for all natural numbers $k$, if $k$ is less than or equal to $n$, then the cardinality of the number of combinations of $A$ and $k$ is equal to the binomial coefficient of $n$ and $k$ for every set $A$.
%% (Scores {tree_length = 83, tree_depth = 16, characters = 286, tokens = 84, subsequent_dollars = 0, initial_dollars = 0, parses = 1},470)

Thm58. For all natural numbers $n$, if the cardinality of $A$ is equal to $n$, then for all natural numbers $k$, if $k$ is less than or equal to $n$, then the cardinality of the number of combinations of $A$ and $k$ is equal to the binomial coefficient of $n$ and $k$ for all sets $A$.
%% (Scores {tree_length = 84, tree_depth = 16, characters = 285, tokens = 84, subsequent_dollars = 0, initial_dollars = 0, parses = 1},470)

Thm58. Let $A$ be a set. Let $n$ be a natural number. Assume that the cardinality of $A$ is equal to $n$. Let $k \in N$. Assume that $k$ is less than or equal to $n$. Then the cardinality of the number of combinations of $A$ and $k$ is equal to the binomial coefficient of $n$ and $k$.
%% (Scores {tree_length = 86, tree_depth = 13, characters = 285, tokens = 87, subsequent_dollars = 0, initial_dollars = 0, parses = 1},472)

Thm58. Let $A$ be a set. Let $n \in N$. Assume that the cardinality of $A$ is equal to $n$. Let $k$ be a natural number. Assume that $k$ is less than or equal to $n$. Then the cardinality of the number of combinations of $A$ and $k$ is equal to the binomial coefficient of $n$ and $k$.
%% (Scores {tree_length = 86, tree_depth = 13, characters = 285, tokens = 87, subsequent_dollars = 0, initial_dollars = 0, parses = 1},472)

Thm58. For all sets $A$, for all natural numbers $n$, if the cardinality of $A$ is equal to $n$, then for all natural numbers $k$, if $k$ is less than or equal to $n$, then the cardinality of the number of combinations of $A$ and $k$ is equal to the binomial coefficient of $n$ and $k$.
%% (Scores {tree_length = 85, tree_depth = 16, characters = 286, tokens = 85, subsequent_dollars = 0, initial_dollars = 0, parses = 1},473)

Thm58. Let $A$ be a set. Then if the cardinality of $A$ is equal to $n$, then for all natural numbers $k$, if $k$ is less than or equal to $n$, then the cardinality of the number of combinations of $A$ and $k$ is equal to the binomial coefficient of $n$ and $k$ for every natural number $n$.
%% (Scores {tree_length = 82, tree_depth = 15, characters = 291, tokens = 86, subsequent_dollars = 0, initial_dollars = 0, parses = 1},475)

Thm58. Let $A$ be a set. Then if the cardinality of $A$ is equal to $n$, then for all natural numbers $k$, if $k$ is less than or equal to $n$, then the cardinality of the number of combinations of $A$ and $k$ is equal to the binomial coefficient of $n$ and $k$ for all natural numbers $n$.
%% (Scores {tree_length = 83, tree_depth = 15, characters = 290, tokens = 86, subsequent_dollars = 0, initial_dollars = 0, parses = 1},475)

Thm58. Let $A$ be a set. Let $n$ be a natural number. Then if the cardinality of $A$ is equal to $n$, then for all natural numbers $k$, if $k$ is less than or equal to $n$, then the cardinality of the number of combinations of $A$ and $k$ is equal to the binomial coefficient of $n$ and $k$.
%% (Scores {tree_length = 83, tree_depth = 14, characters = 291, tokens = 88, subsequent_dollars = 0, initial_dollars = 0, parses = 1},477)

Thm58. Let $A$ be a set. Let $n$ be a natural number. Then the cardinality of $A$ is equal to $n$, only if for all natural numbers $k$, if $k$ is less than or equal to $n$, then the cardinality of the number of combinations of $A$ and $k$ is equal to the binomial coefficient of $n$ and $k$.
%% (Scores {tree_length = 83, tree_depth = 14, characters = 291, tokens = 88, subsequent_dollars = 0, initial_dollars = 0, parses = 1},477)

Thm58. Let $A$ be a set. Then for all natural numbers $n$, if the cardinality of $A$ is equal to $n$, then for all natural numbers $k$, if $k$ is less than or equal to $n$, then the cardinality of the number of combinations of $A$ and $k$ is equal to the binomial coefficient of $n$ and $k$.
%% (Scores {tree_length = 84, tree_depth = 15, characters = 291, tokens = 87, subsequent_dollars = 0, initial_dollars = 0, parses = 1},478)

Thm58. Let $A$ be a set. Let $n$ be a natural number. Assume that the cardinality of $A$ is equal to $n$. Then if $k$ is less than or equal to $n$, then the cardinality of the number of combinations of $A$ and $k$ is equal to the binomial coefficient of $n$ and $k$ for every natural number $k$.
%% (Scores {tree_length = 82, tree_depth = 13, characters = 295, tokens = 87, subsequent_dollars = 0, initial_dollars = 0, parses = 1},478)

Thm58. Let $A$ be a set. Let $n$ be a natural number. Assume that the cardinality of $A$ is equal to $n$. Then if $k$ is less than or equal to $n$, then the cardinality of the number of combinations of $A$ and $k$ is equal to the binomial coefficient of $n$ and $k$ for all natural numbers $k$.
%% (Scores {tree_length = 83, tree_depth = 13, characters = 294, tokens = 87, subsequent_dollars = 0, initial_dollars = 0, parses = 1},478)

Thm58. Let $A$ be a set. Let $n$ be a natural number. Assume that the cardinality of $A$ is equal to $n$. Let $k$ be a natural number. Then if $k$ is less than or equal to $n$, then the cardinality of the number of combinations of $A$ and $k$ is equal to the binomial coefficient of $n$ and $k$.
%% (Scores {tree_length = 83, tree_depth = 12, characters = 295, tokens = 89, subsequent_dollars = 0, initial_dollars = 0, parses = 1},480)

Thm58. Let $A$ be a set. Let $n$ be a natural number. Assume that the cardinality of $A$ is equal to $n$. Let $k$ be a natural number. Then $k$ is less than or equal to $n$, only if the cardinality of the number of combinations of $A$ and $k$ is equal to the binomial coefficient of $n$ and $k$.
%% (Scores {tree_length = 83, tree_depth = 12, characters = 295, tokens = 89, subsequent_dollars = 0, initial_dollars = 0, parses = 1},480)

Thm58. Let $A$ be a set. Let $n$ be a natural number. Assume that the cardinality of $A$ is equal to $n$. Then for all natural numbers $k$, if $k$ is less than or equal to $n$, then the cardinality of the number of combinations of $A$ and $k$ is equal to the binomial coefficient of $n$ and $k$.
%% (Scores {tree_length = 84, tree_depth = 13, characters = 295, tokens = 88, subsequent_dollars = 0, initial_dollars = 0, parses = 1},481)

Thm58. Let $A$ be a set. Let $n$ be a natural number. Assume that the cardinality of $A$ is equal to $n$. Let $k$ be a natural number. Assume that $k$ is less than or equal to $n$. Then the cardinality of the number of combinations of $A$ and $k$ is equal to the binomial coefficient of $n$ and $k$.
%% (Scores {tree_length = 84, tree_depth = 13, characters = 299, tokens = 89, subsequent_dollars = 0, initial_dollars = 0, parses = 1},486)

Thm58. For all instances $n$ of natural numbers, if we can prove that the cardinality of $A$ is equal to $n$, then for all instances $k$ of natural numbers, if we can prove that $k$ is less than or equal to $n$, then we can prove that the cardinality of the number of combinations of $A$ and $k$ is equal to the binomial coefficient of $n$ and $k$ for every set $A$.
%% (Scores {tree_length = 88, tree_depth = 17, characters = 366, tokens = 100, subsequent_dollars = 0, initial_dollars = 0, parses = 1},572)

Thm58. For all instances $n$ of natural numbers, if we can prove that the cardinality of $A$ is equal to $n$, then for all instances $k$ of natural numbers, if we can prove that $k$ is less than or equal to $n$, then we can prove that the cardinality of the number of combinations of $A$ and $k$ is equal to the binomial coefficient of $n$ and $k$ for all sets $A$.
%% (Scores {tree_length = 89, tree_depth = 17, characters = 365, tokens = 100, subsequent_dollars = 0, initial_dollars = 0, parses = 1},572)

Thm58. For all sets $A$, for all instances $n$ of natural numbers, if we can prove that the cardinality of $A$ is equal to $n$, then for all instances $k$ of natural numbers, if we can prove that $k$ is less than or equal to $n$, then we can prove that the cardinality of the number of combinations of $A$ and $k$ is equal to the binomial coefficient of $n$ and $k$.
%% (Scores {tree_length = 90, tree_depth = 17, characters = 366, tokens = 101, subsequent_dollars = 0, initial_dollars = 0, parses = 1},575)

Thm58. Let $A$ be a set. Then if we can prove that the cardinality of $A$ is equal to $n$, then for all instances $k$ of natural numbers, if we can prove that $k$ is less than or equal to $n$, then we can prove that the cardinality of the number of combinations of $A$ and $k$ is equal to the binomial coefficient of $n$ and $k$ for every instance $n$ of natural numbers.
%% (Scores {tree_length = 87, tree_depth = 16, characters = 371, tokens = 102, subsequent_dollars = 0, initial_dollars = 0, parses = 1},577)

Thm58. Let $A$ be a set. Then if we can prove that the cardinality of $A$ is equal to $n$, then for all instances $k$ of natural numbers, if we can prove that $k$ is less than or equal to $n$, then we can prove that the cardinality of the number of combinations of $A$ and $k$ is equal to the binomial coefficient of $n$ and $k$ for all instances $n$ of natural numbers.
%% (Scores {tree_length = 88, tree_depth = 16, characters = 370, tokens = 102, subsequent_dollars = 0, initial_dollars = 0, parses = 1},577)

Thm58. Let $A$ be a set. Then for all instances $n$ of natural numbers, if we can prove that the cardinality of $A$ is equal to $n$, then for all instances $k$ of natural numbers, if we can prove that $k$ is less than or equal to $n$, then we can prove that the cardinality of the number of combinations of $A$ and $k$ is equal to the binomial coefficient of $n$ and $k$.
%% (Scores {tree_length = 89, tree_depth = 16, characters = 371, tokens = 103, subsequent_dollars = 0, initial_dollars = 0, parses = 1},580)

Thm58. Let $A$ be a set. Let $n$ be an instance of natural numbers. Then if we can prove that the cardinality of $A$ is equal to $n$, then for all instances $k$ of natural numbers, if we can prove that $k$ is less than or equal to $n$, then we can prove that the cardinality of the number of combinations of $A$ and $k$ is equal to the binomial coefficient of $n$ and $k$.
%% (Scores {tree_length = 88, tree_depth = 15, characters = 372, tokens = 104, subsequent_dollars = 0, initial_dollars = 0, parses = 1},580)

Thm58. Let $A$ be a set. Let $n$ be an instance of natural numbers. Then we can prove that the cardinality of $A$ is equal to $n$, only if for all instances $k$ of natural numbers, if we can prove that $k$ is less than or equal to $n$, then we can prove that the cardinality of the number of combinations of $A$ and $k$ is equal to the binomial coefficient of $n$ and $k$.
%% (Scores {tree_length = 88, tree_depth = 15, characters = 372, tokens = 104, subsequent_dollars = 0, initial_dollars = 0, parses = 1},580)

Thm58. Let $A$ be a set. Let $n$ be an instance of natural numbers. Assume that we can prove that the cardinality of $A$ is equal to $n$. Then if we can prove that $k$ is less than or equal to $n$, then we can prove that the cardinality of the number of combinations of $A$ and $k$ is equal to the binomial coefficient of $n$ and $k$ for every instance $k$ of natural numbers.
%% (Scores {tree_length = 87, tree_depth = 14, characters = 376, tokens = 103, subsequent_dollars = 0, initial_dollars = 0, parses = 1},581)

Thm58. Let $A$ be a set. Let $n$ be an instance of natural numbers. Assume that we can prove that the cardinality of $A$ is equal to $n$. Then if we can prove that $k$ is less than or equal to $n$, then we can prove that the cardinality of the number of combinations of $A$ and $k$ is equal to the binomial coefficient of $n$ and $k$ for all instances $k$ of natural numbers.
%% (Scores {tree_length = 88, tree_depth = 14, characters = 375, tokens = 103, subsequent_dollars = 0, initial_dollars = 0, parses = 1},581)

Thm58. Let $A$ be a set. Let $n$ be an instance of natural numbers. Assume that we can prove that the cardinality of $A$ is equal to $n$. Then for all instances $k$ of natural numbers, if we can prove that $k$ is less than or equal to $n$, then we can prove that the cardinality of the number of combinations of $A$ and $k$ is equal to the binomial coefficient of $n$ and $k$.
%% (Scores {tree_length = 89, tree_depth = 14, characters = 376, tokens = 104, subsequent_dollars = 0, initial_dollars = 0, parses = 1},584)

Thm58. Let $A$ be a set. Let $n$ be an instance of natural numbers. Assume that we can prove that the cardinality of $A$ is equal to $n$. Let $k$ be an instance of natural numbers. Then if we can prove that $k$ is less than or equal to $n$, then we can prove that the cardinality of the number of combinations of $A$ and $k$ is equal to the binomial coefficient of $n$ and $k$.
%% (Scores {tree_length = 88, tree_depth = 13, characters = 377, tokens = 105, subsequent_dollars = 0, initial_dollars = 0, parses = 1},584)

Thm58. Let $A$ be a set. Let $n$ be an instance of natural numbers. Assume that we can prove that the cardinality of $A$ is equal to $n$. Let $k$ be an instance of natural numbers. Then we can prove that $k$ is less than or equal to $n$, only if we can prove that the cardinality of the number of combinations of $A$ and $k$ is equal to the binomial coefficient of $n$ and $k$.
%% (Scores {tree_length = 88, tree_depth = 13, characters = 377, tokens = 105, subsequent_dollars = 0, initial_dollars = 0, parses = 1},584)

Thm58. Let $A$ be a set. Let $n$ be an instance of natural numbers. Assume that we can prove that the cardinality of $A$ is equal to $n$. Let $k$ be an instance of natural numbers. Assume that we can prove that $k$ is less than or equal to $n$. Then we can prove that the cardinality of the number of combinations of $A$ and $k$ is equal to the binomial coefficient of $n$ and $k$.
%% (Scores {tree_length = 89, tree_depth = 14, characters = 381, tokens = 105, subsequent_dollars = 0, initial_dollars = 0, parses = 1},590)

Thm78. $$u \cdot v \leq \| u \| \| v \|$$ for all vectors $u$ and $v$.
%% (Scores {tree_length = 38, tree_depth = 10, characters = 70, tokens = 25, subsequent_dollars = 0, initial_dollars = 0, parses = 1},144)

Thm78. For all vectors $u$ and $v$, $$u \cdot v \leq \| u \| \| v \|.$$
%% (Scores {tree_length = 39, tree_depth = 10, characters = 71, tokens = 25, subsequent_dollars = 0, initial_dollars = 0, parses = 1},146)

Thm78. Let $u$ and $v$ be vectors. Then $u \cdot v \leq \| u \| \| v \|$.
%% (Scores {tree_length = 38, tree_depth = 9, characters = 73, tokens = 27, subsequent_dollars = 0, initial_dollars = 0, parses = 1},148)

Thm78. Let $u$ and $v$ be vectors. Then $$u \cdot v \leq \| u \| \| v \|.$$
%% (Scores {tree_length = 38, tree_depth = 9, characters = 75, tokens = 26, subsequent_dollars = 0, initial_dollars = 0, parses = 1},149)

Thm78. The dot product of $u$ and $v$ is less than or equal to the product of the norm of $u$ and the norm of $v$ for all vectors $u$ and $v$.
%% (Scores {tree_length = 47, tree_depth = 13, characters = 142, tokens = 46, subsequent_dollars = 0, initial_dollars = 0, parses = 1},249)

Thm78. For all vectors $u$ and $v$, the dot product of $u$ and $v$ is less than or equal to the product of the norm of $u$ and the norm of $v$.
%% (Scores {tree_length = 48, tree_depth = 13, characters = 143, tokens = 47, subsequent_dollars = 0, initial_dollars = 0, parses = 1},252)

Thm78. Let $u$ and $v$ be vectors. Then the dot product of $u$ and $v$ is less than or equal to the product of the norm of $u$ and the norm of $v$.
%% (Scores {tree_length = 47, tree_depth = 12, characters = 147, tokens = 48, subsequent_dollars = 0, initial_dollars = 0, parses = 1},255)

Thm78. We can prove that the dot product of $u$ and $v$ is less than or equal to the product of the norm of $u$ and the norm of $v$ for all instances $u$ and $v$ of vectors.
%% (Scores {tree_length = 49, tree_depth = 14, characters = 173, tokens = 52, subsequent_dollars = 0, initial_dollars = 0, parses = 1},289)

Thm78. For all instances $u$ and $v$ of vectors, we can prove that the dot product of $u$ and $v$ is less than or equal to the product of the norm of $u$ and the norm of $v$.
%% (Scores {tree_length = 50, tree_depth = 14, characters = 174, tokens = 53, subsequent_dollars = 0, initial_dollars = 0, parses = 1},292)

Thm78. Let $u$ and $v$ be instances of vectors. Then we can prove that the dot product of $u$ and $v$ is less than or equal to the product of the norm of $u$ and the norm of $v$.
%% (Scores {tree_length = 49, tree_depth = 13, characters = 178, tokens = 54, subsequent_dollars = 0, initial_dollars = 0, parses = 1},295)

Thm78a. $u \perp v$ implies $u \cdot v = 0$ for all vectors $u$ and $v$.
%% (Scores {tree_length = 39, tree_depth = 9, characters = 72, tokens = 26, subsequent_dollars = 0, initial_dollars = 1, parses = 1},148)

Thm78a. For all vectors $u$ and $v$, $u \perp v$ implies $u \cdot v = 0$.
%% (Scores {tree_length = 40, tree_depth = 9, characters = 73, tokens = 27, subsequent_dollars = 1, initial_dollars = 0, parses = 1},151)

Thm78a. If $u \perp v$, then $u \cdot v = 0$ for all vectors $u$ and $v$.
%% (Scores {tree_length = 41, tree_depth = 10, characters = 73, tokens = 28, subsequent_dollars = 0, initial_dollars = 0, parses = 1},153)

Thm78a. Let $u$ and $v$ be vectors. Then $u \perp v$ implies $u \cdot v = 0$.
%% (Scores {tree_length = 39, tree_depth = 8, characters = 77, tokens = 28, subsequent_dollars = 0, initial_dollars = 0, parses = 1},153)

Thm78a. $u \perp v$, only if $u \cdot v = 0$ for all vectors $u$ and $v$.
%% (Scores {tree_length = 41, tree_depth = 10, characters = 73, tokens = 28, subsequent_dollars = 0, initial_dollars = 1, parses = 1},154)

Thm78a. For all vectors $u$ and $v$, if $u \perp v$, then $u \cdot v = 0$.
%% (Scores {tree_length = 42, tree_depth = 10, characters = 74, tokens = 29, subsequent_dollars = 0, initial_dollars = 0, parses = 1},156)

Thm78a. For all vectors $u$ and $v$, $u \perp v$, only if $u \cdot v = 0$.
%% (Scores {tree_length = 42, tree_depth = 10, characters = 74, tokens = 29, subsequent_dollars = 1, initial_dollars = 0, parses = 1},157)

Thm78a. Let $u$ and $v$ be vectors. Then if $u \perp v$, then $u \cdot v = 0$.
%% (Scores {tree_length = 41, tree_depth = 9, characters = 78, tokens = 30, subsequent_dollars = 0, initial_dollars = 0, parses = 1},159)

Thm78a. Let $u$ and $v$ be vectors. Then $u \perp v$, only if $u \cdot v = 0$.
%% (Scores {tree_length = 41, tree_depth = 9, characters = 78, tokens = 30, subsequent_dollars = 0, initial_dollars = 0, parses = 1},159)

Thm78a. Let $u$ and $v$ be vectors. Then $$u \perp v,$$ only if $$u \cdot v = 0.$$
%% (Scores {tree_length = 41, tree_depth = 9, characters = 82, tokens = 30, subsequent_dollars = 0, initial_dollars = 0, parses = 1},163)

Thm78a. If $u$ is orthogonal to $v$, then the dot product of $u$ and $v$ is equal to $0$ for all vectors $u$ and $v$.
%% (Scores {tree_length = 44, tree_depth = 11, characters = 117, tokens = 42, subsequent_dollars = 0, initial_dollars = 0, parses = 1},215)

Thm78a. $u$ is orthogonal to $v$, only if the dot product of $u$ and $v$ is equal to $0$ for all vectors $u$ and $v$.
%% (Scores {tree_length = 44, tree_depth = 11, characters = 117, tokens = 42, subsequent_dollars = 0, initial_dollars = 1, parses = 1},216)

Thm78a. For all vectors $u$ and $v$, if $u$ is orthogonal to $v$, then the dot product of $u$ and $v$ is equal to $0$.
%% (Scores {tree_length = 45, tree_depth = 11, characters = 118, tokens = 43, subsequent_dollars = 0, initial_dollars = 0, parses = 1},218)

Thm78a. For all vectors $u$ and $v$, $u$ is orthogonal to $v$, only if the dot product of $u$ and $v$ is equal to $0$.
%% (Scores {tree_length = 45, tree_depth = 11, characters = 118, tokens = 43, subsequent_dollars = 1, initial_dollars = 0, parses = 1},219)

Thm78a. Let $u$ and $v$ be vectors. Then if $u$ is orthogonal to $v$, then the dot product of $u$ and $v$ is equal to $0$.
%% (Scores {tree_length = 44, tree_depth = 10, characters = 122, tokens = 44, subsequent_dollars = 0, initial_dollars = 0, parses = 1},221)

Thm78a. Let $u$ and $v$ be vectors. Then $u$ is orthogonal to $v$, only if the dot product of $u$ and $v$ is equal to $0$.
%% (Scores {tree_length = 44, tree_depth = 10, characters = 122, tokens = 44, subsequent_dollars = 0, initial_dollars = 0, parses = 1},221)

Thm78a. We can prove that if $u$ is orthogonal to $v$, then the dot product of $u$ and $v$ is equal to $0$ for all instances $u$ and $v$ of vectors.
%% (Scores {tree_length = 46, tree_depth = 12, characters = 148, tokens = 48, subsequent_dollars = 0, initial_dollars = 0, parses = 1},255)

Thm78a. We can prove that $u$ is orthogonal to $v$, only if the dot product of $u$ and $v$ is equal to $0$ for all instances $u$ and $v$ of vectors.
%% (Scores {tree_length = 46, tree_depth = 12, characters = 148, tokens = 48, subsequent_dollars = 0, initial_dollars = 0, parses = 1},255)

Thm78a. For all instances $u$ and $v$ of vectors, we can prove that if $u$ is orthogonal to $v$, then the dot product of $u$ and $v$ is equal to $0$.
%% (Scores {tree_length = 47, tree_depth = 12, characters = 149, tokens = 49, subsequent_dollars = 0, initial_dollars = 0, parses = 1},258)

Thm78a. For all instances $u$ and $v$ of vectors, we can prove that $u$ is orthogonal to $v$, only if the dot product of $u$ and $v$ is equal to $0$.
%% (Scores {tree_length = 47, tree_depth = 12, characters = 149, tokens = 49, subsequent_dollars = 0, initial_dollars = 0, parses = 1},258)

Thm78a. Let $u$ and $v$ be instances of vectors. Then we can prove that if $u$ is orthogonal to $v$, then the dot product of $u$ and $v$ is equal to $0$.
%% (Scores {tree_length = 46, tree_depth = 11, characters = 153, tokens = 50, subsequent_dollars = 0, initial_dollars = 0, parses = 1},261)

Thm78a. Let $u$ and $v$ be instances of vectors. Then we can prove that $u$ is orthogonal to $v$, only if the dot product of $u$ and $v$ is equal to $0$.
%% (Scores {tree_length = 46, tree_depth = 11, characters = 153, tokens = 50, subsequent_dollars = 0, initial_dollars = 0, parses = 1},261)

Thm91. $$\| u + v \| \leq \| u \| + \| v \|$$ for all vectors $u$ and $v$.
%% (Scores {tree_length = 41, tree_depth = 10, characters = 74, tokens = 28, subsequent_dollars = 0, initial_dollars = 0, parses = 1},154)

Thm91. For all vectors $u$ and $v$, $$\| u + v \| \leq \| u \| + \| v \|.$$
%% (Scores {tree_length = 42, tree_depth = 10, characters = 75, tokens = 28, subsequent_dollars = 0, initial_dollars = 0, parses = 1},156)

Thm91. Let $u$ and $v$ be vectors. Then $\| u + v \| \leq \| u \| + \| v \|$.
%% (Scores {tree_length = 41, tree_depth = 9, characters = 77, tokens = 30, subsequent_dollars = 0, initial_dollars = 0, parses = 1},158)

Thm91. Let $u$ and $v$ be vectors. Then $$\| u + v \| \leq \| u \| + \| v \|.$$
%% (Scores {tree_length = 41, tree_depth = 9, characters = 79, tokens = 29, subsequent_dollars = 0, initial_dollars = 0, parses = 1},159)

Thm91. The norm of the sum of $u$ and $v$ is less than or equal to the sum of the norm of $u$ and the norm of $v$ for all vectors $u$ and $v$.
%% (Scores {tree_length = 50, tree_depth = 13, characters = 142, tokens = 48, subsequent_dollars = 0, initial_dollars = 0, parses = 1},254)

Thm91. For all vectors $u$ and $v$, the norm of the sum of $u$ and $v$ is less than or equal to the sum of the norm of $u$ and the norm of $v$.
%% (Scores {tree_length = 51, tree_depth = 13, characters = 143, tokens = 49, subsequent_dollars = 0, initial_dollars = 0, parses = 1},257)

Thm91. Let $u$ and $v$ be vectors. Then the norm of the sum of $u$ and $v$ is less than or equal to the sum of the norm of $u$ and the norm of $v$.
%% (Scores {tree_length = 50, tree_depth = 12, characters = 147, tokens = 50, subsequent_dollars = 0, initial_dollars = 0, parses = 1},260)

Thm91. We can prove that the norm of the sum of $u$ and $v$ is less than or equal to the sum of the norm of $u$ and the norm of $v$ for all instances $u$ and $v$ of vectors.
%% (Scores {tree_length = 52, tree_depth = 14, characters = 173, tokens = 54, subsequent_dollars = 0, initial_dollars = 0, parses = 1},294)

Thm91. For all instances $u$ and $v$ of vectors, we can prove that the norm of the sum of $u$ and $v$ is less than or equal to the sum of the norm of $u$ and the norm of $v$.
%% (Scores {tree_length = 53, tree_depth = 14, characters = 174, tokens = 55, subsequent_dollars = 0, initial_dollars = 0, parses = 1},297)

Thm91. Let $u$ and $v$ be instances of vectors. Then we can prove that the norm of the sum of $u$ and $v$ is less than or equal to the sum of the norm of $u$ and the norm of $v$.
%% (Scores {tree_length = 52, tree_depth = 13, characters = 178, tokens = 56, subsequent_dollars = 0, initial_dollars = 0, parses = 1},300)

Thm98. Let $n \in N$. Then $$n > 1,$$ only if $p$ is prime and $n < p < 2 n$ for a natural number $p$.
%% (Scores {tree_length = 60, tree_depth = 13, characters = 102, tokens = 40, subsequent_dollars = 0, initial_dollars = 0, parses = 1},216)

Thm98. Let $n \in N$. Then $$n > 1,$$ only if $p$ is prime and $n < p < 2 n$ for some natural number $p$.
%% (Scores {tree_length = 61, tree_depth = 13, characters = 105, tokens = 40, subsequent_dollars = 0, initial_dollars = 0, parses = 1},220)

Thm98. Let $n$ be a natural number. Then $$n > 1,$$ only if $p$ is prime and $n < p < 2 n$ for a natural number $p$.
%% (Scores {tree_length = 58, tree_depth = 13, characters = 116, tokens = 42, subsequent_dollars = 0, initial_dollars = 0, parses = 1},230)

Thm98. Let $n$ be a natural number. Then $$n > 1,$$ only if $p$ is prime and $n < p < 2 n$ for some natural number $p$.
%% (Scores {tree_length = 59, tree_depth = 13, characters = 119, tokens = 42, subsequent_dollars = 0, initial_dollars = 0, parses = 1},234)

Thm98. Let $n \in N$. Then $$n > 1,$$ only if $p$ is prime and $n < p$ and $p < 2 n$ for a natural number $p$.
%% (Scores {tree_length = 67, tree_depth = 14, characters = 110, tokens = 44, subsequent_dollars = 0, initial_dollars = 0, parses = 1},236)

Thm98. Let $n \in N$. Then $$n > 1,$$ only if $p$ is prime and $n < p$ and $p < 2 n$ for some natural number $p$.
%% (Scores {tree_length = 68, tree_depth = 14, characters = 113, tokens = 44, subsequent_dollars = 0, initial_dollars = 0, parses = 1},240)

Thm98. Let $n \in N$. Then if $n > 1$, then there exists a natural number $p$, such that $p$ is prime and $n < p < 2 n$.
%% (Scores {tree_length = 62, tree_depth = 13, characters = 120, tokens = 44, subsequent_dollars = 0, initial_dollars = 0, parses = 1},240)

Thm98. Let $n \in N$. Then $n > 1$, only if there exists a natural number $p$, such that $p$ is prime and $n < p < 2 n$.
%% (Scores {tree_length = 62, tree_depth = 13, characters = 120, tokens = 44, subsequent_dollars = 0, initial_dollars = 0, parses = 1},240)

Thm98. Let $n \in N$. Then $$n > 1,$$ only if there exists a natural number $p$, such that $p$ is prime and $n < p < 2 n$.
%% (Scores {tree_length = 62, tree_depth = 13, characters = 122, tokens = 44, subsequent_dollars = 0, initial_dollars = 0, parses = 1},242)

Thm98. If $n > 1$, then there exists a natural number $p$, such that $p$ is prime and $n < p < 2 n$ for every natural number $n$.
%% (Scores {tree_length = 59, tree_depth = 14, characters = 129, tokens = 43, subsequent_dollars = 0, initial_dollars = 0, parses = 1},246)

Thm98. If $n > 1$, then there exists a natural number $p$, such that $p$ is prime and $n < p < 2 n$ for all natural numbers $n$.
%% (Scores {tree_length = 60, tree_depth = 14, characters = 128, tokens = 43, subsequent_dollars = 0, initial_dollars = 0, parses = 1},246)

Thm98. $n > 1$, only if there exists a natural number $p$, such that $p$ is prime and $n < p < 2 n$ for every natural number $n$.
%% (Scores {tree_length = 59, tree_depth = 14, characters = 129, tokens = 43, subsequent_dollars = 0, initial_dollars = 1, parses = 1},247)

Thm98. $n > 1$, only if there exists a natural number $p$, such that $p$ is prime and $n < p < 2 n$ for all natural numbers $n$.
%% (Scores {tree_length = 60, tree_depth = 14, characters = 128, tokens = 43, subsequent_dollars = 0, initial_dollars = 1, parses = 1},247)

Thm98. For all natural numbers $n$, if $n > 1$, then there exists a natural number $p$, such that $p$ is prime and $n < p < 2 n$.
%% (Scores {tree_length = 61, tree_depth = 14, characters = 129, tokens = 44, subsequent_dollars = 0, initial_dollars = 0, parses = 1},249)

Thm98. Let $n$ be a natural number. Then $$n > 1,$$ only if $p$ is prime and $n < p$ and $p < 2 n$ for a natural number $p$.
%% (Scores {tree_length = 65, tree_depth = 14, characters = 124, tokens = 46, subsequent_dollars = 0, initial_dollars = 0, parses = 1},250)

Thm98. For all natural numbers $n$, $n > 1$, only if there exists a natural number $p$, such that $p$ is prime and $n < p < 2 n$.
%% (Scores {tree_length = 61, tree_depth = 14, characters = 129, tokens = 44, subsequent_dollars = 1, initial_dollars = 0, parses = 1},250)

Thm98. Let $n$ be a natural number. Then $$n > 1,$$ only if $p$ is prime and $n < p$ and $p < 2 n$ for some natural number $p$.
%% (Scores {tree_length = 66, tree_depth = 14, characters = 127, tokens = 46, subsequent_dollars = 0, initial_dollars = 0, parses = 1},254)

Thm98. Let $n$ be a natural number. Then if $n > 1$, then there exists a natural number $p$, such that $p$ is prime and $n < p < 2 n$.
%% (Scores {tree_length = 60, tree_depth = 13, characters = 134, tokens = 46, subsequent_dollars = 0, initial_dollars = 0, parses = 1},254)

Thm98. Let $n$ be a natural number. Then $n > 1$, only if there exists a natural number $p$, such that $p$ is prime and $n < p < 2 n$.
%% (Scores {tree_length = 60, tree_depth = 13, characters = 134, tokens = 46, subsequent_dollars = 0, initial_dollars = 0, parses = 1},254)

Thm98. Let $n$ be a natural number. Then $$n > 1,$$ only if there exists a natural number $p$, such that $p$ is prime and $n < p < 2 n$.
%% (Scores {tree_length = 60, tree_depth = 13, characters = 136, tokens = 46, subsequent_dollars = 0, initial_dollars = 0, parses = 1},256)

Thm98. Let $n \in N$. Then if $n > 1$, then there exists a natural number $p$, such that $p$ is prime and $n < p$ and $p < 2 n$.
%% (Scores {tree_length = 69, tree_depth = 14, characters = 128, tokens = 48, subsequent_dollars = 0, initial_dollars = 0, parses = 1},260)

Thm98. Let $n \in N$. Then $n > 1$, only if there exists a natural number $p$, such that $p$ is prime and $n < p$ and $p < 2 n$.
%% (Scores {tree_length = 69, tree_depth = 14, characters = 128, tokens = 48, subsequent_dollars = 0, initial_dollars = 0, parses = 1},260)

Thm98. Let $n \in N$. Then $$n > 1,$$ only if there exists a natural number $p$, such that $p$ is prime and $n < p$ and $p < 2 n$.
%% (Scores {tree_length = 69, tree_depth = 14, characters = 130, tokens = 48, subsequent_dollars = 0, initial_dollars = 0, parses = 1},262)

Thm98. If $n > 1$, then there exists a natural number $p$, such that $p$ is prime and $n < p$ and $p < 2 n$ for every natural number $n$.
%% (Scores {tree_length = 66, tree_depth = 15, characters = 137, tokens = 47, subsequent_dollars = 0, initial_dollars = 0, parses = 1},266)

Thm98. If $n > 1$, then there exists a natural number $p$, such that $p$ is prime and $n < p$ and $p < 2 n$ for all natural numbers $n$.
%% (Scores {tree_length = 67, tree_depth = 15, characters = 136, tokens = 47, subsequent_dollars = 0, initial_dollars = 0, parses = 1},266)

Thm98. $n > 1$, only if there exists a natural number $p$, such that $p$ is prime and $n < p$ and $p < 2 n$ for every natural number $n$.
%% (Scores {tree_length = 66, tree_depth = 15, characters = 137, tokens = 47, subsequent_dollars = 0, initial_dollars = 1, parses = 1},267)

Thm98. $n > 1$, only if there exists a natural number $p$, such that $p$ is prime and $n < p$ and $p < 2 n$ for all natural numbers $n$.
%% (Scores {tree_length = 67, tree_depth = 15, characters = 136, tokens = 47, subsequent_dollars = 0, initial_dollars = 1, parses = 1},267)

Thm98. For all natural numbers $n$, if $n > 1$, then there exists a natural number $p$, such that $p$ is prime and $n < p$ and $p < 2 n$.
%% (Scores {tree_length = 68, tree_depth = 15, characters = 137, tokens = 48, subsequent_dollars = 0, initial_dollars = 0, parses = 1},269)

Thm98. For all natural numbers $n$, $n > 1$, only if there exists a natural number $p$, such that $p$ is prime and $n < p$ and $p < 2 n$.
%% (Scores {tree_length = 68, tree_depth = 15, characters = 137, tokens = 48, subsequent_dollars = 1, initial_dollars = 0, parses = 1},270)

Thm98. Let $n$ be a natural number. Then if $n > 1$, then there exists a natural number $p$, such that $p$ is prime and $n < p$ and $p < 2 n$.
%% (Scores {tree_length = 67, tree_depth = 14, characters = 142, tokens = 50, subsequent_dollars = 0, initial_dollars = 0, parses = 1},274)

Thm98. Let $n$ be a natural number. Then $n > 1$, only if there exists a natural number $p$, such that $p$ is prime and $n < p$ and $p < 2 n$.
%% (Scores {tree_length = 67, tree_depth = 14, characters = 142, tokens = 50, subsequent_dollars = 0, initial_dollars = 0, parses = 1},274)

Thm98. Let $n$ be a natural number. Then $$n > 1,$$ only if there exists a natural number $p$, such that $p$ is prime and $n < p$ and $p < 2 n$.
%% (Scores {tree_length = 67, tree_depth = 14, characters = 144, tokens = 50, subsequent_dollars = 0, initial_dollars = 0, parses = 1},276)

Thm98. Let $n \in N$. Then $n$ is greater than $1$, only if $p$ is prime and $n$ is less than $p$ and $p$ is less than the product of $2$ and $n$ for a natural number $p$.
%% (Scores {tree_length = 71, tree_depth = 16, characters = 171, tokens = 62, subsequent_dollars = 0, initial_dollars = 0, parses = 1},321)

Thm98. Let $n \in N$. Then $n$ is greater than $1$, only if $p$ is prime and $n$ is less than $p$ and $p$ is less than the product of $2$ and $n$ for some natural number $p$.
%% (Scores {tree_length = 72, tree_depth = 16, characters = 174, tokens = 62, subsequent_dollars = 0, initial_dollars = 0, parses = 1},325)

Thm98. Let $n$ be a natural number. Then $n$ is greater than $1$, only if $p$ is prime and $n$ is less than $p$ and $p$ is less than the product of $2$ and $n$ for a natural number $p$.
%% (Scores {tree_length = 69, tree_depth = 16, characters = 185, tokens = 64, subsequent_dollars = 0, initial_dollars = 0, parses = 1},335)

Thm98. Let $n$ be a natural number. Then $n$ is greater than $1$, only if $p$ is prime and $n$ is less than $p$ and $p$ is less than the product of $2$ and $n$ for some natural number $p$.
%% (Scores {tree_length = 70, tree_depth = 16, characters = 188, tokens = 64, subsequent_dollars = 0, initial_dollars = 0, parses = 1},339)

Thm98. Let $n \in N$. Then if $n$ is greater than $1$, then there exists a natural number $p$, such that $p$ is prime and $n$ is less than $p$ and $p$ is less than the product of $2$ and $n$.
%% (Scores {tree_length = 73, tree_depth = 16, characters = 191, tokens = 66, subsequent_dollars = 0, initial_dollars = 0, parses = 1},347)

Thm98. Let $n \in N$. Then $n$ is greater than $1$, only if there exists a natural number $p$, such that $p$ is prime and $n$ is less than $p$ and $p$ is less than the product of $2$ and $n$.
%% (Scores {tree_length = 73, tree_depth = 16, characters = 191, tokens = 66, subsequent_dollars = 0, initial_dollars = 0, parses = 1},347)

Thm98. If $n$ is greater than $1$, then there exists a natural number $p$, such that $p$ is prime and $n$ is less than $p$ and $p$ is less than the product of $2$ and $n$ for every natural number $n$.
%% (Scores {tree_length = 70, tree_depth = 17, characters = 200, tokens = 65, subsequent_dollars = 0, initial_dollars = 0, parses = 1},353)

Thm98. If $n$ is greater than $1$, then there exists a natural number $p$, such that $p$ is prime and $n$ is less than $p$ and $p$ is less than the product of $2$ and $n$ for all natural numbers $n$.
%% (Scores {tree_length = 71, tree_depth = 17, characters = 199, tokens = 65, subsequent_dollars = 0, initial_dollars = 0, parses = 1},353)

Thm98. $n$ is greater than $1$, only if there exists a natural number $p$, such that $p$ is prime and $n$ is less than $p$ and $p$ is less than the product of $2$ and $n$ for every natural number $n$.
%% (Scores {tree_length = 70, tree_depth = 17, characters = 200, tokens = 65, subsequent_dollars = 0, initial_dollars = 1, parses = 1},354)

Thm98. $n$ is greater than $1$, only if there exists a natural number $p$, such that $p$ is prime and $n$ is less than $p$ and $p$ is less than the product of $2$ and $n$ for all natural numbers $n$.
%% (Scores {tree_length = 71, tree_depth = 17, characters = 199, tokens = 65, subsequent_dollars = 0, initial_dollars = 1, parses = 1},354)

Thm98. For all natural numbers $n$, if $n$ is greater than $1$, then there exists a natural number $p$, such that $p$ is prime and $n$ is less than $p$ and $p$ is less than the product of $2$ and $n$.
%% (Scores {tree_length = 72, tree_depth = 17, characters = 200, tokens = 66, subsequent_dollars = 0, initial_dollars = 0, parses = 1},356)

Thm98. For all natural numbers $n$, $n$ is greater than $1$, only if there exists a natural number $p$, such that $p$ is prime and $n$ is less than $p$ and $p$ is less than the product of $2$ and $n$.
%% (Scores {tree_length = 72, tree_depth = 17, characters = 200, tokens = 66, subsequent_dollars = 1, initial_dollars = 0, parses = 1},357)

Thm98. Let $n$ be a natural number. Then if $n$ is greater than $1$, then there exists a natural number $p$, such that $p$ is prime and $n$ is less than $p$ and $p$ is less than the product of $2$ and $n$.
%% (Scores {tree_length = 71, tree_depth = 16, characters = 205, tokens = 68, subsequent_dollars = 0, initial_dollars = 0, parses = 1},361)

Thm98. Let $n$ be a natural number. Then $n$ is greater than $1$, only if there exists a natural number $p$, such that $p$ is prime and $n$ is less than $p$ and $p$ is less than the product of $2$ and $n$.
%% (Scores {tree_length = 71, tree_depth = 16, characters = 205, tokens = 68, subsequent_dollars = 0, initial_dollars = 0, parses = 1},361)

Thm98. Let $n$ be an instance of natural numbers. Then we can prove that $n$ is greater than $1$, only if $p$ is prime and $n$ is less than $p$ and $p$ is less than the product of $2$ and $n$ for a natural number $p$.
%% (Scores {tree_length = 71, tree_depth = 17, characters = 217, tokens = 70, subsequent_dollars = 0, initial_dollars = 0, parses = 1},376)

Thm98. Let $n$ be an instance of natural numbers. Then we can prove that $n$ is greater than $1$, only if $p$ is prime and $n$ is less than $p$ and $p$ is less than the product of $2$ and $n$ for some natural number $p$.
%% (Scores {tree_length = 72, tree_depth = 17, characters = 220, tokens = 70, subsequent_dollars = 0, initial_dollars = 0, parses = 1},380)

Thm98. We can prove that if $n$ is greater than $1$, then there exists a natural number $p$, such that $p$ is prime and $n$ is less than $p$ and $p$ is less than the product of $2$ and $n$ for every instance $n$ of natural numbers.
%% (Scores {tree_length = 72, tree_depth = 18, characters = 231, tokens = 71, subsequent_dollars = 0, initial_dollars = 0, parses = 1},393)

Thm98. We can prove that if $n$ is greater than $1$, then there exists a natural number $p$, such that $p$ is prime and $n$ is less than $p$ and $p$ is less than the product of $2$ and $n$ for all instances $n$ of natural numbers.
%% (Scores {tree_length = 73, tree_depth = 18, characters = 230, tokens = 71, subsequent_dollars = 0, initial_dollars = 0, parses = 1},393)

Thm98. We can prove that $n$ is greater than $1$, only if there exists a natural number $p$, such that $p$ is prime and $n$ is less than $p$ and $p$ is less than the product of $2$ and $n$ for every instance $n$ of natural numbers.
%% (Scores {tree_length = 72, tree_depth = 18, characters = 231, tokens = 71, subsequent_dollars = 0, initial_dollars = 0, parses = 1},393)

Thm98. We can prove that $n$ is greater than $1$, only if there exists a natural number $p$, such that $p$ is prime and $n$ is less than $p$ and $p$ is less than the product of $2$ and $n$ for all instances $n$ of natural numbers.
%% (Scores {tree_length = 73, tree_depth = 18, characters = 230, tokens = 71, subsequent_dollars = 0, initial_dollars = 0, parses = 1},393)

Thm98. For all instances $n$ of natural numbers, we can prove that if $n$ is greater than $1$, then there exists a natural number $p$, such that $p$ is prime and $n$ is less than $p$ and $p$ is less than the product of $2$ and $n$.
%% (Scores {tree_length = 74, tree_depth = 18, characters = 231, tokens = 72, subsequent_dollars = 0, initial_dollars = 0, parses = 1},396)

Thm98. For all instances $n$ of natural numbers, we can prove that $n$ is greater than $1$, only if there exists a natural number $p$, such that $p$ is prime and $n$ is less than $p$ and $p$ is less than the product of $2$ and $n$.
%% (Scores {tree_length = 74, tree_depth = 18, characters = 231, tokens = 72, subsequent_dollars = 0, initial_dollars = 0, parses = 1},396)

Thm98. Let $n$ be an instance of natural numbers. Then we can prove that if $n$ is greater than $1$, then there exists a natural number $p$, such that $p$ is prime and $n$ is less than $p$ and $p$ is less than the product of $2$ and $n$.
%% (Scores {tree_length = 73, tree_depth = 17, characters = 237, tokens = 74, subsequent_dollars = 0, initial_dollars = 0, parses = 1},402)

Thm98. Let $n$ be an instance of natural numbers. Then we can prove that $n$ is greater than $1$, only if there exists a natural number $p$, such that $p$ is prime and $n$ is less than $p$ and $p$ is less than the product of $2$ and $n$.
%% (Scores {tree_length = 73, tree_depth = 17, characters = 237, tokens = 74, subsequent_dollars = 0, initial_dollars = 0, parses = 1},402)

\end{document}
