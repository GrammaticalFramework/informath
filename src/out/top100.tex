\documentclass{article}
\usepackage{amsfonts}
\usepackage{amssymb}
\setlength\parindent{0pt}
\setlength\parskip{8pt}
\begin{document}

Thm01. For all instances $m$ and $n$ of natural numbers, if we can prove that $n$ is not equal to $0$, then we can prove that the exponentiation of the quotient of $m$ and $n$ and $2$ is not equal to $2$.

Thm01. Let $m$ and $n$ be instances of natural numbers. Then if we can prove that $n$ is not equal to $0$, then we can prove that the exponentiation of the quotient of $m$ and $n$ and $2$ is not equal to $2$.

Thm01. Let $m$ and $n$ be instances of natural numbers. Assume that we can prove that $n$ is not equal to $0$. Then we can prove that the exponentiation of the quotient of $m$ and $n$ and $2$ is not equal to $2$.

Thm01. For all natural numbers $m$ and $n$, if $n$ is not equal to $0$, then the exponentiation of the quotient of $m$ and $n$ and $2$ is not equal to $2$.

Thm01. Let $m$ and $n$ be natural numbers. Then if $n$ is not equal to $0$, then the exponentiation of the quotient of $m$ and $n$ and $2$ is not equal to $2$.

Thm01. Let $m , n \in N$. Then if $n$ is not equal to $0$, then the exponentiation of the quotient of $m$ and $n$ and $2$ is not equal to $2$.

Thm01. Let $m$ and $n$ be natural numbers. Assume that $n$ is not equal to $0$. Then the exponentiation of the quotient of $m$ and $n$ and $2$ is not equal to $2$.

Thm01. Let $m , n \in N$. Assume that $n$ is not equal to $0$. Then the exponentiation of the quotient of $m$ and $n$ and $2$ is not equal to $2$.

Thm01. For all natural numbers $m$ and $n$, if $n \neq 0$, then $(\frac{ m}{n})^ {2}\neq 2$.

Thm01. Let $m$ and $n$ be natural numbers. Then if $n \neq 0$, then $(\frac{ m}{n})^ {2}\neq 2$.

Thm01. Let $m , n \in N$. Then if $n \neq 0$, then $(\frac{ m}{n})^ {2}\neq 2$.

Thm01. Let $m$ and $n$ be natural numbers. Assume that $n \neq 0$. Then $(\frac{ m}{n})^ {2}\neq 2$.

Thm01. Let $m , n \in N$. Assume that $n \neq 0$. Then $(\frac{ m}{n})^ {2}\neq 2$.

Thm01a. For all instances $m$ and $n$ of natural numbers, we can prove that the exponentiation of the quotient of $m$ and the sum of $n$ and $1$ and $2$ is not equal to $2$.

Thm01a. Let $m$ and $n$ be instances of natural numbers. Then we can prove that the exponentiation of the quotient of $m$ and the sum of $n$ and $1$ and $2$ is not equal to $2$.

Thm01a. For all natural numbers $m$ and $n$, the exponentiation of the quotient of $m$ and the sum of $n$ and $1$ and $2$ is not equal to $2$.

Thm01a. Let $m$ and $n$ be natural numbers. Then the exponentiation of the quotient of $m$ and the sum of $n$ and $1$ and $2$ is not equal to $2$.

Thm01a. Let $m , n \in N$. Then the exponentiation of the quotient of $m$ and the sum of $n$ and $1$ and $2$ is not equal to $2$.

Thm01a. For all natural numbers $m$ and $n$, $(\frac{ m}{n + 1})^ {2}\neq 2$.

Thm01a. Let $m$ and $n$ be natural numbers. Then $(\frac{ m}{n + 1})^ {2}\neq 2$.

Thm01a. Let $m , n \in N$. Then $(\frac{ m}{n + 1})^ {2}\neq 2$.

Thm01b. For all instances $q$ of rational numbers, we can prove that the exponentiation of $q$ and $2$ is not equal to $2$.

Thm01b. Let $q$ be an instance of rational numbers. Then we can prove that the exponentiation of $q$ and $2$ is not equal to $2$.

Thm01b. For all rational numbers $q$, the exponentiation of $q$ and $2$ is not equal to $2$.

Thm01b. Let $q$ be a rational number. Then the exponentiation of $q$ and $2$ is not equal to $2$.

Thm01b. Let $q \in Q$. Then the exponentiation of $q$ and $2$ is not equal to $2$.

Thm01b. For all rational numbers $q$, $q ^ {2}\neq 2$.

Thm01b. Let $q$ be a rational number. Then $q ^ {2}\neq 2$.

Thm01b. Let $q \in Q$. Then $q ^ {2}\neq 2$.

Thm01c. For all instances $q$ of rational numbers, we can prove that the square root of $2$ is not equal to $q$.

Thm01c. Let $q$ be an instance of rational numbers. Then we can prove that the square root of $2$ is not equal to $q$.

Thm01c. For all rational numbers $q$, the square root of $2$ is not equal to $q$.

Thm01c. Let $q$ be a rational number. Then the square root of $2$ is not equal to $q$.

Thm01c. Let $q \in Q$. Then the square root of $2$ is not equal to $q$.

Thm01c. For all rational numbers $q$, $\sqrt{ 2}\neq q$.

Thm01c. Let $q$ be a rational number. Then $\sqrt{ 2}\neq q$.

Thm01c. Let $q \in Q$. Then $\sqrt{ 2}\neq q$.

Definition. Let $x$ be an instance of real numbers. Then $x$ is rational, if there exists an integer $p$, such that there exists an integer $q$, such that $q$ is not equal to $0$ and $x$ is equal to the quotient of $p$ and $q$.

Definition. Let $x$ be an instance of real numbers. Then $x$ is rational, if there exists an integer $q$, such that $q$ is not equal to $0$ and $x$ is equal to the quotient of $p$ and $q$ for an integer $p$.

Definition. Let $x$ be an instance of real numbers. Then $x$ is rational, if there exists an integer $q$, such that $q$ is not equal to $0$ and $x$ is equal to the quotient of $p$ and $q$ for some integer $p$.

Definition. Let $x$ be a real number. Then $x$ is rational, if there exists an integer $p$, such that there exists an integer $q$, such that $q$ is not equal to $0$ and $x$ is equal to the quotient of $p$ and $q$.

Definition. Let $x$ be a real number. Then $x$ is rational, if there exists an integer $q$, such that $q$ is not equal to $0$ and $x$ is equal to the quotient of $p$ and $q$ for an integer $p$.

Definition. Let $x$ be a real number. Then $x$ is rational, if there exists an integer $q$, such that $q$ is not equal to $0$ and $x$ is equal to the quotient of $p$ and $q$ for some integer $p$.

Definition. Let $x \in R$. Then $x$ is rational, if there exists an integer $p$, such that there exists an integer $q$, such that $q$ is not equal to $0$ and $x$ is equal to the quotient of $p$ and $q$.

Definition. Let $x \in R$. Then $x$ is rational, if there exists an integer $q$, such that $q$ is not equal to $0$ and $x$ is equal to the quotient of $p$ and $q$ for an integer $p$.

Definition. Let $x \in R$. Then $x$ is rational, if there exists an integer $q$, such that $q$ is not equal to $0$ and $x$ is equal to the quotient of $p$ and $q$ for some integer $p$.

Definition. Let $x$ be a real number. Then $x$ is rational, if there exists an integer $p$, such that there exists an integer $q$, such that $q \neq 0$ and $x = \frac{ p}{q}$.

Definition. Let $x$ be a real number. Then $x$ is rational, if there exists an integer $q$, such that $q \neq 0$ and $x = \frac{ p}{q}$ for an integer $p$.

Definition. Let $x$ be a real number. Then $x$ is rational, if there exists an integer $q$, such that $q \neq 0$ and $x = \frac{ p}{q}$ for some integer $p$.

Definition. Let $x \in R$. Then $x$ is rational, if there exists an integer $p$, such that there exists an integer $q$, such that $q \neq 0$ and $x = \frac{ p}{q}$.

Definition. Let $x \in R$. Then $x$ is rational, if there exists an integer $q$, such that $q \neq 0$ and $x = \frac{ p}{q}$ for an integer $p$.

Definition. Let $x \in R$. Then $x$ is rational, if there exists an integer $q$, such that $q \neq 0$ and $x = \frac{ p}{q}$ for some integer $p$.

Definition. Let $x$ be a real number. Then $x$ is rational, if there exist integers $p$ and $q$, such that $q \neq 0$ and $x = \frac{ p}{q}$.

Definition. Let $x$ be a real number. Then $x$ is rational, if $q \neq 0$ and $x = \frac{ p}{q}$ for some integers $p$ and $q$.

Definition. Let $x \in R$. Then $x$ is rational, if there exist integers $p$ and $q$, such that $q \neq 0$ and $x = \frac{ p}{q}$.

Definition. Let $x \in R$. Then $x$ is rational, if $q \neq 0$ and $x = \frac{ p}{q}$ for some integers $p$ and $q$.

Thm01d. We can prove that the square root of $2$ is not rational.

Thm01d. The square root of $2$ is not rational.

Thm01d. $\sqrt{ 2}$ is not rational.

Definition. Let $x$ be an instance of real numbers. Then $x$ is irrational, if $x$ is not rational.

Definition. Let $x$ be a real number. Then $x$ is irrational, if $x$ is not rational.

Definition. Let $x \in R$. Then $x$ is irrational, if $x$ is not rational.

Thm01e. We can prove that the square root of $2$ is irrational.

Thm01e. The square root of $2$ is irrational.

Thm01e. $\sqrt{ 2}$ is irrational.

Axiom. Polynomials are a basic type.

Axiom. Let $P$ be a polynomial. Then the degree of $P$ is an instance of natural numbers.

Axiom. Let $P$ be a polynomial. Then the degree of $P$ is a natural number.

Axiom. Let $c$ be an instance of complex numbers. Let $P$ be a polynomial. Then we can say that $c$ is a root of $P$.

Axiom. Let $c$ be a complex number. Let $P$ be a polynomial. Then we can say that $c$ is a root of $P$.

Axiom. Let $c \in C$. Let $P$ be a polynomial. Then we can say that $c$ is a root of $P$.

Thm02. For all polynomials $P$, if we can prove that the degree of $P$ is greater than $0$, then we can prove that there exists a complex number $c$, such that $c$ is a root of $P$.

Thm02. For all polynomials $P$, if we can prove that the degree of $P$ is greater than $0$, then we can prove that $c$ is a root of $P$ for a complex number $c$.

Thm02. For all polynomials $P$, if we can prove that the degree of $P$ is greater than $0$, then we can prove that $c$ is a root of $P$ for some complex number $c$.

Thm02. Let $P$ be a polynomial. Then if we can prove that the degree of $P$ is greater than $0$, then we can prove that there exists a complex number $c$, such that $c$ is a root of $P$.

Thm02. Let $P$ be a polynomial. Then if we can prove that the degree of $P$ is greater than $0$, then we can prove that $c$ is a root of $P$ for a complex number $c$.

Thm02. Let $P$ be a polynomial. Then if we can prove that the degree of $P$ is greater than $0$, then we can prove that $c$ is a root of $P$ for some complex number $c$.

Thm02. Let $P$ be a polynomial. Assume that we can prove that the degree of $P$ is greater than $0$. Then we can prove that there exists a complex number $c$, such that $c$ is a root of $P$.

Thm02. Let $P$ be a polynomial. Assume that we can prove that the degree of $P$ is greater than $0$. Then we can prove that $c$ is a root of $P$ for a complex number $c$.

Thm02. Let $P$ be a polynomial. Assume that we can prove that the degree of $P$ is greater than $0$. Then we can prove that $c$ is a root of $P$ for some complex number $c$.

Thm02. For all polynomials $P$, if the degree of $P$ is greater than $0$, then there exists a complex number $c$, such that $c$ is a root of $P$.

Thm02. For all polynomials $P$, if the degree of $P$ is greater than $0$, then $c$ is a root of $P$ for a complex number $c$.

Thm02. For all polynomials $P$, if the degree of $P$ is greater than $0$, then $c$ is a root of $P$ for some complex number $c$.

Thm02. Let $P$ be a polynomial. Then if the degree of $P$ is greater than $0$, then there exists a complex number $c$, such that $c$ is a root of $P$.

Thm02. Let $P$ be a polynomial. Then if the degree of $P$ is greater than $0$, then $c$ is a root of $P$ for a complex number $c$.

Thm02. Let $P$ be a polynomial. Then if the degree of $P$ is greater than $0$, then $c$ is a root of $P$ for some complex number $c$.

Thm02. Let $P$ be a polynomial. Assume that the degree of $P$ is greater than $0$. Then there exists a complex number $c$, such that $c$ is a root of $P$.

Thm02. Let $P$ be a polynomial. Assume that the degree of $P$ is greater than $0$. Then $c$ is a root of $P$ for a complex number $c$.

Thm02. Let $P$ be a polynomial. Assume that the degree of $P$ is greater than $0$. Then $c$ is a root of $P$ for some complex number $c$.

Axiom. Let $A$ be a set. Then the cardinality of $A$ is an instance of real numbers.

Axiom. Let $A$ be a set. Then the cardinality of $A$ is a real number.

Definition. Let $A$ be a set. Then $A$ is denumerable, if the cardinality of $A$ is equal to the cardinality of $Nat$.

Thm03. We can prove that $Rat$ is denumerable.

Thm03. $Rat$ is denumerable.

Thm03a. We can prove that the cardinality of $Nat$ is equal to the cardinality of $Rat$.

Thm03a. The cardinality of $Nat$ is equal to the cardinality of $Rat$.

Axiom. Vectors are a basic type.

Axiom. Let $x$ be an instance of vectors. Then the length of $x$ is an instance of real numbers.

Axiom. Let $x$ be a vector. Then the length of $x$ is a real number.

Axiom. Let $x$ be a vector. Then $\| x \|$ is a real number.

Axiom. Let $x$ and $y$ be instances of vectors. Then we can say that $x$ is perpendicular to $y$.

Axiom. Let $x$ and $y$ be vectors. Then we can say that $x$ is perpendicular to $y$.

Axiom. Let $x$ and $y$ be vectors. Then we can say that $x \perp y$.

Axiom. Let $x$ and $y$ be instances of vectors. Then the sum of $x$ and $y$ is an instance of vectors.

Axiom. Let $x$ and $y$ be vectors. Then the sum of $x$ and $y$ is a vector.

Axiom. Let $x$ and $y$ be vectors. Then $x + y$ is a vector.

Definition. Let $x$ be an instance of real numbers. Then the square of $x$ is an instance of real numbers defined as the function that maps $x$ to the exponentiation of $x$ and $2$.

Definition. Let $x$ be a real number. Then the square of $x$ is a real number defined as the function that maps $x$ to the exponentiation of $x$ and $2$.

Definition. Let $x \in R$. Then the square of $x$ is a real number defined as the function that maps $x$ to the exponentiation of $x$ and $2$.

Definition. Let $x$ be a real number. Then $x ^{ 2}$ is a real number defined as the function that maps $x$ to $x ^ {2}$.

Definition. Let $x \in R$. Then $x ^{ 2}$ is a real number defined as the function that maps $x$ to $x ^ {2}$.

Thm04. For all instances $u$ and $v$ of vectors, if we can prove that $u$ is perpendicular to $v$, then we can prove that the length of the sum of $u$ and $v$ is equal to the square root of the sum of the square of the length of $u$ and the square of the length of $v$.

Thm04. Let $u$ and $v$ be instances of vectors. Then if we can prove that $u$ is perpendicular to $v$, then we can prove that the length of the sum of $u$ and $v$ is equal to the square root of the sum of the square of the length of $u$ and the square of the length of $v$.

Thm04. Let $u$ and $v$ be instances of vectors. Assume that we can prove that $u$ is perpendicular to $v$. Then we can prove that the length of the sum of $u$ and $v$ is equal to the square root of the sum of the square of the length of $u$ and the square of the length of $v$.

Thm04. For all vectors $u$ and $v$, if $u$ is perpendicular to $v$, then the length of the sum of $u$ and $v$ is equal to the square root of the sum of the square of the length of $u$ and the square of the length of $v$.

Thm04. Let $u$ and $v$ be vectors. Then if $u$ is perpendicular to $v$, then the length of the sum of $u$ and $v$ is equal to the square root of the sum of the square of the length of $u$ and the square of the length of $v$.

Thm04. Let $u$ and $v$ be vectors. Assume that $u$ is perpendicular to $v$. Then the length of the sum of $u$ and $v$ is equal to the square root of the sum of the square of the length of $u$ and the square of the length of $v$.

Thm04. For all vectors $u$ and $v$, if $u \perp v$, then $\| u + v \| = \sqrt{ \| u \| ^{ 2}+ \| v \| ^{ 2}}$.

Thm04. Let $u$ and $v$ be vectors. Then if $u \perp v$, then $\| u + v \| = \sqrt{ \| u \| ^{ 2}+ \| v \| ^{ 2}}$.

Thm04. Let $u$ and $v$ be vectors. Assume that $u \perp v$. Then $\| u + v \| = \sqrt{ \| u \| ^{ 2}+ \| v \| ^{ 2}}$.

Axiom. Let $x$ and $y$ be instances of natural numbers. Then the Legendre symbol of $x$ and $y$ is an instance of integers.

Axiom. Let $x$ and $y$ be natural numbers. Then the Legendre symbol of $x$ and $y$ is an integer.

Axiom. Let $x , y \in N$. Then the Legendre symbol of $x$ and $y$ is an integer.

Axiom. Let $x$ and $y$ be natural numbers. Then $\left(\frac{ x }{ y }\right)$ is an integer.

Axiom. Let $x , y \in N$. Then $\left(\frac{ x }{ y }\right)$ is an integer.

Thm07. For all instances $p$ and $q$ of natural numbers, if we can prove that $p$ is prime and $q$ is prime, then we can prove that the product of the Legendre symbol of $p$ and $q$ and the Legendre symbol of $q$ and $p$ is equal to the exponentiation of the negation of $1$ and the product of the quotient of the difference of $p$ and $1$ and $2$ and the quotient of the difference of $q$ and $1$ and $2$.

Thm07. Let $p$ and $q$ be instances of natural numbers. Then if we can prove that $p$ is prime and $q$ is prime, then we can prove that the product of the Legendre symbol of $p$ and $q$ and the Legendre symbol of $q$ and $p$ is equal to the exponentiation of the negation of $1$ and the product of the quotient of the difference of $p$ and $1$ and $2$ and the quotient of the difference of $q$ and $1$ and $2$.

Thm07. Let $p$ and $q$ be instances of natural numbers. Assume that we can prove that $p$ is prime and $q$ is prime. Then we can prove that the product of the Legendre symbol of $p$ and $q$ and the Legendre symbol of $q$ and $p$ is equal to the exponentiation of the negation of $1$ and the product of the quotient of the difference of $p$ and $1$ and $2$ and the quotient of the difference of $q$ and $1$ and $2$.

Thm07. For all natural numbers $p$ and $q$, if $p$ is prime and $q$ is prime, then the product of the Legendre symbol of $p$ and $q$ and the Legendre symbol of $q$ and $p$ is equal to the exponentiation of the negation of $1$ and the product of the quotient of the difference of $p$ and $1$ and $2$ and the quotient of the difference of $q$ and $1$ and $2$.

Thm07. Let $p$ and $q$ be natural numbers. Then if $p$ is prime and $q$ is prime, then the product of the Legendre symbol of $p$ and $q$ and the Legendre symbol of $q$ and $p$ is equal to the exponentiation of the negation of $1$ and the product of the quotient of the difference of $p$ and $1$ and $2$ and the quotient of the difference of $q$ and $1$ and $2$.

Thm07. Let $p , q \in N$. Then if $p$ is prime and $q$ is prime, then the product of the Legendre symbol of $p$ and $q$ and the Legendre symbol of $q$ and $p$ is equal to the exponentiation of the negation of $1$ and the product of the quotient of the difference of $p$ and $1$ and $2$ and the quotient of the difference of $q$ and $1$ and $2$.

Thm07. Let $p$ and $q$ be natural numbers. Assume that $p$ is prime and $q$ is prime. Then the product of the Legendre symbol of $p$ and $q$ and the Legendre symbol of $q$ and $p$ is equal to the exponentiation of the negation of $1$ and the product of the quotient of the difference of $p$ and $1$ and $2$ and the quotient of the difference of $q$ and $1$ and $2$.

Thm07. Let $p , q \in N$. Assume that $p$ is prime and $q$ is prime. Then the product of the Legendre symbol of $p$ and $q$ and the Legendre symbol of $q$ and $p$ is equal to the exponentiation of the negation of $1$ and the product of the quotient of the difference of $p$ and $1$ and $2$ and the quotient of the difference of $q$ and $1$ and $2$.

Thm07. For all natural numbers $p$ and $q$, if $p$ is prime and $q$ is prime, then $\left(\frac{ p }{ q }\right) \left(\frac{ q }{ p }\right) = (- 1)^ {\frac{ p - 1}{2}\frac{ q - 1}{2}}$.

Thm07. Let $p$ and $q$ be natural numbers. Then if $p$ is prime and $q$ is prime, then $\left(\frac{ p }{ q }\right) \left(\frac{ q }{ p }\right) = (- 1)^ {\frac{ p - 1}{2}\frac{ q - 1}{2}}$.

Thm07. Let $p , q \in N$. Then if $p$ is prime and $q$ is prime, then $\left(\frac{ p }{ q }\right) \left(\frac{ q }{ p }\right) = (- 1)^ {\frac{ p - 1}{2}\frac{ q - 1}{2}}$.

Thm07. Let $p$ and $q$ be natural numbers. Assume that $p$ is prime and $q$ is prime. Then $\left(\frac{ p }{ q }\right) \left(\frac{ q }{ p }\right) = (- 1)^ {\frac{ p - 1}{2}\frac{ q - 1}{2}}$.

Thm07. Let $p , q \in N$. Assume that $p$ is prime and $q$ is prime. Then $\left(\frac{ p }{ q }\right) \left(\frac{ q }{ p }\right) = (- 1)^ {\frac{ p - 1}{2}\frac{ q - 1}{2}}$.

Thm07. For all natural numbers $p$ and $q$, if $p$ and $q$ are prime, then $\left(\frac{ p }{ q }\right) \left(\frac{ q }{ p }\right) = (- 1)^ {\frac{ p - 1}{2}\frac{ q - 1}{2}}$.

Thm07. Let $p$ and $q$ be natural numbers. Then if $p$ and $q$ are prime, then $\left(\frac{ p }{ q }\right) \left(\frac{ q }{ p }\right) = (- 1)^ {\frac{ p - 1}{2}\frac{ q - 1}{2}}$.

Thm07. Let $p , q \in N$. Then if $p$ and $q$ are prime, then $\left(\frac{ p }{ q }\right) \left(\frac{ q }{ p }\right) = (- 1)^ {\frac{ p - 1}{2}\frac{ q - 1}{2}}$.

Thm07. Let $p$ and $q$ be natural numbers. Assume that $p$ and $q$ are prime. Then $\left(\frac{ p }{ q }\right) \left(\frac{ q }{ p }\right) = (- 1)^ {\frac{ p - 1}{2}\frac{ q - 1}{2}}$.

Thm07. Let $p , q \in N$. Assume that $p$ and $q$ are prime. Then $\left(\frac{ p }{ q }\right) \left(\frac{ q }{ p }\right) = (- 1)^ {\frac{ p - 1}{2}\frac{ q - 1}{2}}$.

Axiom. The number \(\pi\) is an instance of real numbers.

Axiom. The number \(\pi\) is a real number.

Axiom. $\pi$ is a real number.

Axiom. Circles are a basic type.

Axiom. Let $x$ be a circle. Then the radius of $x$ is an instance of real numbers.

Axiom. Let $x$ be a circle. Then the radius of $x$ is a real number.

Axiom. Let $x$ be a circle. Then the area of $x$ is an instance of real numbers.

Axiom. Let $x$ be a circle. Then the area of $x$ is a real number.

Thm09. For all circles $c$, for all instances $r$ of real numbers, if we can prove that $r$ is equal to the radius of $c$, then we can prove that the area of $c$ is equal to the product of the number \(\pi\) and the exponentiation of $r$ and $2$.

Thm09. Let $c$ be a circle. Then for all instances $r$ of real numbers, if we can prove that $r$ is equal to the radius of $c$, then we can prove that the area of $c$ is equal to the product of the number \(\pi\) and the exponentiation of $r$ and $2$.

Thm09. Let $c$ be a circle. Let $r$ be an instance of real numbers. Then if we can prove that $r$ is equal to the radius of $c$, then we can prove that the area of $c$ is equal to the product of the number \(\pi\) and the exponentiation of $r$ and $2$.

Thm09. Let $c$ be a circle. Let $r$ be an instance of real numbers. Assume that we can prove that $r$ is equal to the radius of $c$. Then we can prove that the area of $c$ is equal to the product of the number \(\pi\) and the exponentiation of $r$ and $2$.

Thm09. For all circles $c$, for all real numbers $r$, if $r$ is equal to the radius of $c$, then the area of $c$ is equal to the product of the number \(\pi\) and the exponentiation of $r$ and $2$.

Thm09. Let $c$ be a circle. Then for all real numbers $r$, if $r$ is equal to the radius of $c$, then the area of $c$ is equal to the product of the number \(\pi\) and the exponentiation of $r$ and $2$.

Thm09. Let $c$ be a circle. Let $r$ be a real number. Then if $r$ is equal to the radius of $c$, then the area of $c$ is equal to the product of the number \(\pi\) and the exponentiation of $r$ and $2$.

Thm09. Let $c$ be a circle. Let $r \in R$. Then if $r$ is equal to the radius of $c$, then the area of $c$ is equal to the product of the number \(\pi\) and the exponentiation of $r$ and $2$.

Thm09. Let $c$ be a circle. Let $r$ be a real number. Assume that $r$ is equal to the radius of $c$. Then the area of $c$ is equal to the product of the number \(\pi\) and the exponentiation of $r$ and $2$.

Thm09. Let $c$ be a circle. Let $r \in R$. Assume that $r$ is equal to the radius of $c$. Then the area of $c$ is equal to the product of the number \(\pi\) and the exponentiation of $r$ and $2$.

Thm09. For all circles $c$, for all real numbers $r$, if $r$ is equal to the radius of $c$, then the area of $c$ is equal to $\pi r ^ {2}$.

Thm09. Let $c$ be a circle. Then for all real numbers $r$, if $r$ is equal to the radius of $c$, then the area of $c$ is equal to $\pi r ^ {2}$.

Thm09. Let $c$ be a circle. Let $r$ be a real number. Then if $r$ is equal to the radius of $c$, then the area of $c$ is equal to $\pi r ^ {2}$.

Thm09. Let $c$ be a circle. Let $r \in R$. Then if $r$ is equal to the radius of $c$, then the area of $c$ is equal to $\pi r ^ {2}$.

Thm09. Let $c$ be a circle. Let $r$ be a real number. Assume that $r$ is equal to the radius of $c$. Then the area of $c$ is equal to $\pi r ^ {2}$.

Thm09. Let $c$ be a circle. Let $r \in R$. Assume that $r$ is equal to the radius of $c$. Then the area of $c$ is equal to $\pi r ^ {2}$.

Thm10FermatLittle. For all instances $p$ of natural numbers, if we can prove that $p$ is prime, then for all instances $a$ of integers, we can prove that there exists an integer $q$, such that the difference of the exponentiation of $a$ and $p$ and $a$ is equal to the product of $p$ and $q$.

Thm10FermatLittle. For all instances $p$ of natural numbers, if we can prove that $p$ is prime, then for all instances $a$ of integers, we can prove that the difference of the exponentiation of $a$ and $p$ and $a$ is equal to the product of $p$ and $q$ for an integer $q$.

Thm10FermatLittle. For all instances $p$ of natural numbers, if we can prove that $p$ is prime, then for all instances $a$ of integers, we can prove that the difference of the exponentiation of $a$ and $p$ and $a$ is equal to the product of $p$ and $q$ for some integer $q$.

Thm10FermatLittle. Let $p$ be an instance of natural numbers. Then if we can prove that $p$ is prime, then for all instances $a$ of integers, we can prove that there exists an integer $q$, such that the difference of the exponentiation of $a$ and $p$ and $a$ is equal to the product of $p$ and $q$.

Thm10FermatLittle. Let $p$ be an instance of natural numbers. Then if we can prove that $p$ is prime, then for all instances $a$ of integers, we can prove that the difference of the exponentiation of $a$ and $p$ and $a$ is equal to the product of $p$ and $q$ for an integer $q$.

Thm10FermatLittle. Let $p$ be an instance of natural numbers. Then if we can prove that $p$ is prime, then for all instances $a$ of integers, we can prove that the difference of the exponentiation of $a$ and $p$ and $a$ is equal to the product of $p$ and $q$ for some integer $q$.

Thm10FermatLittle. Let $p$ be an instance of natural numbers. Assume that we can prove that $p$ is prime. Then for all instances $a$ of integers, we can prove that there exists an integer $q$, such that the difference of the exponentiation of $a$ and $p$ and $a$ is equal to the product of $p$ and $q$.

Thm10FermatLittle. Let $p$ be an instance of natural numbers. Assume that we can prove that $p$ is prime. Then for all instances $a$ of integers, we can prove that the difference of the exponentiation of $a$ and $p$ and $a$ is equal to the product of $p$ and $q$ for an integer $q$.

Thm10FermatLittle. Let $p$ be an instance of natural numbers. Assume that we can prove that $p$ is prime. Then for all instances $a$ of integers, we can prove that the difference of the exponentiation of $a$ and $p$ and $a$ is equal to the product of $p$ and $q$ for some integer $q$.

Thm10FermatLittle. Let $p$ be an instance of natural numbers. Assume that we can prove that $p$ is prime. Let $a$ be an instance of integers. Then we can prove that there exists an integer $q$, such that the difference of the exponentiation of $a$ and $p$ and $a$ is equal to the product of $p$ and $q$.

Thm10FermatLittle. Let $p$ be an instance of natural numbers. Assume that we can prove that $p$ is prime. Let $a$ be an instance of integers. Then we can prove that the difference of the exponentiation of $a$ and $p$ and $a$ is equal to the product of $p$ and $q$ for an integer $q$.

Thm10FermatLittle. Let $p$ be an instance of natural numbers. Assume that we can prove that $p$ is prime. Let $a$ be an instance of integers. Then we can prove that the difference of the exponentiation of $a$ and $p$ and $a$ is equal to the product of $p$ and $q$ for some integer $q$.

Thm10FermatLittle. For all natural numbers $p$, if $p$ is prime, then for all integers $a$, there exists an integer $q$, such that the difference of the exponentiation of $a$ and $p$ and $a$ is equal to the product of $p$ and $q$.

Thm10FermatLittle. For all natural numbers $p$, if $p$ is prime, then for all integers $a$, the difference of the exponentiation of $a$ and $p$ and $a$ is equal to the product of $p$ and $q$ for an integer $q$.

Thm10FermatLittle. For all natural numbers $p$, if $p$ is prime, then for all integers $a$, the difference of the exponentiation of $a$ and $p$ and $a$ is equal to the product of $p$ and $q$ for some integer $q$.

Thm10FermatLittle. Let $p$ be a natural number. Then if $p$ is prime, then for all integers $a$, there exists an integer $q$, such that the difference of the exponentiation of $a$ and $p$ and $a$ is equal to the product of $p$ and $q$.

Thm10FermatLittle. Let $p$ be a natural number. Then if $p$ is prime, then for all integers $a$, the difference of the exponentiation of $a$ and $p$ and $a$ is equal to the product of $p$ and $q$ for an integer $q$.

Thm10FermatLittle. Let $p$ be a natural number. Then if $p$ is prime, then for all integers $a$, the difference of the exponentiation of $a$ and $p$ and $a$ is equal to the product of $p$ and $q$ for some integer $q$.

Thm10FermatLittle. Let $p \in N$. Then if $p$ is prime, then for all integers $a$, there exists an integer $q$, such that the difference of the exponentiation of $a$ and $p$ and $a$ is equal to the product of $p$ and $q$.

Thm10FermatLittle. Let $p \in N$. Then if $p$ is prime, then for all integers $a$, the difference of the exponentiation of $a$ and $p$ and $a$ is equal to the product of $p$ and $q$ for an integer $q$.

Thm10FermatLittle. Let $p \in N$. Then if $p$ is prime, then for all integers $a$, the difference of the exponentiation of $a$ and $p$ and $a$ is equal to the product of $p$ and $q$ for some integer $q$.

Thm10FermatLittle. Let $p$ be a natural number. Assume that $p$ is prime. Then for all integers $a$, there exists an integer $q$, such that the difference of the exponentiation of $a$ and $p$ and $a$ is equal to the product of $p$ and $q$.

Thm10FermatLittle. Let $p$ be a natural number. Assume that $p$ is prime. Then for all integers $a$, the difference of the exponentiation of $a$ and $p$ and $a$ is equal to the product of $p$ and $q$ for an integer $q$.

Thm10FermatLittle. Let $p$ be a natural number. Assume that $p$ is prime. Then for all integers $a$, the difference of the exponentiation of $a$ and $p$ and $a$ is equal to the product of $p$ and $q$ for some integer $q$.

Thm10FermatLittle. Let $p \in N$. Assume that $p$ is prime. Then for all integers $a$, there exists an integer $q$, such that the difference of the exponentiation of $a$ and $p$ and $a$ is equal to the product of $p$ and $q$.

Thm10FermatLittle. Let $p \in N$. Assume that $p$ is prime. Then for all integers $a$, the difference of the exponentiation of $a$ and $p$ and $a$ is equal to the product of $p$ and $q$ for an integer $q$.

Thm10FermatLittle. Let $p \in N$. Assume that $p$ is prime. Then for all integers $a$, the difference of the exponentiation of $a$ and $p$ and $a$ is equal to the product of $p$ and $q$ for some integer $q$.

Thm10FermatLittle. Let $p$ be a natural number. Assume that $p$ is prime. Let $a$ be an integer. Then there exists an integer $q$, such that the difference of the exponentiation of $a$ and $p$ and $a$ is equal to the product of $p$ and $q$.

Thm10FermatLittle. Let $p$ be a natural number. Assume that $p$ is prime. Let $a$ be an integer. Then the difference of the exponentiation of $a$ and $p$ and $a$ is equal to the product of $p$ and $q$ for an integer $q$.

Thm10FermatLittle. Let $p$ be a natural number. Assume that $p$ is prime. Let $a$ be an integer. Then the difference of the exponentiation of $a$ and $p$ and $a$ is equal to the product of $p$ and $q$ for some integer $q$.

Thm10FermatLittle. Let $p$ be a natural number. Assume that $p$ is prime. Let $a \in Z$. Then there exists an integer $q$, such that the difference of the exponentiation of $a$ and $p$ and $a$ is equal to the product of $p$ and $q$.

Thm10FermatLittle. Let $p$ be a natural number. Assume that $p$ is prime. Let $a \in Z$. Then the difference of the exponentiation of $a$ and $p$ and $a$ is equal to the product of $p$ and $q$ for an integer $q$.

Thm10FermatLittle. Let $p$ be a natural number. Assume that $p$ is prime. Let $a \in Z$. Then the difference of the exponentiation of $a$ and $p$ and $a$ is equal to the product of $p$ and $q$ for some integer $q$.

Thm10FermatLittle. Let $p \in N$. Assume that $p$ is prime. Let $a$ be an integer. Then there exists an integer $q$, such that the difference of the exponentiation of $a$ and $p$ and $a$ is equal to the product of $p$ and $q$.

Thm10FermatLittle. Let $p \in N$. Assume that $p$ is prime. Let $a$ be an integer. Then the difference of the exponentiation of $a$ and $p$ and $a$ is equal to the product of $p$ and $q$ for an integer $q$.

Thm10FermatLittle. Let $p \in N$. Assume that $p$ is prime. Let $a$ be an integer. Then the difference of the exponentiation of $a$ and $p$ and $a$ is equal to the product of $p$ and $q$ for some integer $q$.

Thm10FermatLittle. Let $p \in N$. Assume that $p$ is prime. Let $a \in Z$. Then there exists an integer $q$, such that the difference of the exponentiation of $a$ and $p$ and $a$ is equal to the product of $p$ and $q$.

Thm10FermatLittle. Let $p \in N$. Assume that $p$ is prime. Let $a \in Z$. Then the difference of the exponentiation of $a$ and $p$ and $a$ is equal to the product of $p$ and $q$ for an integer $q$.

Thm10FermatLittle. Let $p \in N$. Assume that $p$ is prime. Let $a \in Z$. Then the difference of the exponentiation of $a$ and $p$ and $a$ is equal to the product of $p$ and $q$ for some integer $q$.

Thm10FermatLittle. For all natural numbers $p$, if $p$ is prime, then for all integers $a$, there exists an integer $q$, such that $a ^ {p}- a = p q$.

Thm10FermatLittle. For all natural numbers $p$, if $p$ is prime, then for all integers $a$, $a ^ {p}- a = p q$ for an integer $q$.

Thm10FermatLittle. For all natural numbers $p$, if $p$ is prime, then for all integers $a$, $a ^ {p}- a = p q$ for some integer $q$.

Thm10FermatLittle. Let $p$ be a natural number. Then if $p$ is prime, then for all integers $a$, there exists an integer $q$, such that $a ^ {p}- a = p q$.

Thm10FermatLittle. Let $p$ be a natural number. Then if $p$ is prime, then for all integers $a$, $a ^ {p}- a = p q$ for an integer $q$.

Thm10FermatLittle. Let $p$ be a natural number. Then if $p$ is prime, then for all integers $a$, $a ^ {p}- a = p q$ for some integer $q$.

Thm10FermatLittle. Let $p \in N$. Then if $p$ is prime, then for all integers $a$, there exists an integer $q$, such that $a ^ {p}- a = p q$.

Thm10FermatLittle. Let $p \in N$. Then if $p$ is prime, then for all integers $a$, $a ^ {p}- a = p q$ for an integer $q$.

Thm10FermatLittle. Let $p \in N$. Then if $p$ is prime, then for all integers $a$, $a ^ {p}- a = p q$ for some integer $q$.

Thm10FermatLittle. Let $p$ be a natural number. Assume that $p$ is prime. Then for all integers $a$, there exists an integer $q$, such that $a ^ {p}- a = p q$.

Thm10FermatLittle. Let $p$ be a natural number. Assume that $p$ is prime. Then for all integers $a$, $a ^ {p}- a = p q$ for an integer $q$.

Thm10FermatLittle. Let $p$ be a natural number. Assume that $p$ is prime. Then for all integers $a$, $a ^ {p}- a = p q$ for some integer $q$.

Thm10FermatLittle. Let $p \in N$. Assume that $p$ is prime. Then for all integers $a$, there exists an integer $q$, such that $a ^ {p}- a = p q$.

Thm10FermatLittle. Let $p \in N$. Assume that $p$ is prime. Then for all integers $a$, $a ^ {p}- a = p q$ for an integer $q$.

Thm10FermatLittle. Let $p \in N$. Assume that $p$ is prime. Then for all integers $a$, $a ^ {p}- a = p q$ for some integer $q$.

Thm10FermatLittle. Let $p$ be a natural number. Assume that $p$ is prime. Let $a$ be an integer. Then there exists an integer $q$, such that $a ^ {p}- a = p q$.

Thm10FermatLittle. Let $p$ be a natural number. Assume that $p$ is prime. Let $a$ be an integer. Then $a ^ {p}- a = p q$ for an integer $q$.

Thm10FermatLittle. Let $p$ be a natural number. Assume that $p$ is prime. Let $a$ be an integer. Then $a ^ {p}- a = p q$ for some integer $q$.

Thm10FermatLittle. Let $p$ be a natural number. Assume that $p$ is prime. Let $a \in Z$. Then there exists an integer $q$, such that $a ^ {p}- a = p q$.

Thm10FermatLittle. Let $p$ be a natural number. Assume that $p$ is prime. Let $a \in Z$. Then $a ^ {p}- a = p q$ for an integer $q$.

Thm10FermatLittle. Let $p$ be a natural number. Assume that $p$ is prime. Let $a \in Z$. Then $a ^ {p}- a = p q$ for some integer $q$.

Thm10FermatLittle. Let $p \in N$. Assume that $p$ is prime. Let $a$ be an integer. Then there exists an integer $q$, such that $a ^ {p}- a = p q$.

Thm10FermatLittle. Let $p \in N$. Assume that $p$ is prime. Let $a$ be an integer. Then $a ^ {p}- a = p q$ for an integer $q$.

Thm10FermatLittle. Let $p \in N$. Assume that $p$ is prime. Let $a$ be an integer. Then $a ^ {p}- a = p q$ for some integer $q$.

Thm10FermatLittle. Let $p \in N$. Assume that $p$ is prime. Let $a \in Z$. Then there exists an integer $q$, such that $a ^ {p}- a = p q$.

Thm10FermatLittle. Let $p \in N$. Assume that $p$ is prime. Let $a \in Z$. Then $a ^ {p}- a = p q$ for an integer $q$.

Thm10FermatLittle. Let $p \in N$. Assume that $p$ is prime. Let $a \in Z$. Then $a ^ {p}- a = p q$ for some integer $q$.

Thm11. For all instances $n$ of natural numbers, we can prove that there exists a natural number $p$, such that $p$ is greater than or equal to $n$ and $p$ is prime.

Thm11. For all instances $n$ of natural numbers, we can prove that $p$ is greater than or equal to $n$ and $p$ is prime for a natural number $p$.

Thm11. For all instances $n$ of natural numbers, we can prove that $p$ is greater than or equal to $n$ and $p$ is prime for some natural number $p$.

Thm11. Let $n$ be an instance of natural numbers. Then we can prove that there exists a natural number $p$, such that $p$ is greater than or equal to $n$ and $p$ is prime.

Thm11. Let $n$ be an instance of natural numbers. Then we can prove that $p$ is greater than or equal to $n$ and $p$ is prime for a natural number $p$.

Thm11. Let $n$ be an instance of natural numbers. Then we can prove that $p$ is greater than or equal to $n$ and $p$ is prime for some natural number $p$.

Thm11. For all natural numbers $n$, there exists a natural number $p$, such that $p$ is greater than or equal to $n$ and $p$ is prime.

Thm11. For all natural numbers $n$, $p$ is greater than or equal to $n$ and $p$ is prime for a natural number $p$.

Thm11. For all natural numbers $n$, $p$ is greater than or equal to $n$ and $p$ is prime for some natural number $p$.

Thm11. Let $n$ be a natural number. Then there exists a natural number $p$, such that $p$ is greater than or equal to $n$ and $p$ is prime.

Thm11. Let $n$ be a natural number. Then $p$ is greater than or equal to $n$ and $p$ is prime for a natural number $p$.

Thm11. Let $n$ be a natural number. Then $p$ is greater than or equal to $n$ and $p$ is prime for some natural number $p$.

Thm11. Let $n \in N$. Then there exists a natural number $p$, such that $p$ is greater than or equal to $n$ and $p$ is prime.

Thm11. Let $n \in N$. Then $p$ is greater than or equal to $n$ and $p$ is prime for a natural number $p$.

Thm11. Let $n \in N$. Then $p$ is greater than or equal to $n$ and $p$ is prime for some natural number $p$.

Thm11. For all natural numbers $n$, there exists a natural number $p$, such that $p \geq n$ and $p$ is prime.

Thm11. For all natural numbers $n$, $p \geq n$ and $p$ is prime for a natural number $p$.

Thm11. For all natural numbers $n$, $p \geq n$ and $p$ is prime for some natural number $p$.

Thm11. Let $n$ be a natural number. Then there exists a natural number $p$, such that $p \geq n$ and $p$ is prime.

Thm11. Let $n$ be a natural number. Then $p \geq n$ and $p$ is prime for a natural number $p$.

Thm11. Let $n$ be a natural number. Then $p \geq n$ and $p$ is prime for some natural number $p$.

Thm11. Let $n \in N$. Then there exists a natural number $p$, such that $p \geq n$ and $p$ is prime.

Thm11. Let $n \in N$. Then $p \geq n$ and $p$ is prime for a natural number $p$.

Thm11. Let $n \in N$. Then $p \geq n$ and $p$ is prime for some natural number $p$.

Thm19. For all instances $n$ of natural numbers, we can prove that there exists a natural number $a$, such that there exists a natural number $b$, such that there exists a natural number $c$, such that there exists a natural number $d$, such that $n$ is equal to the sum of the sum of the sum of the square of $a$ and the square of $b$ and the square of $c$ and the square of $d$.

Thm19. For all instances $n$ of natural numbers, we can prove that there exists a natural number $b$, such that there exists a natural number $c$, such that there exists a natural number $d$, such that $n$ is equal to the sum of the sum of the sum of the square of $a$ and the square of $b$ and the square of $c$ and the square of $d$ for a natural number $a$.

Thm19. For all instances $n$ of natural numbers, we can prove that there exists a natural number $b$, such that there exists a natural number $c$, such that there exists a natural number $d$, such that $n$ is equal to the sum of the sum of the sum of the square of $a$ and the square of $b$ and the square of $c$ and the square of $d$ for some natural number $a$.

Thm19. Let $n$ be an instance of natural numbers. Then we can prove that there exists a natural number $a$, such that there exists a natural number $b$, such that there exists a natural number $c$, such that there exists a natural number $d$, such that $n$ is equal to the sum of the sum of the sum of the square of $a$ and the square of $b$ and the square of $c$ and the square of $d$.

Thm19. Let $n$ be an instance of natural numbers. Then we can prove that there exists a natural number $b$, such that there exists a natural number $c$, such that there exists a natural number $d$, such that $n$ is equal to the sum of the sum of the sum of the square of $a$ and the square of $b$ and the square of $c$ and the square of $d$ for a natural number $a$.

Thm19. Let $n$ be an instance of natural numbers. Then we can prove that there exists a natural number $b$, such that there exists a natural number $c$, such that there exists a natural number $d$, such that $n$ is equal to the sum of the sum of the sum of the square of $a$ and the square of $b$ and the square of $c$ and the square of $d$ for some natural number $a$.

Thm19. For all natural numbers $n$, there exists a natural number $a$, such that there exists a natural number $b$, such that there exists a natural number $c$, such that there exists a natural number $d$, such that $n$ is equal to the sum of the sum of the sum of the square of $a$ and the square of $b$ and the square of $c$ and the square of $d$.

Thm19. For all natural numbers $n$, there exists a natural number $b$, such that there exists a natural number $c$, such that there exists a natural number $d$, such that $n$ is equal to the sum of the sum of the sum of the square of $a$ and the square of $b$ and the square of $c$ and the square of $d$ for a natural number $a$.

Thm19. For all natural numbers $n$, there exists a natural number $b$, such that there exists a natural number $c$, such that there exists a natural number $d$, such that $n$ is equal to the sum of the sum of the sum of the square of $a$ and the square of $b$ and the square of $c$ and the square of $d$ for some natural number $a$.

Thm19. Let $n$ be a natural number. Then there exists a natural number $a$, such that there exists a natural number $b$, such that there exists a natural number $c$, such that there exists a natural number $d$, such that $n$ is equal to the sum of the sum of the sum of the square of $a$ and the square of $b$ and the square of $c$ and the square of $d$.

Thm19. Let $n$ be a natural number. Then there exists a natural number $b$, such that there exists a natural number $c$, such that there exists a natural number $d$, such that $n$ is equal to the sum of the sum of the sum of the square of $a$ and the square of $b$ and the square of $c$ and the square of $d$ for a natural number $a$.

Thm19. Let $n$ be a natural number. Then there exists a natural number $b$, such that there exists a natural number $c$, such that there exists a natural number $d$, such that $n$ is equal to the sum of the sum of the sum of the square of $a$ and the square of $b$ and the square of $c$ and the square of $d$ for some natural number $a$.

Thm19. Let $n \in N$. Then there exists a natural number $a$, such that there exists a natural number $b$, such that there exists a natural number $c$, such that there exists a natural number $d$, such that $n$ is equal to the sum of the sum of the sum of the square of $a$ and the square of $b$ and the square of $c$ and the square of $d$.

Thm19. Let $n \in N$. Then there exists a natural number $b$, such that there exists a natural number $c$, such that there exists a natural number $d$, such that $n$ is equal to the sum of the sum of the sum of the square of $a$ and the square of $b$ and the square of $c$ and the square of $d$ for a natural number $a$.

Thm19. Let $n \in N$. Then there exists a natural number $b$, such that there exists a natural number $c$, such that there exists a natural number $d$, such that $n$ is equal to the sum of the sum of the sum of the square of $a$ and the square of $b$ and the square of $c$ and the square of $d$ for some natural number $a$.

Thm19. For all natural numbers $n$, there exists a natural number $a$, such that there exists a natural number $b$, such that there exists a natural number $c$, such that there exists a natural number $d$, such that $n = a ^{ 2}+ b ^{ 2}+ c ^{ 2}+ d ^{ 2}$.

Thm19. For all natural numbers $n$, there exists a natural number $b$, such that there exists a natural number $c$, such that there exists a natural number $d$, such that $n = a ^{ 2}+ b ^{ 2}+ c ^{ 2}+ d ^{ 2}$ for a natural number $a$.

Thm19. For all natural numbers $n$, there exists a natural number $b$, such that there exists a natural number $c$, such that there exists a natural number $d$, such that $n = a ^{ 2}+ b ^{ 2}+ c ^{ 2}+ d ^{ 2}$ for some natural number $a$.

Thm19. Let $n$ be a natural number. Then there exists a natural number $a$, such that there exists a natural number $b$, such that there exists a natural number $c$, such that there exists a natural number $d$, such that $n = a ^{ 2}+ b ^{ 2}+ c ^{ 2}+ d ^{ 2}$.

Thm19. Let $n$ be a natural number. Then there exists a natural number $b$, such that there exists a natural number $c$, such that there exists a natural number $d$, such that $n = a ^{ 2}+ b ^{ 2}+ c ^{ 2}+ d ^{ 2}$ for a natural number $a$.

Thm19. Let $n$ be a natural number. Then there exists a natural number $b$, such that there exists a natural number $c$, such that there exists a natural number $d$, such that $n = a ^{ 2}+ b ^{ 2}+ c ^{ 2}+ d ^{ 2}$ for some natural number $a$.

Thm19. Let $n \in N$. Then there exists a natural number $a$, such that there exists a natural number $b$, such that there exists a natural number $c$, such that there exists a natural number $d$, such that $n = a ^{ 2}+ b ^{ 2}+ c ^{ 2}+ d ^{ 2}$.

Thm19. Let $n \in N$. Then there exists a natural number $b$, such that there exists a natural number $c$, such that there exists a natural number $d$, such that $n = a ^{ 2}+ b ^{ 2}+ c ^{ 2}+ d ^{ 2}$ for a natural number $a$.

Thm19. Let $n \in N$. Then there exists a natural number $b$, such that there exists a natural number $c$, such that there exists a natural number $d$, such that $n = a ^{ 2}+ b ^{ 2}+ c ^{ 2}+ d ^{ 2}$ for some natural number $a$.

Thm19. For all natural numbers $n$, there exist natural numbers $a$, $b$, $c$ and $d$, such that $n = a ^{ 2}+ b ^{ 2}+ c ^{ 2}+ d ^{ 2}$.

Thm19. For all natural numbers $n$, $n = a ^{ 2}+ b ^{ 2}+ c ^{ 2}+ d ^{ 2}$ for some natural numbers $a$, $b$, $c$ and $d$.

Thm19. Let $n$ be a natural number. Then there exist natural numbers $a$, $b$, $c$ and $d$, such that $n = a ^{ 2}+ b ^{ 2}+ c ^{ 2}+ d ^{ 2}$.

Thm19. Let $n$ be a natural number. Then $n = a ^{ 2}+ b ^{ 2}+ c ^{ 2}+ d ^{ 2}$ for some natural numbers $a$, $b$, $c$ and $d$.

Thm19. Let $n \in N$. Then there exist natural numbers $a$, $b$, $c$ and $d$, such that $n = a ^{ 2}+ b ^{ 2}+ c ^{ 2}+ d ^{ 2}$.

Thm19. Let $n \in N$. Then $n = a ^{ 2}+ b ^{ 2}+ c ^{ 2}+ d ^{ 2}$ for some natural numbers $a$, $b$, $c$ and $d$.

\end{document}
