\documentclass{article}
\usepackage{amsfonts}
\usepackage{amssymb}
\usepackage{amsmath}
\setlength\parindent{0pt}
\setlength\parskip{8pt}
\begin{document}

Thm01. Let $m$ and $n$ be instances of natural numbers. Assume that we can prove that $n$ is not equal to $0$. Then we can prove that the exponentiation of the quotient of $m$ and $n$ and $2$ is not equal to $2$.

Thm01. For all instances $m$ and $n$ of natural numbers, if we can prove that $n$ is not equal to $0$, then we can prove that the exponentiation of the quotient of $m$ and $n$ and $2$ is not equal to $2$.

Thm01. If we can prove that $n$ is not equal to $0$, then we can prove that the exponentiation of the quotient of $m$ and $n$ and $2$ is not equal to $2$ for all instances $m$ and $n$ of natural numbers.

Thm01. Let $m$ and $n$ be instances of natural numbers. Then if we can prove that $n$ is not equal to $0$, then we can prove that the exponentiation of the quotient of $m$ and $n$ and $2$ is not equal to $2$.

Thm01. Let $m$ and $n$ be instances of natural numbers. Then we can prove that $n$ is not equal to $0$, only if we can prove that the exponentiation of the quotient of $m$ and $n$ and $2$ is not equal to $2$.

Thm01. Let $m$ and $n$ be natural numbers. Assume that $n$ is not equal to $0$. Then the exponentiation of the quotient of $m$ and $n$ and $2$ is not equal to $2$.

Thm01. For all natural numbers $m$ and $n$, if $n$ is not equal to $0$, then the exponentiation of the quotient of $m$ and $n$ and $2$ is not equal to $2$.

Thm01. If $n$ is not equal to $0$, then the exponentiation of the quotient of $m$ and $n$ and $2$ is not equal to $2$ for all natural numbers $m$ and $n$.

Thm01. Let $m$ and $n$ be natural numbers. Then if $n$ is not equal to $0$, then the exponentiation of the quotient of $m$ and $n$ and $2$ is not equal to $2$.

Thm01. Let $m$ and $n$ be natural numbers. Then $n$ is not equal to $0$, only if the exponentiation of the quotient of $m$ and $n$ and $2$ is not equal to $2$.

Thm01. Let $m , n \in N$. Then if $n$ is not equal to $0$, then the exponentiation of the quotient of $m$ and $n$ and $2$ is not equal to $2$.

Thm01. Let $m , n \in N$. Then $n$ is not equal to $0$, only if the exponentiation of the quotient of $m$ and $n$ and $2$ is not equal to $2$.

Thm01. Let $m , n \in N$. Assume that $n$ is not equal to $0$. Then the exponentiation of the quotient of $m$ and $n$ and $2$ is not equal to $2$.

Thm01. Let $m$ and $n$ be natural numbers. Assume that $n \neq 0$. Then $(\frac{ m}{n})^ {2}\neq 2$.

Thm01. For all natural numbers $m$ and $n$, if $n \neq 0$, then $(\frac{ m}{n})^ {2}\neq 2$.

Thm01. If $n \neq 0$, then $(\frac{ m}{n})^ {2}\neq 2$ for all natural numbers $m$ and $n$.

Thm01. Let $m$ and $n$ be natural numbers. Then if $n \neq 0$, then $(\frac{ m}{n})^ {2}\neq 2$.

Thm01. Let $m$ and $n$ be natural numbers. Then $n \neq 0$, only if $(\frac{ m}{n})^ {2}\neq 2$.

Thm01. Let $m$ and $n$ be natural numbers. Then $n \neq 0$ implies $(\frac{ m}{n})^ {2}\neq 2$.

Thm01. Let $m , n \in N$. Then if $n \neq 0$, then $(\frac{ m}{n})^ {2}\neq 2$.

Thm01. Let $m , n \in N$. Then $n \neq 0$, only if $(\frac{ m}{n})^ {2}\neq 2$.

Thm01. Let $m , n \in N$. Then $n \neq 0$ implies $(\frac{ m}{n})^ {2}\neq 2$.

Thm01. Let $m , n \in N$. Assume that $n \neq 0$. Then $(\frac{ m}{n})^ {2}\neq 2$.

Thm01a. Let $m$ and $n$ be instances of natural numbers. Then we can prove that the exponentiation of the quotient of $m$ and the sum of $n$ and $1$ and $2$ is not equal to $2$.

Thm01a. For all instances $m$ and $n$ of natural numbers, we can prove that the exponentiation of the quotient of $m$ and the sum of $n$ and $1$ and $2$ is not equal to $2$.

Thm01a. We can prove that the exponentiation of the quotient of $m$ and the sum of $n$ and $1$ and $2$ is not equal to $2$ for all instances $m$ and $n$ of natural numbers.

Thm01a. Let $m$ and $n$ be natural numbers. Then the exponentiation of the quotient of $m$ and the sum of $n$ and $1$ and $2$ is not equal to $2$.

Thm01a. For all natural numbers $m$ and $n$, the exponentiation of the quotient of $m$ and the sum of $n$ and $1$ and $2$ is not equal to $2$.

Thm01a. The exponentiation of the quotient of $m$ and the sum of $n$ and $1$ and $2$ is not equal to $2$ for all natural numbers $m$ and $n$.

Thm01a. Let $m , n \in N$. Then the exponentiation of the quotient of $m$ and the sum of $n$ and $1$ and $2$ is not equal to $2$.

Thm01a. Let $m$ and $n$ be natural numbers. Then $(\frac{ m}{n + 1})^ {2}\neq 2$.

Thm01a. For all natural numbers $m$ and $n$, $(\frac{ m}{n + 1})^ {2}\neq 2$.

Thm01a. $(\frac{ m}{n + 1})^ {2}\neq 2$ for all natural numbers $m$ and $n$.

Thm01a. Let $m , n \in N$. Then $(\frac{ m}{n + 1})^ {2}\neq 2$.

Thm01b. Let $q$ be an instance of rational numbers. Then we can prove that the exponentiation of $q$ and $2$ is not equal to $2$.

Thm01b. For all instances $q$ of rational numbers, we can prove that the exponentiation of $q$ and $2$ is not equal to $2$.

Thm01b. We can prove that the exponentiation of $q$ and $2$ is not equal to $2$ for every instance $q$ of rational numbers.

Thm01b. We can prove that the exponentiation of $q$ and $2$ is not equal to $2$ for all instances $q$ of rational numbers.

Thm01b. Let $q$ be a rational number. Then the exponentiation of $q$ and $2$ is not equal to $2$.

Thm01b. For all rational numbers $q$, the exponentiation of $q$ and $2$ is not equal to $2$.

Thm01b. The exponentiation of $q$ and $2$ is not equal to $2$ for every rational number $q$.

Thm01b. The exponentiation of $q$ and $2$ is not equal to $2$ for all rational numbers $q$.

Thm01b. Let $q \in Q$. Then the exponentiation of $q$ and $2$ is not equal to $2$.

Thm01b. Let $q$ be a rational number. Then $q ^ {2}\neq 2$.

Thm01b. For all rational numbers $q$, $q ^ {2}\neq 2$.

Thm01b. $q ^ {2}\neq 2$ for every rational number $q$.

Thm01b. $q ^ {2}\neq 2$ for all rational numbers $q$.

Thm01b. Let $q \in Q$. Then $q ^ {2}\neq 2$.

Thm01c. Let $q$ be an instance of rational numbers. Then we can prove that the square root of $2$ is not equal to $q$.

Thm01c. For all instances $q$ of rational numbers, we can prove that the square root of $2$ is not equal to $q$.

Thm01c. We can prove that the square root of $2$ is not equal to $q$ for every instance $q$ of rational numbers.

Thm01c. We can prove that the square root of $2$ is not equal to $q$ for all instances $q$ of rational numbers.

Thm01c. Let $q$ be a rational number. Then the square root of $2$ is not equal to $q$.

Thm01c. For all rational numbers $q$, the square root of $2$ is not equal to $q$.

Thm01c. The square root of $2$ is not equal to $q$ for every rational number $q$.

Thm01c. The square root of $2$ is not equal to $q$ for all rational numbers $q$.

Thm01c. Let $q \in Q$. Then the square root of $2$ is not equal to $q$.

Thm01c. Let $q$ be a rational number. Then $\sqrt{ 2}\neq q$.

Thm01c. For all rational numbers $q$, $\sqrt{ 2}\neq q$.

Thm01c. $\sqrt{ 2}\neq q$ for every rational number $q$.

Thm01c. $\sqrt{ 2}\neq q$ for all rational numbers $q$.

Thm01c. Let $q \in Q$. Then $\sqrt{ 2}\neq q$.

Thm01d. We can prove that the square root of $2$ is not rational.

Thm01d. The square root of $2$ is not rational.

Thm01d. $\sqrt{ 2}$ is not rational.

Thm01e. We can prove that the square root of $2$ is irrational.

Thm01e. The square root of $2$ is irrational.

Thm01e. $\sqrt{ 2}$ is irrational.

Thm01f. We can prove that it is not the case that there exists a natural number $p$, such that there exists a natural number $q$, such that the square of $p$ is equal to the product of $2$ and the square of $q$.

Thm01f. We can prove that it is not the case that there exists a natural number $q$, such that the square of $p$ is equal to the product of $2$ and the square of $q$ for a natural number $p$.

Thm01f. We can prove that it is not the case that there exists a natural number $q$, such that the square of $p$ is equal to the product of $2$ and the square of $q$ for some natural number $p$.

Thm01f. We can prove that it is not the case that the square of $p$ is equal to the product of $2$ and the square of $q$ for a natural number $q$ for a natural number $p$.

Thm01f. We can prove that it is not the case that the square of $p$ is equal to the product of $2$ and the square of $q$ for some natural number $q$ for a natural number $p$.

Thm01f. We can prove that it is not the case that the square of $p$ is equal to the product of $2$ and the square of $q$ for a natural number $q$ for some natural number $p$.

Thm01f. We can prove that it is not the case that the square of $p$ is equal to the product of $2$ and the square of $q$ for some natural number $q$ for some natural number $p$.

Thm01f. It is not the case that there exists a natural number $p$, such that there exists a natural number $q$, such that the square of $p$ is equal to the product of $2$ and the square of $q$.

Thm01f. It is not the case that there exists a natural number $q$, such that the square of $p$ is equal to the product of $2$ and the square of $q$ for a natural number $p$.

Thm01f. It is not the case that there exists a natural number $q$, such that the square of $p$ is equal to the product of $2$ and the square of $q$ for some natural number $p$.

Thm01f. It is not the case that the square of $p$ is equal to the product of $2$ and the square of $q$ for a natural number $q$ for a natural number $p$.

Thm01f. It is not the case that the square of $p$ is equal to the product of $2$ and the square of $q$ for some natural number $q$ for a natural number $p$.

Thm01f. It is not the case that the square of $p$ is equal to the product of $2$ and the square of $q$ for a natural number $q$ for some natural number $p$.

Thm01f. It is not the case that the square of $p$ is equal to the product of $2$ and the square of $q$ for some natural number $q$ for some natural number $p$.

Thm01f. It is not the case that there exists a natural number $p$, such that there exists a natural number $q$, such that $p ^{ 2}= 2 (q ^{ 2})$.

Thm01f. It is not the case that there exists a natural number $q$, such that $p ^{ 2}= 2 (q ^{ 2})$ for a natural number $p$.

Thm01f. It is not the case that there exists a natural number $q$, such that $p ^{ 2}= 2 (q ^{ 2})$ for some natural number $p$.

Thm01f. It is not the case that $p ^{ 2}= 2 (q ^{ 2})$ for a natural number $q$ for a natural number $p$.

Thm01f. It is not the case that $p ^{ 2}= 2 (q ^{ 2})$ for some natural number $q$ for a natural number $p$.

Thm01f. It is not the case that $p ^{ 2}= 2 (q ^{ 2})$ for a natural number $q$ for some natural number $p$.

Thm01f. It is not the case that $p ^{ 2}= 2 (q ^{ 2})$ for some natural number $q$ for some natural number $p$.

Thm01f. It is not the case that there exist natural numbers $p$ and $q$, such that $p ^{ 2}= 2 (q ^{ 2})$.

Thm01f. It is not the case that $p ^{ 2}= 2 (q ^{ 2})$ for some natural numbers $p$ and $q$.

Thm02. Let $P$ be a polynomial. Assume that we can prove that the degree of $P$ is greater than $0$. Then we can prove that there exists a complex number $c$, such that $c$ is a root of $P$.

Thm02. For all polynomials $P$, if we can prove that the degree of $P$ is greater than $0$, then we can prove that there exists a complex number $c$, such that $c$ is a root of $P$.

Thm02. If we can prove that the degree of $P$ is greater than $0$, then we can prove that there exists a complex number $c$, such that $c$ is a root of $P$ for every polynomial $P$.

Thm02. If we can prove that the degree of $P$ is greater than $0$, then we can prove that there exists a complex number $c$, such that $c$ is a root of $P$ for all polynomials $P$.

Thm02. For all polynomials $P$, if we can prove that the degree of $P$ is greater than $0$, then we can prove that $c$ is a root of $P$ for a complex number $c$.

Thm02. If we can prove that the degree of $P$ is greater than $0$, then we can prove that $c$ is a root of $P$ for a complex number $c$ for every polynomial $P$.

Thm02. If we can prove that the degree of $P$ is greater than $0$, then we can prove that $c$ is a root of $P$ for a complex number $c$ for all polynomials $P$.

Thm02. For all polynomials $P$, if we can prove that the degree of $P$ is greater than $0$, then we can prove that $c$ is a root of $P$ for some complex number $c$.

Thm02. If we can prove that the degree of $P$ is greater than $0$, then we can prove that $c$ is a root of $P$ for some complex number $c$ for every polynomial $P$.

Thm02. If we can prove that the degree of $P$ is greater than $0$, then we can prove that $c$ is a root of $P$ for some complex number $c$ for all polynomials $P$.

Thm02. Let $P$ be a polynomial. Then if we can prove that the degree of $P$ is greater than $0$, then we can prove that there exists a complex number $c$, such that $c$ is a root of $P$.

Thm02. Let $P$ be a polynomial. Then we can prove that the degree of $P$ is greater than $0$, only if we can prove that there exists a complex number $c$, such that $c$ is a root of $P$.

Thm02. Let $P$ be a polynomial. Then if we can prove that the degree of $P$ is greater than $0$, then we can prove that $c$ is a root of $P$ for a complex number $c$.

Thm02. Let $P$ be a polynomial. Then we can prove that the degree of $P$ is greater than $0$, only if we can prove that $c$ is a root of $P$ for a complex number $c$.

Thm02. Let $P$ be a polynomial. Then if we can prove that the degree of $P$ is greater than $0$, then we can prove that $c$ is a root of $P$ for some complex number $c$.

Thm02. Let $P$ be a polynomial. Then we can prove that the degree of $P$ is greater than $0$, only if we can prove that $c$ is a root of $P$ for some complex number $c$.

Thm02. Let $P$ be a polynomial. Assume that we can prove that the degree of $P$ is greater than $0$. Then we can prove that $c$ is a root of $P$ for a complex number $c$.

Thm02. Let $P$ be a polynomial. Assume that we can prove that the degree of $P$ is greater than $0$. Then we can prove that $c$ is a root of $P$ for some complex number $c$.

Thm02. Let $P$ be a polynomial. Assume that the degree of $P$ is greater than $0$. Then there exists a complex number $c$, such that $c$ is a root of $P$.

Thm02. For all polynomials $P$, if the degree of $P$ is greater than $0$, then there exists a complex number $c$, such that $c$ is a root of $P$.

Thm02. If the degree of $P$ is greater than $0$, then there exists a complex number $c$, such that $c$ is a root of $P$ for every polynomial $P$.

Thm02. If the degree of $P$ is greater than $0$, then there exists a complex number $c$, such that $c$ is a root of $P$ for all polynomials $P$.

Thm02. For all polynomials $P$, if the degree of $P$ is greater than $0$, then $c$ is a root of $P$ for a complex number $c$.

Thm02. If the degree of $P$ is greater than $0$, then $c$ is a root of $P$ for a complex number $c$ for every polynomial $P$.

Thm02. If the degree of $P$ is greater than $0$, then $c$ is a root of $P$ for a complex number $c$ for all polynomials $P$.

Thm02. For all polynomials $P$, if the degree of $P$ is greater than $0$, then $c$ is a root of $P$ for some complex number $c$.

Thm02. If the degree of $P$ is greater than $0$, then $c$ is a root of $P$ for some complex number $c$ for every polynomial $P$.

Thm02. If the degree of $P$ is greater than $0$, then $c$ is a root of $P$ for some complex number $c$ for all polynomials $P$.

Thm02. Let $P$ be a polynomial. Then if the degree of $P$ is greater than $0$, then there exists a complex number $c$, such that $c$ is a root of $P$.

Thm02. Let $P$ be a polynomial. Then the degree of $P$ is greater than $0$, only if there exists a complex number $c$, such that $c$ is a root of $P$.

Thm02. Let $P$ be a polynomial. Then if the degree of $P$ is greater than $0$, then $c$ is a root of $P$ for a complex number $c$.

Thm02. Let $P$ be a polynomial. Then the degree of $P$ is greater than $0$, only if $c$ is a root of $P$ for a complex number $c$.

Thm02. Let $P$ be a polynomial. Then if the degree of $P$ is greater than $0$, then $c$ is a root of $P$ for some complex number $c$.

Thm02. Let $P$ be a polynomial. Then the degree of $P$ is greater than $0$, only if $c$ is a root of $P$ for some complex number $c$.

Thm02. Let $P$ be a polynomial. Assume that the degree of $P$ is greater than $0$. Then $c$ is a root of $P$ for a complex number $c$.

Thm02. Let $P$ be a polynomial. Assume that the degree of $P$ is greater than $0$. Then $c$ is a root of $P$ for some complex number $c$.

Thm03. We can prove that $Rat$ is denumerable.

Thm03. $Rat$ is denumerable.

Thm03a. We can prove that the cardinality of $Nat$ is equal to the cardinality of $Rat$.

Thm03a. The cardinality of $Nat$ is equal to the cardinality of $Rat$.

Thm03a. $| Nat | = | Rat |$.

Thm04. Let $u$ and $v$ be instances of vectors. Assume that we can prove that $u$ is perpendicular to $v$. Then we can prove that the length of the resultant of $u$ and $v$ is equal to the square root of the sum of the square of the length of $u$ and the square of the length of $v$.

Thm04. For all instances $u$ and $v$ of vectors, if we can prove that $u$ is perpendicular to $v$, then we can prove that the length of the resultant of $u$ and $v$ is equal to the square root of the sum of the square of the length of $u$ and the square of the length of $v$.

Thm04. If we can prove that $u$ is perpendicular to $v$, then we can prove that the length of the resultant of $u$ and $v$ is equal to the square root of the sum of the square of the length of $u$ and the square of the length of $v$ for all instances $u$ and $v$ of vectors.

Thm04. Let $u$ and $v$ be instances of vectors. Then if we can prove that $u$ is perpendicular to $v$, then we can prove that the length of the resultant of $u$ and $v$ is equal to the square root of the sum of the square of the length of $u$ and the square of the length of $v$.

Thm04. Let $u$ and $v$ be instances of vectors. Then we can prove that $u$ is perpendicular to $v$, only if we can prove that the length of the resultant of $u$ and $v$ is equal to the square root of the sum of the square of the length of $u$ and the square of the length of $v$.

Thm04. Let $u$ and $v$ be vectors. Assume that $u$ is perpendicular to $v$. Then the length of the resultant of $u$ and $v$ is equal to the square root of the sum of the square of the length of $u$ and the square of the length of $v$.

Thm04. For all vectors $u$ and $v$, if $u$ is perpendicular to $v$, then the length of the resultant of $u$ and $v$ is equal to the square root of the sum of the square of the length of $u$ and the square of the length of $v$.

Thm04. If $u$ is perpendicular to $v$, then the length of the resultant of $u$ and $v$ is equal to the square root of the sum of the square of the length of $u$ and the square of the length of $v$ for all vectors $u$ and $v$.

Thm04. Let $u$ and $v$ be vectors. Then if $u$ is perpendicular to $v$, then the length of the resultant of $u$ and $v$ is equal to the square root of the sum of the square of the length of $u$ and the square of the length of $v$.

Thm04. Let $u$ and $v$ be vectors. Then $u$ is perpendicular to $v$, only if the length of the resultant of $u$ and $v$ is equal to the square root of the sum of the square of the length of $u$ and the square of the length of $v$.

Thm04. Let $u$ and $v$ be vectors. Assume that $u \perp v$. Then $\| u + v \| = \sqrt{ \| u \| ^{ 2}+ \| v \| ^{ 2}}$.

Thm04. For all vectors $u$ and $v$, if $u \perp v$, then $\| u + v \| = \sqrt{ \| u \| ^{ 2}+ \| v \| ^{ 2}}$.

Thm04. If $u \perp v$, then $\| u + v \| = \sqrt{ \| u \| ^{ 2}+ \| v \| ^{ 2}}$ for all vectors $u$ and $v$.

Thm04. Let $u$ and $v$ be vectors. Then if $u \perp v$, then $\| u + v \| = \sqrt{ \| u \| ^{ 2}+ \| v \| ^{ 2}}$.

Thm04. Let $u$ and $v$ be vectors. Then $u \perp v$, only if $\| u + v \| = \sqrt{ \| u \| ^{ 2}+ \| v \| ^{ 2}}$.

Thm04. Let $u$ and $v$ be vectors. Then $u \perp v$ implies $\| u + v \| = \sqrt{ \| u \| ^{ 2}+ \| v \| ^{ 2}}$.

Thm07. Let $p$ and $q$ be instances of natural numbers. Assume that we can prove that $p$ is prime and $q$ is prime. Then we can prove that the product of the Legendre symbol of $p$ and $q$ and the Legendre symbol of $q$ and $p$ is equal to the exponentiation of the negation of $1$ and the product of the quotient of the difference of $p$ and $1$ and $2$ and the quotient of the difference of $q$ and $1$ and $2$.

Thm07. For all instances $p$ and $q$ of natural numbers, if we can prove that $p$ is prime and $q$ is prime, then we can prove that the product of the Legendre symbol of $p$ and $q$ and the Legendre symbol of $q$ and $p$ is equal to the exponentiation of the negation of $1$ and the product of the quotient of the difference of $p$ and $1$ and $2$ and the quotient of the difference of $q$ and $1$ and $2$.

Thm07. If we can prove that $p$ is prime and $q$ is prime, then we can prove that the product of the Legendre symbol of $p$ and $q$ and the Legendre symbol of $q$ and $p$ is equal to the exponentiation of the negation of $1$ and the product of the quotient of the difference of $p$ and $1$ and $2$ and the quotient of the difference of $q$ and $1$ and $2$ for all instances $p$ and $q$ of natural numbers.

Thm07. Let $p$ and $q$ be instances of natural numbers. Then if we can prove that $p$ is prime and $q$ is prime, then we can prove that the product of the Legendre symbol of $p$ and $q$ and the Legendre symbol of $q$ and $p$ is equal to the exponentiation of the negation of $1$ and the product of the quotient of the difference of $p$ and $1$ and $2$ and the quotient of the difference of $q$ and $1$ and $2$.

Thm07. Let $p$ and $q$ be instances of natural numbers. Then we can prove that $p$ is prime and $q$ is prime, only if we can prove that the product of the Legendre symbol of $p$ and $q$ and the Legendre symbol of $q$ and $p$ is equal to the exponentiation of the negation of $1$ and the product of the quotient of the difference of $p$ and $1$ and $2$ and the quotient of the difference of $q$ and $1$ and $2$.

Thm07. Let $p$ and $q$ be natural numbers. Assume that $p$ is prime and $q$ is prime. Then the product of the Legendre symbol of $p$ and $q$ and the Legendre symbol of $q$ and $p$ is equal to the exponentiation of the negation of $1$ and the product of the quotient of the difference of $p$ and $1$ and $2$ and the quotient of the difference of $q$ and $1$ and $2$.

Thm07. For all natural numbers $p$ and $q$, if $p$ is prime and $q$ is prime, then the product of the Legendre symbol of $p$ and $q$ and the Legendre symbol of $q$ and $p$ is equal to the exponentiation of the negation of $1$ and the product of the quotient of the difference of $p$ and $1$ and $2$ and the quotient of the difference of $q$ and $1$ and $2$.

Thm07. If $p$ is prime and $q$ is prime, then the product of the Legendre symbol of $p$ and $q$ and the Legendre symbol of $q$ and $p$ is equal to the exponentiation of the negation of $1$ and the product of the quotient of the difference of $p$ and $1$ and $2$ and the quotient of the difference of $q$ and $1$ and $2$ for all natural numbers $p$ and $q$.

Thm07. Let $p$ and $q$ be natural numbers. Then if $p$ is prime and $q$ is prime, then the product of the Legendre symbol of $p$ and $q$ and the Legendre symbol of $q$ and $p$ is equal to the exponentiation of the negation of $1$ and the product of the quotient of the difference of $p$ and $1$ and $2$ and the quotient of the difference of $q$ and $1$ and $2$.

Thm07. Let $p$ and $q$ be natural numbers. Then $p$ is prime and $q$ is prime, only if the product of the Legendre symbol of $p$ and $q$ and the Legendre symbol of $q$ and $p$ is equal to the exponentiation of the negation of $1$ and the product of the quotient of the difference of $p$ and $1$ and $2$ and the quotient of the difference of $q$ and $1$ and $2$.

Thm07. Let $p , q \in N$. Then if $p$ is prime and $q$ is prime, then the product of the Legendre symbol of $p$ and $q$ and the Legendre symbol of $q$ and $p$ is equal to the exponentiation of the negation of $1$ and the product of the quotient of the difference of $p$ and $1$ and $2$ and the quotient of the difference of $q$ and $1$ and $2$.

Thm07. Let $p , q \in N$. Then $p$ is prime and $q$ is prime, only if the product of the Legendre symbol of $p$ and $q$ and the Legendre symbol of $q$ and $p$ is equal to the exponentiation of the negation of $1$ and the product of the quotient of the difference of $p$ and $1$ and $2$ and the quotient of the difference of $q$ and $1$ and $2$.

Thm07. Let $p , q \in N$. Assume that $p$ is prime and $q$ is prime. Then the product of the Legendre symbol of $p$ and $q$ and the Legendre symbol of $q$ and $p$ is equal to the exponentiation of the negation of $1$ and the product of the quotient of the difference of $p$ and $1$ and $2$ and the quotient of the difference of $q$ and $1$ and $2$.

Thm07. Let $p$ and $q$ be natural numbers. Assume that $p$ is prime and $q$ is prime. Then $\left(\frac{ p }{ q }\right) \left(\frac{ q }{ p }\right) = (- 1)^ {\frac{ p - 1}{2}\frac{ q - 1}{2}}$.

Thm07. For all natural numbers $p$ and $q$, if $p$ is prime and $q$ is prime, then $\left(\frac{ p }{ q }\right) \left(\frac{ q }{ p }\right) = (- 1)^ {\frac{ p - 1}{2}\frac{ q - 1}{2}}$.

Thm07. If $p$ is prime and $q$ is prime, then $\left(\frac{ p }{ q }\right) \left(\frac{ q }{ p }\right) = (- 1)^ {\frac{ p - 1}{2}\frac{ q - 1}{2}}$ for all natural numbers $p$ and $q$.

Thm07. Let $p$ and $q$ be natural numbers. Then if $p$ is prime and $q$ is prime, then $\left(\frac{ p }{ q }\right) \left(\frac{ q }{ p }\right) = (- 1)^ {\frac{ p - 1}{2}\frac{ q - 1}{2}}$.

Thm07. Let $p$ and $q$ be natural numbers. Then $p$ is prime and $q$ is prime, only if $\left(\frac{ p }{ q }\right) \left(\frac{ q }{ p }\right) = (- 1)^ {\frac{ p - 1}{2}\frac{ q - 1}{2}}$.

Thm07. Let $p , q \in N$. Then if $p$ is prime and $q$ is prime, then $\left(\frac{ p }{ q }\right) \left(\frac{ q }{ p }\right) = (- 1)^ {\frac{ p - 1}{2}\frac{ q - 1}{2}}$.

Thm07. Let $p , q \in N$. Then $p$ is prime and $q$ is prime, only if $\left(\frac{ p }{ q }\right) \left(\frac{ q }{ p }\right) = (- 1)^ {\frac{ p - 1}{2}\frac{ q - 1}{2}}$.

Thm07. Let $p , q \in N$. Assume that $p$ is prime and $q$ is prime. Then $\left(\frac{ p }{ q }\right) \left(\frac{ q }{ p }\right) = (- 1)^ {\frac{ p - 1}{2}\frac{ q - 1}{2}}$.

Thm07. Let $p$ and $q$ be natural numbers. Assume that $p$ and $q$ are prime. Then $\left(\frac{ p }{ q }\right) \left(\frac{ q }{ p }\right) = (- 1)^ {\frac{ p - 1}{2}\frac{ q - 1}{2}}$.

Thm07. For all natural numbers $p$ and $q$, if $p$ and $q$ are prime, then $\left(\frac{ p }{ q }\right) \left(\frac{ q }{ p }\right) = (- 1)^ {\frac{ p - 1}{2}\frac{ q - 1}{2}}$.

Thm07. If $p$ and $q$ are prime, then $\left(\frac{ p }{ q }\right) \left(\frac{ q }{ p }\right) = (- 1)^ {\frac{ p - 1}{2}\frac{ q - 1}{2}}$ for all natural numbers $p$ and $q$.

Thm07. Let $p$ and $q$ be natural numbers. Then if $p$ and $q$ are prime, then $\left(\frac{ p }{ q }\right) \left(\frac{ q }{ p }\right) = (- 1)^ {\frac{ p - 1}{2}\frac{ q - 1}{2}}$.

Thm07. Let $p$ and $q$ be natural numbers. Then $p$ and $q$ are prime, only if $\left(\frac{ p }{ q }\right) \left(\frac{ q }{ p }\right) = (- 1)^ {\frac{ p - 1}{2}\frac{ q - 1}{2}}$.

Thm07. Let $p , q \in N$. Then if $p$ and $q$ are prime, then $\left(\frac{ p }{ q }\right) \left(\frac{ q }{ p }\right) = (- 1)^ {\frac{ p - 1}{2}\frac{ q - 1}{2}}$.

Thm07. Let $p , q \in N$. Then $p$ and $q$ are prime, only if $\left(\frac{ p }{ q }\right) \left(\frac{ q }{ p }\right) = (- 1)^ {\frac{ p - 1}{2}\frac{ q - 1}{2}}$.

Thm07. Let $p , q \in N$. Assume that $p$ and $q$ are prime. Then $\left(\frac{ p }{ q }\right) \left(\frac{ q }{ p }\right) = (- 1)^ {\frac{ p - 1}{2}\frac{ q - 1}{2}}$.

Thm09. Let $c$ be a circle. Let $r$ be an instance of real numbers. Assume that we can prove that $r$ is equal to the radius of $c$. Then we can prove that the area of $c$ is equal to the product of the number \(\pi\) and the exponentiation of $r$ and $2$.

Thm09. For all circles $c$, for all instances $r$ of real numbers, if we can prove that $r$ is equal to the radius of $c$, then we can prove that the area of $c$ is equal to the product of the number \(\pi\) and the exponentiation of $r$ and $2$.

Thm09. For all instances $r$ of real numbers, if we can prove that $r$ is equal to the radius of $c$, then we can prove that the area of $c$ is equal to the product of the number \(\pi\) and the exponentiation of $r$ and $2$ for every circle $c$.

Thm09. For all instances $r$ of real numbers, if we can prove that $r$ is equal to the radius of $c$, then we can prove that the area of $c$ is equal to the product of the number \(\pi\) and the exponentiation of $r$ and $2$ for all circles $c$.

Thm09. Let $c$ be a circle. Then for all instances $r$ of real numbers, if we can prove that $r$ is equal to the radius of $c$, then we can prove that the area of $c$ is equal to the product of the number \(\pi\) and the exponentiation of $r$ and $2$.

Thm09. Let $c$ be a circle. Then if we can prove that $r$ is equal to the radius of $c$, then we can prove that the area of $c$ is equal to the product of the number \(\pi\) and the exponentiation of $r$ and $2$ for every instance $r$ of real numbers.

Thm09. Let $c$ be a circle. Then if we can prove that $r$ is equal to the radius of $c$, then we can prove that the area of $c$ is equal to the product of the number \(\pi\) and the exponentiation of $r$ and $2$ for all instances $r$ of real numbers.

Thm09. Let $c$ be a circle. Let $r$ be an instance of real numbers. Then if we can prove that $r$ is equal to the radius of $c$, then we can prove that the area of $c$ is equal to the product of the number \(\pi\) and the exponentiation of $r$ and $2$.

Thm09. Let $c$ be a circle. Let $r$ be an instance of real numbers. Then we can prove that $r$ is equal to the radius of $c$, only if we can prove that the area of $c$ is equal to the product of the number \(\pi\) and the exponentiation of $r$ and $2$.

Thm09. Let $c$ be a circle. Let $r$ be a real number. Assume that $r$ is equal to the radius of $c$. Then the area of $c$ is equal to the product of the number \(\pi\) and the exponentiation of $r$ and $2$.

Thm09. For all circles $c$, for all real numbers $r$, if $r$ is equal to the radius of $c$, then the area of $c$ is equal to the product of the number \(\pi\) and the exponentiation of $r$ and $2$.

Thm09. For all real numbers $r$, if $r$ is equal to the radius of $c$, then the area of $c$ is equal to the product of the number \(\pi\) and the exponentiation of $r$ and $2$ for every circle $c$.

Thm09. For all real numbers $r$, if $r$ is equal to the radius of $c$, then the area of $c$ is equal to the product of the number \(\pi\) and the exponentiation of $r$ and $2$ for all circles $c$.

Thm09. Let $c$ be a circle. Then for all real numbers $r$, if $r$ is equal to the radius of $c$, then the area of $c$ is equal to the product of the number \(\pi\) and the exponentiation of $r$ and $2$.

Thm09. Let $c$ be a circle. Then if $r$ is equal to the radius of $c$, then the area of $c$ is equal to the product of the number \(\pi\) and the exponentiation of $r$ and $2$ for every real number $r$.

Thm09. Let $c$ be a circle. Then if $r$ is equal to the radius of $c$, then the area of $c$ is equal to the product of the number \(\pi\) and the exponentiation of $r$ and $2$ for all real numbers $r$.

Thm09. Let $c$ be a circle. Let $r$ be a real number. Then if $r$ is equal to the radius of $c$, then the area of $c$ is equal to the product of the number \(\pi\) and the exponentiation of $r$ and $2$.

Thm09. Let $c$ be a circle. Let $r$ be a real number. Then $r$ is equal to the radius of $c$, only if the area of $c$ is equal to the product of the number \(\pi\) and the exponentiation of $r$ and $2$.

Thm09. Let $c$ be a circle. Let $r \in R$. Then if $r$ is equal to the radius of $c$, then the area of $c$ is equal to the product of the number \(\pi\) and the exponentiation of $r$ and $2$.

Thm09. Let $c$ be a circle. Let $r \in R$. Then $r$ is equal to the radius of $c$, only if the area of $c$ is equal to the product of the number \(\pi\) and the exponentiation of $r$ and $2$.

Thm09. Let $c$ be a circle. Let $r \in R$. Assume that $r$ is equal to the radius of $c$. Then the area of $c$ is equal to the product of the number \(\pi\) and the exponentiation of $r$ and $2$.

Thm09. Let $c$ be a circle. Let $r$ be a real number. Assume that $r$ is equal to the radius of $c$. Then the area of $c$ is equal to $\pi r ^ {2}$.

Thm09. For all circles $c$, for all real numbers $r$, if $r$ is equal to the radius of $c$, then the area of $c$ is equal to $\pi r ^ {2}$.

Thm09. For all real numbers $r$, if $r$ is equal to the radius of $c$, then the area of $c$ is equal to $\pi r ^ {2}$ for every circle $c$.

Thm09. For all real numbers $r$, if $r$ is equal to the radius of $c$, then the area of $c$ is equal to $\pi r ^ {2}$ for all circles $c$.

Thm09. Let $c$ be a circle. Then for all real numbers $r$, if $r$ is equal to the radius of $c$, then the area of $c$ is equal to $\pi r ^ {2}$.

Thm09. Let $c$ be a circle. Then if $r$ is equal to the radius of $c$, then the area of $c$ is equal to $\pi r ^ {2}$ for every real number $r$.

Thm09. Let $c$ be a circle. Then if $r$ is equal to the radius of $c$, then the area of $c$ is equal to $\pi r ^ {2}$ for all real numbers $r$.

Thm09. Let $c$ be a circle. Let $r$ be a real number. Then if $r$ is equal to the radius of $c$, then the area of $c$ is equal to $\pi r ^ {2}$.

Thm09. Let $c$ be a circle. Let $r$ be a real number. Then $r$ is equal to the radius of $c$, only if the area of $c$ is equal to $\pi r ^ {2}$.

Thm09. Let $c$ be a circle. Let $r \in R$. Then if $r$ is equal to the radius of $c$, then the area of $c$ is equal to $\pi r ^ {2}$.

Thm09. Let $c$ be a circle. Let $r \in R$. Then $r$ is equal to the radius of $c$, only if the area of $c$ is equal to $\pi r ^ {2}$.

Thm09. Let $c$ be a circle. Let $r \in R$. Assume that $r$ is equal to the radius of $c$. Then the area of $c$ is equal to $\pi r ^ {2}$.

Thm10FermatLittle. Let $p$ be an instance of natural numbers. Assume that we can prove that $p$ is prime. Let $a$ be an instance of integers. Then we can prove that there exists an integer $q$, such that the difference of the exponentiation of $a$ and $p$ and $a$ is equal to the product of $p$ and $q$.

Thm10FermatLittle. For all instances $p$ of natural numbers, if we can prove that $p$ is prime, then for all instances $a$ of integers, we can prove that there exists an integer $q$, such that the difference of the exponentiation of $a$ and $p$ and $a$ is equal to the product of $p$ and $q$.

Thm10FermatLittle. If we can prove that $p$ is prime, then for all instances $a$ of integers, we can prove that there exists an integer $q$, such that the difference of the exponentiation of $a$ and $p$ and $a$ is equal to the product of $p$ and $q$ for every instance $p$ of natural numbers.

Thm10FermatLittle. If we can prove that $p$ is prime, then for all instances $a$ of integers, we can prove that there exists an integer $q$, such that the difference of the exponentiation of $a$ and $p$ and $a$ is equal to the product of $p$ and $q$ for all instances $p$ of natural numbers.

Thm10FermatLittle. For all instances $p$ of natural numbers, if we can prove that $p$ is prime, then for all instances $a$ of integers, we can prove that the difference of the exponentiation of $a$ and $p$ and $a$ is equal to the product of $p$ and $q$ for an integer $q$.

Thm10FermatLittle. If we can prove that $p$ is prime, then for all instances $a$ of integers, we can prove that the difference of the exponentiation of $a$ and $p$ and $a$ is equal to the product of $p$ and $q$ for an integer $q$ for every instance $p$ of natural numbers.

Thm10FermatLittle. If we can prove that $p$ is prime, then for all instances $a$ of integers, we can prove that the difference of the exponentiation of $a$ and $p$ and $a$ is equal to the product of $p$ and $q$ for an integer $q$ for all instances $p$ of natural numbers.

Thm10FermatLittle. For all instances $p$ of natural numbers, if we can prove that $p$ is prime, then for all instances $a$ of integers, we can prove that the difference of the exponentiation of $a$ and $p$ and $a$ is equal to the product of $p$ and $q$ for some integer $q$.

Thm10FermatLittle. If we can prove that $p$ is prime, then for all instances $a$ of integers, we can prove that the difference of the exponentiation of $a$ and $p$ and $a$ is equal to the product of $p$ and $q$ for some integer $q$ for every instance $p$ of natural numbers.

Thm10FermatLittle. If we can prove that $p$ is prime, then for all instances $a$ of integers, we can prove that the difference of the exponentiation of $a$ and $p$ and $a$ is equal to the product of $p$ and $q$ for some integer $q$ for all instances $p$ of natural numbers.

Thm10FermatLittle. Let $p$ be an instance of natural numbers. Then if we can prove that $p$ is prime, then for all instances $a$ of integers, we can prove that there exists an integer $q$, such that the difference of the exponentiation of $a$ and $p$ and $a$ is equal to the product of $p$ and $q$.

Thm10FermatLittle. Let $p$ be an instance of natural numbers. Then we can prove that $p$ is prime, only if for all instances $a$ of integers, we can prove that there exists an integer $q$, such that the difference of the exponentiation of $a$ and $p$ and $a$ is equal to the product of $p$ and $q$.

Thm10FermatLittle. Let $p$ be an instance of natural numbers. Then if we can prove that $p$ is prime, then for all instances $a$ of integers, we can prove that the difference of the exponentiation of $a$ and $p$ and $a$ is equal to the product of $p$ and $q$ for an integer $q$.

Thm10FermatLittle. Let $p$ be an instance of natural numbers. Then we can prove that $p$ is prime, only if for all instances $a$ of integers, we can prove that the difference of the exponentiation of $a$ and $p$ and $a$ is equal to the product of $p$ and $q$ for an integer $q$.

Thm10FermatLittle. Let $p$ be an instance of natural numbers. Then if we can prove that $p$ is prime, then for all instances $a$ of integers, we can prove that the difference of the exponentiation of $a$ and $p$ and $a$ is equal to the product of $p$ and $q$ for some integer $q$.

Thm10FermatLittle. Let $p$ be an instance of natural numbers. Then we can prove that $p$ is prime, only if for all instances $a$ of integers, we can prove that the difference of the exponentiation of $a$ and $p$ and $a$ is equal to the product of $p$ and $q$ for some integer $q$.

Thm10FermatLittle. Let $p$ be an instance of natural numbers. Assume that we can prove that $p$ is prime. Then for all instances $a$ of integers, we can prove that there exists an integer $q$, such that the difference of the exponentiation of $a$ and $p$ and $a$ is equal to the product of $p$ and $q$.

Thm10FermatLittle. Let $p$ be an instance of natural numbers. Assume that we can prove that $p$ is prime. Then we can prove that there exists an integer $q$, such that the difference of the exponentiation of $a$ and $p$ and $a$ is equal to the product of $p$ and $q$ for every instance $a$ of integers.

Thm10FermatLittle. Let $p$ be an instance of natural numbers. Assume that we can prove that $p$ is prime. Then we can prove that there exists an integer $q$, such that the difference of the exponentiation of $a$ and $p$ and $a$ is equal to the product of $p$ and $q$ for all instances $a$ of integers.

Thm10FermatLittle. Let $p$ be an instance of natural numbers. Assume that we can prove that $p$ is prime. Then for all instances $a$ of integers, we can prove that the difference of the exponentiation of $a$ and $p$ and $a$ is equal to the product of $p$ and $q$ for an integer $q$.

Thm10FermatLittle. Let $p$ be an instance of natural numbers. Assume that we can prove that $p$ is prime. Then we can prove that the difference of the exponentiation of $a$ and $p$ and $a$ is equal to the product of $p$ and $q$ for an integer $q$ for every instance $a$ of integers.

Thm10FermatLittle. Let $p$ be an instance of natural numbers. Assume that we can prove that $p$ is prime. Then we can prove that the difference of the exponentiation of $a$ and $p$ and $a$ is equal to the product of $p$ and $q$ for an integer $q$ for all instances $a$ of integers.

Thm10FermatLittle. Let $p$ be an instance of natural numbers. Assume that we can prove that $p$ is prime. Then for all instances $a$ of integers, we can prove that the difference of the exponentiation of $a$ and $p$ and $a$ is equal to the product of $p$ and $q$ for some integer $q$.

Thm10FermatLittle. Let $p$ be an instance of natural numbers. Assume that we can prove that $p$ is prime. Then we can prove that the difference of the exponentiation of $a$ and $p$ and $a$ is equal to the product of $p$ and $q$ for some integer $q$ for every instance $a$ of integers.

Thm10FermatLittle. Let $p$ be an instance of natural numbers. Assume that we can prove that $p$ is prime. Then we can prove that the difference of the exponentiation of $a$ and $p$ and $a$ is equal to the product of $p$ and $q$ for some integer $q$ for all instances $a$ of integers.

Thm10FermatLittle. Let $p$ be an instance of natural numbers. Assume that we can prove that $p$ is prime. Let $a$ be an instance of integers. Then we can prove that the difference of the exponentiation of $a$ and $p$ and $a$ is equal to the product of $p$ and $q$ for an integer $q$.

Thm10FermatLittle. Let $p$ be an instance of natural numbers. Assume that we can prove that $p$ is prime. Let $a$ be an instance of integers. Then we can prove that the difference of the exponentiation of $a$ and $p$ and $a$ is equal to the product of $p$ and $q$ for some integer $q$.

Thm10FermatLittle. Let $p$ be a natural number. Assume that $p$ is prime. Let $a$ be an integer. Then there exists an integer $q$, such that the difference of the exponentiation of $a$ and $p$ and $a$ is equal to the product of $p$ and $q$.

Thm10FermatLittle. For all natural numbers $p$, if $p$ is prime, then for all integers $a$, there exists an integer $q$, such that the difference of the exponentiation of $a$ and $p$ and $a$ is equal to the product of $p$ and $q$.

Thm10FermatLittle. If $p$ is prime, then for all integers $a$, there exists an integer $q$, such that the difference of the exponentiation of $a$ and $p$ and $a$ is equal to the product of $p$ and $q$ for every natural number $p$.

Thm10FermatLittle. If $p$ is prime, then for all integers $a$, there exists an integer $q$, such that the difference of the exponentiation of $a$ and $p$ and $a$ is equal to the product of $p$ and $q$ for all natural numbers $p$.

Thm10FermatLittle. For all natural numbers $p$, if $p$ is prime, then for all integers $a$, the difference of the exponentiation of $a$ and $p$ and $a$ is equal to the product of $p$ and $q$ for an integer $q$.

Thm10FermatLittle. If $p$ is prime, then for all integers $a$, the difference of the exponentiation of $a$ and $p$ and $a$ is equal to the product of $p$ and $q$ for an integer $q$ for every natural number $p$.

Thm10FermatLittle. If $p$ is prime, then for all integers $a$, the difference of the exponentiation of $a$ and $p$ and $a$ is equal to the product of $p$ and $q$ for an integer $q$ for all natural numbers $p$.

Thm10FermatLittle. For all natural numbers $p$, if $p$ is prime, then for all integers $a$, the difference of the exponentiation of $a$ and $p$ and $a$ is equal to the product of $p$ and $q$ for some integer $q$.

Thm10FermatLittle. If $p$ is prime, then for all integers $a$, the difference of the exponentiation of $a$ and $p$ and $a$ is equal to the product of $p$ and $q$ for some integer $q$ for every natural number $p$.

Thm10FermatLittle. If $p$ is prime, then for all integers $a$, the difference of the exponentiation of $a$ and $p$ and $a$ is equal to the product of $p$ and $q$ for some integer $q$ for all natural numbers $p$.

Thm10FermatLittle. Let $p$ be a natural number. Then if $p$ is prime, then for all integers $a$, there exists an integer $q$, such that the difference of the exponentiation of $a$ and $p$ and $a$ is equal to the product of $p$ and $q$.

Thm10FermatLittle. Let $p$ be a natural number. Then $p$ is prime, only if for all integers $a$, there exists an integer $q$, such that the difference of the exponentiation of $a$ and $p$ and $a$ is equal to the product of $p$ and $q$.

Thm10FermatLittle. Let $p$ be a natural number. Then if $p$ is prime, then for all integers $a$, the difference of the exponentiation of $a$ and $p$ and $a$ is equal to the product of $p$ and $q$ for an integer $q$.

Thm10FermatLittle. Let $p$ be a natural number. Then $p$ is prime, only if for all integers $a$, the difference of the exponentiation of $a$ and $p$ and $a$ is equal to the product of $p$ and $q$ for an integer $q$.

Thm10FermatLittle. Let $p$ be a natural number. Then if $p$ is prime, then for all integers $a$, the difference of the exponentiation of $a$ and $p$ and $a$ is equal to the product of $p$ and $q$ for some integer $q$.

Thm10FermatLittle. Let $p$ be a natural number. Then $p$ is prime, only if for all integers $a$, the difference of the exponentiation of $a$ and $p$ and $a$ is equal to the product of $p$ and $q$ for some integer $q$.

Thm10FermatLittle. Let $p \in N$. Then if $p$ is prime, then for all integers $a$, there exists an integer $q$, such that the difference of the exponentiation of $a$ and $p$ and $a$ is equal to the product of $p$ and $q$.

Thm10FermatLittle. Let $p \in N$. Then $p$ is prime, only if for all integers $a$, there exists an integer $q$, such that the difference of the exponentiation of $a$ and $p$ and $a$ is equal to the product of $p$ and $q$.

Thm10FermatLittle. Let $p \in N$. Then if $p$ is prime, then for all integers $a$, the difference of the exponentiation of $a$ and $p$ and $a$ is equal to the product of $p$ and $q$ for an integer $q$.

Thm10FermatLittle. Let $p \in N$. Then $p$ is prime, only if for all integers $a$, the difference of the exponentiation of $a$ and $p$ and $a$ is equal to the product of $p$ and $q$ for an integer $q$.

Thm10FermatLittle. Let $p \in N$. Then if $p$ is prime, then for all integers $a$, the difference of the exponentiation of $a$ and $p$ and $a$ is equal to the product of $p$ and $q$ for some integer $q$.

Thm10FermatLittle. Let $p \in N$. Then $p$ is prime, only if for all integers $a$, the difference of the exponentiation of $a$ and $p$ and $a$ is equal to the product of $p$ and $q$ for some integer $q$.

Thm10FermatLittle. Let $p$ be a natural number. Assume that $p$ is prime. Then for all integers $a$, there exists an integer $q$, such that the difference of the exponentiation of $a$ and $p$ and $a$ is equal to the product of $p$ and $q$.

Thm10FermatLittle. Let $p$ be a natural number. Assume that $p$ is prime. Then there exists an integer $q$, such that the difference of the exponentiation of $a$ and $p$ and $a$ is equal to the product of $p$ and $q$ for every integer $a$.

Thm10FermatLittle. Let $p$ be a natural number. Assume that $p$ is prime. Then there exists an integer $q$, such that the difference of the exponentiation of $a$ and $p$ and $a$ is equal to the product of $p$ and $q$ for all integers $a$.

Thm10FermatLittle. Let $p$ be a natural number. Assume that $p$ is prime. Then for all integers $a$, the difference of the exponentiation of $a$ and $p$ and $a$ is equal to the product of $p$ and $q$ for an integer $q$.

Thm10FermatLittle. Let $p$ be a natural number. Assume that $p$ is prime. Then the difference of the exponentiation of $a$ and $p$ and $a$ is equal to the product of $p$ and $q$ for an integer $q$ for every integer $a$.

Thm10FermatLittle. Let $p$ be a natural number. Assume that $p$ is prime. Then the difference of the exponentiation of $a$ and $p$ and $a$ is equal to the product of $p$ and $q$ for an integer $q$ for all integers $a$.

Thm10FermatLittle. Let $p$ be a natural number. Assume that $p$ is prime. Then for all integers $a$, the difference of the exponentiation of $a$ and $p$ and $a$ is equal to the product of $p$ and $q$ for some integer $q$.

Thm10FermatLittle. Let $p$ be a natural number. Assume that $p$ is prime. Then the difference of the exponentiation of $a$ and $p$ and $a$ is equal to the product of $p$ and $q$ for some integer $q$ for every integer $a$.

Thm10FermatLittle. Let $p$ be a natural number. Assume that $p$ is prime. Then the difference of the exponentiation of $a$ and $p$ and $a$ is equal to the product of $p$ and $q$ for some integer $q$ for all integers $a$.

Thm10FermatLittle. Let $p \in N$. Assume that $p$ is prime. Then for all integers $a$, there exists an integer $q$, such that the difference of the exponentiation of $a$ and $p$ and $a$ is equal to the product of $p$ and $q$.

Thm10FermatLittle. Let $p \in N$. Assume that $p$ is prime. Then there exists an integer $q$, such that the difference of the exponentiation of $a$ and $p$ and $a$ is equal to the product of $p$ and $q$ for every integer $a$.

Thm10FermatLittle. Let $p \in N$. Assume that $p$ is prime. Then there exists an integer $q$, such that the difference of the exponentiation of $a$ and $p$ and $a$ is equal to the product of $p$ and $q$ for all integers $a$.

Thm10FermatLittle. Let $p \in N$. Assume that $p$ is prime. Then for all integers $a$, the difference of the exponentiation of $a$ and $p$ and $a$ is equal to the product of $p$ and $q$ for an integer $q$.

Thm10FermatLittle. Let $p \in N$. Assume that $p$ is prime. Then the difference of the exponentiation of $a$ and $p$ and $a$ is equal to the product of $p$ and $q$ for an integer $q$ for every integer $a$.

Thm10FermatLittle. Let $p \in N$. Assume that $p$ is prime. Then the difference of the exponentiation of $a$ and $p$ and $a$ is equal to the product of $p$ and $q$ for an integer $q$ for all integers $a$.

Thm10FermatLittle. Let $p \in N$. Assume that $p$ is prime. Then for all integers $a$, the difference of the exponentiation of $a$ and $p$ and $a$ is equal to the product of $p$ and $q$ for some integer $q$.

Thm10FermatLittle. Let $p \in N$. Assume that $p$ is prime. Then the difference of the exponentiation of $a$ and $p$ and $a$ is equal to the product of $p$ and $q$ for some integer $q$ for every integer $a$.

Thm10FermatLittle. Let $p \in N$. Assume that $p$ is prime. Then the difference of the exponentiation of $a$ and $p$ and $a$ is equal to the product of $p$ and $q$ for some integer $q$ for all integers $a$.

Thm10FermatLittle. Let $p$ be a natural number. Assume that $p$ is prime. Let $a$ be an integer. Then the difference of the exponentiation of $a$ and $p$ and $a$ is equal to the product of $p$ and $q$ for an integer $q$.

Thm10FermatLittle. Let $p$ be a natural number. Assume that $p$ is prime. Let $a$ be an integer. Then the difference of the exponentiation of $a$ and $p$ and $a$ is equal to the product of $p$ and $q$ for some integer $q$.

Thm10FermatLittle. Let $p$ be a natural number. Assume that $p$ is prime. Let $a \in Z$. Then there exists an integer $q$, such that the difference of the exponentiation of $a$ and $p$ and $a$ is equal to the product of $p$ and $q$.

Thm10FermatLittle. Let $p$ be a natural number. Assume that $p$ is prime. Let $a \in Z$. Then the difference of the exponentiation of $a$ and $p$ and $a$ is equal to the product of $p$ and $q$ for an integer $q$.

Thm10FermatLittle. Let $p$ be a natural number. Assume that $p$ is prime. Let $a \in Z$. Then the difference of the exponentiation of $a$ and $p$ and $a$ is equal to the product of $p$ and $q$ for some integer $q$.

Thm10FermatLittle. Let $p \in N$. Assume that $p$ is prime. Let $a$ be an integer. Then there exists an integer $q$, such that the difference of the exponentiation of $a$ and $p$ and $a$ is equal to the product of $p$ and $q$.

Thm10FermatLittle. Let $p \in N$. Assume that $p$ is prime. Let $a$ be an integer. Then the difference of the exponentiation of $a$ and $p$ and $a$ is equal to the product of $p$ and $q$ for an integer $q$.

Thm10FermatLittle. Let $p \in N$. Assume that $p$ is prime. Let $a$ be an integer. Then the difference of the exponentiation of $a$ and $p$ and $a$ is equal to the product of $p$ and $q$ for some integer $q$.

Thm10FermatLittle. Let $p \in N$. Assume that $p$ is prime. Let $a \in Z$. Then there exists an integer $q$, such that the difference of the exponentiation of $a$ and $p$ and $a$ is equal to the product of $p$ and $q$.

Thm10FermatLittle. Let $p \in N$. Assume that $p$ is prime. Let $a \in Z$. Then the difference of the exponentiation of $a$ and $p$ and $a$ is equal to the product of $p$ and $q$ for an integer $q$.

Thm10FermatLittle. Let $p \in N$. Assume that $p$ is prime. Let $a \in Z$. Then the difference of the exponentiation of $a$ and $p$ and $a$ is equal to the product of $p$ and $q$ for some integer $q$.

Thm10FermatLittle. Let $p$ be a natural number. Assume that $p$ is prime. Let $a$ be an integer. Then there exists an integer $q$, such that $a ^ {p}- a = p q$.

Thm10FermatLittle. For all natural numbers $p$, if $p$ is prime, then for all integers $a$, there exists an integer $q$, such that $a ^ {p}- a = p q$.

Thm10FermatLittle. If $p$ is prime, then for all integers $a$, there exists an integer $q$, such that $a ^ {p}- a = p q$ for every natural number $p$.

Thm10FermatLittle. If $p$ is prime, then for all integers $a$, there exists an integer $q$, such that $a ^ {p}- a = p q$ for all natural numbers $p$.

Thm10FermatLittle. For all natural numbers $p$, if $p$ is prime, then for all integers $a$, $a ^ {p}- a = p q$ for an integer $q$.

Thm10FermatLittle. If $p$ is prime, then for all integers $a$, $a ^ {p}- a = p q$ for an integer $q$ for every natural number $p$.

Thm10FermatLittle. If $p$ is prime, then for all integers $a$, $a ^ {p}- a = p q$ for an integer $q$ for all natural numbers $p$.

Thm10FermatLittle. For all natural numbers $p$, if $p$ is prime, then for all integers $a$, $a ^ {p}- a = p q$ for some integer $q$.

Thm10FermatLittle. If $p$ is prime, then for all integers $a$, $a ^ {p}- a = p q$ for some integer $q$ for every natural number $p$.

Thm10FermatLittle. If $p$ is prime, then for all integers $a$, $a ^ {p}- a = p q$ for some integer $q$ for all natural numbers $p$.

Thm10FermatLittle. Let $p$ be a natural number. Then if $p$ is prime, then for all integers $a$, there exists an integer $q$, such that $a ^ {p}- a = p q$.

Thm10FermatLittle. Let $p$ be a natural number. Then $p$ is prime, only if for all integers $a$, there exists an integer $q$, such that $a ^ {p}- a = p q$.

Thm10FermatLittle. Let $p$ be a natural number. Then if $p$ is prime, then for all integers $a$, $a ^ {p}- a = p q$ for an integer $q$.

Thm10FermatLittle. Let $p$ be a natural number. Then $p$ is prime, only if for all integers $a$, $a ^ {p}- a = p q$ for an integer $q$.

Thm10FermatLittle. Let $p$ be a natural number. Then if $p$ is prime, then for all integers $a$, $a ^ {p}- a = p q$ for some integer $q$.

Thm10FermatLittle. Let $p$ be a natural number. Then $p$ is prime, only if for all integers $a$, $a ^ {p}- a = p q$ for some integer $q$.

Thm10FermatLittle. Let $p \in N$. Then if $p$ is prime, then for all integers $a$, there exists an integer $q$, such that $a ^ {p}- a = p q$.

Thm10FermatLittle. Let $p \in N$. Then $p$ is prime, only if for all integers $a$, there exists an integer $q$, such that $a ^ {p}- a = p q$.

Thm10FermatLittle. Let $p \in N$. Then if $p$ is prime, then for all integers $a$, $a ^ {p}- a = p q$ for an integer $q$.

Thm10FermatLittle. Let $p \in N$. Then $p$ is prime, only if for all integers $a$, $a ^ {p}- a = p q$ for an integer $q$.

Thm10FermatLittle. Let $p \in N$. Then if $p$ is prime, then for all integers $a$, $a ^ {p}- a = p q$ for some integer $q$.

Thm10FermatLittle. Let $p \in N$. Then $p$ is prime, only if for all integers $a$, $a ^ {p}- a = p q$ for some integer $q$.

Thm10FermatLittle. Let $p$ be a natural number. Assume that $p$ is prime. Then for all integers $a$, there exists an integer $q$, such that $a ^ {p}- a = p q$.

Thm10FermatLittle. Let $p$ be a natural number. Assume that $p$ is prime. Then there exists an integer $q$, such that $a ^ {p}- a = p q$ for every integer $a$.

Thm10FermatLittle. Let $p$ be a natural number. Assume that $p$ is prime. Then there exists an integer $q$, such that $a ^ {p}- a = p q$ for all integers $a$.

Thm10FermatLittle. Let $p$ be a natural number. Assume that $p$ is prime. Then for all integers $a$, $a ^ {p}- a = p q$ for an integer $q$.

Thm10FermatLittle. Let $p$ be a natural number. Assume that $p$ is prime. Then $a ^ {p}- a = p q$ for an integer $q$ for every integer $a$.

Thm10FermatLittle. Let $p$ be a natural number. Assume that $p$ is prime. Then $a ^ {p}- a = p q$ for an integer $q$ for all integers $a$.

Thm10FermatLittle. Let $p$ be a natural number. Assume that $p$ is prime. Then for all integers $a$, $a ^ {p}- a = p q$ for some integer $q$.

Thm10FermatLittle. Let $p$ be a natural number. Assume that $p$ is prime. Then $a ^ {p}- a = p q$ for some integer $q$ for every integer $a$.

Thm10FermatLittle. Let $p$ be a natural number. Assume that $p$ is prime. Then $a ^ {p}- a = p q$ for some integer $q$ for all integers $a$.

Thm10FermatLittle. Let $p \in N$. Assume that $p$ is prime. Then for all integers $a$, there exists an integer $q$, such that $a ^ {p}- a = p q$.

Thm10FermatLittle. Let $p \in N$. Assume that $p$ is prime. Then there exists an integer $q$, such that $a ^ {p}- a = p q$ for every integer $a$.

Thm10FermatLittle. Let $p \in N$. Assume that $p$ is prime. Then there exists an integer $q$, such that $a ^ {p}- a = p q$ for all integers $a$.

Thm10FermatLittle. Let $p \in N$. Assume that $p$ is prime. Then for all integers $a$, $a ^ {p}- a = p q$ for an integer $q$.

Thm10FermatLittle. Let $p \in N$. Assume that $p$ is prime. Then $a ^ {p}- a = p q$ for an integer $q$ for every integer $a$.

Thm10FermatLittle. Let $p \in N$. Assume that $p$ is prime. Then $a ^ {p}- a = p q$ for an integer $q$ for all integers $a$.

Thm10FermatLittle. Let $p \in N$. Assume that $p$ is prime. Then for all integers $a$, $a ^ {p}- a = p q$ for some integer $q$.

Thm10FermatLittle. Let $p \in N$. Assume that $p$ is prime. Then $a ^ {p}- a = p q$ for some integer $q$ for every integer $a$.

Thm10FermatLittle. Let $p \in N$. Assume that $p$ is prime. Then $a ^ {p}- a = p q$ for some integer $q$ for all integers $a$.

Thm10FermatLittle. Let $p$ be a natural number. Assume that $p$ is prime. Let $a$ be an integer. Then $a ^ {p}- a = p q$ for an integer $q$.

Thm10FermatLittle. Let $p$ be a natural number. Assume that $p$ is prime. Let $a$ be an integer. Then $a ^ {p}- a = p q$ for some integer $q$.

Thm10FermatLittle. Let $p$ be a natural number. Assume that $p$ is prime. Let $a \in Z$. Then there exists an integer $q$, such that $a ^ {p}- a = p q$.

Thm10FermatLittle. Let $p$ be a natural number. Assume that $p$ is prime. Let $a \in Z$. Then $a ^ {p}- a = p q$ for an integer $q$.

Thm10FermatLittle. Let $p$ be a natural number. Assume that $p$ is prime. Let $a \in Z$. Then $a ^ {p}- a = p q$ for some integer $q$.

Thm10FermatLittle. Let $p \in N$. Assume that $p$ is prime. Let $a$ be an integer. Then there exists an integer $q$, such that $a ^ {p}- a = p q$.

Thm10FermatLittle. Let $p \in N$. Assume that $p$ is prime. Let $a$ be an integer. Then $a ^ {p}- a = p q$ for an integer $q$.

Thm10FermatLittle. Let $p \in N$. Assume that $p$ is prime. Let $a$ be an integer. Then $a ^ {p}- a = p q$ for some integer $q$.

Thm10FermatLittle. Let $p \in N$. Assume that $p$ is prime. Let $a \in Z$. Then there exists an integer $q$, such that $a ^ {p}- a = p q$.

Thm10FermatLittle. Let $p \in N$. Assume that $p$ is prime. Let $a \in Z$. Then $a ^ {p}- a = p q$ for an integer $q$.

Thm10FermatLittle. Let $p \in N$. Assume that $p$ is prime. Let $a \in Z$. Then $a ^ {p}- a = p q$ for some integer $q$.

Thm11. Let $n$ be an instance of natural numbers. Then we can prove that there exists a natural number $p$, such that $p$ is greater than or equal to $n$ and $p$ is prime.

Thm11. For all instances $n$ of natural numbers, we can prove that there exists a natural number $p$, such that $p$ is greater than or equal to $n$ and $p$ is prime.

Thm11. We can prove that there exists a natural number $p$, such that $p$ is greater than or equal to $n$ and $p$ is prime for every instance $n$ of natural numbers.

Thm11. We can prove that there exists a natural number $p$, such that $p$ is greater than or equal to $n$ and $p$ is prime for all instances $n$ of natural numbers.

Thm11. For all instances $n$ of natural numbers, we can prove that $p$ is greater than or equal to $n$ and $p$ is prime for a natural number $p$.

Thm11. We can prove that $p$ is greater than or equal to $n$ and $p$ is prime for a natural number $p$ for every instance $n$ of natural numbers.

Thm11. We can prove that $p$ is greater than or equal to $n$ and $p$ is prime for a natural number $p$ for all instances $n$ of natural numbers.

Thm11. For all instances $n$ of natural numbers, we can prove that $p$ is greater than or equal to $n$ and $p$ is prime for some natural number $p$.

Thm11. We can prove that $p$ is greater than or equal to $n$ and $p$ is prime for some natural number $p$ for every instance $n$ of natural numbers.

Thm11. We can prove that $p$ is greater than or equal to $n$ and $p$ is prime for some natural number $p$ for all instances $n$ of natural numbers.

Thm11. Let $n$ be an instance of natural numbers. Then we can prove that $p$ is greater than or equal to $n$ and $p$ is prime for a natural number $p$.

Thm11. Let $n$ be an instance of natural numbers. Then we can prove that $p$ is greater than or equal to $n$ and $p$ is prime for some natural number $p$.

Thm11. Let $n$ be a natural number. Then there exists a natural number $p$, such that $p$ is greater than or equal to $n$ and $p$ is prime.

Thm11. For all natural numbers $n$, there exists a natural number $p$, such that $p$ is greater than or equal to $n$ and $p$ is prime.

Thm11. There exists a natural number $p$, such that $p$ is greater than or equal to $n$ and $p$ is prime for every natural number $n$.

Thm11. There exists a natural number $p$, such that $p$ is greater than or equal to $n$ and $p$ is prime for all natural numbers $n$.

Thm11. For all natural numbers $n$, $p$ is greater than or equal to $n$ and $p$ is prime for a natural number $p$.

Thm11. $p$ is greater than or equal to $n$ and $p$ is prime for a natural number $p$ for every natural number $n$.

Thm11. $p$ is greater than or equal to $n$ and $p$ is prime for a natural number $p$ for all natural numbers $n$.

Thm11. For all natural numbers $n$, $p$ is greater than or equal to $n$ and $p$ is prime for some natural number $p$.

Thm11. $p$ is greater than or equal to $n$ and $p$ is prime for some natural number $p$ for every natural number $n$.

Thm11. $p$ is greater than or equal to $n$ and $p$ is prime for some natural number $p$ for all natural numbers $n$.

Thm11. Let $n$ be a natural number. Then $p$ is greater than or equal to $n$ and $p$ is prime for a natural number $p$.

Thm11. Let $n$ be a natural number. Then $p$ is greater than or equal to $n$ and $p$ is prime for some natural number $p$.

Thm11. Let $n \in N$. Then there exists a natural number $p$, such that $p$ is greater than or equal to $n$ and $p$ is prime.

Thm11. Let $n \in N$. Then $p$ is greater than or equal to $n$ and $p$ is prime for a natural number $p$.

Thm11. Let $n \in N$. Then $p$ is greater than or equal to $n$ and $p$ is prime for some natural number $p$.

Thm11. Let $n$ be a natural number. Then there exists a natural number $p$, such that $p \geq n$ and $p$ is prime.

Thm11. For all natural numbers $n$, there exists a natural number $p$, such that $p \geq n$ and $p$ is prime.

Thm11. There exists a natural number $p$, such that $p \geq n$ and $p$ is prime for every natural number $n$.

Thm11. There exists a natural number $p$, such that $p \geq n$ and $p$ is prime for all natural numbers $n$.

Thm11. For all natural numbers $n$, $p \geq n$ and $p$ is prime for a natural number $p$.

Thm11. $p \geq n$ and $p$ is prime for a natural number $p$ for every natural number $n$.

Thm11. $p \geq n$ and $p$ is prime for a natural number $p$ for all natural numbers $n$.

Thm11. For all natural numbers $n$, $p \geq n$ and $p$ is prime for some natural number $p$.

Thm11. $p \geq n$ and $p$ is prime for some natural number $p$ for every natural number $n$.

Thm11. $p \geq n$ and $p$ is prime for some natural number $p$ for all natural numbers $n$.

Thm11. Let $n$ be a natural number. Then $p \geq n$ and $p$ is prime for a natural number $p$.

Thm11. Let $n$ be a natural number. Then $p \geq n$ and $p$ is prime for some natural number $p$.

Thm11. Let $n \in N$. Then there exists a natural number $p$, such that $p \geq n$ and $p$ is prime.

Thm11. Let $n \in N$. Then $p \geq n$ and $p$ is prime for a natural number $p$.

Thm11. Let $n \in N$. Then $p \geq n$ and $p$ is prime for some natural number $p$.

Thm19. Let $n$ be an instance of natural numbers. Then we can prove that there exists a natural number $a$, such that there exists a natural number $b$, such that there exists a natural number $c$, such that there exists a natural number $d$, such that $n$ is equal to the sum of the sum of the sum of the square of $a$ and the square of $b$ and the square of $c$ and the square of $d$.

Thm19. For all instances $n$ of natural numbers, we can prove that there exists a natural number $a$, such that there exists a natural number $b$, such that there exists a natural number $c$, such that there exists a natural number $d$, such that $n$ is equal to the sum of the sum of the sum of the square of $a$ and the square of $b$ and the square of $c$ and the square of $d$.

Thm19. We can prove that there exists a natural number $a$, such that there exists a natural number $b$, such that there exists a natural number $c$, such that there exists a natural number $d$, such that $n$ is equal to the sum of the sum of the sum of the square of $a$ and the square of $b$ and the square of $c$ and the square of $d$ for every instance $n$ of natural numbers.

Thm19. We can prove that there exists a natural number $a$, such that there exists a natural number $b$, such that there exists a natural number $c$, such that there exists a natural number $d$, such that $n$ is equal to the sum of the sum of the sum of the square of $a$ and the square of $b$ and the square of $c$ and the square of $d$ for all instances $n$ of natural numbers.

Thm19. For all instances $n$ of natural numbers, we can prove that there exists a natural number $b$, such that there exists a natural number $c$, such that there exists a natural number $d$, such that $n$ is equal to the sum of the sum of the sum of the square of $a$ and the square of $b$ and the square of $c$ and the square of $d$ for a natural number $a$.

Thm19. We can prove that there exists a natural number $b$, such that there exists a natural number $c$, such that there exists a natural number $d$, such that $n$ is equal to the sum of the sum of the sum of the square of $a$ and the square of $b$ and the square of $c$ and the square of $d$ for a natural number $a$ for every instance $n$ of natural numbers.

Thm19. We can prove that there exists a natural number $b$, such that there exists a natural number $c$, such that there exists a natural number $d$, such that $n$ is equal to the sum of the sum of the sum of the square of $a$ and the square of $b$ and the square of $c$ and the square of $d$ for a natural number $a$ for all instances $n$ of natural numbers.

Thm19. For all instances $n$ of natural numbers, we can prove that there exists a natural number $b$, such that there exists a natural number $c$, such that there exists a natural number $d$, such that $n$ is equal to the sum of the sum of the sum of the square of $a$ and the square of $b$ and the square of $c$ and the square of $d$ for some natural number $a$.

Thm19. We can prove that there exists a natural number $b$, such that there exists a natural number $c$, such that there exists a natural number $d$, such that $n$ is equal to the sum of the sum of the sum of the square of $a$ and the square of $b$ and the square of $c$ and the square of $d$ for some natural number $a$ for every instance $n$ of natural numbers.

Thm19. We can prove that there exists a natural number $b$, such that there exists a natural number $c$, such that there exists a natural number $d$, such that $n$ is equal to the sum of the sum of the sum of the square of $a$ and the square of $b$ and the square of $c$ and the square of $d$ for some natural number $a$ for all instances $n$ of natural numbers.

Thm19. Let $n$ be an instance of natural numbers. Then we can prove that there exists a natural number $b$, such that there exists a natural number $c$, such that there exists a natural number $d$, such that $n$ is equal to the sum of the sum of the sum of the square of $a$ and the square of $b$ and the square of $c$ and the square of $d$ for a natural number $a$.

Thm19. Let $n$ be an instance of natural numbers. Then we can prove that there exists a natural number $b$, such that there exists a natural number $c$, such that there exists a natural number $d$, such that $n$ is equal to the sum of the sum of the sum of the square of $a$ and the square of $b$ and the square of $c$ and the square of $d$ for some natural number $a$.

Thm19. Let $n$ be an instance of natural numbers. Then we can prove that there exists a natural number $c$, such that there exists a natural number $d$, such that $n$ is equal to the sum of the sum of the sum of the square of $a$ and the square of $b$ and the square of $c$ and the square of $d$ for a natural number $b$ for a natural number $a$.

Thm19. Let $n$ be an instance of natural numbers. Then we can prove that there exists a natural number $c$, such that there exists a natural number $d$, such that $n$ is equal to the sum of the sum of the sum of the square of $a$ and the square of $b$ and the square of $c$ and the square of $d$ for some natural number $b$ for a natural number $a$.

Thm19. Let $n$ be an instance of natural numbers. Then we can prove that there exists a natural number $c$, such that there exists a natural number $d$, such that $n$ is equal to the sum of the sum of the sum of the square of $a$ and the square of $b$ and the square of $c$ and the square of $d$ for a natural number $b$ for some natural number $a$.

Thm19. Let $n$ be an instance of natural numbers. Then we can prove that there exists a natural number $c$, such that there exists a natural number $d$, such that $n$ is equal to the sum of the sum of the sum of the square of $a$ and the square of $b$ and the square of $c$ and the square of $d$ for some natural number $b$ for some natural number $a$.

Thm19. Let $n$ be a natural number. Then there exists a natural number $a$, such that there exists a natural number $b$, such that there exists a natural number $c$, such that there exists a natural number $d$, such that $n$ is equal to the sum of the sum of the sum of the square of $a$ and the square of $b$ and the square of $c$ and the square of $d$.

Thm19. For all natural numbers $n$, there exists a natural number $a$, such that there exists a natural number $b$, such that there exists a natural number $c$, such that there exists a natural number $d$, such that $n$ is equal to the sum of the sum of the sum of the square of $a$ and the square of $b$ and the square of $c$ and the square of $d$.

Thm19. There exists a natural number $a$, such that there exists a natural number $b$, such that there exists a natural number $c$, such that there exists a natural number $d$, such that $n$ is equal to the sum of the sum of the sum of the square of $a$ and the square of $b$ and the square of $c$ and the square of $d$ for every natural number $n$.

Thm19. There exists a natural number $a$, such that there exists a natural number $b$, such that there exists a natural number $c$, such that there exists a natural number $d$, such that $n$ is equal to the sum of the sum of the sum of the square of $a$ and the square of $b$ and the square of $c$ and the square of $d$ for all natural numbers $n$.

Thm19. For all natural numbers $n$, there exists a natural number $b$, such that there exists a natural number $c$, such that there exists a natural number $d$, such that $n$ is equal to the sum of the sum of the sum of the square of $a$ and the square of $b$ and the square of $c$ and the square of $d$ for a natural number $a$.

Thm19. There exists a natural number $b$, such that there exists a natural number $c$, such that there exists a natural number $d$, such that $n$ is equal to the sum of the sum of the sum of the square of $a$ and the square of $b$ and the square of $c$ and the square of $d$ for a natural number $a$ for every natural number $n$.

Thm19. There exists a natural number $b$, such that there exists a natural number $c$, such that there exists a natural number $d$, such that $n$ is equal to the sum of the sum of the sum of the square of $a$ and the square of $b$ and the square of $c$ and the square of $d$ for a natural number $a$ for all natural numbers $n$.

Thm19. For all natural numbers $n$, there exists a natural number $b$, such that there exists a natural number $c$, such that there exists a natural number $d$, such that $n$ is equal to the sum of the sum of the sum of the square of $a$ and the square of $b$ and the square of $c$ and the square of $d$ for some natural number $a$.

Thm19. There exists a natural number $b$, such that there exists a natural number $c$, such that there exists a natural number $d$, such that $n$ is equal to the sum of the sum of the sum of the square of $a$ and the square of $b$ and the square of $c$ and the square of $d$ for some natural number $a$ for every natural number $n$.

Thm19. There exists a natural number $b$, such that there exists a natural number $c$, such that there exists a natural number $d$, such that $n$ is equal to the sum of the sum of the sum of the square of $a$ and the square of $b$ and the square of $c$ and the square of $d$ for some natural number $a$ for all natural numbers $n$.

Thm19. Let $n$ be a natural number. Then there exists a natural number $b$, such that there exists a natural number $c$, such that there exists a natural number $d$, such that $n$ is equal to the sum of the sum of the sum of the square of $a$ and the square of $b$ and the square of $c$ and the square of $d$ for a natural number $a$.

Thm19. Let $n$ be a natural number. Then there exists a natural number $b$, such that there exists a natural number $c$, such that there exists a natural number $d$, such that $n$ is equal to the sum of the sum of the sum of the square of $a$ and the square of $b$ and the square of $c$ and the square of $d$ for some natural number $a$.

Thm19. Let $n$ be a natural number. Then there exists a natural number $c$, such that there exists a natural number $d$, such that $n$ is equal to the sum of the sum of the sum of the square of $a$ and the square of $b$ and the square of $c$ and the square of $d$ for a natural number $b$ for a natural number $a$.

Thm19. Let $n$ be a natural number. Then there exists a natural number $c$, such that there exists a natural number $d$, such that $n$ is equal to the sum of the sum of the sum of the square of $a$ and the square of $b$ and the square of $c$ and the square of $d$ for some natural number $b$ for a natural number $a$.

Thm19. Let $n$ be a natural number. Then there exists a natural number $c$, such that there exists a natural number $d$, such that $n$ is equal to the sum of the sum of the sum of the square of $a$ and the square of $b$ and the square of $c$ and the square of $d$ for a natural number $b$ for some natural number $a$.

Thm19. Let $n$ be a natural number. Then there exists a natural number $c$, such that there exists a natural number $d$, such that $n$ is equal to the sum of the sum of the sum of the square of $a$ and the square of $b$ and the square of $c$ and the square of $d$ for some natural number $b$ for some natural number $a$.

Thm19. Let $n \in N$. Then there exists a natural number $a$, such that there exists a natural number $b$, such that there exists a natural number $c$, such that there exists a natural number $d$, such that $n$ is equal to the sum of the sum of the sum of the square of $a$ and the square of $b$ and the square of $c$ and the square of $d$.

Thm19. Let $n \in N$. Then there exists a natural number $b$, such that there exists a natural number $c$, such that there exists a natural number $d$, such that $n$ is equal to the sum of the sum of the sum of the square of $a$ and the square of $b$ and the square of $c$ and the square of $d$ for a natural number $a$.

Thm19. Let $n \in N$. Then there exists a natural number $b$, such that there exists a natural number $c$, such that there exists a natural number $d$, such that $n$ is equal to the sum of the sum of the sum of the square of $a$ and the square of $b$ and the square of $c$ and the square of $d$ for some natural number $a$.

Thm19. Let $n \in N$. Then there exists a natural number $c$, such that there exists a natural number $d$, such that $n$ is equal to the sum of the sum of the sum of the square of $a$ and the square of $b$ and the square of $c$ and the square of $d$ for a natural number $b$ for a natural number $a$.

Thm19. Let $n \in N$. Then there exists a natural number $c$, such that there exists a natural number $d$, such that $n$ is equal to the sum of the sum of the sum of the square of $a$ and the square of $b$ and the square of $c$ and the square of $d$ for some natural number $b$ for a natural number $a$.

Thm19. Let $n \in N$. Then there exists a natural number $c$, such that there exists a natural number $d$, such that $n$ is equal to the sum of the sum of the sum of the square of $a$ and the square of $b$ and the square of $c$ and the square of $d$ for a natural number $b$ for some natural number $a$.

Thm19. Let $n \in N$. Then there exists a natural number $c$, such that there exists a natural number $d$, such that $n$ is equal to the sum of the sum of the sum of the square of $a$ and the square of $b$ and the square of $c$ and the square of $d$ for some natural number $b$ for some natural number $a$.

Thm19. Let $n$ be a natural number. Then there exists a natural number $a$, such that there exists a natural number $b$, such that there exists a natural number $c$, such that there exists a natural number $d$, such that $n = a ^{ 2}+ b ^{ 2}+ c ^{ 2}+ d ^{ 2}$.

Thm19. For all natural numbers $n$, there exists a natural number $a$, such that there exists a natural number $b$, such that there exists a natural number $c$, such that there exists a natural number $d$, such that $n = a ^{ 2}+ b ^{ 2}+ c ^{ 2}+ d ^{ 2}$.

Thm19. There exists a natural number $a$, such that there exists a natural number $b$, such that there exists a natural number $c$, such that there exists a natural number $d$, such that $n = a ^{ 2}+ b ^{ 2}+ c ^{ 2}+ d ^{ 2}$ for every natural number $n$.

Thm19. There exists a natural number $a$, such that there exists a natural number $b$, such that there exists a natural number $c$, such that there exists a natural number $d$, such that $n = a ^{ 2}+ b ^{ 2}+ c ^{ 2}+ d ^{ 2}$ for all natural numbers $n$.

Thm19. For all natural numbers $n$, there exists a natural number $b$, such that there exists a natural number $c$, such that there exists a natural number $d$, such that $n = a ^{ 2}+ b ^{ 2}+ c ^{ 2}+ d ^{ 2}$ for a natural number $a$.

Thm19. There exists a natural number $b$, such that there exists a natural number $c$, such that there exists a natural number $d$, such that $n = a ^{ 2}+ b ^{ 2}+ c ^{ 2}+ d ^{ 2}$ for a natural number $a$ for every natural number $n$.

Thm19. There exists a natural number $b$, such that there exists a natural number $c$, such that there exists a natural number $d$, such that $n = a ^{ 2}+ b ^{ 2}+ c ^{ 2}+ d ^{ 2}$ for a natural number $a$ for all natural numbers $n$.

Thm19. For all natural numbers $n$, there exists a natural number $b$, such that there exists a natural number $c$, such that there exists a natural number $d$, such that $n = a ^{ 2}+ b ^{ 2}+ c ^{ 2}+ d ^{ 2}$ for some natural number $a$.

Thm19. There exists a natural number $b$, such that there exists a natural number $c$, such that there exists a natural number $d$, such that $n = a ^{ 2}+ b ^{ 2}+ c ^{ 2}+ d ^{ 2}$ for some natural number $a$ for every natural number $n$.

Thm19. There exists a natural number $b$, such that there exists a natural number $c$, such that there exists a natural number $d$, such that $n = a ^{ 2}+ b ^{ 2}+ c ^{ 2}+ d ^{ 2}$ for some natural number $a$ for all natural numbers $n$.

Thm19. Let $n$ be a natural number. Then there exists a natural number $b$, such that there exists a natural number $c$, such that there exists a natural number $d$, such that $n = a ^{ 2}+ b ^{ 2}+ c ^{ 2}+ d ^{ 2}$ for a natural number $a$.

Thm19. Let $n$ be a natural number. Then there exists a natural number $b$, such that there exists a natural number $c$, such that there exists a natural number $d$, such that $n = a ^{ 2}+ b ^{ 2}+ c ^{ 2}+ d ^{ 2}$ for some natural number $a$.

Thm19. Let $n$ be a natural number. Then there exists a natural number $c$, such that there exists a natural number $d$, such that $n = a ^{ 2}+ b ^{ 2}+ c ^{ 2}+ d ^{ 2}$ for a natural number $b$ for a natural number $a$.

Thm19. Let $n$ be a natural number. Then there exists a natural number $c$, such that there exists a natural number $d$, such that $n = a ^{ 2}+ b ^{ 2}+ c ^{ 2}+ d ^{ 2}$ for some natural number $b$ for a natural number $a$.

Thm19. Let $n$ be a natural number. Then there exists a natural number $c$, such that there exists a natural number $d$, such that $n = a ^{ 2}+ b ^{ 2}+ c ^{ 2}+ d ^{ 2}$ for a natural number $b$ for some natural number $a$.

Thm19. Let $n$ be a natural number. Then there exists a natural number $c$, such that there exists a natural number $d$, such that $n = a ^{ 2}+ b ^{ 2}+ c ^{ 2}+ d ^{ 2}$ for some natural number $b$ for some natural number $a$.

Thm19. Let $n \in N$. Then there exists a natural number $a$, such that there exists a natural number $b$, such that there exists a natural number $c$, such that there exists a natural number $d$, such that $n = a ^{ 2}+ b ^{ 2}+ c ^{ 2}+ d ^{ 2}$.

Thm19. Let $n \in N$. Then there exists a natural number $b$, such that there exists a natural number $c$, such that there exists a natural number $d$, such that $n = a ^{ 2}+ b ^{ 2}+ c ^{ 2}+ d ^{ 2}$ for a natural number $a$.

Thm19. Let $n \in N$. Then there exists a natural number $b$, such that there exists a natural number $c$, such that there exists a natural number $d$, such that $n = a ^{ 2}+ b ^{ 2}+ c ^{ 2}+ d ^{ 2}$ for some natural number $a$.

Thm19. Let $n \in N$. Then there exists a natural number $c$, such that there exists a natural number $d$, such that $n = a ^{ 2}+ b ^{ 2}+ c ^{ 2}+ d ^{ 2}$ for a natural number $b$ for a natural number $a$.

Thm19. Let $n \in N$. Then there exists a natural number $c$, such that there exists a natural number $d$, such that $n = a ^{ 2}+ b ^{ 2}+ c ^{ 2}+ d ^{ 2}$ for some natural number $b$ for a natural number $a$.

Thm19. Let $n \in N$. Then there exists a natural number $c$, such that there exists a natural number $d$, such that $n = a ^{ 2}+ b ^{ 2}+ c ^{ 2}+ d ^{ 2}$ for a natural number $b$ for some natural number $a$.

Thm19. Let $n \in N$. Then there exists a natural number $c$, such that there exists a natural number $d$, such that $n = a ^{ 2}+ b ^{ 2}+ c ^{ 2}+ d ^{ 2}$ for some natural number $b$ for some natural number $a$.

Thm19. Let $n$ be a natural number. Then there exist natural numbers $a$, $b$, $c$ and $d$, such that $n = a ^{ 2}+ b ^{ 2}+ c ^{ 2}+ d ^{ 2}$.

Thm19. For all natural numbers $n$, there exist natural numbers $a$, $b$, $c$ and $d$, such that $n = a ^{ 2}+ b ^{ 2}+ c ^{ 2}+ d ^{ 2}$.

Thm19. There exist natural numbers $a$, $b$, $c$ and $d$, such that $n = a ^{ 2}+ b ^{ 2}+ c ^{ 2}+ d ^{ 2}$ for every natural number $n$.

Thm19. There exist natural numbers $a$, $b$, $c$ and $d$, such that $n = a ^{ 2}+ b ^{ 2}+ c ^{ 2}+ d ^{ 2}$ for all natural numbers $n$.

Thm19. For all natural numbers $n$, $n = a ^{ 2}+ b ^{ 2}+ c ^{ 2}+ d ^{ 2}$ for some natural numbers $a$, $b$, $c$ and $d$.

Thm19. $n = a ^{ 2}+ b ^{ 2}+ c ^{ 2}+ d ^{ 2}$ for some natural numbers $a$, $b$, $c$ and $d$ for every natural number $n$.

Thm19. $n = a ^{ 2}+ b ^{ 2}+ c ^{ 2}+ d ^{ 2}$ for some natural numbers $a$, $b$, $c$ and $d$ for all natural numbers $n$.

Thm19. Let $n$ be a natural number. Then $n = a ^{ 2}+ b ^{ 2}+ c ^{ 2}+ d ^{ 2}$ for some natural numbers $a$, $b$, $c$ and $d$.

Thm19. Let $n \in N$. Then there exist natural numbers $a$, $b$, $c$ and $d$, such that $n = a ^{ 2}+ b ^{ 2}+ c ^{ 2}+ d ^{ 2}$.

Thm19. Let $n \in N$. Then $n = a ^{ 2}+ b ^{ 2}+ c ^{ 2}+ d ^{ 2}$ for some natural numbers $a$, $b$, $c$ and $d$.

Thm20a. Let $p$ be an instance of natural numbers. Assume that we can prove that $p$ is prime. Let $k$ be an instance of natural numbers. Assume that we can prove that $p$ is equal to the sum of the product of $4$ and $k$ and $1$. Then we can prove that there exists a natural number $x$, such that there exists a natural number $y$, such that $p$ is equal to the sum of the square of $x$ and the square of $y$.

Thm20a. For all instances $p$ of natural numbers, if we can prove that $p$ is prime, then for all instances $k$ of natural numbers, if we can prove that $p$ is equal to the sum of the product of $4$ and $k$ and $1$, then we can prove that there exists a natural number $x$, such that there exists a natural number $y$, such that $p$ is equal to the sum of the square of $x$ and the square of $y$.

Thm20a. If we can prove that $p$ is prime, then for all instances $k$ of natural numbers, if we can prove that $p$ is equal to the sum of the product of $4$ and $k$ and $1$, then we can prove that there exists a natural number $x$, such that there exists a natural number $y$, such that $p$ is equal to the sum of the square of $x$ and the square of $y$ for every instance $p$ of natural numbers.

Thm20a. If we can prove that $p$ is prime, then for all instances $k$ of natural numbers, if we can prove that $p$ is equal to the sum of the product of $4$ and $k$ and $1$, then we can prove that there exists a natural number $x$, such that there exists a natural number $y$, such that $p$ is equal to the sum of the square of $x$ and the square of $y$ for all instances $p$ of natural numbers.

Thm20a. For all instances $p$ of natural numbers, if we can prove that $p$ is prime, then for all instances $k$ of natural numbers, if we can prove that $p$ is equal to the sum of the product of $4$ and $k$ and $1$, then we can prove that there exists a natural number $y$, such that $p$ is equal to the sum of the square of $x$ and the square of $y$ for a natural number $x$.

Thm20a. If we can prove that $p$ is prime, then for all instances $k$ of natural numbers, if we can prove that $p$ is equal to the sum of the product of $4$ and $k$ and $1$, then we can prove that there exists a natural number $y$, such that $p$ is equal to the sum of the square of $x$ and the square of $y$ for a natural number $x$ for every instance $p$ of natural numbers.

Thm20a. If we can prove that $p$ is prime, then for all instances $k$ of natural numbers, if we can prove that $p$ is equal to the sum of the product of $4$ and $k$ and $1$, then we can prove that there exists a natural number $y$, such that $p$ is equal to the sum of the square of $x$ and the square of $y$ for a natural number $x$ for all instances $p$ of natural numbers.

Thm20a. For all instances $p$ of natural numbers, if we can prove that $p$ is prime, then for all instances $k$ of natural numbers, if we can prove that $p$ is equal to the sum of the product of $4$ and $k$ and $1$, then we can prove that there exists a natural number $y$, such that $p$ is equal to the sum of the square of $x$ and the square of $y$ for some natural number $x$.

Thm20a. If we can prove that $p$ is prime, then for all instances $k$ of natural numbers, if we can prove that $p$ is equal to the sum of the product of $4$ and $k$ and $1$, then we can prove that there exists a natural number $y$, such that $p$ is equal to the sum of the square of $x$ and the square of $y$ for some natural number $x$ for every instance $p$ of natural numbers.

Thm20a. If we can prove that $p$ is prime, then for all instances $k$ of natural numbers, if we can prove that $p$ is equal to the sum of the product of $4$ and $k$ and $1$, then we can prove that there exists a natural number $y$, such that $p$ is equal to the sum of the square of $x$ and the square of $y$ for some natural number $x$ for all instances $p$ of natural numbers.

Thm20a. Let $p$ be an instance of natural numbers. Then if we can prove that $p$ is prime, then for all instances $k$ of natural numbers, if we can prove that $p$ is equal to the sum of the product of $4$ and $k$ and $1$, then we can prove that there exists a natural number $x$, such that there exists a natural number $y$, such that $p$ is equal to the sum of the square of $x$ and the square of $y$.

Thm20a. Let $p$ be an instance of natural numbers. Then we can prove that $p$ is prime, only if for all instances $k$ of natural numbers, if we can prove that $p$ is equal to the sum of the product of $4$ and $k$ and $1$, then we can prove that there exists a natural number $x$, such that there exists a natural number $y$, such that $p$ is equal to the sum of the square of $x$ and the square of $y$.

Thm20a. Let $p$ be an instance of natural numbers. Then if we can prove that $p$ is prime, then for all instances $k$ of natural numbers, if we can prove that $p$ is equal to the sum of the product of $4$ and $k$ and $1$, then we can prove that there exists a natural number $y$, such that $p$ is equal to the sum of the square of $x$ and the square of $y$ for a natural number $x$.

Thm20a. Let $p$ be an instance of natural numbers. Then we can prove that $p$ is prime, only if for all instances $k$ of natural numbers, if we can prove that $p$ is equal to the sum of the product of $4$ and $k$ and $1$, then we can prove that there exists a natural number $y$, such that $p$ is equal to the sum of the square of $x$ and the square of $y$ for a natural number $x$.

Thm20a. Let $p$ be an instance of natural numbers. Then if we can prove that $p$ is prime, then for all instances $k$ of natural numbers, if we can prove that $p$ is equal to the sum of the product of $4$ and $k$ and $1$, then we can prove that there exists a natural number $y$, such that $p$ is equal to the sum of the square of $x$ and the square of $y$ for some natural number $x$.

Thm20a. Let $p$ be an instance of natural numbers. Then we can prove that $p$ is prime, only if for all instances $k$ of natural numbers, if we can prove that $p$ is equal to the sum of the product of $4$ and $k$ and $1$, then we can prove that there exists a natural number $y$, such that $p$ is equal to the sum of the square of $x$ and the square of $y$ for some natural number $x$.

Thm20a. Let $p$ be an instance of natural numbers. Assume that we can prove that $p$ is prime. Then for all instances $k$ of natural numbers, if we can prove that $p$ is equal to the sum of the product of $4$ and $k$ and $1$, then we can prove that there exists a natural number $x$, such that there exists a natural number $y$, such that $p$ is equal to the sum of the square of $x$ and the square of $y$.

Thm20a. Let $p$ be an instance of natural numbers. Assume that we can prove that $p$ is prime. Then if we can prove that $p$ is equal to the sum of the product of $4$ and $k$ and $1$, then we can prove that there exists a natural number $x$, such that there exists a natural number $y$, such that $p$ is equal to the sum of the square of $x$ and the square of $y$ for every instance $k$ of natural numbers.

Thm20a. Let $p$ be an instance of natural numbers. Assume that we can prove that $p$ is prime. Then if we can prove that $p$ is equal to the sum of the product of $4$ and $k$ and $1$, then we can prove that there exists a natural number $x$, such that there exists a natural number $y$, such that $p$ is equal to the sum of the square of $x$ and the square of $y$ for all instances $k$ of natural numbers.

Thm20a. Let $p$ be an instance of natural numbers. Assume that we can prove that $p$ is prime. Then for all instances $k$ of natural numbers, if we can prove that $p$ is equal to the sum of the product of $4$ and $k$ and $1$, then we can prove that there exists a natural number $y$, such that $p$ is equal to the sum of the square of $x$ and the square of $y$ for a natural number $x$.

Thm20a. Let $p$ be an instance of natural numbers. Assume that we can prove that $p$ is prime. Then if we can prove that $p$ is equal to the sum of the product of $4$ and $k$ and $1$, then we can prove that there exists a natural number $y$, such that $p$ is equal to the sum of the square of $x$ and the square of $y$ for a natural number $x$ for every instance $k$ of natural numbers.

Thm20a. Let $p$ be an instance of natural numbers. Assume that we can prove that $p$ is prime. Then if we can prove that $p$ is equal to the sum of the product of $4$ and $k$ and $1$, then we can prove that there exists a natural number $y$, such that $p$ is equal to the sum of the square of $x$ and the square of $y$ for a natural number $x$ for all instances $k$ of natural numbers.

Thm20a. Let $p$ be an instance of natural numbers. Assume that we can prove that $p$ is prime. Then for all instances $k$ of natural numbers, if we can prove that $p$ is equal to the sum of the product of $4$ and $k$ and $1$, then we can prove that there exists a natural number $y$, such that $p$ is equal to the sum of the square of $x$ and the square of $y$ for some natural number $x$.

Thm20a. Let $p$ be an instance of natural numbers. Assume that we can prove that $p$ is prime. Then if we can prove that $p$ is equal to the sum of the product of $4$ and $k$ and $1$, then we can prove that there exists a natural number $y$, such that $p$ is equal to the sum of the square of $x$ and the square of $y$ for some natural number $x$ for every instance $k$ of natural numbers.

Thm20a. Let $p$ be an instance of natural numbers. Assume that we can prove that $p$ is prime. Then if we can prove that $p$ is equal to the sum of the product of $4$ and $k$ and $1$, then we can prove that there exists a natural number $y$, such that $p$ is equal to the sum of the square of $x$ and the square of $y$ for some natural number $x$ for all instances $k$ of natural numbers.

Thm20a. Let $p$ be an instance of natural numbers. Assume that we can prove that $p$ is prime. Let $k$ be an instance of natural numbers. Then if we can prove that $p$ is equal to the sum of the product of $4$ and $k$ and $1$, then we can prove that there exists a natural number $x$, such that there exists a natural number $y$, such that $p$ is equal to the sum of the square of $x$ and the square of $y$.

Thm20a. Let $p$ be an instance of natural numbers. Assume that we can prove that $p$ is prime. Let $k$ be an instance of natural numbers. Then we can prove that $p$ is equal to the sum of the product of $4$ and $k$ and $1$, only if we can prove that there exists a natural number $x$, such that there exists a natural number $y$, such that $p$ is equal to the sum of the square of $x$ and the square of $y$.

Thm20a. Let $p$ be an instance of natural numbers. Assume that we can prove that $p$ is prime. Let $k$ be an instance of natural numbers. Then if we can prove that $p$ is equal to the sum of the product of $4$ and $k$ and $1$, then we can prove that there exists a natural number $y$, such that $p$ is equal to the sum of the square of $x$ and the square of $y$ for a natural number $x$.

Thm20a. Let $p$ be an instance of natural numbers. Assume that we can prove that $p$ is prime. Let $k$ be an instance of natural numbers. Then we can prove that $p$ is equal to the sum of the product of $4$ and $k$ and $1$, only if we can prove that there exists a natural number $y$, such that $p$ is equal to the sum of the square of $x$ and the square of $y$ for a natural number $x$.

Thm20a. Let $p$ be an instance of natural numbers. Assume that we can prove that $p$ is prime. Let $k$ be an instance of natural numbers. Then if we can prove that $p$ is equal to the sum of the product of $4$ and $k$ and $1$, then we can prove that there exists a natural number $y$, such that $p$ is equal to the sum of the square of $x$ and the square of $y$ for some natural number $x$.

Thm20a. Let $p$ be an instance of natural numbers. Assume that we can prove that $p$ is prime. Let $k$ be an instance of natural numbers. Then we can prove that $p$ is equal to the sum of the product of $4$ and $k$ and $1$, only if we can prove that there exists a natural number $y$, such that $p$ is equal to the sum of the square of $x$ and the square of $y$ for some natural number $x$.

Thm20a. Let $p$ be an instance of natural numbers. Assume that we can prove that $p$ is prime. Let $k$ be an instance of natural numbers. Assume that we can prove that $p$ is equal to the sum of the product of $4$ and $k$ and $1$. Then we can prove that there exists a natural number $y$, such that $p$ is equal to the sum of the square of $x$ and the square of $y$ for a natural number $x$.

Thm20a. Let $p$ be an instance of natural numbers. Assume that we can prove that $p$ is prime. Let $k$ be an instance of natural numbers. Assume that we can prove that $p$ is equal to the sum of the product of $4$ and $k$ and $1$. Then we can prove that there exists a natural number $y$, such that $p$ is equal to the sum of the square of $x$ and the square of $y$ for some natural number $x$.

Thm20a. Let $p$ be an instance of natural numbers. Assume that we can prove that $p$ is prime. Let $k$ be an instance of natural numbers. Assume that we can prove that $p$ is equal to the sum of the product of $4$ and $k$ and $1$. Then we can prove that $p$ is equal to the sum of the square of $x$ and the square of $y$ for a natural number $y$ for a natural number $x$.

Thm20a. Let $p$ be an instance of natural numbers. Assume that we can prove that $p$ is prime. Let $k$ be an instance of natural numbers. Assume that we can prove that $p$ is equal to the sum of the product of $4$ and $k$ and $1$. Then we can prove that $p$ is equal to the sum of the square of $x$ and the square of $y$ for some natural number $y$ for a natural number $x$.

Thm20a. Let $p$ be an instance of natural numbers. Assume that we can prove that $p$ is prime. Let $k$ be an instance of natural numbers. Assume that we can prove that $p$ is equal to the sum of the product of $4$ and $k$ and $1$. Then we can prove that $p$ is equal to the sum of the square of $x$ and the square of $y$ for a natural number $y$ for some natural number $x$.

Thm20a. Let $p$ be an instance of natural numbers. Assume that we can prove that $p$ is prime. Let $k$ be an instance of natural numbers. Assume that we can prove that $p$ is equal to the sum of the product of $4$ and $k$ and $1$. Then we can prove that $p$ is equal to the sum of the square of $x$ and the square of $y$ for some natural number $y$ for some natural number $x$.

Thm20a. Let $p$ be a natural number. Assume that $p$ is prime. Let $k$ be a natural number. Assume that $p$ is equal to the sum of the product of $4$ and $k$ and $1$. Then there exists a natural number $x$, such that there exists a natural number $y$, such that $p$ is equal to the sum of the square of $x$ and the square of $y$.

Thm20a. For all natural numbers $p$, if $p$ is prime, then for all natural numbers $k$, if $p$ is equal to the sum of the product of $4$ and $k$ and $1$, then there exists a natural number $x$, such that there exists a natural number $y$, such that $p$ is equal to the sum of the square of $x$ and the square of $y$.

Thm20a. If $p$ is prime, then for all natural numbers $k$, if $p$ is equal to the sum of the product of $4$ and $k$ and $1$, then there exists a natural number $x$, such that there exists a natural number $y$, such that $p$ is equal to the sum of the square of $x$ and the square of $y$ for every natural number $p$.

Thm20a. If $p$ is prime, then for all natural numbers $k$, if $p$ is equal to the sum of the product of $4$ and $k$ and $1$, then there exists a natural number $x$, such that there exists a natural number $y$, such that $p$ is equal to the sum of the square of $x$ and the square of $y$ for all natural numbers $p$.

Thm20a. For all natural numbers $p$, if $p$ is prime, then for all natural numbers $k$, if $p$ is equal to the sum of the product of $4$ and $k$ and $1$, then there exists a natural number $y$, such that $p$ is equal to the sum of the square of $x$ and the square of $y$ for a natural number $x$.

Thm20a. If $p$ is prime, then for all natural numbers $k$, if $p$ is equal to the sum of the product of $4$ and $k$ and $1$, then there exists a natural number $y$, such that $p$ is equal to the sum of the square of $x$ and the square of $y$ for a natural number $x$ for every natural number $p$.

Thm20a. If $p$ is prime, then for all natural numbers $k$, if $p$ is equal to the sum of the product of $4$ and $k$ and $1$, then there exists a natural number $y$, such that $p$ is equal to the sum of the square of $x$ and the square of $y$ for a natural number $x$ for all natural numbers $p$.

Thm20a. For all natural numbers $p$, if $p$ is prime, then for all natural numbers $k$, if $p$ is equal to the sum of the product of $4$ and $k$ and $1$, then there exists a natural number $y$, such that $p$ is equal to the sum of the square of $x$ and the square of $y$ for some natural number $x$.

Thm20a. If $p$ is prime, then for all natural numbers $k$, if $p$ is equal to the sum of the product of $4$ and $k$ and $1$, then there exists a natural number $y$, such that $p$ is equal to the sum of the square of $x$ and the square of $y$ for some natural number $x$ for every natural number $p$.

Thm20a. If $p$ is prime, then for all natural numbers $k$, if $p$ is equal to the sum of the product of $4$ and $k$ and $1$, then there exists a natural number $y$, such that $p$ is equal to the sum of the square of $x$ and the square of $y$ for some natural number $x$ for all natural numbers $p$.

Thm20a. Let $p$ be a natural number. Then if $p$ is prime, then for all natural numbers $k$, if $p$ is equal to the sum of the product of $4$ and $k$ and $1$, then there exists a natural number $x$, such that there exists a natural number $y$, such that $p$ is equal to the sum of the square of $x$ and the square of $y$.

Thm20a. Let $p$ be a natural number. Then $p$ is prime, only if for all natural numbers $k$, if $p$ is equal to the sum of the product of $4$ and $k$ and $1$, then there exists a natural number $x$, such that there exists a natural number $y$, such that $p$ is equal to the sum of the square of $x$ and the square of $y$.

Thm20a. Let $p$ be a natural number. Then if $p$ is prime, then for all natural numbers $k$, if $p$ is equal to the sum of the product of $4$ and $k$ and $1$, then there exists a natural number $y$, such that $p$ is equal to the sum of the square of $x$ and the square of $y$ for a natural number $x$.

Thm20a. Let $p$ be a natural number. Then $p$ is prime, only if for all natural numbers $k$, if $p$ is equal to the sum of the product of $4$ and $k$ and $1$, then there exists a natural number $y$, such that $p$ is equal to the sum of the square of $x$ and the square of $y$ for a natural number $x$.

Thm20a. Let $p$ be a natural number. Then if $p$ is prime, then for all natural numbers $k$, if $p$ is equal to the sum of the product of $4$ and $k$ and $1$, then there exists a natural number $y$, such that $p$ is equal to the sum of the square of $x$ and the square of $y$ for some natural number $x$.

Thm20a. Let $p$ be a natural number. Then $p$ is prime, only if for all natural numbers $k$, if $p$ is equal to the sum of the product of $4$ and $k$ and $1$, then there exists a natural number $y$, such that $p$ is equal to the sum of the square of $x$ and the square of $y$ for some natural number $x$.

Thm20a. Let $p \in N$. Then if $p$ is prime, then for all natural numbers $k$, if $p$ is equal to the sum of the product of $4$ and $k$ and $1$, then there exists a natural number $x$, such that there exists a natural number $y$, such that $p$ is equal to the sum of the square of $x$ and the square of $y$.

Thm20a. Let $p \in N$. Then $p$ is prime, only if for all natural numbers $k$, if $p$ is equal to the sum of the product of $4$ and $k$ and $1$, then there exists a natural number $x$, such that there exists a natural number $y$, such that $p$ is equal to the sum of the square of $x$ and the square of $y$.

Thm20a. Let $p \in N$. Then if $p$ is prime, then for all natural numbers $k$, if $p$ is equal to the sum of the product of $4$ and $k$ and $1$, then there exists a natural number $y$, such that $p$ is equal to the sum of the square of $x$ and the square of $y$ for a natural number $x$.

Thm20a. Let $p \in N$. Then $p$ is prime, only if for all natural numbers $k$, if $p$ is equal to the sum of the product of $4$ and $k$ and $1$, then there exists a natural number $y$, such that $p$ is equal to the sum of the square of $x$ and the square of $y$ for a natural number $x$.

Thm20a. Let $p \in N$. Then if $p$ is prime, then for all natural numbers $k$, if $p$ is equal to the sum of the product of $4$ and $k$ and $1$, then there exists a natural number $y$, such that $p$ is equal to the sum of the square of $x$ and the square of $y$ for some natural number $x$.

Thm20a. Let $p \in N$. Then $p$ is prime, only if for all natural numbers $k$, if $p$ is equal to the sum of the product of $4$ and $k$ and $1$, then there exists a natural number $y$, such that $p$ is equal to the sum of the square of $x$ and the square of $y$ for some natural number $x$.

Thm20a. Let $p$ be a natural number. Assume that $p$ is prime. Then for all natural numbers $k$, if $p$ is equal to the sum of the product of $4$ and $k$ and $1$, then there exists a natural number $x$, such that there exists a natural number $y$, such that $p$ is equal to the sum of the square of $x$ and the square of $y$.

Thm20a. Let $p$ be a natural number. Assume that $p$ is prime. Then if $p$ is equal to the sum of the product of $4$ and $k$ and $1$, then there exists a natural number $x$, such that there exists a natural number $y$, such that $p$ is equal to the sum of the square of $x$ and the square of $y$ for every natural number $k$.

Thm20a. Let $p$ be a natural number. Assume that $p$ is prime. Then if $p$ is equal to the sum of the product of $4$ and $k$ and $1$, then there exists a natural number $x$, such that there exists a natural number $y$, such that $p$ is equal to the sum of the square of $x$ and the square of $y$ for all natural numbers $k$.

Thm20a. Let $p$ be a natural number. Assume that $p$ is prime. Then for all natural numbers $k$, if $p$ is equal to the sum of the product of $4$ and $k$ and $1$, then there exists a natural number $y$, such that $p$ is equal to the sum of the square of $x$ and the square of $y$ for a natural number $x$.

Thm20a. Let $p$ be a natural number. Assume that $p$ is prime. Then if $p$ is equal to the sum of the product of $4$ and $k$ and $1$, then there exists a natural number $y$, such that $p$ is equal to the sum of the square of $x$ and the square of $y$ for a natural number $x$ for every natural number $k$.

Thm20a. Let $p$ be a natural number. Assume that $p$ is prime. Then if $p$ is equal to the sum of the product of $4$ and $k$ and $1$, then there exists a natural number $y$, such that $p$ is equal to the sum of the square of $x$ and the square of $y$ for a natural number $x$ for all natural numbers $k$.

Thm20a. Let $p$ be a natural number. Assume that $p$ is prime. Then for all natural numbers $k$, if $p$ is equal to the sum of the product of $4$ and $k$ and $1$, then there exists a natural number $y$, such that $p$ is equal to the sum of the square of $x$ and the square of $y$ for some natural number $x$.

Thm20a. Let $p$ be a natural number. Assume that $p$ is prime. Then if $p$ is equal to the sum of the product of $4$ and $k$ and $1$, then there exists a natural number $y$, such that $p$ is equal to the sum of the square of $x$ and the square of $y$ for some natural number $x$ for every natural number $k$.

Thm20a. Let $p$ be a natural number. Assume that $p$ is prime. Then if $p$ is equal to the sum of the product of $4$ and $k$ and $1$, then there exists a natural number $y$, such that $p$ is equal to the sum of the square of $x$ and the square of $y$ for some natural number $x$ for all natural numbers $k$.

Thm20a. Let $p \in N$. Assume that $p$ is prime. Then for all natural numbers $k$, if $p$ is equal to the sum of the product of $4$ and $k$ and $1$, then there exists a natural number $x$, such that there exists a natural number $y$, such that $p$ is equal to the sum of the square of $x$ and the square of $y$.

Thm20a. Let $p \in N$. Assume that $p$ is prime. Then if $p$ is equal to the sum of the product of $4$ and $k$ and $1$, then there exists a natural number $x$, such that there exists a natural number $y$, such that $p$ is equal to the sum of the square of $x$ and the square of $y$ for every natural number $k$.

Thm20a. Let $p \in N$. Assume that $p$ is prime. Then if $p$ is equal to the sum of the product of $4$ and $k$ and $1$, then there exists a natural number $x$, such that there exists a natural number $y$, such that $p$ is equal to the sum of the square of $x$ and the square of $y$ for all natural numbers $k$.

Thm20a. Let $p \in N$. Assume that $p$ is prime. Then for all natural numbers $k$, if $p$ is equal to the sum of the product of $4$ and $k$ and $1$, then there exists a natural number $y$, such that $p$ is equal to the sum of the square of $x$ and the square of $y$ for a natural number $x$.

Thm20a. Let $p \in N$. Assume that $p$ is prime. Then if $p$ is equal to the sum of the product of $4$ and $k$ and $1$, then there exists a natural number $y$, such that $p$ is equal to the sum of the square of $x$ and the square of $y$ for a natural number $x$ for every natural number $k$.

Thm20a. Let $p \in N$. Assume that $p$ is prime. Then if $p$ is equal to the sum of the product of $4$ and $k$ and $1$, then there exists a natural number $y$, such that $p$ is equal to the sum of the square of $x$ and the square of $y$ for a natural number $x$ for all natural numbers $k$.

Thm20a. Let $p \in N$. Assume that $p$ is prime. Then for all natural numbers $k$, if $p$ is equal to the sum of the product of $4$ and $k$ and $1$, then there exists a natural number $y$, such that $p$ is equal to the sum of the square of $x$ and the square of $y$ for some natural number $x$.

Thm20a. Let $p \in N$. Assume that $p$ is prime. Then if $p$ is equal to the sum of the product of $4$ and $k$ and $1$, then there exists a natural number $y$, such that $p$ is equal to the sum of the square of $x$ and the square of $y$ for some natural number $x$ for every natural number $k$.

Thm20a. Let $p \in N$. Assume that $p$ is prime. Then if $p$ is equal to the sum of the product of $4$ and $k$ and $1$, then there exists a natural number $y$, such that $p$ is equal to the sum of the square of $x$ and the square of $y$ for some natural number $x$ for all natural numbers $k$.

Thm20a. Let $p$ be a natural number. Assume that $p$ is prime. Let $k$ be a natural number. Then if $p$ is equal to the sum of the product of $4$ and $k$ and $1$, then there exists a natural number $x$, such that there exists a natural number $y$, such that $p$ is equal to the sum of the square of $x$ and the square of $y$.

Thm20a. Let $p$ be a natural number. Assume that $p$ is prime. Let $k$ be a natural number. Then $p$ is equal to the sum of the product of $4$ and $k$ and $1$, only if there exists a natural number $x$, such that there exists a natural number $y$, such that $p$ is equal to the sum of the square of $x$ and the square of $y$.

Thm20a. Let $p$ be a natural number. Assume that $p$ is prime. Let $k$ be a natural number. Then if $p$ is equal to the sum of the product of $4$ and $k$ and $1$, then there exists a natural number $y$, such that $p$ is equal to the sum of the square of $x$ and the square of $y$ for a natural number $x$.

Thm20a. Let $p$ be a natural number. Assume that $p$ is prime. Let $k$ be a natural number. Then $p$ is equal to the sum of the product of $4$ and $k$ and $1$, only if there exists a natural number $y$, such that $p$ is equal to the sum of the square of $x$ and the square of $y$ for a natural number $x$.

Thm20a. Let $p$ be a natural number. Assume that $p$ is prime. Let $k$ be a natural number. Then if $p$ is equal to the sum of the product of $4$ and $k$ and $1$, then there exists a natural number $y$, such that $p$ is equal to the sum of the square of $x$ and the square of $y$ for some natural number $x$.

Thm20a. Let $p$ be a natural number. Assume that $p$ is prime. Let $k$ be a natural number. Then $p$ is equal to the sum of the product of $4$ and $k$ and $1$, only if there exists a natural number $y$, such that $p$ is equal to the sum of the square of $x$ and the square of $y$ for some natural number $x$.

Thm20a. Let $p$ be a natural number. Assume that $p$ is prime. Let $k \in N$. Then if $p$ is equal to the sum of the product of $4$ and $k$ and $1$, then there exists a natural number $x$, such that there exists a natural number $y$, such that $p$ is equal to the sum of the square of $x$ and the square of $y$.

Thm20a. Let $p$ be a natural number. Assume that $p$ is prime. Let $k \in N$. Then $p$ is equal to the sum of the product of $4$ and $k$ and $1$, only if there exists a natural number $x$, such that there exists a natural number $y$, such that $p$ is equal to the sum of the square of $x$ and the square of $y$.

Thm20a. Let $p$ be a natural number. Assume that $p$ is prime. Let $k \in N$. Then if $p$ is equal to the sum of the product of $4$ and $k$ and $1$, then there exists a natural number $y$, such that $p$ is equal to the sum of the square of $x$ and the square of $y$ for a natural number $x$.

Thm20a. Let $p$ be a natural number. Assume that $p$ is prime. Let $k \in N$. Then $p$ is equal to the sum of the product of $4$ and $k$ and $1$, only if there exists a natural number $y$, such that $p$ is equal to the sum of the square of $x$ and the square of $y$ for a natural number $x$.

Thm20a. Let $p$ be a natural number. Assume that $p$ is prime. Let $k \in N$. Then if $p$ is equal to the sum of the product of $4$ and $k$ and $1$, then there exists a natural number $y$, such that $p$ is equal to the sum of the square of $x$ and the square of $y$ for some natural number $x$.

Thm20a. Let $p$ be a natural number. Assume that $p$ is prime. Let $k \in N$. Then $p$ is equal to the sum of the product of $4$ and $k$ and $1$, only if there exists a natural number $y$, such that $p$ is equal to the sum of the square of $x$ and the square of $y$ for some natural number $x$.

Thm20a. Let $p \in N$. Assume that $p$ is prime. Let $k$ be a natural number. Then if $p$ is equal to the sum of the product of $4$ and $k$ and $1$, then there exists a natural number $x$, such that there exists a natural number $y$, such that $p$ is equal to the sum of the square of $x$ and the square of $y$.

Thm20a. Let $p \in N$. Assume that $p$ is prime. Let $k$ be a natural number. Then $p$ is equal to the sum of the product of $4$ and $k$ and $1$, only if there exists a natural number $x$, such that there exists a natural number $y$, such that $p$ is equal to the sum of the square of $x$ and the square of $y$.

Thm20a. Let $p \in N$. Assume that $p$ is prime. Let $k$ be a natural number. Then if $p$ is equal to the sum of the product of $4$ and $k$ and $1$, then there exists a natural number $y$, such that $p$ is equal to the sum of the square of $x$ and the square of $y$ for a natural number $x$.

Thm20a. Let $p \in N$. Assume that $p$ is prime. Let $k$ be a natural number. Then $p$ is equal to the sum of the product of $4$ and $k$ and $1$, only if there exists a natural number $y$, such that $p$ is equal to the sum of the square of $x$ and the square of $y$ for a natural number $x$.

Thm20a. Let $p \in N$. Assume that $p$ is prime. Let $k$ be a natural number. Then if $p$ is equal to the sum of the product of $4$ and $k$ and $1$, then there exists a natural number $y$, such that $p$ is equal to the sum of the square of $x$ and the square of $y$ for some natural number $x$.

Thm20a. Let $p \in N$. Assume that $p$ is prime. Let $k$ be a natural number. Then $p$ is equal to the sum of the product of $4$ and $k$ and $1$, only if there exists a natural number $y$, such that $p$ is equal to the sum of the square of $x$ and the square of $y$ for some natural number $x$.

Thm20a. Let $p \in N$. Assume that $p$ is prime. Let $k \in N$. Then if $p$ is equal to the sum of the product of $4$ and $k$ and $1$, then there exists a natural number $x$, such that there exists a natural number $y$, such that $p$ is equal to the sum of the square of $x$ and the square of $y$.

Thm20a. Let $p \in N$. Assume that $p$ is prime. Let $k \in N$. Then $p$ is equal to the sum of the product of $4$ and $k$ and $1$, only if there exists a natural number $x$, such that there exists a natural number $y$, such that $p$ is equal to the sum of the square of $x$ and the square of $y$.

Thm20a. Let $p \in N$. Assume that $p$ is prime. Let $k \in N$. Then if $p$ is equal to the sum of the product of $4$ and $k$ and $1$, then there exists a natural number $y$, such that $p$ is equal to the sum of the square of $x$ and the square of $y$ for a natural number $x$.

Thm20a. Let $p \in N$. Assume that $p$ is prime. Let $k \in N$. Then $p$ is equal to the sum of the product of $4$ and $k$ and $1$, only if there exists a natural number $y$, such that $p$ is equal to the sum of the square of $x$ and the square of $y$ for a natural number $x$.

Thm20a. Let $p \in N$. Assume that $p$ is prime. Let $k \in N$. Then if $p$ is equal to the sum of the product of $4$ and $k$ and $1$, then there exists a natural number $y$, such that $p$ is equal to the sum of the square of $x$ and the square of $y$ for some natural number $x$.

Thm20a. Let $p \in N$. Assume that $p$ is prime. Let $k \in N$. Then $p$ is equal to the sum of the product of $4$ and $k$ and $1$, only if there exists a natural number $y$, such that $p$ is equal to the sum of the square of $x$ and the square of $y$ for some natural number $x$.

Thm20a. Let $p$ be a natural number. Assume that $p$ is prime. Let $k$ be a natural number. Assume that $p$ is equal to the sum of the product of $4$ and $k$ and $1$. Then there exists a natural number $y$, such that $p$ is equal to the sum of the square of $x$ and the square of $y$ for a natural number $x$.

Thm20a. Let $p$ be a natural number. Assume that $p$ is prime. Let $k$ be a natural number. Assume that $p$ is equal to the sum of the product of $4$ and $k$ and $1$. Then there exists a natural number $y$, such that $p$ is equal to the sum of the square of $x$ and the square of $y$ for some natural number $x$.

Thm20a. Let $p$ be a natural number. Assume that $p$ is prime. Let $k$ be a natural number. Assume that $p$ is equal to the sum of the product of $4$ and $k$ and $1$. Then $p$ is equal to the sum of the square of $x$ and the square of $y$ for a natural number $y$ for a natural number $x$.

Thm20a. Let $p$ be a natural number. Assume that $p$ is prime. Let $k$ be a natural number. Assume that $p$ is equal to the sum of the product of $4$ and $k$ and $1$. Then $p$ is equal to the sum of the square of $x$ and the square of $y$ for some natural number $y$ for a natural number $x$.

Thm20a. Let $p$ be a natural number. Assume that $p$ is prime. Let $k$ be a natural number. Assume that $p$ is equal to the sum of the product of $4$ and $k$ and $1$. Then $p$ is equal to the sum of the square of $x$ and the square of $y$ for a natural number $y$ for some natural number $x$.

Thm20a. Let $p$ be a natural number. Assume that $p$ is prime. Let $k$ be a natural number. Assume that $p$ is equal to the sum of the product of $4$ and $k$ and $1$. Then $p$ is equal to the sum of the square of $x$ and the square of $y$ for some natural number $y$ for some natural number $x$.

Thm20a. Let $p$ be a natural number. Assume that $p$ is prime. Let $k \in N$. Assume that $p$ is equal to the sum of the product of $4$ and $k$ and $1$. Then there exists a natural number $x$, such that there exists a natural number $y$, such that $p$ is equal to the sum of the square of $x$ and the square of $y$.

Thm20a. Let $p$ be a natural number. Assume that $p$ is prime. Let $k \in N$. Assume that $p$ is equal to the sum of the product of $4$ and $k$ and $1$. Then there exists a natural number $y$, such that $p$ is equal to the sum of the square of $x$ and the square of $y$ for a natural number $x$.

Thm20a. Let $p$ be a natural number. Assume that $p$ is prime. Let $k \in N$. Assume that $p$ is equal to the sum of the product of $4$ and $k$ and $1$. Then there exists a natural number $y$, such that $p$ is equal to the sum of the square of $x$ and the square of $y$ for some natural number $x$.

Thm20a. Let $p$ be a natural number. Assume that $p$ is prime. Let $k \in N$. Assume that $p$ is equal to the sum of the product of $4$ and $k$ and $1$. Then $p$ is equal to the sum of the square of $x$ and the square of $y$ for a natural number $y$ for a natural number $x$.

Thm20a. Let $p$ be a natural number. Assume that $p$ is prime. Let $k \in N$. Assume that $p$ is equal to the sum of the product of $4$ and $k$ and $1$. Then $p$ is equal to the sum of the square of $x$ and the square of $y$ for some natural number $y$ for a natural number $x$.

Thm20a. Let $p$ be a natural number. Assume that $p$ is prime. Let $k \in N$. Assume that $p$ is equal to the sum of the product of $4$ and $k$ and $1$. Then $p$ is equal to the sum of the square of $x$ and the square of $y$ for a natural number $y$ for some natural number $x$.

Thm20a. Let $p$ be a natural number. Assume that $p$ is prime. Let $k \in N$. Assume that $p$ is equal to the sum of the product of $4$ and $k$ and $1$. Then $p$ is equal to the sum of the square of $x$ and the square of $y$ for some natural number $y$ for some natural number $x$.

Thm20a. Let $p \in N$. Assume that $p$ is prime. Let $k$ be a natural number. Assume that $p$ is equal to the sum of the product of $4$ and $k$ and $1$. Then there exists a natural number $x$, such that there exists a natural number $y$, such that $p$ is equal to the sum of the square of $x$ and the square of $y$.

Thm20a. Let $p \in N$. Assume that $p$ is prime. Let $k$ be a natural number. Assume that $p$ is equal to the sum of the product of $4$ and $k$ and $1$. Then there exists a natural number $y$, such that $p$ is equal to the sum of the square of $x$ and the square of $y$ for a natural number $x$.

Thm20a. Let $p \in N$. Assume that $p$ is prime. Let $k$ be a natural number. Assume that $p$ is equal to the sum of the product of $4$ and $k$ and $1$. Then there exists a natural number $y$, such that $p$ is equal to the sum of the square of $x$ and the square of $y$ for some natural number $x$.

Thm20a. Let $p \in N$. Assume that $p$ is prime. Let $k$ be a natural number. Assume that $p$ is equal to the sum of the product of $4$ and $k$ and $1$. Then $p$ is equal to the sum of the square of $x$ and the square of $y$ for a natural number $y$ for a natural number $x$.

Thm20a. Let $p \in N$. Assume that $p$ is prime. Let $k$ be a natural number. Assume that $p$ is equal to the sum of the product of $4$ and $k$ and $1$. Then $p$ is equal to the sum of the square of $x$ and the square of $y$ for some natural number $y$ for a natural number $x$.

Thm20a. Let $p \in N$. Assume that $p$ is prime. Let $k$ be a natural number. Assume that $p$ is equal to the sum of the product of $4$ and $k$ and $1$. Then $p$ is equal to the sum of the square of $x$ and the square of $y$ for a natural number $y$ for some natural number $x$.

Thm20a. Let $p \in N$. Assume that $p$ is prime. Let $k$ be a natural number. Assume that $p$ is equal to the sum of the product of $4$ and $k$ and $1$. Then $p$ is equal to the sum of the square of $x$ and the square of $y$ for some natural number $y$ for some natural number $x$.

Thm20a. Let $p \in N$. Assume that $p$ is prime. Let $k \in N$. Assume that $p$ is equal to the sum of the product of $4$ and $k$ and $1$. Then there exists a natural number $x$, such that there exists a natural number $y$, such that $p$ is equal to the sum of the square of $x$ and the square of $y$.

Thm20a. Let $p \in N$. Assume that $p$ is prime. Let $k \in N$. Assume that $p$ is equal to the sum of the product of $4$ and $k$ and $1$. Then there exists a natural number $y$, such that $p$ is equal to the sum of the square of $x$ and the square of $y$ for a natural number $x$.

Thm20a. Let $p \in N$. Assume that $p$ is prime. Let $k \in N$. Assume that $p$ is equal to the sum of the product of $4$ and $k$ and $1$. Then there exists a natural number $y$, such that $p$ is equal to the sum of the square of $x$ and the square of $y$ for some natural number $x$.

Thm20a. Let $p \in N$. Assume that $p$ is prime. Let $k \in N$. Assume that $p$ is equal to the sum of the product of $4$ and $k$ and $1$. Then $p$ is equal to the sum of the square of $x$ and the square of $y$ for a natural number $y$ for a natural number $x$.

Thm20a. Let $p \in N$. Assume that $p$ is prime. Let $k \in N$. Assume that $p$ is equal to the sum of the product of $4$ and $k$ and $1$. Then $p$ is equal to the sum of the square of $x$ and the square of $y$ for some natural number $y$ for a natural number $x$.

Thm20a. Let $p \in N$. Assume that $p$ is prime. Let $k \in N$. Assume that $p$ is equal to the sum of the product of $4$ and $k$ and $1$. Then $p$ is equal to the sum of the square of $x$ and the square of $y$ for a natural number $y$ for some natural number $x$.

Thm20a. Let $p \in N$. Assume that $p$ is prime. Let $k \in N$. Assume that $p$ is equal to the sum of the product of $4$ and $k$ and $1$. Then $p$ is equal to the sum of the square of $x$ and the square of $y$ for some natural number $y$ for some natural number $x$.

Thm20a. Let $p$ be a natural number. Assume that $p$ is prime. Let $k$ be a natural number. Assume that $p = 4 k + 1$. Then there exists a natural number $x$, such that there exists a natural number $y$, such that $p = x ^{ 2}+ y ^{ 2}$.

Thm20a. For all natural numbers $p$, if $p$ is prime, then for all natural numbers $k$, if $p = 4 k + 1$, then there exists a natural number $x$, such that there exists a natural number $y$, such that $p = x ^{ 2}+ y ^{ 2}$.

Thm20a. If $p$ is prime, then for all natural numbers $k$, if $p = 4 k + 1$, then there exists a natural number $x$, such that there exists a natural number $y$, such that $p = x ^{ 2}+ y ^{ 2}$ for every natural number $p$.

Thm20a. If $p$ is prime, then for all natural numbers $k$, if $p = 4 k + 1$, then there exists a natural number $x$, such that there exists a natural number $y$, such that $p = x ^{ 2}+ y ^{ 2}$ for all natural numbers $p$.

Thm20a. For all natural numbers $p$, if $p$ is prime, then for all natural numbers $k$, if $p = 4 k + 1$, then there exists a natural number $y$, such that $p = x ^{ 2}+ y ^{ 2}$ for a natural number $x$.

Thm20a. If $p$ is prime, then for all natural numbers $k$, if $p = 4 k + 1$, then there exists a natural number $y$, such that $p = x ^{ 2}+ y ^{ 2}$ for a natural number $x$ for every natural number $p$.

Thm20a. If $p$ is prime, then for all natural numbers $k$, if $p = 4 k + 1$, then there exists a natural number $y$, such that $p = x ^{ 2}+ y ^{ 2}$ for a natural number $x$ for all natural numbers $p$.

Thm20a. For all natural numbers $p$, if $p$ is prime, then for all natural numbers $k$, if $p = 4 k + 1$, then there exists a natural number $y$, such that $p = x ^{ 2}+ y ^{ 2}$ for some natural number $x$.

Thm20a. If $p$ is prime, then for all natural numbers $k$, if $p = 4 k + 1$, then there exists a natural number $y$, such that $p = x ^{ 2}+ y ^{ 2}$ for some natural number $x$ for every natural number $p$.

Thm20a. If $p$ is prime, then for all natural numbers $k$, if $p = 4 k + 1$, then there exists a natural number $y$, such that $p = x ^{ 2}+ y ^{ 2}$ for some natural number $x$ for all natural numbers $p$.

Thm20a. Let $p$ be a natural number. Then if $p$ is prime, then for all natural numbers $k$, if $p = 4 k + 1$, then there exists a natural number $x$, such that there exists a natural number $y$, such that $p = x ^{ 2}+ y ^{ 2}$.

Thm20a. Let $p$ be a natural number. Then $p$ is prime, only if for all natural numbers $k$, if $p = 4 k + 1$, then there exists a natural number $x$, such that there exists a natural number $y$, such that $p = x ^{ 2}+ y ^{ 2}$.

Thm20a. Let $p$ be a natural number. Then if $p$ is prime, then for all natural numbers $k$, if $p = 4 k + 1$, then there exists a natural number $y$, such that $p = x ^{ 2}+ y ^{ 2}$ for a natural number $x$.

Thm20a. Let $p$ be a natural number. Then $p$ is prime, only if for all natural numbers $k$, if $p = 4 k + 1$, then there exists a natural number $y$, such that $p = x ^{ 2}+ y ^{ 2}$ for a natural number $x$.

Thm20a. Let $p$ be a natural number. Then if $p$ is prime, then for all natural numbers $k$, if $p = 4 k + 1$, then there exists a natural number $y$, such that $p = x ^{ 2}+ y ^{ 2}$ for some natural number $x$.

Thm20a. Let $p$ be a natural number. Then $p$ is prime, only if for all natural numbers $k$, if $p = 4 k + 1$, then there exists a natural number $y$, such that $p = x ^{ 2}+ y ^{ 2}$ for some natural number $x$.

Thm20a. Let $p \in N$. Then if $p$ is prime, then for all natural numbers $k$, if $p = 4 k + 1$, then there exists a natural number $x$, such that there exists a natural number $y$, such that $p = x ^{ 2}+ y ^{ 2}$.

Thm20a. Let $p \in N$. Then $p$ is prime, only if for all natural numbers $k$, if $p = 4 k + 1$, then there exists a natural number $x$, such that there exists a natural number $y$, such that $p = x ^{ 2}+ y ^{ 2}$.

Thm20a. Let $p \in N$. Then if $p$ is prime, then for all natural numbers $k$, if $p = 4 k + 1$, then there exists a natural number $y$, such that $p = x ^{ 2}+ y ^{ 2}$ for a natural number $x$.

Thm20a. Let $p \in N$. Then $p$ is prime, only if for all natural numbers $k$, if $p = 4 k + 1$, then there exists a natural number $y$, such that $p = x ^{ 2}+ y ^{ 2}$ for a natural number $x$.

Thm20a. Let $p \in N$. Then if $p$ is prime, then for all natural numbers $k$, if $p = 4 k + 1$, then there exists a natural number $y$, such that $p = x ^{ 2}+ y ^{ 2}$ for some natural number $x$.

Thm20a. Let $p \in N$. Then $p$ is prime, only if for all natural numbers $k$, if $p = 4 k + 1$, then there exists a natural number $y$, such that $p = x ^{ 2}+ y ^{ 2}$ for some natural number $x$.

Thm20a. Let $p$ be a natural number. Assume that $p$ is prime. Then for all natural numbers $k$, if $p = 4 k + 1$, then there exists a natural number $x$, such that there exists a natural number $y$, such that $p = x ^{ 2}+ y ^{ 2}$.

Thm20a. Let $p$ be a natural number. Assume that $p$ is prime. Then if $p = 4 k + 1$, then there exists a natural number $x$, such that there exists a natural number $y$, such that $p = x ^{ 2}+ y ^{ 2}$ for every natural number $k$.

Thm20a. Let $p$ be a natural number. Assume that $p$ is prime. Then if $p = 4 k + 1$, then there exists a natural number $x$, such that there exists a natural number $y$, such that $p = x ^{ 2}+ y ^{ 2}$ for all natural numbers $k$.

Thm20a. Let $p$ be a natural number. Assume that $p$ is prime. Then for all natural numbers $k$, if $p = 4 k + 1$, then there exists a natural number $y$, such that $p = x ^{ 2}+ y ^{ 2}$ for a natural number $x$.

Thm20a. Let $p$ be a natural number. Assume that $p$ is prime. Then if $p = 4 k + 1$, then there exists a natural number $y$, such that $p = x ^{ 2}+ y ^{ 2}$ for a natural number $x$ for every natural number $k$.

Thm20a. Let $p$ be a natural number. Assume that $p$ is prime. Then if $p = 4 k + 1$, then there exists a natural number $y$, such that $p = x ^{ 2}+ y ^{ 2}$ for a natural number $x$ for all natural numbers $k$.

Thm20a. Let $p$ be a natural number. Assume that $p$ is prime. Then for all natural numbers $k$, if $p = 4 k + 1$, then there exists a natural number $y$, such that $p = x ^{ 2}+ y ^{ 2}$ for some natural number $x$.

Thm20a. Let $p$ be a natural number. Assume that $p$ is prime. Then if $p = 4 k + 1$, then there exists a natural number $y$, such that $p = x ^{ 2}+ y ^{ 2}$ for some natural number $x$ for every natural number $k$.

Thm20a. Let $p$ be a natural number. Assume that $p$ is prime. Then if $p = 4 k + 1$, then there exists a natural number $y$, such that $p = x ^{ 2}+ y ^{ 2}$ for some natural number $x$ for all natural numbers $k$.

Thm20a. Let $p \in N$. Assume that $p$ is prime. Then for all natural numbers $k$, if $p = 4 k + 1$, then there exists a natural number $x$, such that there exists a natural number $y$, such that $p = x ^{ 2}+ y ^{ 2}$.

Thm20a. Let $p \in N$. Assume that $p$ is prime. Then if $p = 4 k + 1$, then there exists a natural number $x$, such that there exists a natural number $y$, such that $p = x ^{ 2}+ y ^{ 2}$ for every natural number $k$.

Thm20a. Let $p \in N$. Assume that $p$ is prime. Then if $p = 4 k + 1$, then there exists a natural number $x$, such that there exists a natural number $y$, such that $p = x ^{ 2}+ y ^{ 2}$ for all natural numbers $k$.

Thm20a. Let $p \in N$. Assume that $p$ is prime. Then for all natural numbers $k$, if $p = 4 k + 1$, then there exists a natural number $y$, such that $p = x ^{ 2}+ y ^{ 2}$ for a natural number $x$.

Thm20a. Let $p \in N$. Assume that $p$ is prime. Then if $p = 4 k + 1$, then there exists a natural number $y$, such that $p = x ^{ 2}+ y ^{ 2}$ for a natural number $x$ for every natural number $k$.

Thm20a. Let $p \in N$. Assume that $p$ is prime. Then if $p = 4 k + 1$, then there exists a natural number $y$, such that $p = x ^{ 2}+ y ^{ 2}$ for a natural number $x$ for all natural numbers $k$.

Thm20a. Let $p \in N$. Assume that $p$ is prime. Then for all natural numbers $k$, if $p = 4 k + 1$, then there exists a natural number $y$, such that $p = x ^{ 2}+ y ^{ 2}$ for some natural number $x$.

Thm20a. Let $p \in N$. Assume that $p$ is prime. Then if $p = 4 k + 1$, then there exists a natural number $y$, such that $p = x ^{ 2}+ y ^{ 2}$ for some natural number $x$ for every natural number $k$.

Thm20a. Let $p \in N$. Assume that $p$ is prime. Then if $p = 4 k + 1$, then there exists a natural number $y$, such that $p = x ^{ 2}+ y ^{ 2}$ for some natural number $x$ for all natural numbers $k$.

Thm20a. Let $p$ be a natural number. Assume that $p$ is prime. Let $k$ be a natural number. Then if $p = 4 k + 1$, then there exists a natural number $x$, such that there exists a natural number $y$, such that $p = x ^{ 2}+ y ^{ 2}$.

Thm20a. Let $p$ be a natural number. Assume that $p$ is prime. Let $k$ be a natural number. Then $p = 4 k + 1$, only if there exists a natural number $x$, such that there exists a natural number $y$, such that $p = x ^{ 2}+ y ^{ 2}$.

Thm20a. Let $p$ be a natural number. Assume that $p$ is prime. Let $k$ be a natural number. Then if $p = 4 k + 1$, then there exists a natural number $y$, such that $p = x ^{ 2}+ y ^{ 2}$ for a natural number $x$.

Thm20a. Let $p$ be a natural number. Assume that $p$ is prime. Let $k$ be a natural number. Then $p = 4 k + 1$, only if there exists a natural number $y$, such that $p = x ^{ 2}+ y ^{ 2}$ for a natural number $x$.

Thm20a. Let $p$ be a natural number. Assume that $p$ is prime. Let $k$ be a natural number. Then if $p = 4 k + 1$, then there exists a natural number $y$, such that $p = x ^{ 2}+ y ^{ 2}$ for some natural number $x$.

Thm20a. Let $p$ be a natural number. Assume that $p$ is prime. Let $k$ be a natural number. Then $p = 4 k + 1$, only if there exists a natural number $y$, such that $p = x ^{ 2}+ y ^{ 2}$ for some natural number $x$.

Thm20a. Let $p$ be a natural number. Assume that $p$ is prime. Let $k \in N$. Then if $p = 4 k + 1$, then there exists a natural number $x$, such that there exists a natural number $y$, such that $p = x ^{ 2}+ y ^{ 2}$.

Thm20a. Let $p$ be a natural number. Assume that $p$ is prime. Let $k \in N$. Then $p = 4 k + 1$, only if there exists a natural number $x$, such that there exists a natural number $y$, such that $p = x ^{ 2}+ y ^{ 2}$.

Thm20a. Let $p$ be a natural number. Assume that $p$ is prime. Let $k \in N$. Then if $p = 4 k + 1$, then there exists a natural number $y$, such that $p = x ^{ 2}+ y ^{ 2}$ for a natural number $x$.

Thm20a. Let $p$ be a natural number. Assume that $p$ is prime. Let $k \in N$. Then $p = 4 k + 1$, only if there exists a natural number $y$, such that $p = x ^{ 2}+ y ^{ 2}$ for a natural number $x$.

Thm20a. Let $p$ be a natural number. Assume that $p$ is prime. Let $k \in N$. Then if $p = 4 k + 1$, then there exists a natural number $y$, such that $p = x ^{ 2}+ y ^{ 2}$ for some natural number $x$.

Thm20a. Let $p$ be a natural number. Assume that $p$ is prime. Let $k \in N$. Then $p = 4 k + 1$, only if there exists a natural number $y$, such that $p = x ^{ 2}+ y ^{ 2}$ for some natural number $x$.

Thm20a. Let $p \in N$. Assume that $p$ is prime. Let $k$ be a natural number. Then if $p = 4 k + 1$, then there exists a natural number $x$, such that there exists a natural number $y$, such that $p = x ^{ 2}+ y ^{ 2}$.

Thm20a. Let $p \in N$. Assume that $p$ is prime. Let $k$ be a natural number. Then $p = 4 k + 1$, only if there exists a natural number $x$, such that there exists a natural number $y$, such that $p = x ^{ 2}+ y ^{ 2}$.

Thm20a. Let $p \in N$. Assume that $p$ is prime. Let $k$ be a natural number. Then if $p = 4 k + 1$, then there exists a natural number $y$, such that $p = x ^{ 2}+ y ^{ 2}$ for a natural number $x$.

Thm20a. Let $p \in N$. Assume that $p$ is prime. Let $k$ be a natural number. Then $p = 4 k + 1$, only if there exists a natural number $y$, such that $p = x ^{ 2}+ y ^{ 2}$ for a natural number $x$.

Thm20a. Let $p \in N$. Assume that $p$ is prime. Let $k$ be a natural number. Then if $p = 4 k + 1$, then there exists a natural number $y$, such that $p = x ^{ 2}+ y ^{ 2}$ for some natural number $x$.

Thm20a. Let $p \in N$. Assume that $p$ is prime. Let $k$ be a natural number. Then $p = 4 k + 1$, only if there exists a natural number $y$, such that $p = x ^{ 2}+ y ^{ 2}$ for some natural number $x$.

Thm20a. Let $p \in N$. Assume that $p$ is prime. Let $k \in N$. Then if $p = 4 k + 1$, then there exists a natural number $x$, such that there exists a natural number $y$, such that $p = x ^{ 2}+ y ^{ 2}$.

Thm20a. Let $p \in N$. Assume that $p$ is prime. Let $k \in N$. Then $p = 4 k + 1$, only if there exists a natural number $x$, such that there exists a natural number $y$, such that $p = x ^{ 2}+ y ^{ 2}$.

Thm20a. Let $p \in N$. Assume that $p$ is prime. Let $k \in N$. Then if $p = 4 k + 1$, then there exists a natural number $y$, such that $p = x ^{ 2}+ y ^{ 2}$ for a natural number $x$.

Thm20a. Let $p \in N$. Assume that $p$ is prime. Let $k \in N$. Then $p = 4 k + 1$, only if there exists a natural number $y$, such that $p = x ^{ 2}+ y ^{ 2}$ for a natural number $x$.

Thm20a. Let $p \in N$. Assume that $p$ is prime. Let $k \in N$. Then if $p = 4 k + 1$, then there exists a natural number $y$, such that $p = x ^{ 2}+ y ^{ 2}$ for some natural number $x$.

Thm20a. Let $p \in N$. Assume that $p$ is prime. Let $k \in N$. Then $p = 4 k + 1$, only if there exists a natural number $y$, such that $p = x ^{ 2}+ y ^{ 2}$ for some natural number $x$.

Thm20a. Let $p$ be a natural number. Assume that $p$ is prime. Let $k$ be a natural number. Assume that $p = 4 k + 1$. Then there exists a natural number $y$, such that $p = x ^{ 2}+ y ^{ 2}$ for a natural number $x$.

Thm20a. Let $p$ be a natural number. Assume that $p$ is prime. Let $k$ be a natural number. Assume that $p = 4 k + 1$. Then there exists a natural number $y$, such that $p = x ^{ 2}+ y ^{ 2}$ for some natural number $x$.

Thm20a. Let $p$ be a natural number. Assume that $p$ is prime. Let $k$ be a natural number. Assume that $p = 4 k + 1$. Then $p = x ^{ 2}+ y ^{ 2}$ for a natural number $y$ for a natural number $x$.

Thm20a. Let $p$ be a natural number. Assume that $p$ is prime. Let $k$ be a natural number. Assume that $p = 4 k + 1$. Then $p = x ^{ 2}+ y ^{ 2}$ for some natural number $y$ for a natural number $x$.

Thm20a. Let $p$ be a natural number. Assume that $p$ is prime. Let $k$ be a natural number. Assume that $p = 4 k + 1$. Then $p = x ^{ 2}+ y ^{ 2}$ for a natural number $y$ for some natural number $x$.

Thm20a. Let $p$ be a natural number. Assume that $p$ is prime. Let $k$ be a natural number. Assume that $p = 4 k + 1$. Then $p = x ^{ 2}+ y ^{ 2}$ for some natural number $y$ for some natural number $x$.

Thm20a. Let $p$ be a natural number. Assume that $p$ is prime. Let $k \in N$. Assume that $p = 4 k + 1$. Then there exists a natural number $x$, such that there exists a natural number $y$, such that $p = x ^{ 2}+ y ^{ 2}$.

Thm20a. Let $p$ be a natural number. Assume that $p$ is prime. Let $k \in N$. Assume that $p = 4 k + 1$. Then there exists a natural number $y$, such that $p = x ^{ 2}+ y ^{ 2}$ for a natural number $x$.

Thm20a. Let $p$ be a natural number. Assume that $p$ is prime. Let $k \in N$. Assume that $p = 4 k + 1$. Then there exists a natural number $y$, such that $p = x ^{ 2}+ y ^{ 2}$ for some natural number $x$.

Thm20a. Let $p$ be a natural number. Assume that $p$ is prime. Let $k \in N$. Assume that $p = 4 k + 1$. Then $p = x ^{ 2}+ y ^{ 2}$ for a natural number $y$ for a natural number $x$.

Thm20a. Let $p$ be a natural number. Assume that $p$ is prime. Let $k \in N$. Assume that $p = 4 k + 1$. Then $p = x ^{ 2}+ y ^{ 2}$ for some natural number $y$ for a natural number $x$.

Thm20a. Let $p$ be a natural number. Assume that $p$ is prime. Let $k \in N$. Assume that $p = 4 k + 1$. Then $p = x ^{ 2}+ y ^{ 2}$ for a natural number $y$ for some natural number $x$.

Thm20a. Let $p$ be a natural number. Assume that $p$ is prime. Let $k \in N$. Assume that $p = 4 k + 1$. Then $p = x ^{ 2}+ y ^{ 2}$ for some natural number $y$ for some natural number $x$.

Thm20a. Let $p \in N$. Assume that $p$ is prime. Let $k$ be a natural number. Assume that $p = 4 k + 1$. Then there exists a natural number $x$, such that there exists a natural number $y$, such that $p = x ^{ 2}+ y ^{ 2}$.

Thm20a. Let $p \in N$. Assume that $p$ is prime. Let $k$ be a natural number. Assume that $p = 4 k + 1$. Then there exists a natural number $y$, such that $p = x ^{ 2}+ y ^{ 2}$ for a natural number $x$.

Thm20a. Let $p \in N$. Assume that $p$ is prime. Let $k$ be a natural number. Assume that $p = 4 k + 1$. Then there exists a natural number $y$, such that $p = x ^{ 2}+ y ^{ 2}$ for some natural number $x$.

Thm20a. Let $p \in N$. Assume that $p$ is prime. Let $k$ be a natural number. Assume that $p = 4 k + 1$. Then $p = x ^{ 2}+ y ^{ 2}$ for a natural number $y$ for a natural number $x$.

Thm20a. Let $p \in N$. Assume that $p$ is prime. Let $k$ be a natural number. Assume that $p = 4 k + 1$. Then $p = x ^{ 2}+ y ^{ 2}$ for some natural number $y$ for a natural number $x$.

Thm20a. Let $p \in N$. Assume that $p$ is prime. Let $k$ be a natural number. Assume that $p = 4 k + 1$. Then $p = x ^{ 2}+ y ^{ 2}$ for a natural number $y$ for some natural number $x$.

Thm20a. Let $p \in N$. Assume that $p$ is prime. Let $k$ be a natural number. Assume that $p = 4 k + 1$. Then $p = x ^{ 2}+ y ^{ 2}$ for some natural number $y$ for some natural number $x$.

Thm20a. Let $p \in N$. Assume that $p$ is prime. Let $k \in N$. Assume that $p = 4 k + 1$. Then there exists a natural number $x$, such that there exists a natural number $y$, such that $p = x ^{ 2}+ y ^{ 2}$.

Thm20a. Let $p \in N$. Assume that $p$ is prime. Let $k \in N$. Assume that $p = 4 k + 1$. Then there exists a natural number $y$, such that $p = x ^{ 2}+ y ^{ 2}$ for a natural number $x$.

Thm20a. Let $p \in N$. Assume that $p$ is prime. Let $k \in N$. Assume that $p = 4 k + 1$. Then there exists a natural number $y$, such that $p = x ^{ 2}+ y ^{ 2}$ for some natural number $x$.

Thm20a. Let $p \in N$. Assume that $p$ is prime. Let $k \in N$. Assume that $p = 4 k + 1$. Then $p = x ^{ 2}+ y ^{ 2}$ for a natural number $y$ for a natural number $x$.

Thm20a. Let $p \in N$. Assume that $p$ is prime. Let $k \in N$. Assume that $p = 4 k + 1$. Then $p = x ^{ 2}+ y ^{ 2}$ for some natural number $y$ for a natural number $x$.

Thm20a. Let $p \in N$. Assume that $p$ is prime. Let $k \in N$. Assume that $p = 4 k + 1$. Then $p = x ^{ 2}+ y ^{ 2}$ for a natural number $y$ for some natural number $x$.

Thm20a. Let $p \in N$. Assume that $p$ is prime. Let $k \in N$. Assume that $p = 4 k + 1$. Then $p = x ^{ 2}+ y ^{ 2}$ for some natural number $y$ for some natural number $x$.

Thm20a. Let $p$ be a natural number. Assume that $p$ is prime. Let $k$ be a natural number. Assume that $p = 4 k + 1$. Then there exist natural numbers $x$ and $y$, such that $p = x ^{ 2}+ y ^{ 2}$.

Thm20a. For all natural numbers $p$, if $p$ is prime, then for all natural numbers $k$, if $p = 4 k + 1$, then there exist natural numbers $x$ and $y$, such that $p = x ^{ 2}+ y ^{ 2}$.

Thm20a. If $p$ is prime, then for all natural numbers $k$, if $p = 4 k + 1$, then there exist natural numbers $x$ and $y$, such that $p = x ^{ 2}+ y ^{ 2}$ for every natural number $p$.

Thm20a. If $p$ is prime, then for all natural numbers $k$, if $p = 4 k + 1$, then there exist natural numbers $x$ and $y$, such that $p = x ^{ 2}+ y ^{ 2}$ for all natural numbers $p$.

Thm20a. For all natural numbers $p$, if $p$ is prime, then for all natural numbers $k$, if $p = 4 k + 1$, then $p = x ^{ 2}+ y ^{ 2}$ for some natural numbers $x$ and $y$.

Thm20a. If $p$ is prime, then for all natural numbers $k$, if $p = 4 k + 1$, then $p = x ^{ 2}+ y ^{ 2}$ for some natural numbers $x$ and $y$ for every natural number $p$.

Thm20a. If $p$ is prime, then for all natural numbers $k$, if $p = 4 k + 1$, then $p = x ^{ 2}+ y ^{ 2}$ for some natural numbers $x$ and $y$ for all natural numbers $p$.

Thm20a. Let $p$ be a natural number. Then if $p$ is prime, then for all natural numbers $k$, if $p = 4 k + 1$, then there exist natural numbers $x$ and $y$, such that $p = x ^{ 2}+ y ^{ 2}$.

Thm20a. Let $p$ be a natural number. Then $p$ is prime, only if for all natural numbers $k$, if $p = 4 k + 1$, then there exist natural numbers $x$ and $y$, such that $p = x ^{ 2}+ y ^{ 2}$.

Thm20a. Let $p$ be a natural number. Then if $p$ is prime, then for all natural numbers $k$, if $p = 4 k + 1$, then $p = x ^{ 2}+ y ^{ 2}$ for some natural numbers $x$ and $y$.

Thm20a. Let $p$ be a natural number. Then $p$ is prime, only if for all natural numbers $k$, if $p = 4 k + 1$, then $p = x ^{ 2}+ y ^{ 2}$ for some natural numbers $x$ and $y$.

Thm20a. Let $p \in N$. Then if $p$ is prime, then for all natural numbers $k$, if $p = 4 k + 1$, then there exist natural numbers $x$ and $y$, such that $p = x ^{ 2}+ y ^{ 2}$.

Thm20a. Let $p \in N$. Then $p$ is prime, only if for all natural numbers $k$, if $p = 4 k + 1$, then there exist natural numbers $x$ and $y$, such that $p = x ^{ 2}+ y ^{ 2}$.

Thm20a. Let $p \in N$. Then if $p$ is prime, then for all natural numbers $k$, if $p = 4 k + 1$, then $p = x ^{ 2}+ y ^{ 2}$ for some natural numbers $x$ and $y$.

Thm20a. Let $p \in N$. Then $p$ is prime, only if for all natural numbers $k$, if $p = 4 k + 1$, then $p = x ^{ 2}+ y ^{ 2}$ for some natural numbers $x$ and $y$.

Thm20a. Let $p$ be a natural number. Assume that $p$ is prime. Then for all natural numbers $k$, if $p = 4 k + 1$, then there exist natural numbers $x$ and $y$, such that $p = x ^{ 2}+ y ^{ 2}$.

Thm20a. Let $p$ be a natural number. Assume that $p$ is prime. Then if $p = 4 k + 1$, then there exist natural numbers $x$ and $y$, such that $p = x ^{ 2}+ y ^{ 2}$ for every natural number $k$.

Thm20a. Let $p$ be a natural number. Assume that $p$ is prime. Then if $p = 4 k + 1$, then there exist natural numbers $x$ and $y$, such that $p = x ^{ 2}+ y ^{ 2}$ for all natural numbers $k$.

Thm20a. Let $p$ be a natural number. Assume that $p$ is prime. Then for all natural numbers $k$, if $p = 4 k + 1$, then $p = x ^{ 2}+ y ^{ 2}$ for some natural numbers $x$ and $y$.

Thm20a. Let $p$ be a natural number. Assume that $p$ is prime. Then if $p = 4 k + 1$, then $p = x ^{ 2}+ y ^{ 2}$ for some natural numbers $x$ and $y$ for every natural number $k$.

Thm20a. Let $p$ be a natural number. Assume that $p$ is prime. Then if $p = 4 k + 1$, then $p = x ^{ 2}+ y ^{ 2}$ for some natural numbers $x$ and $y$ for all natural numbers $k$.

Thm20a. Let $p \in N$. Assume that $p$ is prime. Then for all natural numbers $k$, if $p = 4 k + 1$, then there exist natural numbers $x$ and $y$, such that $p = x ^{ 2}+ y ^{ 2}$.

Thm20a. Let $p \in N$. Assume that $p$ is prime. Then if $p = 4 k + 1$, then there exist natural numbers $x$ and $y$, such that $p = x ^{ 2}+ y ^{ 2}$ for every natural number $k$.

Thm20a. Let $p \in N$. Assume that $p$ is prime. Then if $p = 4 k + 1$, then there exist natural numbers $x$ and $y$, such that $p = x ^{ 2}+ y ^{ 2}$ for all natural numbers $k$.

Thm20a. Let $p \in N$. Assume that $p$ is prime. Then for all natural numbers $k$, if $p = 4 k + 1$, then $p = x ^{ 2}+ y ^{ 2}$ for some natural numbers $x$ and $y$.

Thm20a. Let $p \in N$. Assume that $p$ is prime. Then if $p = 4 k + 1$, then $p = x ^{ 2}+ y ^{ 2}$ for some natural numbers $x$ and $y$ for every natural number $k$.

Thm20a. Let $p \in N$. Assume that $p$ is prime. Then if $p = 4 k + 1$, then $p = x ^{ 2}+ y ^{ 2}$ for some natural numbers $x$ and $y$ for all natural numbers $k$.

Thm20a. Let $p$ be a natural number. Assume that $p$ is prime. Let $k$ be a natural number. Then if $p = 4 k + 1$, then there exist natural numbers $x$ and $y$, such that $p = x ^{ 2}+ y ^{ 2}$.

Thm20a. Let $p$ be a natural number. Assume that $p$ is prime. Let $k$ be a natural number. Then $p = 4 k + 1$, only if there exist natural numbers $x$ and $y$, such that $p = x ^{ 2}+ y ^{ 2}$.

Thm20a. Let $p$ be a natural number. Assume that $p$ is prime. Let $k$ be a natural number. Then if $p = 4 k + 1$, then $p = x ^{ 2}+ y ^{ 2}$ for some natural numbers $x$ and $y$.

Thm20a. Let $p$ be a natural number. Assume that $p$ is prime. Let $k$ be a natural number. Then $p = 4 k + 1$, only if $p = x ^{ 2}+ y ^{ 2}$ for some natural numbers $x$ and $y$.

Thm20a. Let $p$ be a natural number. Assume that $p$ is prime. Let $k \in N$. Then if $p = 4 k + 1$, then there exist natural numbers $x$ and $y$, such that $p = x ^{ 2}+ y ^{ 2}$.

Thm20a. Let $p$ be a natural number. Assume that $p$ is prime. Let $k \in N$. Then $p = 4 k + 1$, only if there exist natural numbers $x$ and $y$, such that $p = x ^{ 2}+ y ^{ 2}$.

Thm20a. Let $p$ be a natural number. Assume that $p$ is prime. Let $k \in N$. Then if $p = 4 k + 1$, then $p = x ^{ 2}+ y ^{ 2}$ for some natural numbers $x$ and $y$.

Thm20a. Let $p$ be a natural number. Assume that $p$ is prime. Let $k \in N$. Then $p = 4 k + 1$, only if $p = x ^{ 2}+ y ^{ 2}$ for some natural numbers $x$ and $y$.

Thm20a. Let $p \in N$. Assume that $p$ is prime. Let $k$ be a natural number. Then if $p = 4 k + 1$, then there exist natural numbers $x$ and $y$, such that $p = x ^{ 2}+ y ^{ 2}$.

Thm20a. Let $p \in N$. Assume that $p$ is prime. Let $k$ be a natural number. Then $p = 4 k + 1$, only if there exist natural numbers $x$ and $y$, such that $p = x ^{ 2}+ y ^{ 2}$.

Thm20a. Let $p \in N$. Assume that $p$ is prime. Let $k$ be a natural number. Then if $p = 4 k + 1$, then $p = x ^{ 2}+ y ^{ 2}$ for some natural numbers $x$ and $y$.

Thm20a. Let $p \in N$. Assume that $p$ is prime. Let $k$ be a natural number. Then $p = 4 k + 1$, only if $p = x ^{ 2}+ y ^{ 2}$ for some natural numbers $x$ and $y$.

Thm20a. Let $p \in N$. Assume that $p$ is prime. Let $k \in N$. Then if $p = 4 k + 1$, then there exist natural numbers $x$ and $y$, such that $p = x ^{ 2}+ y ^{ 2}$.

Thm20a. Let $p \in N$. Assume that $p$ is prime. Let $k \in N$. Then $p = 4 k + 1$, only if there exist natural numbers $x$ and $y$, such that $p = x ^{ 2}+ y ^{ 2}$.

Thm20a. Let $p \in N$. Assume that $p$ is prime. Let $k \in N$. Then if $p = 4 k + 1$, then $p = x ^{ 2}+ y ^{ 2}$ for some natural numbers $x$ and $y$.

Thm20a. Let $p \in N$. Assume that $p$ is prime. Let $k \in N$. Then $p = 4 k + 1$, only if $p = x ^{ 2}+ y ^{ 2}$ for some natural numbers $x$ and $y$.

Thm20a. Let $p$ be a natural number. Assume that $p$ is prime. Let $k$ be a natural number. Assume that $p = 4 k + 1$. Then $p = x ^{ 2}+ y ^{ 2}$ for some natural numbers $x$ and $y$.

Thm20a. Let $p$ be a natural number. Assume that $p$ is prime. Let $k \in N$. Assume that $p = 4 k + 1$. Then there exist natural numbers $x$ and $y$, such that $p = x ^{ 2}+ y ^{ 2}$.

Thm20a. Let $p$ be a natural number. Assume that $p$ is prime. Let $k \in N$. Assume that $p = 4 k + 1$. Then $p = x ^{ 2}+ y ^{ 2}$ for some natural numbers $x$ and $y$.

Thm20a. Let $p \in N$. Assume that $p$ is prime. Let $k$ be a natural number. Assume that $p = 4 k + 1$. Then there exist natural numbers $x$ and $y$, such that $p = x ^{ 2}+ y ^{ 2}$.

Thm20a. Let $p \in N$. Assume that $p$ is prime. Let $k$ be a natural number. Assume that $p = 4 k + 1$. Then $p = x ^{ 2}+ y ^{ 2}$ for some natural numbers $x$ and $y$.

Thm20a. Let $p \in N$. Assume that $p$ is prime. Let $k \in N$. Assume that $p = 4 k + 1$. Then there exist natural numbers $x$ and $y$, such that $p = x ^{ 2}+ y ^{ 2}$.

Thm20a. Let $p \in N$. Assume that $p$ is prime. Let $k \in N$. Assume that $p = 4 k + 1$. Then $p = x ^{ 2}+ y ^{ 2}$ for some natural numbers $x$ and $y$.

Thm20b. Let $p$ be an instance of natural numbers. Assume that we can prove that $p$ is prime. Assume that we can prove that $p$ is congruent to $1$ modulo $4$. Then we can prove that there exists a natural number $x$, such that there exists a natural number $y$, such that $p$ is equal to the sum of the square of $x$ and the square of $y$.

Thm20b. For all instances $p$ of natural numbers, if we can prove that $p$ is prime, then if we can prove that $p$ is congruent to $1$ modulo $4$, then we can prove that there exists a natural number $x$, such that there exists a natural number $y$, such that $p$ is equal to the sum of the square of $x$ and the square of $y$.

Thm20b. If we can prove that $p$ is prime, then if we can prove that $p$ is congruent to $1$ modulo $4$, then we can prove that there exists a natural number $x$, such that there exists a natural number $y$, such that $p$ is equal to the sum of the square of $x$ and the square of $y$ for every instance $p$ of natural numbers.

Thm20b. If we can prove that $p$ is prime, then if we can prove that $p$ is congruent to $1$ modulo $4$, then we can prove that there exists a natural number $x$, such that there exists a natural number $y$, such that $p$ is equal to the sum of the square of $x$ and the square of $y$ for all instances $p$ of natural numbers.

Thm20b. For all instances $p$ of natural numbers, if we can prove that $p$ is prime, then if we can prove that $p$ is congruent to $1$ modulo $4$, then we can prove that there exists a natural number $y$, such that $p$ is equal to the sum of the square of $x$ and the square of $y$ for a natural number $x$.

Thm20b. If we can prove that $p$ is prime, then if we can prove that $p$ is congruent to $1$ modulo $4$, then we can prove that there exists a natural number $y$, such that $p$ is equal to the sum of the square of $x$ and the square of $y$ for a natural number $x$ for every instance $p$ of natural numbers.

Thm20b. If we can prove that $p$ is prime, then if we can prove that $p$ is congruent to $1$ modulo $4$, then we can prove that there exists a natural number $y$, such that $p$ is equal to the sum of the square of $x$ and the square of $y$ for a natural number $x$ for all instances $p$ of natural numbers.

Thm20b. For all instances $p$ of natural numbers, if we can prove that $p$ is prime, then if we can prove that $p$ is congruent to $1$ modulo $4$, then we can prove that there exists a natural number $y$, such that $p$ is equal to the sum of the square of $x$ and the square of $y$ for some natural number $x$.

Thm20b. If we can prove that $p$ is prime, then if we can prove that $p$ is congruent to $1$ modulo $4$, then we can prove that there exists a natural number $y$, such that $p$ is equal to the sum of the square of $x$ and the square of $y$ for some natural number $x$ for every instance $p$ of natural numbers.

Thm20b. If we can prove that $p$ is prime, then if we can prove that $p$ is congruent to $1$ modulo $4$, then we can prove that there exists a natural number $y$, such that $p$ is equal to the sum of the square of $x$ and the square of $y$ for some natural number $x$ for all instances $p$ of natural numbers.

Thm20b. Let $p$ be an instance of natural numbers. Then if we can prove that $p$ is prime, then if we can prove that $p$ is congruent to $1$ modulo $4$, then we can prove that there exists a natural number $x$, such that there exists a natural number $y$, such that $p$ is equal to the sum of the square of $x$ and the square of $y$.

Thm20b. Let $p$ be an instance of natural numbers. Then we can prove that $p$ is prime, only if if we can prove that $p$ is congruent to $1$ modulo $4$, then we can prove that there exists a natural number $x$, such that there exists a natural number $y$, such that $p$ is equal to the sum of the square of $x$ and the square of $y$.

Thm20b. Let $p$ be an instance of natural numbers. Then if we can prove that $p$ is prime, then if we can prove that $p$ is congruent to $1$ modulo $4$, then we can prove that there exists a natural number $y$, such that $p$ is equal to the sum of the square of $x$ and the square of $y$ for a natural number $x$.

Thm20b. Let $p$ be an instance of natural numbers. Then we can prove that $p$ is prime, only if if we can prove that $p$ is congruent to $1$ modulo $4$, then we can prove that there exists a natural number $y$, such that $p$ is equal to the sum of the square of $x$ and the square of $y$ for a natural number $x$.

Thm20b. Let $p$ be an instance of natural numbers. Then if we can prove that $p$ is prime, then if we can prove that $p$ is congruent to $1$ modulo $4$, then we can prove that there exists a natural number $y$, such that $p$ is equal to the sum of the square of $x$ and the square of $y$ for some natural number $x$.

Thm20b. Let $p$ be an instance of natural numbers. Then we can prove that $p$ is prime, only if if we can prove that $p$ is congruent to $1$ modulo $4$, then we can prove that there exists a natural number $y$, such that $p$ is equal to the sum of the square of $x$ and the square of $y$ for some natural number $x$.

Thm20b. Let $p$ be an instance of natural numbers. Assume that we can prove that $p$ is prime. Then if we can prove that $p$ is congruent to $1$ modulo $4$, then we can prove that there exists a natural number $x$, such that there exists a natural number $y$, such that $p$ is equal to the sum of the square of $x$ and the square of $y$.

Thm20b. Let $p$ be an instance of natural numbers. Assume that we can prove that $p$ is prime. Then we can prove that $p$ is congruent to $1$ modulo $4$, only if we can prove that there exists a natural number $x$, such that there exists a natural number $y$, such that $p$ is equal to the sum of the square of $x$ and the square of $y$.

Thm20b. Let $p$ be an instance of natural numbers. Assume that we can prove that $p$ is prime. Then if we can prove that $p$ is congruent to $1$ modulo $4$, then we can prove that there exists a natural number $y$, such that $p$ is equal to the sum of the square of $x$ and the square of $y$ for a natural number $x$.

Thm20b. Let $p$ be an instance of natural numbers. Assume that we can prove that $p$ is prime. Then we can prove that $p$ is congruent to $1$ modulo $4$, only if we can prove that there exists a natural number $y$, such that $p$ is equal to the sum of the square of $x$ and the square of $y$ for a natural number $x$.

Thm20b. Let $p$ be an instance of natural numbers. Assume that we can prove that $p$ is prime. Then if we can prove that $p$ is congruent to $1$ modulo $4$, then we can prove that there exists a natural number $y$, such that $p$ is equal to the sum of the square of $x$ and the square of $y$ for some natural number $x$.

Thm20b. Let $p$ be an instance of natural numbers. Assume that we can prove that $p$ is prime. Then we can prove that $p$ is congruent to $1$ modulo $4$, only if we can prove that there exists a natural number $y$, such that $p$ is equal to the sum of the square of $x$ and the square of $y$ for some natural number $x$.

Thm20b. Let $p$ be an instance of natural numbers. Assume that we can prove that $p$ is prime. Assume that we can prove that $p$ is congruent to $1$ modulo $4$. Then we can prove that there exists a natural number $y$, such that $p$ is equal to the sum of the square of $x$ and the square of $y$ for a natural number $x$.

Thm20b. Let $p$ be an instance of natural numbers. Assume that we can prove that $p$ is prime. Assume that we can prove that $p$ is congruent to $1$ modulo $4$. Then we can prove that there exists a natural number $y$, such that $p$ is equal to the sum of the square of $x$ and the square of $y$ for some natural number $x$.

Thm20b. Let $p$ be an instance of natural numbers. Assume that we can prove that $p$ is prime. Assume that we can prove that $p$ is congruent to $1$ modulo $4$. Then we can prove that $p$ is equal to the sum of the square of $x$ and the square of $y$ for a natural number $y$ for a natural number $x$.

Thm20b. Let $p$ be an instance of natural numbers. Assume that we can prove that $p$ is prime. Assume that we can prove that $p$ is congruent to $1$ modulo $4$. Then we can prove that $p$ is equal to the sum of the square of $x$ and the square of $y$ for some natural number $y$ for a natural number $x$.

Thm20b. Let $p$ be an instance of natural numbers. Assume that we can prove that $p$ is prime. Assume that we can prove that $p$ is congruent to $1$ modulo $4$. Then we can prove that $p$ is equal to the sum of the square of $x$ and the square of $y$ for a natural number $y$ for some natural number $x$.

Thm20b. Let $p$ be an instance of natural numbers. Assume that we can prove that $p$ is prime. Assume that we can prove that $p$ is congruent to $1$ modulo $4$. Then we can prove that $p$ is equal to the sum of the square of $x$ and the square of $y$ for some natural number $y$ for some natural number $x$.

Thm20b. Let $p$ be a natural number. Assume that $p$ is prime. Assume that $p$ is congruent to $1$ modulo $4$. Then there exists a natural number $x$, such that there exists a natural number $y$, such that $p$ is equal to the sum of the square of $x$ and the square of $y$.

Thm20b. For all natural numbers $p$, if $p$ is prime, then if $p$ is congruent to $1$ modulo $4$, then there exists a natural number $x$, such that there exists a natural number $y$, such that $p$ is equal to the sum of the square of $x$ and the square of $y$.

Thm20b. If $p$ is prime, then if $p$ is congruent to $1$ modulo $4$, then there exists a natural number $x$, such that there exists a natural number $y$, such that $p$ is equal to the sum of the square of $x$ and the square of $y$ for every natural number $p$.

Thm20b. If $p$ is prime, then if $p$ is congruent to $1$ modulo $4$, then there exists a natural number $x$, such that there exists a natural number $y$, such that $p$ is equal to the sum of the square of $x$ and the square of $y$ for all natural numbers $p$.

Thm20b. For all natural numbers $p$, if $p$ is prime, then if $p$ is congruent to $1$ modulo $4$, then there exists a natural number $y$, such that $p$ is equal to the sum of the square of $x$ and the square of $y$ for a natural number $x$.

Thm20b. If $p$ is prime, then if $p$ is congruent to $1$ modulo $4$, then there exists a natural number $y$, such that $p$ is equal to the sum of the square of $x$ and the square of $y$ for a natural number $x$ for every natural number $p$.

Thm20b. If $p$ is prime, then if $p$ is congruent to $1$ modulo $4$, then there exists a natural number $y$, such that $p$ is equal to the sum of the square of $x$ and the square of $y$ for a natural number $x$ for all natural numbers $p$.

Thm20b. For all natural numbers $p$, if $p$ is prime, then if $p$ is congruent to $1$ modulo $4$, then there exists a natural number $y$, such that $p$ is equal to the sum of the square of $x$ and the square of $y$ for some natural number $x$.

Thm20b. If $p$ is prime, then if $p$ is congruent to $1$ modulo $4$, then there exists a natural number $y$, such that $p$ is equal to the sum of the square of $x$ and the square of $y$ for some natural number $x$ for every natural number $p$.

Thm20b. If $p$ is prime, then if $p$ is congruent to $1$ modulo $4$, then there exists a natural number $y$, such that $p$ is equal to the sum of the square of $x$ and the square of $y$ for some natural number $x$ for all natural numbers $p$.

Thm20b. Let $p$ be a natural number. Then if $p$ is prime, then if $p$ is congruent to $1$ modulo $4$, then there exists a natural number $x$, such that there exists a natural number $y$, such that $p$ is equal to the sum of the square of $x$ and the square of $y$.

Thm20b. Let $p$ be a natural number. Then $p$ is prime, only if if $p$ is congruent to $1$ modulo $4$, then there exists a natural number $x$, such that there exists a natural number $y$, such that $p$ is equal to the sum of the square of $x$ and the square of $y$.

Thm20b. Let $p$ be a natural number. Then if $p$ is prime, then if $p$ is congruent to $1$ modulo $4$, then there exists a natural number $y$, such that $p$ is equal to the sum of the square of $x$ and the square of $y$ for a natural number $x$.

Thm20b. Let $p$ be a natural number. Then $p$ is prime, only if if $p$ is congruent to $1$ modulo $4$, then there exists a natural number $y$, such that $p$ is equal to the sum of the square of $x$ and the square of $y$ for a natural number $x$.

Thm20b. Let $p$ be a natural number. Then if $p$ is prime, then if $p$ is congruent to $1$ modulo $4$, then there exists a natural number $y$, such that $p$ is equal to the sum of the square of $x$ and the square of $y$ for some natural number $x$.

Thm20b. Let $p$ be a natural number. Then $p$ is prime, only if if $p$ is congruent to $1$ modulo $4$, then there exists a natural number $y$, such that $p$ is equal to the sum of the square of $x$ and the square of $y$ for some natural number $x$.

Thm20b. Let $p \in N$. Then if $p$ is prime, then if $p$ is congruent to $1$ modulo $4$, then there exists a natural number $x$, such that there exists a natural number $y$, such that $p$ is equal to the sum of the square of $x$ and the square of $y$.

Thm20b. Let $p \in N$. Then $p$ is prime, only if if $p$ is congruent to $1$ modulo $4$, then there exists a natural number $x$, such that there exists a natural number $y$, such that $p$ is equal to the sum of the square of $x$ and the square of $y$.

Thm20b. Let $p \in N$. Then if $p$ is prime, then if $p$ is congruent to $1$ modulo $4$, then there exists a natural number $y$, such that $p$ is equal to the sum of the square of $x$ and the square of $y$ for a natural number $x$.

Thm20b. Let $p \in N$. Then $p$ is prime, only if if $p$ is congruent to $1$ modulo $4$, then there exists a natural number $y$, such that $p$ is equal to the sum of the square of $x$ and the square of $y$ for a natural number $x$.

Thm20b. Let $p \in N$. Then if $p$ is prime, then if $p$ is congruent to $1$ modulo $4$, then there exists a natural number $y$, such that $p$ is equal to the sum of the square of $x$ and the square of $y$ for some natural number $x$.

Thm20b. Let $p \in N$. Then $p$ is prime, only if if $p$ is congruent to $1$ modulo $4$, then there exists a natural number $y$, such that $p$ is equal to the sum of the square of $x$ and the square of $y$ for some natural number $x$.

Thm20b. Let $p$ be a natural number. Assume that $p$ is prime. Then if $p$ is congruent to $1$ modulo $4$, then there exists a natural number $x$, such that there exists a natural number $y$, such that $p$ is equal to the sum of the square of $x$ and the square of $y$.

Thm20b. Let $p$ be a natural number. Assume that $p$ is prime. Then $p$ is congruent to $1$ modulo $4$, only if there exists a natural number $x$, such that there exists a natural number $y$, such that $p$ is equal to the sum of the square of $x$ and the square of $y$.

Thm20b. Let $p$ be a natural number. Assume that $p$ is prime. Then if $p$ is congruent to $1$ modulo $4$, then there exists a natural number $y$, such that $p$ is equal to the sum of the square of $x$ and the square of $y$ for a natural number $x$.

Thm20b. Let $p$ be a natural number. Assume that $p$ is prime. Then $p$ is congruent to $1$ modulo $4$, only if there exists a natural number $y$, such that $p$ is equal to the sum of the square of $x$ and the square of $y$ for a natural number $x$.

Thm20b. Let $p$ be a natural number. Assume that $p$ is prime. Then if $p$ is congruent to $1$ modulo $4$, then there exists a natural number $y$, such that $p$ is equal to the sum of the square of $x$ and the square of $y$ for some natural number $x$.

Thm20b. Let $p$ be a natural number. Assume that $p$ is prime. Then $p$ is congruent to $1$ modulo $4$, only if there exists a natural number $y$, such that $p$ is equal to the sum of the square of $x$ and the square of $y$ for some natural number $x$.

Thm20b. Let $p \in N$. Assume that $p$ is prime. Then if $p$ is congruent to $1$ modulo $4$, then there exists a natural number $x$, such that there exists a natural number $y$, such that $p$ is equal to the sum of the square of $x$ and the square of $y$.

Thm20b. Let $p \in N$. Assume that $p$ is prime. Then $p$ is congruent to $1$ modulo $4$, only if there exists a natural number $x$, such that there exists a natural number $y$, such that $p$ is equal to the sum of the square of $x$ and the square of $y$.

Thm20b. Let $p \in N$. Assume that $p$ is prime. Then if $p$ is congruent to $1$ modulo $4$, then there exists a natural number $y$, such that $p$ is equal to the sum of the square of $x$ and the square of $y$ for a natural number $x$.

Thm20b. Let $p \in N$. Assume that $p$ is prime. Then $p$ is congruent to $1$ modulo $4$, only if there exists a natural number $y$, such that $p$ is equal to the sum of the square of $x$ and the square of $y$ for a natural number $x$.

Thm20b. Let $p \in N$. Assume that $p$ is prime. Then if $p$ is congruent to $1$ modulo $4$, then there exists a natural number $y$, such that $p$ is equal to the sum of the square of $x$ and the square of $y$ for some natural number $x$.

Thm20b. Let $p \in N$. Assume that $p$ is prime. Then $p$ is congruent to $1$ modulo $4$, only if there exists a natural number $y$, such that $p$ is equal to the sum of the square of $x$ and the square of $y$ for some natural number $x$.

Thm20b. Let $p$ be a natural number. Assume that $p$ is prime. Assume that $p$ is congruent to $1$ modulo $4$. Then there exists a natural number $y$, such that $p$ is equal to the sum of the square of $x$ and the square of $y$ for a natural number $x$.

Thm20b. Let $p$ be a natural number. Assume that $p$ is prime. Assume that $p$ is congruent to $1$ modulo $4$. Then there exists a natural number $y$, such that $p$ is equal to the sum of the square of $x$ and the square of $y$ for some natural number $x$.

Thm20b. Let $p$ be a natural number. Assume that $p$ is prime. Assume that $p$ is congruent to $1$ modulo $4$. Then $p$ is equal to the sum of the square of $x$ and the square of $y$ for a natural number $y$ for a natural number $x$.

Thm20b. Let $p$ be a natural number. Assume that $p$ is prime. Assume that $p$ is congruent to $1$ modulo $4$. Then $p$ is equal to the sum of the square of $x$ and the square of $y$ for some natural number $y$ for a natural number $x$.

Thm20b. Let $p$ be a natural number. Assume that $p$ is prime. Assume that $p$ is congruent to $1$ modulo $4$. Then $p$ is equal to the sum of the square of $x$ and the square of $y$ for a natural number $y$ for some natural number $x$.

Thm20b. Let $p$ be a natural number. Assume that $p$ is prime. Assume that $p$ is congruent to $1$ modulo $4$. Then $p$ is equal to the sum of the square of $x$ and the square of $y$ for some natural number $y$ for some natural number $x$.

Thm20b. Let $p \in N$. Assume that $p$ is prime. Assume that $p$ is congruent to $1$ modulo $4$. Then there exists a natural number $x$, such that there exists a natural number $y$, such that $p$ is equal to the sum of the square of $x$ and the square of $y$.

Thm20b. Let $p \in N$. Assume that $p$ is prime. Assume that $p$ is congruent to $1$ modulo $4$. Then there exists a natural number $y$, such that $p$ is equal to the sum of the square of $x$ and the square of $y$ for a natural number $x$.

Thm20b. Let $p \in N$. Assume that $p$ is prime. Assume that $p$ is congruent to $1$ modulo $4$. Then there exists a natural number $y$, such that $p$ is equal to the sum of the square of $x$ and the square of $y$ for some natural number $x$.

Thm20b. Let $p \in N$. Assume that $p$ is prime. Assume that $p$ is congruent to $1$ modulo $4$. Then $p$ is equal to the sum of the square of $x$ and the square of $y$ for a natural number $y$ for a natural number $x$.

Thm20b. Let $p \in N$. Assume that $p$ is prime. Assume that $p$ is congruent to $1$ modulo $4$. Then $p$ is equal to the sum of the square of $x$ and the square of $y$ for some natural number $y$ for a natural number $x$.

Thm20b. Let $p \in N$. Assume that $p$ is prime. Assume that $p$ is congruent to $1$ modulo $4$. Then $p$ is equal to the sum of the square of $x$ and the square of $y$ for a natural number $y$ for some natural number $x$.

Thm20b. Let $p \in N$. Assume that $p$ is prime. Assume that $p$ is congruent to $1$ modulo $4$. Then $p$ is equal to the sum of the square of $x$ and the square of $y$ for some natural number $y$ for some natural number $x$.

Thm20b. Let $p$ be a natural number. Assume that $p$ is prime. Assume that $p \equiv 1 \pmod{ 4}$. Then there exists a natural number $x$, such that there exists a natural number $y$, such that $p = x ^{ 2}+ y ^{ 2}$.

Thm20b. For all natural numbers $p$, if $p$ is prime, then if $p \equiv 1 \pmod{ 4}$, then there exists a natural number $x$, such that there exists a natural number $y$, such that $p = x ^{ 2}+ y ^{ 2}$.

Thm20b. If $p$ is prime, then if $p \equiv 1 \pmod{ 4}$, then there exists a natural number $x$, such that there exists a natural number $y$, such that $p = x ^{ 2}+ y ^{ 2}$ for every natural number $p$.

Thm20b. If $p$ is prime, then if $p \equiv 1 \pmod{ 4}$, then there exists a natural number $x$, such that there exists a natural number $y$, such that $p = x ^{ 2}+ y ^{ 2}$ for all natural numbers $p$.

Thm20b. For all natural numbers $p$, if $p$ is prime, then if $p \equiv 1 \pmod{ 4}$, then there exists a natural number $y$, such that $p = x ^{ 2}+ y ^{ 2}$ for a natural number $x$.

Thm20b. If $p$ is prime, then if $p \equiv 1 \pmod{ 4}$, then there exists a natural number $y$, such that $p = x ^{ 2}+ y ^{ 2}$ for a natural number $x$ for every natural number $p$.

Thm20b. If $p$ is prime, then if $p \equiv 1 \pmod{ 4}$, then there exists a natural number $y$, such that $p = x ^{ 2}+ y ^{ 2}$ for a natural number $x$ for all natural numbers $p$.

Thm20b. For all natural numbers $p$, if $p$ is prime, then if $p \equiv 1 \pmod{ 4}$, then there exists a natural number $y$, such that $p = x ^{ 2}+ y ^{ 2}$ for some natural number $x$.

Thm20b. If $p$ is prime, then if $p \equiv 1 \pmod{ 4}$, then there exists a natural number $y$, such that $p = x ^{ 2}+ y ^{ 2}$ for some natural number $x$ for every natural number $p$.

Thm20b. If $p$ is prime, then if $p \equiv 1 \pmod{ 4}$, then there exists a natural number $y$, such that $p = x ^{ 2}+ y ^{ 2}$ for some natural number $x$ for all natural numbers $p$.

Thm20b. Let $p$ be a natural number. Then if $p$ is prime, then if $p \equiv 1 \pmod{ 4}$, then there exists a natural number $x$, such that there exists a natural number $y$, such that $p = x ^{ 2}+ y ^{ 2}$.

Thm20b. Let $p$ be a natural number. Then $p$ is prime, only if if $p \equiv 1 \pmod{ 4}$, then there exists a natural number $x$, such that there exists a natural number $y$, such that $p = x ^{ 2}+ y ^{ 2}$.

Thm20b. Let $p$ be a natural number. Then if $p$ is prime, then if $p \equiv 1 \pmod{ 4}$, then there exists a natural number $y$, such that $p = x ^{ 2}+ y ^{ 2}$ for a natural number $x$.

Thm20b. Let $p$ be a natural number. Then $p$ is prime, only if if $p \equiv 1 \pmod{ 4}$, then there exists a natural number $y$, such that $p = x ^{ 2}+ y ^{ 2}$ for a natural number $x$.

Thm20b. Let $p$ be a natural number. Then if $p$ is prime, then if $p \equiv 1 \pmod{ 4}$, then there exists a natural number $y$, such that $p = x ^{ 2}+ y ^{ 2}$ for some natural number $x$.

Thm20b. Let $p$ be a natural number. Then $p$ is prime, only if if $p \equiv 1 \pmod{ 4}$, then there exists a natural number $y$, such that $p = x ^{ 2}+ y ^{ 2}$ for some natural number $x$.

Thm20b. Let $p \in N$. Then if $p$ is prime, then if $p \equiv 1 \pmod{ 4}$, then there exists a natural number $x$, such that there exists a natural number $y$, such that $p = x ^{ 2}+ y ^{ 2}$.

Thm20b. Let $p \in N$. Then $p$ is prime, only if if $p \equiv 1 \pmod{ 4}$, then there exists a natural number $x$, such that there exists a natural number $y$, such that $p = x ^{ 2}+ y ^{ 2}$.

Thm20b. Let $p \in N$. Then if $p$ is prime, then if $p \equiv 1 \pmod{ 4}$, then there exists a natural number $y$, such that $p = x ^{ 2}+ y ^{ 2}$ for a natural number $x$.

Thm20b. Let $p \in N$. Then $p$ is prime, only if if $p \equiv 1 \pmod{ 4}$, then there exists a natural number $y$, such that $p = x ^{ 2}+ y ^{ 2}$ for a natural number $x$.

Thm20b. Let $p \in N$. Then if $p$ is prime, then if $p \equiv 1 \pmod{ 4}$, then there exists a natural number $y$, such that $p = x ^{ 2}+ y ^{ 2}$ for some natural number $x$.

Thm20b. Let $p \in N$. Then $p$ is prime, only if if $p \equiv 1 \pmod{ 4}$, then there exists a natural number $y$, such that $p = x ^{ 2}+ y ^{ 2}$ for some natural number $x$.

Thm20b. Let $p$ be a natural number. Assume that $p$ is prime. Then if $p \equiv 1 \pmod{ 4}$, then there exists a natural number $x$, such that there exists a natural number $y$, such that $p = x ^{ 2}+ y ^{ 2}$.

Thm20b. Let $p$ be a natural number. Assume that $p$ is prime. Then $p \equiv 1 \pmod{ 4}$, only if there exists a natural number $x$, such that there exists a natural number $y$, such that $p = x ^{ 2}+ y ^{ 2}$.

Thm20b. Let $p$ be a natural number. Assume that $p$ is prime. Then if $p \equiv 1 \pmod{ 4}$, then there exists a natural number $y$, such that $p = x ^{ 2}+ y ^{ 2}$ for a natural number $x$.

Thm20b. Let $p$ be a natural number. Assume that $p$ is prime. Then $p \equiv 1 \pmod{ 4}$, only if there exists a natural number $y$, such that $p = x ^{ 2}+ y ^{ 2}$ for a natural number $x$.

Thm20b. Let $p$ be a natural number. Assume that $p$ is prime. Then if $p \equiv 1 \pmod{ 4}$, then there exists a natural number $y$, such that $p = x ^{ 2}+ y ^{ 2}$ for some natural number $x$.

Thm20b. Let $p$ be a natural number. Assume that $p$ is prime. Then $p \equiv 1 \pmod{ 4}$, only if there exists a natural number $y$, such that $p = x ^{ 2}+ y ^{ 2}$ for some natural number $x$.

Thm20b. Let $p \in N$. Assume that $p$ is prime. Then if $p \equiv 1 \pmod{ 4}$, then there exists a natural number $x$, such that there exists a natural number $y$, such that $p = x ^{ 2}+ y ^{ 2}$.

Thm20b. Let $p \in N$. Assume that $p$ is prime. Then $p \equiv 1 \pmod{ 4}$, only if there exists a natural number $x$, such that there exists a natural number $y$, such that $p = x ^{ 2}+ y ^{ 2}$.

Thm20b. Let $p \in N$. Assume that $p$ is prime. Then if $p \equiv 1 \pmod{ 4}$, then there exists a natural number $y$, such that $p = x ^{ 2}+ y ^{ 2}$ for a natural number $x$.

Thm20b. Let $p \in N$. Assume that $p$ is prime. Then $p \equiv 1 \pmod{ 4}$, only if there exists a natural number $y$, such that $p = x ^{ 2}+ y ^{ 2}$ for a natural number $x$.

Thm20b. Let $p \in N$. Assume that $p$ is prime. Then if $p \equiv 1 \pmod{ 4}$, then there exists a natural number $y$, such that $p = x ^{ 2}+ y ^{ 2}$ for some natural number $x$.

Thm20b. Let $p \in N$. Assume that $p$ is prime. Then $p \equiv 1 \pmod{ 4}$, only if there exists a natural number $y$, such that $p = x ^{ 2}+ y ^{ 2}$ for some natural number $x$.

Thm20b. Let $p$ be a natural number. Assume that $p$ is prime. Assume that $p \equiv 1 \pmod{ 4}$. Then there exists a natural number $y$, such that $p = x ^{ 2}+ y ^{ 2}$ for a natural number $x$.

Thm20b. Let $p$ be a natural number. Assume that $p$ is prime. Assume that $p \equiv 1 \pmod{ 4}$. Then there exists a natural number $y$, such that $p = x ^{ 2}+ y ^{ 2}$ for some natural number $x$.

Thm20b. Let $p$ be a natural number. Assume that $p$ is prime. Assume that $p \equiv 1 \pmod{ 4}$. Then $p = x ^{ 2}+ y ^{ 2}$ for a natural number $y$ for a natural number $x$.

Thm20b. Let $p$ be a natural number. Assume that $p$ is prime. Assume that $p \equiv 1 \pmod{ 4}$. Then $p = x ^{ 2}+ y ^{ 2}$ for some natural number $y$ for a natural number $x$.

Thm20b. Let $p$ be a natural number. Assume that $p$ is prime. Assume that $p \equiv 1 \pmod{ 4}$. Then $p = x ^{ 2}+ y ^{ 2}$ for a natural number $y$ for some natural number $x$.

Thm20b. Let $p$ be a natural number. Assume that $p$ is prime. Assume that $p \equiv 1 \pmod{ 4}$. Then $p = x ^{ 2}+ y ^{ 2}$ for some natural number $y$ for some natural number $x$.

Thm20b. Let $p \in N$. Assume that $p$ is prime. Assume that $p \equiv 1 \pmod{ 4}$. Then there exists a natural number $x$, such that there exists a natural number $y$, such that $p = x ^{ 2}+ y ^{ 2}$.

Thm20b. Let $p \in N$. Assume that $p$ is prime. Assume that $p \equiv 1 \pmod{ 4}$. Then there exists a natural number $y$, such that $p = x ^{ 2}+ y ^{ 2}$ for a natural number $x$.

Thm20b. Let $p \in N$. Assume that $p$ is prime. Assume that $p \equiv 1 \pmod{ 4}$. Then there exists a natural number $y$, such that $p = x ^{ 2}+ y ^{ 2}$ for some natural number $x$.

Thm20b. Let $p \in N$. Assume that $p$ is prime. Assume that $p \equiv 1 \pmod{ 4}$. Then $p = x ^{ 2}+ y ^{ 2}$ for a natural number $y$ for a natural number $x$.

Thm20b. Let $p \in N$. Assume that $p$ is prime. Assume that $p \equiv 1 \pmod{ 4}$. Then $p = x ^{ 2}+ y ^{ 2}$ for some natural number $y$ for a natural number $x$.

Thm20b. Let $p \in N$. Assume that $p$ is prime. Assume that $p \equiv 1 \pmod{ 4}$. Then $p = x ^{ 2}+ y ^{ 2}$ for a natural number $y$ for some natural number $x$.

Thm20b. Let $p \in N$. Assume that $p$ is prime. Assume that $p \equiv 1 \pmod{ 4}$. Then $p = x ^{ 2}+ y ^{ 2}$ for some natural number $y$ for some natural number $x$.

Thm20b. Let $p$ be a natural number. Assume that $p$ is prime and $p \equiv 1 \pmod{ 4}$. Then there exist natural numbers $x$ and $y$, such that $p = x ^{ 2}+ y ^{ 2}$.

Thm20b. For all natural numbers $p$, if $p$ is prime and $p \equiv 1 \pmod{ 4}$, then there exist natural numbers $x$ and $y$, such that $p = x ^{ 2}+ y ^{ 2}$.

Thm20b. If $p$ is prime and $p \equiv 1 \pmod{ 4}$, then there exist natural numbers $x$ and $y$, such that $p = x ^{ 2}+ y ^{ 2}$ for every natural number $p$.

Thm20b. If $p$ is prime and $p \equiv 1 \pmod{ 4}$, then there exist natural numbers $x$ and $y$, such that $p = x ^{ 2}+ y ^{ 2}$ for all natural numbers $p$.

Thm20b. For all natural numbers $p$, if $p$ is prime and $p \equiv 1 \pmod{ 4}$, then $p = x ^{ 2}+ y ^{ 2}$ for some natural numbers $x$ and $y$.

Thm20b. If $p$ is prime and $p \equiv 1 \pmod{ 4}$, then $p = x ^{ 2}+ y ^{ 2}$ for some natural numbers $x$ and $y$ for every natural number $p$.

Thm20b. If $p$ is prime and $p \equiv 1 \pmod{ 4}$, then $p = x ^{ 2}+ y ^{ 2}$ for some natural numbers $x$ and $y$ for all natural numbers $p$.

Thm20b. Let $p$ be a natural number. Then if $p$ is prime and $p \equiv 1 \pmod{ 4}$, then there exist natural numbers $x$ and $y$, such that $p = x ^{ 2}+ y ^{ 2}$.

Thm20b. Let $p$ be a natural number. Then $p$ is prime and $p \equiv 1 \pmod{ 4}$, only if there exist natural numbers $x$ and $y$, such that $p = x ^{ 2}+ y ^{ 2}$.

Thm20b. Let $p$ be a natural number. Then if $p$ is prime and $p \equiv 1 \pmod{ 4}$, then $p = x ^{ 2}+ y ^{ 2}$ for some natural numbers $x$ and $y$.

Thm20b. Let $p$ be a natural number. Then $p$ is prime and $p \equiv 1 \pmod{ 4}$, only if $p = x ^{ 2}+ y ^{ 2}$ for some natural numbers $x$ and $y$.

Thm20b. Let $p \in N$. Then if $p$ is prime and $p \equiv 1 \pmod{ 4}$, then there exist natural numbers $x$ and $y$, such that $p = x ^{ 2}+ y ^{ 2}$.

Thm20b. Let $p \in N$. Then $p$ is prime and $p \equiv 1 \pmod{ 4}$, only if there exist natural numbers $x$ and $y$, such that $p = x ^{ 2}+ y ^{ 2}$.

Thm20b. Let $p \in N$. Then if $p$ is prime and $p \equiv 1 \pmod{ 4}$, then $p = x ^{ 2}+ y ^{ 2}$ for some natural numbers $x$ and $y$.

Thm20b. Let $p \in N$. Then $p$ is prime and $p \equiv 1 \pmod{ 4}$, only if $p = x ^{ 2}+ y ^{ 2}$ for some natural numbers $x$ and $y$.

Thm20b. Let $p$ be a natural number. Assume that $p$ is prime and $p \equiv 1 \pmod{ 4}$. Then $p = x ^{ 2}+ y ^{ 2}$ for some natural numbers $x$ and $y$.

Thm20b. Let $p \in N$. Assume that $p$ is prime and $p \equiv 1 \pmod{ 4}$. Then there exist natural numbers $x$ and $y$, such that $p = x ^{ 2}+ y ^{ 2}$.

Thm20b. Let $p \in N$. Assume that $p$ is prime and $p \equiv 1 \pmod{ 4}$. Then $p = x ^{ 2}+ y ^{ 2}$ for some natural numbers $x$ and $y$.

Thm22. We can prove that $Real$ is not denumerable.

Thm22. $Real$ is not denumerable.

Thm51wilson. Let $n$ be an instance of natural numbers. Then we can prove that $n$ is prime, if and only if the factorial of the difference of $n$ and $1$ is congruent to the negation of $1$ modulo $n$.

Thm51wilson. For all instances $n$ of natural numbers, we can prove that $n$ is prime, if and only if the factorial of the difference of $n$ and $1$ is congruent to the negation of $1$ modulo $n$.

Thm51wilson. We can prove that $n$ is prime, if and only if the factorial of the difference of $n$ and $1$ is congruent to the negation of $1$ modulo $n$ for every instance $n$ of natural numbers.

Thm51wilson. We can prove that $n$ is prime, if and only if the factorial of the difference of $n$ and $1$ is congruent to the negation of $1$ modulo $n$ for all instances $n$ of natural numbers.

Thm51wilson. Let $n$ be a natural number. Then $n$ is prime, if and only if the factorial of the difference of $n$ and $1$ is congruent to the negation of $1$ modulo $n$.

Thm51wilson. For all natural numbers $n$, $n$ is prime, if and only if the factorial of the difference of $n$ and $1$ is congruent to the negation of $1$ modulo $n$.

Thm51wilson. $n$ is prime, if and only if the factorial of the difference of $n$ and $1$ is congruent to the negation of $1$ modulo $n$ for every natural number $n$.

Thm51wilson. $n$ is prime, if and only if the factorial of the difference of $n$ and $1$ is congruent to the negation of $1$ modulo $n$ for all natural numbers $n$.

Thm51wilson. Let $n \in N$. Then $n$ is prime, if and only if the factorial of the difference of $n$ and $1$ is congruent to the negation of $1$ modulo $n$.

Thm51wilson. Let $n$ be a natural number. Then $n$ is prime, if and only if $(n - 1)! \equiv - 1 \pmod{ n}$.

Thm51wilson. For all natural numbers $n$, $n$ is prime, if and only if $(n - 1)! \equiv - 1 \pmod{ n}$.

Thm51wilson. $n$ is prime, if and only if $(n - 1)! \equiv - 1 \pmod{ n}$ for every natural number $n$.

Thm51wilson. $n$ is prime, if and only if $(n - 1)! \equiv - 1 \pmod{ n}$ for all natural numbers $n$.

Thm51wilson. Let $n \in N$. Then $n$ is prime, if and only if $(n - 1)! \equiv - 1 \pmod{ n}$.

Thm51b. Let $n$ be an instance of natural numbers. Then we can prove that $n$ is prime, if and only if the sum of the factorial of the difference of $n$ and $1$ and $1$ is divisible by $n$.

Thm51b. For all instances $n$ of natural numbers, we can prove that $n$ is prime, if and only if the sum of the factorial of the difference of $n$ and $1$ and $1$ is divisible by $n$.

Thm51b. We can prove that $n$ is prime, if and only if the sum of the factorial of the difference of $n$ and $1$ and $1$ is divisible by $n$ for every instance $n$ of natural numbers.

Thm51b. We can prove that $n$ is prime, if and only if the sum of the factorial of the difference of $n$ and $1$ and $1$ is divisible by $n$ for all instances $n$ of natural numbers.

Thm51b. Let $n$ be a natural number. Then $n$ is prime, if and only if the sum of the factorial of the difference of $n$ and $1$ and $1$ is divisible by $n$.

Thm51b. For all natural numbers $n$, $n$ is prime, if and only if the sum of the factorial of the difference of $n$ and $1$ and $1$ is divisible by $n$.

Thm51b. $n$ is prime, if and only if the sum of the factorial of the difference of $n$ and $1$ and $1$ is divisible by $n$ for every natural number $n$.

Thm51b. $n$ is prime, if and only if the sum of the factorial of the difference of $n$ and $1$ and $1$ is divisible by $n$ for all natural numbers $n$.

Thm51b. Let $n \in N$. Then $n$ is prime, if and only if the sum of the factorial of the difference of $n$ and $1$ and $1$ is divisible by $n$.

Thm51b. Let $n$ be a natural number. Then $n$ is prime, if and only if $(n - 1)! + 1$ is divisible by $n$.

Thm51b. For all natural numbers $n$, $n$ is prime, if and only if $(n - 1)! + 1$ is divisible by $n$.

Thm51b. $n$ is prime, if and only if $(n - 1)! + 1$ is divisible by $n$ for every natural number $n$.

Thm51b. $n$ is prime, if and only if $(n - 1)! + 1$ is divisible by $n$ for all natural numbers $n$.

Thm51b. Let $n \in N$. Then $n$ is prime, if and only if $(n - 1)! + 1$ is divisible by $n$.

Thm52. Let $A$ be a set. Assume that we can prove that $A$ is finite. Then we can prove that the cardinality of the power set of $A$ is equal to the exponentiation of $2$ and the cardinality of $A$.

Thm52. For all sets $A$, if we can prove that $A$ is finite, then we can prove that the cardinality of the power set of $A$ is equal to the exponentiation of $2$ and the cardinality of $A$.

Thm52. If we can prove that $A$ is finite, then we can prove that the cardinality of the power set of $A$ is equal to the exponentiation of $2$ and the cardinality of $A$ for every set $A$.

Thm52. If we can prove that $A$ is finite, then we can prove that the cardinality of the power set of $A$ is equal to the exponentiation of $2$ and the cardinality of $A$ for all sets $A$.

Thm52. Let $A$ be a set. Then if we can prove that $A$ is finite, then we can prove that the cardinality of the power set of $A$ is equal to the exponentiation of $2$ and the cardinality of $A$.

Thm52. Let $A$ be a set. Then we can prove that $A$ is finite, only if we can prove that the cardinality of the power set of $A$ is equal to the exponentiation of $2$ and the cardinality of $A$.

Thm52. Let $A$ be a set. Assume that $A$ is finite. Then the cardinality of the power set of $A$ is equal to the exponentiation of $2$ and the cardinality of $A$.

Thm52. For all sets $A$, if $A$ is finite, then the cardinality of the power set of $A$ is equal to the exponentiation of $2$ and the cardinality of $A$.

Thm52. If $A$ is finite, then the cardinality of the power set of $A$ is equal to the exponentiation of $2$ and the cardinality of $A$ for every set $A$.

Thm52. If $A$ is finite, then the cardinality of the power set of $A$ is equal to the exponentiation of $2$ and the cardinality of $A$ for all sets $A$.

Thm52. Let $A$ be a set. Then if $A$ is finite, then the cardinality of the power set of $A$ is equal to the exponentiation of $2$ and the cardinality of $A$.

Thm52. Let $A$ be a set. Then $A$ is finite, only if the cardinality of the power set of $A$ is equal to the exponentiation of $2$ and the cardinality of $A$.

Thm52. Let $A$ be a set. Assume that $A$ is finite. Then $| \wp A | = 2 ^ {| A |}$.

Thm52. For all sets $A$, if $A$ is finite, then $| \wp A | = 2 ^ {| A |}$.

Thm52. If $A$ is finite, then $| \wp A | = 2 ^ {| A |}$ for every set $A$.

Thm52. If $A$ is finite, then $| \wp A | = 2 ^ {| A |}$ for all sets $A$.

Thm52. Let $A$ be a set. Then if $A$ is finite, then $| \wp A | = 2 ^ {| A |}$.

Thm52. Let $A$ be a set. Then $A$ is finite, only if $| \wp A | = 2 ^ {| A |}$.

Thm58. Let $A$ be a set. Let $n$ be an instance of natural numbers. Assume that we can prove that the cardinality of $A$ is equal to $n$. Let $k$ be an instance of natural numbers. Assume that we can prove that $k$ is less than or equal to $n$. Then we can prove that the cardinality of the number of combinations of $A$ and $k$ is equal to the binomial coefficient of $n$ and $k$.

Thm58. For all sets $A$, for all instances $n$ of natural numbers, if we can prove that the cardinality of $A$ is equal to $n$, then for all instances $k$ of natural numbers, if we can prove that $k$ is less than or equal to $n$, then we can prove that the cardinality of the number of combinations of $A$ and $k$ is equal to the binomial coefficient of $n$ and $k$.

Thm58. For all instances $n$ of natural numbers, if we can prove that the cardinality of $A$ is equal to $n$, then for all instances $k$ of natural numbers, if we can prove that $k$ is less than or equal to $n$, then we can prove that the cardinality of the number of combinations of $A$ and $k$ is equal to the binomial coefficient of $n$ and $k$ for every set $A$.

Thm58. For all instances $n$ of natural numbers, if we can prove that the cardinality of $A$ is equal to $n$, then for all instances $k$ of natural numbers, if we can prove that $k$ is less than or equal to $n$, then we can prove that the cardinality of the number of combinations of $A$ and $k$ is equal to the binomial coefficient of $n$ and $k$ for all sets $A$.

Thm58. Let $A$ be a set. Then for all instances $n$ of natural numbers, if we can prove that the cardinality of $A$ is equal to $n$, then for all instances $k$ of natural numbers, if we can prove that $k$ is less than or equal to $n$, then we can prove that the cardinality of the number of combinations of $A$ and $k$ is equal to the binomial coefficient of $n$ and $k$.

Thm58. Let $A$ be a set. Then if we can prove that the cardinality of $A$ is equal to $n$, then for all instances $k$ of natural numbers, if we can prove that $k$ is less than or equal to $n$, then we can prove that the cardinality of the number of combinations of $A$ and $k$ is equal to the binomial coefficient of $n$ and $k$ for every instance $n$ of natural numbers.

Thm58. Let $A$ be a set. Then if we can prove that the cardinality of $A$ is equal to $n$, then for all instances $k$ of natural numbers, if we can prove that $k$ is less than or equal to $n$, then we can prove that the cardinality of the number of combinations of $A$ and $k$ is equal to the binomial coefficient of $n$ and $k$ for all instances $n$ of natural numbers.

Thm58. Let $A$ be a set. Let $n$ be an instance of natural numbers. Then if we can prove that the cardinality of $A$ is equal to $n$, then for all instances $k$ of natural numbers, if we can prove that $k$ is less than or equal to $n$, then we can prove that the cardinality of the number of combinations of $A$ and $k$ is equal to the binomial coefficient of $n$ and $k$.

Thm58. Let $A$ be a set. Let $n$ be an instance of natural numbers. Then we can prove that the cardinality of $A$ is equal to $n$, only if for all instances $k$ of natural numbers, if we can prove that $k$ is less than or equal to $n$, then we can prove that the cardinality of the number of combinations of $A$ and $k$ is equal to the binomial coefficient of $n$ and $k$.

Thm58. Let $A$ be a set. Let $n$ be an instance of natural numbers. Assume that we can prove that the cardinality of $A$ is equal to $n$. Then for all instances $k$ of natural numbers, if we can prove that $k$ is less than or equal to $n$, then we can prove that the cardinality of the number of combinations of $A$ and $k$ is equal to the binomial coefficient of $n$ and $k$.

Thm58. Let $A$ be a set. Let $n$ be an instance of natural numbers. Assume that we can prove that the cardinality of $A$ is equal to $n$. Then if we can prove that $k$ is less than or equal to $n$, then we can prove that the cardinality of the number of combinations of $A$ and $k$ is equal to the binomial coefficient of $n$ and $k$ for every instance $k$ of natural numbers.

Thm58. Let $A$ be a set. Let $n$ be an instance of natural numbers. Assume that we can prove that the cardinality of $A$ is equal to $n$. Then if we can prove that $k$ is less than or equal to $n$, then we can prove that the cardinality of the number of combinations of $A$ and $k$ is equal to the binomial coefficient of $n$ and $k$ for all instances $k$ of natural numbers.

Thm58. Let $A$ be a set. Let $n$ be an instance of natural numbers. Assume that we can prove that the cardinality of $A$ is equal to $n$. Let $k$ be an instance of natural numbers. Then if we can prove that $k$ is less than or equal to $n$, then we can prove that the cardinality of the number of combinations of $A$ and $k$ is equal to the binomial coefficient of $n$ and $k$.

Thm58. Let $A$ be a set. Let $n$ be an instance of natural numbers. Assume that we can prove that the cardinality of $A$ is equal to $n$. Let $k$ be an instance of natural numbers. Then we can prove that $k$ is less than or equal to $n$, only if we can prove that the cardinality of the number of combinations of $A$ and $k$ is equal to the binomial coefficient of $n$ and $k$.

Thm58. Let $A$ be a set. Let $n$ be a natural number. Assume that the cardinality of $A$ is equal to $n$. Let $k$ be a natural number. Assume that $k$ is less than or equal to $n$. Then the cardinality of the number of combinations of $A$ and $k$ is equal to the binomial coefficient of $n$ and $k$.

Thm58. For all sets $A$, for all natural numbers $n$, if the cardinality of $A$ is equal to $n$, then for all natural numbers $k$, if $k$ is less than or equal to $n$, then the cardinality of the number of combinations of $A$ and $k$ is equal to the binomial coefficient of $n$ and $k$.

Thm58. For all natural numbers $n$, if the cardinality of $A$ is equal to $n$, then for all natural numbers $k$, if $k$ is less than or equal to $n$, then the cardinality of the number of combinations of $A$ and $k$ is equal to the binomial coefficient of $n$ and $k$ for every set $A$.

Thm58. For all natural numbers $n$, if the cardinality of $A$ is equal to $n$, then for all natural numbers $k$, if $k$ is less than or equal to $n$, then the cardinality of the number of combinations of $A$ and $k$ is equal to the binomial coefficient of $n$ and $k$ for all sets $A$.

Thm58. Let $A$ be a set. Then for all natural numbers $n$, if the cardinality of $A$ is equal to $n$, then for all natural numbers $k$, if $k$ is less than or equal to $n$, then the cardinality of the number of combinations of $A$ and $k$ is equal to the binomial coefficient of $n$ and $k$.

Thm58. Let $A$ be a set. Then if the cardinality of $A$ is equal to $n$, then for all natural numbers $k$, if $k$ is less than or equal to $n$, then the cardinality of the number of combinations of $A$ and $k$ is equal to the binomial coefficient of $n$ and $k$ for every natural number $n$.

Thm58. Let $A$ be a set. Then if the cardinality of $A$ is equal to $n$, then for all natural numbers $k$, if $k$ is less than or equal to $n$, then the cardinality of the number of combinations of $A$ and $k$ is equal to the binomial coefficient of $n$ and $k$ for all natural numbers $n$.

Thm58. Let $A$ be a set. Let $n$ be a natural number. Then if the cardinality of $A$ is equal to $n$, then for all natural numbers $k$, if $k$ is less than or equal to $n$, then the cardinality of the number of combinations of $A$ and $k$ is equal to the binomial coefficient of $n$ and $k$.

Thm58. Let $A$ be a set. Let $n$ be a natural number. Then the cardinality of $A$ is equal to $n$, only if for all natural numbers $k$, if $k$ is less than or equal to $n$, then the cardinality of the number of combinations of $A$ and $k$ is equal to the binomial coefficient of $n$ and $k$.

Thm58. Let $A$ be a set. Let $n \in N$. Then if the cardinality of $A$ is equal to $n$, then for all natural numbers $k$, if $k$ is less than or equal to $n$, then the cardinality of the number of combinations of $A$ and $k$ is equal to the binomial coefficient of $n$ and $k$.

Thm58. Let $A$ be a set. Let $n \in N$. Then the cardinality of $A$ is equal to $n$, only if for all natural numbers $k$, if $k$ is less than or equal to $n$, then the cardinality of the number of combinations of $A$ and $k$ is equal to the binomial coefficient of $n$ and $k$.

Thm58. Let $A$ be a set. Let $n$ be a natural number. Assume that the cardinality of $A$ is equal to $n$. Then for all natural numbers $k$, if $k$ is less than or equal to $n$, then the cardinality of the number of combinations of $A$ and $k$ is equal to the binomial coefficient of $n$ and $k$.

Thm58. Let $A$ be a set. Let $n$ be a natural number. Assume that the cardinality of $A$ is equal to $n$. Then if $k$ is less than or equal to $n$, then the cardinality of the number of combinations of $A$ and $k$ is equal to the binomial coefficient of $n$ and $k$ for every natural number $k$.

Thm58. Let $A$ be a set. Let $n$ be a natural number. Assume that the cardinality of $A$ is equal to $n$. Then if $k$ is less than or equal to $n$, then the cardinality of the number of combinations of $A$ and $k$ is equal to the binomial coefficient of $n$ and $k$ for all natural numbers $k$.

Thm58. Let $A$ be a set. Let $n \in N$. Assume that the cardinality of $A$ is equal to $n$. Then for all natural numbers $k$, if $k$ is less than or equal to $n$, then the cardinality of the number of combinations of $A$ and $k$ is equal to the binomial coefficient of $n$ and $k$.

Thm58. Let $A$ be a set. Let $n \in N$. Assume that the cardinality of $A$ is equal to $n$. Then if $k$ is less than or equal to $n$, then the cardinality of the number of combinations of $A$ and $k$ is equal to the binomial coefficient of $n$ and $k$ for every natural number $k$.

Thm58. Let $A$ be a set. Let $n \in N$. Assume that the cardinality of $A$ is equal to $n$. Then if $k$ is less than or equal to $n$, then the cardinality of the number of combinations of $A$ and $k$ is equal to the binomial coefficient of $n$ and $k$ for all natural numbers $k$.

Thm58. Let $A$ be a set. Let $n$ be a natural number. Assume that the cardinality of $A$ is equal to $n$. Let $k$ be a natural number. Then if $k$ is less than or equal to $n$, then the cardinality of the number of combinations of $A$ and $k$ is equal to the binomial coefficient of $n$ and $k$.

Thm58. Let $A$ be a set. Let $n$ be a natural number. Assume that the cardinality of $A$ is equal to $n$. Let $k$ be a natural number. Then $k$ is less than or equal to $n$, only if the cardinality of the number of combinations of $A$ and $k$ is equal to the binomial coefficient of $n$ and $k$.

Thm58. Let $A$ be a set. Let $n$ be a natural number. Assume that the cardinality of $A$ is equal to $n$. Let $k \in N$. Then if $k$ is less than or equal to $n$, then the cardinality of the number of combinations of $A$ and $k$ is equal to the binomial coefficient of $n$ and $k$.

Thm58. Let $A$ be a set. Let $n$ be a natural number. Assume that the cardinality of $A$ is equal to $n$. Let $k \in N$. Then $k$ is less than or equal to $n$, only if the cardinality of the number of combinations of $A$ and $k$ is equal to the binomial coefficient of $n$ and $k$.

Thm58. Let $A$ be a set. Let $n \in N$. Assume that the cardinality of $A$ is equal to $n$. Let $k$ be a natural number. Then if $k$ is less than or equal to $n$, then the cardinality of the number of combinations of $A$ and $k$ is equal to the binomial coefficient of $n$ and $k$.

Thm58. Let $A$ be a set. Let $n \in N$. Assume that the cardinality of $A$ is equal to $n$. Let $k$ be a natural number. Then $k$ is less than or equal to $n$, only if the cardinality of the number of combinations of $A$ and $k$ is equal to the binomial coefficient of $n$ and $k$.

Thm58. Let $A$ be a set. Let $n \in N$. Assume that the cardinality of $A$ is equal to $n$. Let $k \in N$. Then if $k$ is less than or equal to $n$, then the cardinality of the number of combinations of $A$ and $k$ is equal to the binomial coefficient of $n$ and $k$.

Thm58. Let $A$ be a set. Let $n \in N$. Assume that the cardinality of $A$ is equal to $n$. Let $k \in N$. Then $k$ is less than or equal to $n$, only if the cardinality of the number of combinations of $A$ and $k$ is equal to the binomial coefficient of $n$ and $k$.

Thm58. Let $A$ be a set. Let $n$ be a natural number. Assume that the cardinality of $A$ is equal to $n$. Let $k \in N$. Assume that $k$ is less than or equal to $n$. Then the cardinality of the number of combinations of $A$ and $k$ is equal to the binomial coefficient of $n$ and $k$.

Thm58. Let $A$ be a set. Let $n \in N$. Assume that the cardinality of $A$ is equal to $n$. Let $k$ be a natural number. Assume that $k$ is less than or equal to $n$. Then the cardinality of the number of combinations of $A$ and $k$ is equal to the binomial coefficient of $n$ and $k$.

Thm58. Let $A$ be a set. Let $n \in N$. Assume that the cardinality of $A$ is equal to $n$. Let $k \in N$. Assume that $k$ is less than or equal to $n$. Then the cardinality of the number of combinations of $A$ and $k$ is equal to the binomial coefficient of $n$ and $k$.

Thm58. Let $A$ be a set. Let $n$ be a natural number. Assume that $| A | = n$. Let $k$ be a natural number. Assume that $k \leq n$. Then $| \binom{ A }{ k}| = \binom{ n }{ k}$.

Thm58. For all sets $A$, for all natural numbers $n$, if $| A | = n$, then for all natural numbers $k$, if $k \leq n$, then $| \binom{ A }{ k}| = \binom{ n }{ k}$.

Thm58. For all natural numbers $n$, if $| A | = n$, then for all natural numbers $k$, if $k \leq n$, then $| \binom{ A }{ k}| = \binom{ n }{ k}$ for every set $A$.

Thm58. For all natural numbers $n$, if $| A | = n$, then for all natural numbers $k$, if $k \leq n$, then $| \binom{ A }{ k}| = \binom{ n }{ k}$ for all sets $A$.

Thm58. Let $A$ be a set. Then for all natural numbers $n$, if $| A | = n$, then for all natural numbers $k$, if $k \leq n$, then $| \binom{ A }{ k}| = \binom{ n }{ k}$.

Thm58. Let $A$ be a set. Then if $| A | = n$, then for all natural numbers $k$, if $k \leq n$, then $| \binom{ A }{ k}| = \binom{ n }{ k}$ for every natural number $n$.

Thm58. Let $A$ be a set. Then if $| A | = n$, then for all natural numbers $k$, if $k \leq n$, then $| \binom{ A }{ k}| = \binom{ n }{ k}$ for all natural numbers $n$.

Thm58. Let $A$ be a set. Let $n$ be a natural number. Then if $| A | = n$, then for all natural numbers $k$, if $k \leq n$, then $| \binom{ A }{ k}| = \binom{ n }{ k}$.

Thm58. Let $A$ be a set. Let $n$ be a natural number. Then $| A | = n$, only if for all natural numbers $k$, if $k \leq n$, then $| \binom{ A }{ k}| = \binom{ n }{ k}$.

Thm58. Let $A$ be a set. Let $n \in N$. Then if $| A | = n$, then for all natural numbers $k$, if $k \leq n$, then $| \binom{ A }{ k}| = \binom{ n }{ k}$.

Thm58. Let $A$ be a set. Let $n \in N$. Then $| A | = n$, only if for all natural numbers $k$, if $k \leq n$, then $| \binom{ A }{ k}| = \binom{ n }{ k}$.

Thm58. Let $A$ be a set. Let $n$ be a natural number. Assume that $| A | = n$. Then for all natural numbers $k$, if $k \leq n$, then $| \binom{ A }{ k}| = \binom{ n }{ k}$.

Thm58. Let $A$ be a set. Let $n$ be a natural number. Assume that $| A | = n$. Then if $k \leq n$, then $| \binom{ A }{ k}| = \binom{ n }{ k}$ for every natural number $k$.

Thm58. Let $A$ be a set. Let $n$ be a natural number. Assume that $| A | = n$. Then if $k \leq n$, then $| \binom{ A }{ k}| = \binom{ n }{ k}$ for all natural numbers $k$.

Thm58. Let $A$ be a set. Let $n \in N$. Assume that $| A | = n$. Then for all natural numbers $k$, if $k \leq n$, then $| \binom{ A }{ k}| = \binom{ n }{ k}$.

Thm58. Let $A$ be a set. Let $n \in N$. Assume that $| A | = n$. Then if $k \leq n$, then $| \binom{ A }{ k}| = \binom{ n }{ k}$ for every natural number $k$.

Thm58. Let $A$ be a set. Let $n \in N$. Assume that $| A | = n$. Then if $k \leq n$, then $| \binom{ A }{ k}| = \binom{ n }{ k}$ for all natural numbers $k$.

Thm58. Let $A$ be a set. Let $n$ be a natural number. Assume that $| A | = n$. Let $k$ be a natural number. Then if $k \leq n$, then $| \binom{ A }{ k}| = \binom{ n }{ k}$.

Thm58. Let $A$ be a set. Let $n$ be a natural number. Assume that $| A | = n$. Let $k$ be a natural number. Then $k \leq n$, only if $| \binom{ A }{ k}| = \binom{ n }{ k}$.

Thm58. Let $A$ be a set. Let $n$ be a natural number. Assume that $| A | = n$. Let $k$ be a natural number. Then $k \leq n$ implies $| \binom{ A }{ k}| = \binom{ n }{ k}$.

Thm58. Let $A$ be a set. Let $n$ be a natural number. Assume that $| A | = n$. Let $k \in N$. Then if $k \leq n$, then $| \binom{ A }{ k}| = \binom{ n }{ k}$.

Thm58. Let $A$ be a set. Let $n$ be a natural number. Assume that $| A | = n$. Let $k \in N$. Then $k \leq n$, only if $| \binom{ A }{ k}| = \binom{ n }{ k}$.

Thm58. Let $A$ be a set. Let $n$ be a natural number. Assume that $| A | = n$. Let $k \in N$. Then $k \leq n$ implies $| \binom{ A }{ k}| = \binom{ n }{ k}$.

Thm58. Let $A$ be a set. Let $n \in N$. Assume that $| A | = n$. Let $k$ be a natural number. Then if $k \leq n$, then $| \binom{ A }{ k}| = \binom{ n }{ k}$.

Thm58. Let $A$ be a set. Let $n \in N$. Assume that $| A | = n$. Let $k$ be a natural number. Then $k \leq n$, only if $| \binom{ A }{ k}| = \binom{ n }{ k}$.

Thm58. Let $A$ be a set. Let $n \in N$. Assume that $| A | = n$. Let $k$ be a natural number. Then $k \leq n$ implies $| \binom{ A }{ k}| = \binom{ n }{ k}$.

Thm58. Let $A$ be a set. Let $n \in N$. Assume that $| A | = n$. Let $k \in N$. Then if $k \leq n$, then $| \binom{ A }{ k}| = \binom{ n }{ k}$.

Thm58. Let $A$ be a set. Let $n \in N$. Assume that $| A | = n$. Let $k \in N$. Then $k \leq n$, only if $| \binom{ A }{ k}| = \binom{ n }{ k}$.

Thm58. Let $A$ be a set. Let $n \in N$. Assume that $| A | = n$. Let $k \in N$. Then $k \leq n$ implies $| \binom{ A }{ k}| = \binom{ n }{ k}$.

Thm58. Let $A$ be a set. Let $n$ be a natural number. Assume that $| A | = n$. Let $k \in N$. Assume that $k \leq n$. Then $| \binom{ A }{ k}| = \binom{ n }{ k}$.

Thm58. Let $A$ be a set. Let $n \in N$. Assume that $| A | = n$. Let $k$ be a natural number. Assume that $k \leq n$. Then $| \binom{ A }{ k}| = \binom{ n }{ k}$.

Thm58. Let $A$ be a set. Let $n \in N$. Assume that $| A | = n$. Let $k \in N$. Assume that $k \leq n$. Then $| \binom{ A }{ k}| = \binom{ n }{ k}$.

Thm78. Let $u$ and $v$ be instances of vectors. Then we can prove that the dot product of $u$ and $v$ is less than or equal to the product of the norm of $u$ and the norm of $v$.

Thm78. For all instances $u$ and $v$ of vectors, we can prove that the dot product of $u$ and $v$ is less than or equal to the product of the norm of $u$ and the norm of $v$.

Thm78. We can prove that the dot product of $u$ and $v$ is less than or equal to the product of the norm of $u$ and the norm of $v$ for all instances $u$ and $v$ of vectors.

Thm78. Let $u$ and $v$ be vectors. Then the dot product of $u$ and $v$ is less than or equal to the product of the norm of $u$ and the norm of $v$.

Thm78. For all vectors $u$ and $v$, the dot product of $u$ and $v$ is less than or equal to the product of the norm of $u$ and the norm of $v$.

Thm78. The dot product of $u$ and $v$ is less than or equal to the product of the norm of $u$ and the norm of $v$ for all vectors $u$ and $v$.

Thm78. Let $u$ and $v$ be vectors. Then $u \cdot v \leq \| u \| \| v \|$.

Thm78. For all vectors $u$ and $v$, $u \cdot v \leq \| u \| \| v \|$.

Thm78. $u \cdot v \leq \| u \| \| v \|$ for all vectors $u$ and $v$.

Thm78a. Let $u$ and $v$ be instances of vectors. Then we can prove that if $u$ is orthogonal to $v$, then the dot product of $u$ and $v$ is equal to $0$.

Thm78a. For all instances $u$ and $v$ of vectors, we can prove that if $u$ is orthogonal to $v$, then the dot product of $u$ and $v$ is equal to $0$.

Thm78a. We can prove that if $u$ is orthogonal to $v$, then the dot product of $u$ and $v$ is equal to $0$ for all instances $u$ and $v$ of vectors.

Thm78a. For all instances $u$ and $v$ of vectors, we can prove that $u$ is orthogonal to $v$, only if the dot product of $u$ and $v$ is equal to $0$.

Thm78a. We can prove that $u$ is orthogonal to $v$, only if the dot product of $u$ and $v$ is equal to $0$ for all instances $u$ and $v$ of vectors.

Thm78a. Let $u$ and $v$ be instances of vectors. Then we can prove that $u$ is orthogonal to $v$, only if the dot product of $u$ and $v$ is equal to $0$.

Thm78a. Let $u$ and $v$ be vectors. Then if $u$ is orthogonal to $v$, then the dot product of $u$ and $v$ is equal to $0$.

Thm78a. For all vectors $u$ and $v$, if $u$ is orthogonal to $v$, then the dot product of $u$ and $v$ is equal to $0$.

Thm78a. If $u$ is orthogonal to $v$, then the dot product of $u$ and $v$ is equal to $0$ for all vectors $u$ and $v$.

Thm78a. For all vectors $u$ and $v$, $u$ is orthogonal to $v$, only if the dot product of $u$ and $v$ is equal to $0$.

Thm78a. $u$ is orthogonal to $v$, only if the dot product of $u$ and $v$ is equal to $0$ for all vectors $u$ and $v$.

Thm78a. Let $u$ and $v$ be vectors. Then $u$ is orthogonal to $v$, only if the dot product of $u$ and $v$ is equal to $0$.

Thm78a. Let $u$ and $v$ be vectors. Then if $u \perp v$, then $u \cdot v = 0$.

Thm78a. For all vectors $u$ and $v$, if $u \perp v$, then $u \cdot v = 0$.

Thm78a. If $u \perp v$, then $u \cdot v = 0$ for all vectors $u$ and $v$.

Thm78a. For all vectors $u$ and $v$, $u \perp v$, only if $u \cdot v = 0$.

Thm78a. $u \perp v$, only if $u \cdot v = 0$ for all vectors $u$ and $v$.

Thm78a. For all vectors $u$ and $v$, $u \perp v$ implies $u \cdot v = 0$.

Thm78a. $u \perp v$ implies $u \cdot v = 0$ for all vectors $u$ and $v$.

Thm78a. Let $u$ and $v$ be vectors. Then $u \perp v$, only if $u \cdot v = 0$.

Thm78a. Let $u$ and $v$ be vectors. Then $u \perp v$ implies $u \cdot v = 0$.

Thm91. Let $u$ and $v$ be instances of vectors. Then we can prove that the norm of the sum of $u$ and $v$ is less than or equal to the sum of the norm of $u$ and the norm of $v$.

Thm91. For all instances $u$ and $v$ of vectors, we can prove that the norm of the sum of $u$ and $v$ is less than or equal to the sum of the norm of $u$ and the norm of $v$.

Thm91. We can prove that the norm of the sum of $u$ and $v$ is less than or equal to the sum of the norm of $u$ and the norm of $v$ for all instances $u$ and $v$ of vectors.

Thm91. Let $u$ and $v$ be vectors. Then the norm of the sum of $u$ and $v$ is less than or equal to the sum of the norm of $u$ and the norm of $v$.

Thm91. For all vectors $u$ and $v$, the norm of the sum of $u$ and $v$ is less than or equal to the sum of the norm of $u$ and the norm of $v$.

Thm91. The norm of the sum of $u$ and $v$ is less than or equal to the sum of the norm of $u$ and the norm of $v$ for all vectors $u$ and $v$.

Thm91. Let $u$ and $v$ be vectors. Then $\| u + v \| \leq \| u \| + \| v \|$.

Thm91. For all vectors $u$ and $v$, $\| u + v \| \leq \| u \| + \| v \|$.

Thm91. $\| u + v \| \leq \| u \| + \| v \|$ for all vectors $u$ and $v$.

Thm98. Let $n$ be an instance of natural numbers. Then we can prove that there exists a natural number $p$, such that $p$ is prime and $n$ is less than $p$ and $p$ is less than the product of $2$ and $n$.

Thm98. For all instances $n$ of natural numbers, we can prove that there exists a natural number $p$, such that $p$ is prime and $n$ is less than $p$ and $p$ is less than the product of $2$ and $n$.

Thm98. We can prove that there exists a natural number $p$, such that $p$ is prime and $n$ is less than $p$ and $p$ is less than the product of $2$ and $n$ for every instance $n$ of natural numbers.

Thm98. We can prove that there exists a natural number $p$, such that $p$ is prime and $n$ is less than $p$ and $p$ is less than the product of $2$ and $n$ for all instances $n$ of natural numbers.

Thm98. For all instances $n$ of natural numbers, we can prove that $p$ is prime and $n$ is less than $p$ and $p$ is less than the product of $2$ and $n$ for a natural number $p$.

Thm98. We can prove that $p$ is prime and $n$ is less than $p$ and $p$ is less than the product of $2$ and $n$ for a natural number $p$ for every instance $n$ of natural numbers.

Thm98. We can prove that $p$ is prime and $n$ is less than $p$ and $p$ is less than the product of $2$ and $n$ for a natural number $p$ for all instances $n$ of natural numbers.

Thm98. For all instances $n$ of natural numbers, we can prove that $p$ is prime and $n$ is less than $p$ and $p$ is less than the product of $2$ and $n$ for some natural number $p$.

Thm98. We can prove that $p$ is prime and $n$ is less than $p$ and $p$ is less than the product of $2$ and $n$ for some natural number $p$ for every instance $n$ of natural numbers.

Thm98. We can prove that $p$ is prime and $n$ is less than $p$ and $p$ is less than the product of $2$ and $n$ for some natural number $p$ for all instances $n$ of natural numbers.

Thm98. Let $n$ be an instance of natural numbers. Then we can prove that $p$ is prime and $n$ is less than $p$ and $p$ is less than the product of $2$ and $n$ for a natural number $p$.

Thm98. Let $n$ be an instance of natural numbers. Then we can prove that $p$ is prime and $n$ is less than $p$ and $p$ is less than the product of $2$ and $n$ for some natural number $p$.

Thm98. Let $n$ be a natural number. Then there exists a natural number $p$, such that $p$ is prime and $n$ is less than $p$ and $p$ is less than the product of $2$ and $n$.

Thm98. For all natural numbers $n$, there exists a natural number $p$, such that $p$ is prime and $n$ is less than $p$ and $p$ is less than the product of $2$ and $n$.

Thm98. There exists a natural number $p$, such that $p$ is prime and $n$ is less than $p$ and $p$ is less than the product of $2$ and $n$ for every natural number $n$.

Thm98. There exists a natural number $p$, such that $p$ is prime and $n$ is less than $p$ and $p$ is less than the product of $2$ and $n$ for all natural numbers $n$.

Thm98. For all natural numbers $n$, $p$ is prime and $n$ is less than $p$ and $p$ is less than the product of $2$ and $n$ for a natural number $p$.

Thm98. $p$ is prime and $n$ is less than $p$ and $p$ is less than the product of $2$ and $n$ for a natural number $p$ for every natural number $n$.

Thm98. $p$ is prime and $n$ is less than $p$ and $p$ is less than the product of $2$ and $n$ for a natural number $p$ for all natural numbers $n$.

Thm98. For all natural numbers $n$, $p$ is prime and $n$ is less than $p$ and $p$ is less than the product of $2$ and $n$ for some natural number $p$.

Thm98. $p$ is prime and $n$ is less than $p$ and $p$ is less than the product of $2$ and $n$ for some natural number $p$ for every natural number $n$.

Thm98. $p$ is prime and $n$ is less than $p$ and $p$ is less than the product of $2$ and $n$ for some natural number $p$ for all natural numbers $n$.

Thm98. Let $n$ be a natural number. Then $p$ is prime and $n$ is less than $p$ and $p$ is less than the product of $2$ and $n$ for a natural number $p$.

Thm98. Let $n$ be a natural number. Then $p$ is prime and $n$ is less than $p$ and $p$ is less than the product of $2$ and $n$ for some natural number $p$.

Thm98. Let $n \in N$. Then there exists a natural number $p$, such that $p$ is prime and $n$ is less than $p$ and $p$ is less than the product of $2$ and $n$.

Thm98. Let $n \in N$. Then $p$ is prime and $n$ is less than $p$ and $p$ is less than the product of $2$ and $n$ for a natural number $p$.

Thm98. Let $n \in N$. Then $p$ is prime and $n$ is less than $p$ and $p$ is less than the product of $2$ and $n$ for some natural number $p$.

Thm98. Let $n$ be a natural number. Then there exists a natural number $p$, such that $p$ is prime and $n < p$ and $p < 2 n$.

Thm98. For all natural numbers $n$, there exists a natural number $p$, such that $p$ is prime and $n < p$ and $p < 2 n$.

Thm98. There exists a natural number $p$, such that $p$ is prime and $n < p$ and $p < 2 n$ for every natural number $n$.

Thm98. There exists a natural number $p$, such that $p$ is prime and $n < p$ and $p < 2 n$ for all natural numbers $n$.

Thm98. For all natural numbers $n$, $p$ is prime and $n < p$ and $p < 2 n$ for a natural number $p$.

Thm98. $p$ is prime and $n < p$ and $p < 2 n$ for a natural number $p$ for every natural number $n$.

Thm98. $p$ is prime and $n < p$ and $p < 2 n$ for a natural number $p$ for all natural numbers $n$.

Thm98. For all natural numbers $n$, $p$ is prime and $n < p$ and $p < 2 n$ for some natural number $p$.

Thm98. $p$ is prime and $n < p$ and $p < 2 n$ for some natural number $p$ for every natural number $n$.

Thm98. $p$ is prime and $n < p$ and $p < 2 n$ for some natural number $p$ for all natural numbers $n$.

Thm98. Let $n$ be a natural number. Then $p$ is prime and $n < p$ and $p < 2 n$ for a natural number $p$.

Thm98. Let $n$ be a natural number. Then $p$ is prime and $n < p$ and $p < 2 n$ for some natural number $p$.

Thm98. Let $n$ be a natural number. Then $p$ is prime and $n < p < 2 n$ for a natural number $p$.

Thm98. Let $n$ be a natural number. Then $p$ is prime and $n < p < 2 n$ for some natural number $p$.

Thm98. Let $n \in N$. Then there exists a natural number $p$, such that $p$ is prime and $n < p$ and $p < 2 n$.

Thm98. Let $n \in N$. Then $p$ is prime and $n < p$ and $p < 2 n$ for a natural number $p$.

Thm98. Let $n \in N$. Then $p$ is prime and $n < p$ and $p < 2 n$ for some natural number $p$.

Thm98. Let $n \in N$. Then $p$ is prime and $n < p < 2 n$ for a natural number $p$.

Thm98. Let $n \in N$. Then $p$ is prime and $n < p < 2 n$ for some natural number $p$.

Thm98. Let $n$ be a natural number. Then there exists a natural number $p$, such that $p$ is prime, $n < p$ and $p < 2 n$.

Thm98. For all natural numbers $n$, there exists a natural number $p$, such that $p$ is prime, $n < p$ and $p < 2 n$.

Thm98. There exists a natural number $p$, such that $p$ is prime, $n < p$ and $p < 2 n$ for every natural number $n$.

Thm98. There exists a natural number $p$, such that $p$ is prime, $n < p$ and $p < 2 n$ for all natural numbers $n$.

Thm98. For all natural numbers $n$, $p$ is prime, $n < p$ and $p < 2 n$ for a natural number $p$.

Thm98. $p$ is prime, $n < p$ and $p < 2 n$ for a natural number $p$ for every natural number $n$.

Thm98. $p$ is prime, $n < p$ and $p < 2 n$ for a natural number $p$ for all natural numbers $n$.

Thm98. For all natural numbers $n$, $p$ is prime, $n < p$ and $p < 2 n$ for some natural number $p$.

Thm98. $p$ is prime, $n < p$ and $p < 2 n$ for some natural number $p$ for every natural number $n$.

Thm98. $p$ is prime, $n < p$ and $p < 2 n$ for some natural number $p$ for all natural numbers $n$.

Thm98. Let $n$ be a natural number. Then $p$ is prime, $n < p$ and $p < 2 n$ for a natural number $p$.

Thm98. Let $n$ be a natural number. Then $p$ is prime, $n < p$ and $p < 2 n$ for some natural number $p$.

Thm98. Let $n \in N$. Then there exists a natural number $p$, such that $p$ is prime, $n < p$ and $p < 2 n$.

Thm98. Let $n \in N$. Then $p$ is prime, $n < p$ and $p < 2 n$ for a natural number $p$.

Thm98. Let $n \in N$. Then $p$ is prime, $n < p$ and $p < 2 n$ for some natural number $p$.

\end{document}
