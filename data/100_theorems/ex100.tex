\documentclass{article}

%\usepackage[margin=1.0in,footskip=0.25in]{geometry}

%\usepackage[english]{babel}
\usepackage{naproche}
\usepackage{amssymb}

\setlength\parindent{0pt}

\newcommand{\Prod}[3]{#1_{#2} \cdots #1_{#3}}
\newcommand{\Seq}[2]{\{#1,\dots,#2\}}
\newcommand{\FinSet}[3]{\{#1_{#2},\dots,#1_{#3}\}}
\newcommand{\Primes}{\mathbb{P}}
\newcommand{\pow}{{\cal P}}
\newcommand{\range}{\operatorname{ran}}
\newcommand{\inv}[1]{#1^{-1}}
\newcommand{\sset}[2]{\{#1\}_{#2}}
\newcommand{\sumgeom}[2]{\sum_{0 \leq i < #2} {#1}^i}
\newcommand{\sumarith}[3]{\sum_{i = 1}^{#3}(#1 + #2 \cdot i)}

\begin{document}
\subsection*{1}

$ q^{2} = p $ for no positive rational number $ q $.

$ q^{2} = p $ millekään positiiviselle luvulle $ q $.

$ q^{2} = p $ pour aucun nombre rationnel positif $ q $.

$ q^{2} = p $ für keine positive rationale Zahl $ q $.

$ q^{2} = p $ per nessuno numero razionale positivo $ q $.

$ q^{2} = p $ para nenhum número racional positivo $ q $.

$ q^{2} = p $ para ningún número racional positivo $ q $.

$ q^{2} = p $ för inget positivt rationellt tal $ q $.

\subsection*{2}

The collection of prime numbers is infinite.

Kokoelma alkulukuja on ääretön.

La collection de nombres premiers est infinie.

Die Gesamtheit von Primzahlen ist unendlich.

La collezione di numeri primi è infinita.

O collection de números primos é infinito.

La colección de números primos es infinita.

Samlingen av primtal är oändlig.

\subsection*{3}

Let x, y be sets. $ x $ and $ y $ are equinumerous iff there exists a injective map from $ x $ to $ y $ and there exists an injective map from $ y $ to $ x $.

Olkoot x, y joukkoja. $ x $ ja $ y $ ovat yhtämahtavia jos ja vain jos on olemassa injektiivinen kuvaus $ x $:sta $ y $:an ja on olemassa moduli kuvaus $ y $:sta $ x $:an.

Soient x, y des ensembles. $ x $ et $ y $ sont équinombreux si et seulement si il existe une correspondance injective de $ x $ à $ y $ et il existe une application injective de $ y $ à $ x $.

Seien x, y Mengen. $ x $ und $ y $ sind gleichzahlig wenn und genau dann wenn es eine injektive Abbildung aus $ x $ nach $ y $ gibt und es eine injektive Abbildung aus $ y $ nach $ x $ gibt.

Siano insiemi x, y. $ x $ e $ y $ sono equinumerosi se e solo se esiste una mappa iniettiva da $ x $ a $ y $ ed esiste una mappa iniettiva da $ y $ a $ x $.

Deixe x, y ser conjuntos. $ x $ e $ y $ são equinumeiros se e só se existe uma função injetiva de $ x $ a $ y $ e existe uma aplicação injetiva de $ y $ a $ x $.

Supongamosnos que x, y son conjuntos. $ x $ y $ y $ son equinumerosos si y solo si existe una función inyectiva de $ x $ a $ y $ y existe una función inyectiva de $ y $ a $ x $.

Låt x, y vara mängder. $ x $ och $ y $ är liktaliga om och endast om det finns en injektiv avbildning från $ x $ till $ y $ och det finns en injektiv avbildning från $ y $ till $ x $.

\subsection*{4}

For all finite sets $ X $ and all natural numbers $ n $, if $ |X| = n $, then $ \pow(X) $ is finite and $ |\pow(X)| = 2^{n} $.

Kaikille äärellisille joukoille $ X $ ja kaikille luonnollisille luvuille $ n $, jos $ |X| = n $, niin $ \pow(X) $ on äärellinen ja $ |\pow(X)| = 2^{n} $.

Pour tous les ensembles finis $ X $ et tous les entiers naturels $ n $, si $ |X| = n $, alors $ \pow(X) $ est fini et $ |\pow(X)| = 2^{n} $.

Für alle endlichen Mengen $ X $ und alle natürlichen Zahlen $ n $, wenn $ |X| = n $, dann ist $ \pow(X) $ endlich und $ |\pow(X)| = 2^{n} $.

Per tutti gli insiemi finiti $ X $ e tutti i numeri naturali $ n $, se $ |X| = n $, allora $ \pow(X) $ è finito e $ |\pow(X)| = 2^{n} $.

Para todos os conjuntos finitos $ X $ e todos os números naturais $ n $, se $ |X| = n $, então $ \pow(X) $ é finito e $ |\pow(X)| = 2^{n} $.

Para todos los conjuntos finitos $ X $ y todos los números naturales $ n $, si $ |X| = n $, entonces $ \pow(X) $ es finito y $ |\pow(X)| = 2^{n} $.

För alla ändliga mängder $ X $ och alla naturliga tal $ n $, om $ |X| = n $, så är ändligt $ \pow(X) $ och $ |\pow(X)| = 2^{n} $.

\subsection*{5}

Let $ s, t $ be real numbers such that $ s < t $. Then there exists a real number $ r $ such that $ s < r < t $.

Olkoot $ s, t $ reaalilukuja siten että $ s < t $. Silloin on olemassa reaaliluku $ r $ siten että $ s < r < t $.

Soient $ s, t $ des nombres tel que $ s < t $. Alors il existe un nombre $ r $ tel que $ s < r < t $.

Seien $ s, t $ reelle Zahlen derart dass $ s < t $. Dann gibt es eine reelle Zahl $ r $ derart dass $ s < r < t $.

Siano numeri tale che $ s < t $ $ s, t $. Allora esiste un numero $ r $ tale che $ s < r < t $.

Deixe $ s, t $ ser números tal que $ s < t $. Então existe um número $ r $ tal que $ s < r < t $.

Supongamosnos que $ s, t $ son números tal que $ s < t $. Entonces existe un número $ r $ tal que $ s < r < t $.

Låt $ s, t $ vara tal så att $ s < t $. Då finns det ett tal $ r $ så att $ s < r < t $.

\subsection*{6}

Let $ M $ be a set. Then there exists no surjection from $ M $ onto the powerset of $ M $.

Olkoon $ M $ joukko. Silloin ei ole olemassa mitään surjektiota $ M $:sta potenssijoukolle $ M $:n.

Soit $ M $ un ensemble. Alors il n'existe aucune surjection de $ M $ sur l'ensemble puissance de $ M $.

Sei $ M $ eine Menge. Dann gibt es keine Surjektion aus $ M $ auf die Potenzmenge $ M $.

Sia un insieme $ M $. Allora non esiste nessuna suriezione da $ M $ sull'insieme delle parti di $ M $.

Deixe $ M $ ser um conjunto. Então não existe nenhuma sobrejecção de $ M $ sobre o conjunto de potência de $ M $.

Supongamosnos que $ M $ es un conjunto. Entonces no existe ninguna sobreyección de $ M $ sobre el conjunto potencia de $ M $.

Låt $ M $ vara en mängd. Då finns det ingen surjektion från $ M $ på potensmängden av $ M $.

\subsection*{7}

$ \sumgeom{x}{n} = \frac{1 - x^{n}}{1 - x} $ for all natural numbers $ n $.

$ \sumgeom{x}{n} = \frac{1 - x^{n}}{1 - x} $ kaikille luonnollisille luvuille $ n $.

$ \sumgeom{x}{n} = \frac{1 - x^{n}}{1 - x} $ pour tous les entiers naturels $ n $.

$ \sumgeom{x}{n} = \frac{1 - x^{n}}{1 - x} $ für alle natürlichen Zahlen $ n $.

$ \sumgeom{x}{n} = \frac{1 - x^{n}}{1 - x} $ per tutti i numeri naturali $ n $.

$ \sumgeom{x}{n} = \frac{1 - x^{n}}{1 - x} $ para todos os números naturais $ n $.

$ \sumgeom{x}{n} = \frac{1 - x^{n}}{1 - x} $ para todos los números naturales $ n $.

$ \sumgeom{x}{n} = \frac{1 - x^{n}}{1 - x} $ för alla naturliga tal $ n $.

\subsection*{8}

$ \sumarith{a}{d}{n} = n \cdot ( a + \frac{(n + 1) \cdot d}{2}). $.

$ \sumarith{a}{d}{n} = n \cdot ( a + \frac{(n + 1) \cdot d}{2}). $.

$ \sumarith{a}{d}{n} = n \cdot ( a + \frac{(n + 1) \cdot d}{2}). $.

$ \sumarith{a}{d}{n} = n \cdot ( a + \frac{(n + 1) \cdot d}{2}). $.

$ \sumarith{a}{d}{n} = n \cdot ( a + \frac{(n + 1) \cdot d}{2}). $.

$ \sumarith{a}{d}{n} = n \cdot ( a + \frac{(n + 1) \cdot d}{2}). $.

$ \sumarith{a}{d}{n} = n \cdot ( a + \frac{(n + 1) \cdot d}{2}). $.

$ \sumarith{a}{d}{n} = n \cdot ( a + \frac{(n + 1) \cdot d}{2}). $.

\subsection*{9}

Let $ m, n $ be natural numbers such that $ m < n $. Then the greatest common divisor of $ m $ and $ n $ is the greatest common divisor of $ n-m $ and $ m $.

Olkoot $ m, n $ luonnollisia lukuja siten että $ m < n $. Silloin suurin yhteinen tekijä $ m $:n ja $ n $:n on suurin yhteinen tekijä $ n-m $:n ja $ m $:n.

Soient $ m, n $ des entiers naturels tel que $ m < n $. Alors le plus grand commun diviseur de $ m $ et de $ n $ est le plus grand commun diviseur de $ n-m $ et de $ m $.

Seien $ m, n $ natürliche Zahlen derart dass $ m < n $. Dann ist der größte gemeinsame Teiler $ m $ und $ n $ der größte gemeinsame Teiler $ n-m $ und $ m $.

Siano numeri naturali tale che $ m < n $ $ m, n $. Allora il massimo comun divisore di $ m $ e $ n $ è il massimo comun divisore di $ n-m $ e $ m $.

Deixe $ m, n $ ser números naturais tal que $ m < n $. Então o máximo divisor comum de $ m $ e $ n $ é o máximo divisor comum de $ n-m $ e $ m $.

Supongamosnos que $ m, n $ son números naturales tal que $ m < n $. Entonces el máximo común divisor de $ m $ y $ n $ es el máximo común divisor de $ n-m $ y $ m $.

Låt $ m, n $ vara naturliga tal så att $ m < n $. Då är den größta gemeinsama Teilern av $ n-m $ och $ m $ den größta gemeinsama Teilern av $ m $ och $ n $.

\subsection*{10}

Assume $ A \subseteq \mathbb{N} $ and $ 0 \in A $ and for all $ n \in A $, $ n + 1 \in A $. Then $ A = \mathbb{N} $.

Oleta, että $ A \subseteq \mathbb{N} $ ja $ 0 \in A $ ja kaikelle $ n \in A $:lle, $ n + 1 \in A $. Silloin $ A = \mathbb{N} $.

Supposons que $ A \subseteq \mathbb{N} $ et $ 0 \in A $ et pour tout $ n \in A $, $ n + 1 \in A $. Alors $ A = \mathbb{N} $.

Wir nehmen an, dass $ A \subseteq \mathbb{N} $ und $ 0 \in A $ und für alle $ n \in A $, $ n + 1 \in A $. Dann $ A = \mathbb{N} $.

Supponiamo che $ A \subseteq \mathbb{N} $ e $ 0 \in A $ e per tutto $ n \in A $, $ n + 1 \in A $. Allora $ A = \mathbb{N} $.

Admitemos que $ A \subseteq \mathbb{N} $ e $ 0 \in A $ e para todo $ n \in A $, $ n + 1 \in A $. Então $ A = \mathbb{N} $.

Supongamosnos que $ A \subseteq \mathbb{N} $ y $ 0 \in A $ y para todo $ n \in A $, $ n + 1 \in A $. Entonces $ A = \mathbb{N} $.

Vi antar att $ A \subseteq \mathbb{N} $ och $ 0 \in A $ och för allt $ n \in A $, $ n + 1 \in A $. Då $ A = \mathbb{N} $.

\end{document}
