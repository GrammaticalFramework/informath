\documentclass{article}

%\usepackage[english]{babel}
\usepackage{naproche}
\usepackage{amssymb}

\begin{document}
\setlength\parindent{0pt}

\newcommand{\Prod}[3]{#1_{#2} \cdots #1_{#3}}
\newcommand{\Seq}[2]{\{#1,\dots,#2\}}
\newcommand{\FinSet}[3]{\{#1_{#2},\dots,#1_{#3}\}}
\newcommand{\Primes}{\mathbb{P}}
\newcommand{\pow}{{\cal P}}
\newcommand{\range}{\operatorname{ran}}
\newcommand{\inv}[1]{#1^{-1}}
\newcommand{\sset}[2]{\{#1\}_{#2}}
\newcommand{\sumgeom}[2]{\sum_{0 \leq i < #2} {#1}^i}
\newcommand{\sumarith}[3]{\sum_{i = 1}^{#3}(#1 + #2 \cdot i)}

 $q^{2} = p$ for no positive rational number $q$.

 $q^{2} = p$ pour aucun nombre rationnel positif $q $.

 $q^{2} = p$ für keine positive rationale Zahl $q $.


The collection of prime natural numbers is infinite.

La collection d'entiers naturels primaires est infinie.

Die Sammlung von unteilbaren natürlichen Zahlen ist unendlich.


Let x, y be sets. $x$ and $y$ are equinumerous iff there exists a injective map from $x$ to $y$ and there exists an injective map from $y$ to $x$.

Soient x, y des ensembles. $x$ et $y$ sont équinombreux si et seulement si il existe une correspondance injective de $x$ à $y$ et il existe une application injective de $y$ à $x $.

Seien x, y Mengen. $x$ und $y$ sind gleichzahlig wenn und genau dann wenn es eine injektive Abbildung aus $x$ nach $y$ gibt und es eine injektive Abbildung aus $y$ nach $x$ gibt.


For all finite sets $X$ and all natural numbers $n$, if $|X| = n$, then $\pow(X)$ is finite and $|\pow(X)| = 2^{n}$.

Pour tous les ensembles finis $X$ et tous les entiers naturels $n $, si$ |X| = n $, alors $\pow(X)$ est fini et $|\pow(X)| = 2^{n} $.

Für alle endlichen Mengen $X$ und alle natürlichen Zahlen $n$, wenn $|X| = n$, dann ist $\pow(X)$ endlich und $|\pow(X)| = 2^{n} $.


Let $s , t$ be real numbers such that $s < t$. then there exists a real number $z$ such that $s < r < t$.

Soient $s , t$ des nombres tel que $s < t $. alors il existe un nombre$ z $tel que$ s < r < t $.

Seien $s , t$ reelle Zahlen derart dass $s < t $. dann gibt es eine reelle Zahl$ z $derart dass$ s < r < t $.


Let $M$ be a set. Then there exists no surjection from $M$ onto the powerset of $M$.

Soit $M$ un ensemble. alors il n'existe aucune surjection de $M$ sur l'ensemble puissance de $M $.

Sei $M$ eine Menge. dann gibt es keine Surjektion aus $M$ auf die Potenzmenge $M $.


 $\sumgeom{x}{n} = \frac{1 - x^{n}}{1 - x}$ for all natural numbers $n$.

 $\sumgeom{x}{n} = \frac{1 - x^{n}}{1 - x}$ pour tous les entiers naturels $n $.

 $\sumgeom{x}{n} = \frac{1 - x^{n}}{1 - x}$ für alle natürlichen Zahlen $n $.


 $\sumarith{a}{d}{n} = n \cdot (a + \frac{(n + 1) \cdot d}{2}).$.

 $\sumarith{a}{d}{n} = n \cdot (a + \frac{(n + 1) \cdot d}{2}). $.

 $\sumarith{a}{d}{n} = n \cdot (a + \frac{(n + 1) \cdot d}{2}). $.


Let $m , n$ be natural numbers such that $m < n$. then the greatest common divisor of $m$ and $n$ is the greatest common divisor of $n-m$ and $m$.

Soient $m , n$ des entiers naturels tel que $m < n $. alors le plus grand commun diviseur de$ m $et de$ n $est le plus grand commun diviseur de$ n-m $et de$ m $.

Seien $m , n$ natürliche Zahlen derart dass $m < n $. dann ist der größte gemeinsame Teiler$ m $und$ n $der größte gemeinsame Teiler$ n-m $und$ m $.


Assume $A \subseteq \mathbb{N}$ and $0 \in A$ and for all $n \in A$, $n + 1 \in A$. then $A = \mathbb{N}$.

Supposons que $A \subseteq \mathbb{N}$ et $0 \in A$ et pour tout $n \in A $,$ n + 1 \in A $. alors $A = \mathbb{N} $.

Wir nehmen an, dass $A \subseteq \mathbb{N}$ und $0 \in A$ und für alle $n \in A$, $n + 1 \in A $. dann$ a = \mathbb{N} $.


Success 10 failure 0


\end{document}

success 10 failure 0
